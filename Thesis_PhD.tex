% \documentclass[draft]{book}
\documentclass[draft]{book}

\usepackage{graphicx}

\usepackage{epstopdf}
\usepackage{epsfig}

\usepackage{float}
\usepackage{amssymb,amsmath}
\newcommand{\bm}[1]{\boldsymbol{#1}}

\usepackage[makeroom]{cancel} %for cancel out in eq.

\usepackage{multirow}
\usepackage{fullpage}
\usepackage{appendix}
\usepackage{setspace}

\usepackage[bf, small, center]{caption}
\setlength{\belowcaptionskip}{10pt}

\usepackage{longtable}

\usepackage{geometry}
\geometry{
%  top=1.0in,  
%  inner=1.5in,
%  outer=1.5in,
  bottom=1.2in,
%  headheight=3ex, 
  headsep=3ex,         
}

\usepackage{fancyhdr}
\pagestyle{fancy}
\fancyhead[LO, RE]{\rightmark}
\fancyhead[LE, RO]{\thepage}
\fancyfoot[]{}
%\renewcommand{\headrulewidth}{0.3pt}

\usepackage{array}
\newcolumntype{L}[1]{>{\raggedright\let\newline\\\arraybackslash\hspace{0pt}}m{#1}}
\newcolumntype{C}[1]{>{\centering\let\newline\\\arraybackslash\hspace{0pt}}m{#1}}
\newcolumntype{R}[1]{>{\raggedleft\let\newline\\\arraybackslash\hspace{0pt}}m{#1}}



\usepackage[phd]{thesisfrontmatter}

% *** NOTE WHICH PROGRAM GENERATED IMAGES OR WHICH POWER POINT THEY ARE FROM
%compare methodology with script for final process
%how much detail in methodology?
%some sort of processing flowchart?
%how much do we care about processing?

\usepackage[parfill]{parskip}    % Activate to begin paragraphs with an empty line rather than an indent


\usepackage{mathtools}


\usepackage{color}
\usepackage{soul}



\DeclareGraphicsRule{.tif}{png}{.png}{`convert #1 `dirname #1`/`basename #1 .tif`.png}

\let\stdsection\chapter  
\renewcommand\chapter{\newpage\stdsection}  

%%%%%%%%%%% for Appendix
\makeatletter
\newcommand\appendix@section[1]{%
  \refstepcounter{section}%
  \orig@section*{Appendix \@Alph\c@section: #1}%
  \addcontentsline{toc}{section}{Appendix \@Alph\c@section: #1}%
}
\let\orig@section\section
\g@addto@macro\appendix{\let\section\appendix@section}
\makeatother

% Added 12-03-13 ----------------------------------------
\usepackage{imakeidx} % to create index
\indexsetup{othercode=\small}
\makeindex[program=makeindex,columns=2,intoc=true,options={-s MyIndex.ist}]
% -------------------------------------------------------
%%%%%%%%%%%%%%

\usepackage{hyperref}
\hypersetup{
    % bookmarks=true,         % show bookmarks bar?
    unicode=false,          % non-Latin characters in AcrobatÕs bookmarks
    pdftoolbar=true,        % show AcrobatÕs toolbar?
    pdfmenubar=true,        % show AcrobatÕs menu?
    pdffitwindow=false,     % window fit to page when opened
    pdfstartview={FitH},    % fits the width of the page to the window
    pdftitle={L8's potential for water constituents retrieval },    % title
    pdfauthor={Javier Concha},     % author
    pdfsubject={Subject},   % subject of the document
    pdfcreator={Creator},   % creator of the document
    pdfproducer={Producer}, % producer of the document
    pdfkeywords={keyword1} {key2} {key3}, % list of keywords
    pdfnewwindow=true,      % links in new window
    colorlinks=true,       % false: boxed links; true: colored links
    linkcolor=blue,          % color of internal links
    citecolor=cyan,        % color of links to bibliography
    filecolor=magenta,      % color of file links
    urlcolor=blue           % color of external links
}

\usepackage[all]{hypcap} % to see figure with hyper ref

\setcounter{secnumdepth}{5}
\setcounter{tocdepth}{5}

% Added 11-12-13 ----------------------------------------

\usepackage{titlesec} % For the spacing after and before chapter title

\titleformat{\chapter}[display]
    {\normalfont\huge\bfseries}{\chaptertitlename\ \thechapter}{10pt}{\Huge}
\titlespacing*{\chapter}{0pt}{40pt}{20pt}

\titlespacing*{\section}{0ex}{2.5ex}{2ex}
\titlespacing*{\subsection}{0ex}{1.5ex}{1ex}

\usepackage{indentfirst}
\setlength{\parindent}{20pt}
\doublespacing

% Added 11-17-13 ----------------------------------------
% Select what to do with todonotes: 
% \usepackage[disable]{todonotes} % notes not showed
\usepackage[draft]{todonotes}   % notes showed

\setlength{\marginparwidth}{3.7cm}

% Added 11-15-13 ----------------------------------------
\includeonly{Introduction,Objectives,Background_and_Theory,Methodology,Results,Summary,Appendix} 
% \includeonly{Methodology} 
% \includeonly{Introduction,Objectives,Background_and_Theory,Methodology,Results} 
\usepackage{tikz} % for flow charts
  \usetikzlibrary{shapes,arrows,positioning,shadows,calc}
\usepackage{colortbl}
\usepackage{caption}
\usepackage{subcaption}
\usepackage{multirow}

\listfiles

\setlength{\headheight}{15pt} % to avoid "Package Fancyhdr Warning: \headheight is too small (0.0pt): Make it at least 12.0pt."

% Added 02-18-14 ----------------------------------------
% \setlength{\abovecaptionskip}{-2ex}
% \setlength{\belowcaptionskip}{-4ex} % space after caption
%******************************************************************************************************


% Added 08-10-14 
\usepackage{enumitem}

% Added 07-28-15 
\usepackage[percent]{overpic}
\usepackage{morefloats} % for the error "Too many unprocessed floats"

%******************************************************************************************************
%******************************************************************************************************
%******************************************************************************************************
\begin{document}

\degreetitle{The Use of Landsat-8 for Monitoring of Fresh and Coastal Water}
\degreeauthor{Javier A. Concha}
% \degreedate{October 7, 2014}
\degreedate{\today{}}
\prevdegreeA{M.S. Rochester Institute of Technology, 2012}
\advisor{Dr. John R. Schott}
\memberA{Dr. Anthony Vodacek}
\memberB{Dr. Charles Bachmann}
\memberC{Dr. Christy Tyler}
% \memberD{Dr. Christy Tyler}

\makeproposaldeclaration
\makePHDproposalapproval
% \makecopyright


% \maketitle

\pagenumbering{roman}

% \chapter*{Abstract}
% \addcontentsline{toc}{chapter}{Abstract}

\begin{abstract}
\setlength{\parindent}{20pt}
The Landsat Data Continuity Mission (LDCM; a.k.a. Landsat-8), recently launched (February 2013), is the next generation of Landsat satellite and continues more than 40 years of uninterrupted imaging acquisition, playing a critical role in monitoring, understanding and managing natural resources such as water. Landsat-8, with its improved spectral bands and radiometric resolution, has the potential to dramatically improve our ability to simultaneously retrieve the three primary coloring agents, chlorophyll (Chl), colored dissolved organic material (CDOM) and suspended material (SM) from water bodies. This work presents an approach to obtain these coloring agents in coastal and fresh waters.

In the Case 2 water problem, the sensor-reaching signal due to water is very small when compared to the signal due to the atmospheric effects. Therefore, adequate atmospheric correction becomes an important first step to accurately retrieving water parameters. As a first approach, a model based empirical line method (ELM) atmospheric correction method converts sensor-reaching radiance to water leaving reflectance. This model employs pseudo invariant feature (PIF) pixels extracted from Landsat images along with an in water radiative transfer model (HydroLight) to obtain the field spectra. Further atmospheric compensation technique based in algorithms currently used in Ocean Color sensors will be investigated.

A look-up-table (LUT) methodology is implemented to retrieve the water parameters. The LUT is created using HydroLight.

Collections of water samples when the satellite passes over the Rochester area are planned for summer 2013 and 2014. Concentration and inherent optical properties (IOPs) measurement will help to validate the methods.

\end{abstract}

% \chapter*{Acknowledgements}
% %*****************************************************************************************************
% \begin{acknowledgements}
% \setlength{\parindent}{20pt}
% To my parents, Osvaldo Concha and Nelly Sepulveda, sister Loretto Concha and brother Alvaro Concha. USGS. Monroe County Environment: Scott, Providencia and Gary. Nina and Rolo. Dr. Christy Taylor. Paul. Aaron Gerace. Alan Wiedmman. Ocean Color community: Emmanuel Boss and attendants of the 2013 Ocean color and instrumentation summer course at the Darling Marine Center. Fulbright commission. My advisor Dr. John Schott. Amanda and Cindy. Classmates: Aly, Bikash, Viraj, Madurima, Peter Sun. Professor Dr. John Kerekes. Collection help: Harold Valdivia. My dear friends: Juan Saldana, Kader, Patrick, Sarada, Susan, Melisa, Sasha, Jeremy. The people at Crossfit Rochester: Joe, Andrew. The Latin Rhythm Dance Club at RIT. My best support here Aixa de Jesus. My adopted Chilean family and dancers in Rochester: Marcia, Rosa and Doña Rosita, Raul, Willy and Marisol, Harold and Nancy. My Chilean friends in U. of R. Felipe, Maria Eugenia, Emilio, Brenda and Pauly. Kimberley Thoms.
% \end{acknowledgements}
%*****************************************************************************************************

\tableofcontents

\listoffigures
\addcontentsline{toc}{chapter}{List of Figures}

\listoftables
\addcontentsline{toc}{chapter}{List of Tables}

% % !TEX root=Thesis_PhD.tex  
% the previous is to reference main .bib
%% CHAPTER
\chapter{Introduction}
\label{ch:introduction} 
\pagenumbering{arabic} 
% What? and So what? What is the paper about, and why should the reader care?
% Topic, observation/discovery, explanation
% What research questions is this paper trying to answer?

% - Establish a territory (what is the field of the work, why is this field important, what has already been done?)
% - Establish a niche (indicate a gap, raise a question, or challenge prior work in this territory)
% - Occupy that niche (outline the purpose and announce the present research; optionally summarize the results).

% Not enough spatial resolution
Ocean color studies at a global scale, such as chlorophyll-{\it a} level trends in oceans, can be investigated using the heritage Ocean Color satellites (e.g. \acrfull{seawifs}, \acrfull{modis}). These satellites satisfied the spatial requirement for these kinds of studies. However, when the region of interest includes coastal or inland waters, which could be considered Case 2 water, their spatial resolution of roughly a thousand meters are not enough to resolve smaller water bodies. These kinds of waters are important for us, because it is these kinds of waters that we have the most interaction with, for drinking and recreation. 

% New medium spatial resolution
The launching of a new generation of high spatial resolution satellites (e.g. \acrshort{nasa}'s Landsat 8 \citep{Irons:2012}, \acrshort{esa}'s Sentinel 2 \citep{Malenovsky:2012}) is opening a complete new era in the remote sensing of coastal and inland waters. New sensor specifications could meet the requirements needed to have available the same kinds of tools (e.g. Chl-{\it a} product) that the first generation of ocean color satellites (a.k.a. heritage Ocean Color satellites), such as MODIS \citep{Esaias1998} and SeaWiFS \citep{McClain2004}, made available for open ocean science more than a decade and half ago. The hope is to have these kinds of tools for coastal and inland waters available at a global scale and on a regular basis, similar to MODIS capabilities for open oceans (e.g. MODIS's Chl-{\it a} product). Although Landsat 8 does not have a daily frequency as MODIS does, its 16-day frequency makes it a good candidate to accomplish this endeavor, and most importantly, prepares the way for future missions with similar spatial resolution specifications (e.g. Sentinel 2, \acrfull{hyspiri}).

% Landsat
The Landsat project has been monitoring the earth for over four decades, being the longest uninterrupted data set available. The Landsat satellites' main mission is to image the land areas of the earth and therefore there are typically no open ocean (case 1 water) images available. This is one of the reason why Landsat satellites have been underestimated by the ocean color community for the study of water bodies. In addition, the Landsat instruments have generally had broad bands and low \acrfull{snr} when compared to heritage ocean color satellites such as SeaWiFS and MODIS. Carrying two instruments onboard, the \acrfull{oli} and the \acrfull{tirs}, Landsat 8 is the first of a new generation of Landsat satellites with state-of-the-art technology \citep{Irons:2012}. Since its launch in 2013, the OLI instrument onboard Landsat 8 has created high expectations in the ocean color community. With its 12-bit quantization and improved SNR, OLI is a big improvement to the Landsat mission. In addition, OLI includes a new coastal band that increased the spectral resolution of the instrument. These improvements are the main drivers to hypothesize that the Landsat 8 satellite will have a better performance in water quality studies than its predecessors. \citet{Roy:2014} stated this potential use of Landsat 8 for fresh and coastal water studies, mainly due to a reported SNR that exceeded expectations and the new coastal band. Therefore, the overall objective of this research is to investigate the performance of Landsat 8 for accurately retrieving water constituents.

% Work done with Landsat 8
\citet{Gerace:2013} demonstrated that the spectral coverage and radiometric resolution of OLI should dramatically improve our ability to simultaneously retrieve the \acrfull{cpas} (chlorophyll-{\it a}, sediments and \acrfull{cdom}) concentrations from water bodies. \citet{Vanhellemont2014} and \citet{Vanhellemont:2015} created a tool to apply the standard algorithm for atmospheric correction over water developed by \citet{Gordon:1994} to Landsat 8 over turbid and extremely turbid waters. In \citet{Vanhellemont:2015}, after atmospheric correction, they used the end product for detection of high concentrations of black sediments, but neither concentration values nor comparison with field measurements were reported. \citet{Franz:2015} describes an implementation of the atmospheric corrections developed by \citet{Gordon:1994} applied to Landsat 8 in the \acrfull{seadas} software package (URL: \url{http://seadas.gsfc.nasa.gov/}). A comparison of the Landsat 8's retrieved \acrfull{rrs} and chlorophyll-{\it a} concentration over Chesapeake Bay with results from MODIS, SeaWiFS and {\it in situ} historical chlorophyll-{\it a} measurements is presented showing a relatively good agreement. Though again, no direct comparison to simultaneously measured values were available.

% Atmospheric Correction done
The retrieval of water components is in general performed in the reflectance domain, so the very first step in this work is to perform a high quality atmospheric correction to the radiance image from Landsat 8. This is a complex task to perform over water because the signal leaving the water that reaches the sensor is very small when compared to the signal reaching the sensor produced by atmospheric scattering. Most of the atmospheric correction algorithms for open oceans (Case 2 waters) are based on the methods developed for ocean color satellites by \citet{Gordon:1994}. These methods are based on the fact that the signal leaving the water does not contribute to the overall signal beyond the \acrfull{nir} part of the spectrum; so the signal reaching the sensor is caused only by atmospheric scattering \citep{Gordon:1994} in those wavelengths. This is known as the {\it black pixel assumption}. This concept can be expanded to the \acrfull{swir} bands when the black pixel assumption is not valid in the NIR bands, which is the case for Case 2 and highly productive Case 1 waters \citep{Wang:2007}. Unfortunately, most of these methods are not suitable for highly turbid coastal water, although different modifications to these algorithms have been suggested \citep{Patt2003}.

% Atmospheric Correction developed in this work
In this work, a different approach for atmospheric correction was developed based on \citet{Raqueno:2003}, \citet{Gerace:2013} and \citet{Pahlevan:2012}. This atmospheric correction algorithm is the \acrfull{mobelm} that uses a combination of an in-water radiative transfer model over water and a Landsat reflectance product to determine the bright and dark pixels in the image \citep{Concha2014SPIE,Concha2015_SPIE}. The MoB-ELM is compared with the standard algorithms based on \citet{Gordon:1994} and with {\it in situ} measurements.

% Retrieval: What it has be done
After having atmospherically corrected the image, the next step is to apply a retrieval algorithm that outputs maps of CPA concentrations. The standard algorithms for CPA retrieval used by the heritage ocean color missions do not retrieve the CPAs simultaneously. An example of these algorithms is the \acrfull{oc3} algorithm for retrieving chlorophyll-{\it a} concentrations \citep{OReilly2000}. This is an empirical algorithm that finds the best fit for a function between a ratio of two bands and a {\it in situ} chlorophyll-{\it a} concentration data set.

% Retrieval: What we are proposing
\citet{Gerace:2013} developed an algorithm for the retrieval of CPAs based on a \acrfull{lut} inversion using simulated Landsat 8 imagery. In this work, we extend Gerace's approach to real Landsat 8 imagery \citep{Concha2013IGARSS}. The retrieval algorithm uses a combination of least square error minimization algorithm and a non-linear optimization routine to find the best match for a specific reflectance signal in a LUT of spectral water-leaving reflectance curves. The LUT is created using HydroLight 5, an in-water radiative transfer model \citep{Mobley:2005}. Each curve in the LUT has a specific set of water component concentrations. This is performed on a pixel-by-pixel basis. 

In order to have outputs that are representative of the water bodies that are being studied, \acrfull{iops} of those specific waters have to be input to the HydroLight model. To accomplish this, collections of water samples were conducted at the same time that the Landsat 8 satellite passed over the area of study (the Rochester Embayment, Rochester, NY). After collection, these water samples were analyzed in the lab to obtain the IOPs for the CPAs. Furthermore, \acrfull{aops}, specifically remote-sensing reflectances, and backscattering measurements were also collected for further comparison and to pursue closure between HydroLight AOP results and {\it in situ} AOP measurements.

% Data Collection and Lab Measurements
The concentration of CPAs obtained from the retrieval algorithm are validated by comparison with concentrations measured in the lab from {\it in situ} water samples collected from water bodies present in the Landsat 8 image. Also, the chlorophyll-{\it a} retrieval is compared with the NASA's standard algorithms \citep{OReilly1998_Chl,OReilly2000,Hu:2012fv}.
% !TEX root=Thesis_PhD.tex  
% the previous is to reference main .bib
%% CHAPTER
\chapter{Introduction}
\label{ch:introduction} 
\pagenumbering{arabic} 
% What? and So what? What is the paper about, and why should the reader care?
% Topic, observation/discovery, explanation
% What research questions is this paper trying to answer?

% - Establish a territory (what is the field of the work, why is this field important, what has already been done?)
% - Establish a niche (indicate a gap, raise a question, or challenge prior work in this territory)
% - Occupy that niche (outline the purpose and announce the present research; optionally summarize the results).

% Not enough spatial resolution
Ocean color studies at a global scale, such as chlorophyll-{\it a} level trends in oceans, can be investigated using the heritage Ocean Color satellites (e.g. \acrfull{seawifs}, \acrfull{modis}). These satellites satisfied the spatial requirement for these kinds of studies. However, when the region of interest includes coastal or inland waters, which could be considered Case 2 water, their spatial resolution of roughly a thousand meters are not enough to resolve smaller water bodies. These kinds of waters are important for us, because it is these kinds of waters that we have the most interaction with, for drinking and recreation. 

% New medium spatial resolution
The launching of a new generation of high spatial resolution satellites (e.g. \acrshort{nasa}'s Landsat 8 \citep{Irons:2012}, \acrshort{esa}'s Sentinel 2 \citep{Malenovsky:2012}) is opening a complete new era in the remote sensing of coastal and inland waters. New sensor specifications could meet the requirements needed to have available the same kinds of tools (e.g. Chl-{\it a} product) that the first generation of ocean color satellites (a.k.a. heritage Ocean Color satellites), such as MODIS \citep{Esaias1998} and SeaWiFS \citep{McClain2004}, made available for open ocean science more than a decade and half ago. The hope is to have these kinds of tools for coastal and inland waters available at a global scale and on a regular basis, similar to MODIS capabilities for open oceans (e.g. MODIS's Chl-{\it a} product). Although Landsat 8 does not have a daily frequency as MODIS does, its 16-day frequency makes it a good candidate to accomplish this endeavor, and most importantly, prepares the way for future missions with similar spatial resolution specifications (e.g. Sentinel 2, \acrfull{hyspiri}).

% Landsat
The Landsat project has been monitoring the earth for over four decades, being the longest uninterrupted data set available. The Landsat satellites' main mission is to image the land areas of the earth and therefore there are typically no open ocean (case 1 water) images available. This is one of the reason why Landsat satellites have been underestimated by the ocean color community for the study of water bodies. In addition, the Landsat instruments have generally had broad bands and low \acrfull{snr} when compared to heritage ocean color satellites such as SeaWiFS and MODIS. Carrying two instruments onboard, the \acrfull{oli} and the \acrfull{tirs}, Landsat 8 is the first of a new generation of Landsat satellites with state-of-the-art technology \citep{Irons:2012}. Since its launch in 2013, the OLI instrument onboard Landsat 8 has created high expectations in the ocean color community. With its 12-bit quantization and improved SNR, OLI is a big improvement to the Landsat mission. In addition, OLI includes a new coastal band that increased the spectral resolution of the instrument. These improvements are the main drivers to hypothesize that the Landsat 8 satellite will have a better performance in water quality studies than its predecessors. \citet{Roy:2014} stated this potential use of Landsat 8 for fresh and coastal water studies, mainly due to a reported SNR that exceeded expectations and the new coastal band. Therefore, the overall objective of this research is to investigate the performance of Landsat 8 for accurately retrieving water constituents.

% Work done with Landsat 8
\citet{Gerace:2013} demonstrated that the spectral coverage and radiometric resolution of OLI should dramatically improve our ability to simultaneously retrieve the \acrfull{cpas} (chlorophyll-{\it a}, sediments and \acrfull{cdom}) concentrations from water bodies. \citet{Vanhellemont2014} and \citet{Vanhellemont:2015} created a tool to apply the standard algorithm for atmospheric correction over water developed by \citet{Gordon:1994} to Landsat 8 over turbid and extremely turbid waters. In \citet{Vanhellemont:2015}, after atmospheric correction, they used the end product for detection of high concentrations of black sediments, but neither concentration values nor comparison with field measurements were reported. \citet{Franz:2015} describes an implementation of the atmospheric corrections developed by \citet{Gordon:1994} applied to Landsat 8 in the \acrfull{seadas} software package (URL: \url{http://seadas.gsfc.nasa.gov/}). A comparison of the Landsat 8's retrieved \acrfull{rrs} and chlorophyll-{\it a} concentration over Chesapeake Bay with results from MODIS, SeaWiFS and {\it in situ} historical chlorophyll-{\it a} measurements is presented showing a relatively good agreement. Though again, no direct comparison to simultaneously measured values were available.

% Atmospheric Correction done
The retrieval of water components is in general performed in the reflectance domain, so the very first step in this work is to perform a high quality atmospheric correction to the radiance image from Landsat 8. This is a complex task to perform over water because the signal leaving the water that reaches the sensor is very small when compared to the signal reaching the sensor produced by atmospheric scattering. Most of the atmospheric correction algorithms for open oceans (Case 2 waters) are based on the methods developed for ocean color satellites by \citet{Gordon:1994}. These methods are based on the fact that the signal leaving the water does not contribute to the overall signal beyond the \acrfull{nir} part of the spectrum; so the signal reaching the sensor is caused only by atmospheric scattering \citep{Gordon:1994} in those wavelengths. This is known as the {\it black pixel assumption}. This concept can be expanded to the \acrfull{swir} bands when the black pixel assumption is not valid in the NIR bands, which is the case for Case 2 and highly productive Case 1 waters \citep{Wang:2007}. Unfortunately, most of these methods are not suitable for highly turbid coastal water, although different modifications to these algorithms have been suggested \citep{Patt2003}.

% Atmospheric Correction developed in this work
In this work, a different approach for atmospheric correction was developed based on \citet{Raqueno:2003}, \citet{Gerace:2013} and \citet{Pahlevan:2012}. This atmospheric correction algorithm is the \acrfull{mobelm} that uses a combination of an in-water radiative transfer model over water and a Landsat reflectance product to determine the bright and dark pixels in the image \citep{Concha2014SPIE,Concha2015_SPIE}. The MoB-ELM is compared with the standard algorithms based on \citet{Gordon:1994} and with {\it in situ} measurements.

% Retrieval: What it has be done
After having atmospherically corrected the image, the next step is to apply a retrieval algorithm that outputs maps of CPA concentrations. The standard algorithms for CPA retrieval used by the heritage ocean color missions do not retrieve the CPAs simultaneously. An example of these algorithms is the \acrfull{oc3} algorithm for retrieving chlorophyll-{\it a} concentrations \citep{OReilly2000}. This is an empirical algorithm that finds the best fit for a function between a ratio of two bands and a {\it in situ} chlorophyll-{\it a} concentration data set.

% Retrieval: What we are proposing
\citet{Gerace:2013} developed an algorithm for the retrieval of CPAs based on a \acrfull{lut} inversion using simulated Landsat 8 imagery. In this work, we extend Gerace's approach to real Landsat 8 imagery \citep{Concha2013IGARSS}. The retrieval algorithm uses a combination of least square error minimization algorithm and a non-linear optimization routine to find the best match for a specific reflectance signal in a LUT of spectral water-leaving reflectance curves. The LUT is created using HydroLight 5, an in-water radiative transfer model \citep{Mobley:2005}. Each curve in the LUT has a specific set of water component concentrations. This is performed on a pixel-by-pixel basis. 

In order to have outputs that are representative of the water bodies that are being studied, \acrfull{iops} of those specific waters have to be input to the HydroLight model. To accomplish this, collections of water samples were conducted at the same time that the Landsat 8 satellite passed over the area of study (the Rochester Embayment, Rochester, NY). After collection, these water samples were analyzed in the lab to obtain the IOPs for the CPAs. Furthermore, \acrfull{aops}, specifically remote-sensing reflectances, and backscattering measurements were also collected for further comparison and to pursue closure between HydroLight AOP results and {\it in situ} AOP measurements.

% Data Collection and Lab Measurements
The concentration of CPAs obtained from the retrieval algorithm are validated by comparison with concentrations measured in the lab from {\it in situ} water samples collected from water bodies present in the Landsat 8 image. Also, the chlorophyll-{\it a} retrieval is compared with the NASA's standard algorithms \citep{OReilly1998_Chl,OReilly2000,Hu:2012fv}.
\chapter{Objectives}
\section{Problem Statement}

\section{Statement of Objectives}

\section{Description of Tasks}
\subsection{Primary Requirements}
\subsection{Future Objectives}

\section{Contribution to Field}
% !TEX root=Thesis_PhD.tex 
% the previous is to reference main .bib
%% CHAPTER
\chapter{Background and Theory}
\label{ch:background}
When the target is a water body, besides the traditional concepts applied to remote sensing over land, additional concepts need to be introduced. This chapter will first explain the basics concepts of remote sensing such as radiometric quantities and the governing equation. Later, the additional concepts needed to complete the remote sensing over water are introduced. These concepts include energy interaction with the water column and paths associated with it. Also, this chapter describes the atmospheric correction of satellite imagery, and the atmospheric correction methods used over water. Some tools such as Hydrolight, ACOLITE and the Landsat surface reflectance products are described as well. Furthermore, the concept of light propagation and interaction over the water column and its constituents are addressed, along with the optical properties of these constituents.  
% ----------------------------------------------------
\section{Remote Sensing and Water}
\subsection{Radiometric Quantities}
\autoref{fig:radiance} shows the geometry used to define the radiometric quantities. We need to define a fundamental radiometric quantity that specifies completely positional ($x,y,z$), temporal ($t$), directional ($\theta,\phi$) and spectral ($\lambda$) structure of the light field. This is accomplished by the spectral radiance $L$. The {\it spectral radiance} $L$ \index{radiance!spectral, $L$} can be defined as \citep{Mobley:2001}
\begin{equation}\label{eq:rad1}
  L(x,y,z,t,\theta,\varphi,\lambda)\equiv\frac{\Delta Q}{\Delta t\Delta A\Delta\Omega\Delta\lambda}~~\left[ Js^{-1}m^{-2}sr^{-1}nm^{-1} \right]
\end{equation}
where:\\
      \noindent $\Delta Q$: radiant energy incident \\
      $\Delta t$: time interval \\
      $\Delta A$: surface area at location (x,y,z)\\
      $\Delta\Omega$: solid angle in direction ($\theta$,$\varphi$) \\
      $\Delta\lambda$: photons wavelength interval

\begin{figure}[htb]
  \centering
  \includegraphics[height=8cm]{/Users/javier/Desktop/Javier/PHD_RIT/20123_Spring/Modeling/HydroLight/Beamer/RadianceDef.png}
\caption{Radiance (Note: image taken from \citet{Mobley:2001})}
\label{fig:radiance} 
\end{figure}

In the conceptual limit of infinitesimal parameter interval \citep{Mobley:2001}, \autoref{eq:rad1} becomes
\begin{equation}
  L(x,y,z,t,\theta,\varphi,\lambda)\equiv\frac{\partial^4 Q}{\partial t\partial A\partial\Omega\partial\lambda}~~\left[ Js^{-1}m^{-2}sr^{-1}nm^{-1} \right]~or~\left[ W m^{-2}sr^{-1}nm^{-1} \right]
\end{equation}

In most oceanographic applications, we can assume that there is horizontal homogeneity and time independence, then the spectral radiance can be written as $L(z,\theta,\phi,\lambda)$, which is considered a one dimensional (1-D) quantity.

Even though the spectral radiance describes completely the light field, it is not commonly used due to instrumentation difficulties or because all that information is probably not needed. Irradiance is a radiometric quantity that is easier to measure and often more useful. {\it Irradiance}\index{irradiance! $E$} $E$ is formally defined as
\begin{equation}
  E(x,y,z,t,\lambda) \equiv \frac{\Delta Q}{\Delta t\Delta A\Delta\lambda}~~\left[ W m^{-2}nm^{-1} \right]
\end{equation}

The most commonly measured radiometric variable is the spectral plane irradiance. A collector surface (detector) is equally sensitive to light from any direction. However, the effective area (a.k.a. projected area) of the detector as seen by light in direction $\theta$ is $\Delta A|\cos{\theta}|$, where $\theta$ is the angle between the photon direction and the normal to the surface of the detector (see \autoref{fig:radiance}). If such a detector is placed at depth $z$ and oriented facing upward, so it detects photon traveling downward, then the detector will be measuring the spectral downwelling plane irradiance $E_d(z,\lambda)$. What this detector is really doing is adding the downwelling radiance weighted by the cosine of the photon direction. Therefore, the {\it spectral downwelling plane irradiance}\index{irradiance!spectral downwelling plane, $E_d(z,\lambda)$} $E_d(z,\lambda)$ can be defined as
\begin{equation}
  E_{d}(z,\lambda)=\int_{2\pi_d} L(z,\theta,\varphi,\lambda)|cos\theta|d\Omega~~\left[Wm^{-2}nm^{-1} \right]
\end{equation}
where $2\pi_d$ denotes the hemisphere of downward directions, i.e. the set of directions ($\theta,\phi$) such that $0\leq\theta\leq\pi/2$ and $0\leq\phi\leq2\pi$, if $\theta$ is measured from $+z$ or nadir direction. If the instrument is placed facing downward, then the spectral upwelling plane irradiance $E_u(z,\lambda)$ is measured\index{irradiance! spectral upwelling plane, $Eud(z,\lambda)$}. The difference $E_{net}=E_d-E_u$ is named the net downward irradiance\index{irradiance!net downward, $E_{net}$}.

Now, consider a light detector that is equally sensitive to photons in any direction within a hemisphere of directions, i.e. the detector has the same effective area for radiance in any downward direction, so no $\cos{\theta}$ factor is needed. If that detector is placed at depth $z$, oriented facing upward and therefore collection photons coming from the downward direction, then this detector is measuring the spectral downwelling scalar irradiance at depth $z$, $E_{od}(x,\lambda)$. This spectral downwelling scalar irradiance can be defined as
\begin{equation}\label{eq:Eod}
  E_{od}(z,\lambda)=\int_{2\pi_d} L(z,\theta,\varphi,\lambda)d\Omega~~\left[Wm^{-2}nm^{-1} \right]
\end{equation}
because this instrument is summing radiance over all directions in the downward hemisphere $2\pi_d$.

If the same instrument is placed facing downward, it will be measuring the spectral upwelling scalar irradiance $ E_{ou}(z,\lambda)$, defined as
\begin{equation}
  E_{ou}(z,\lambda)=\int_{2\pi_u} L(z,\theta,\varphi,\lambda)d\Omega~~\left[Wm^{-2}nm^{-1} \right]
\end{equation}
where $2\pi_u$ denotes the hemisphere of upward direction.

The sum of the downwelling and upwelling components can be defined as the {\it spectral scalar irradiance} $E_o(z,\lambda)$\index{irradiance!spectral scalar, $E_o(z,\lambda)$}, i.e.
\begin{align}
  E_{o}(z,\lambda) &\equiv E_{od}(z,\lambda)+E_{ou}(z,\lambda)\notag \\
           &=\int_{4\pi} L(z,\theta,\varphi,\lambda)d\Omega
\end{align}
The spectral scalar irradiance is the radiometric variable that is most relevant to photosynthesis because photosynthesis is independent of the traveling direction of the light.

Another important radiometric quantity in oceanography is the \gls{par}\index{PAR}, defined as
\begin{equation}
  PAR(z)\equiv \int_{350nm}^{700nm} \frac{\lambda}{hc}E_o(z,\lambda)d\lambda~~~\left[photons~s^{-1}m^{-2} \right]
\end{equation}
where $h=6.6255\times10^{-34}J~s$ is the Planck constant and $c=3.0\times10^{17}nm~s^{-1}$ gives the number of available photons rather than the amount of radiant energy. This is relevant because photosynthesis depends on the number of photons absorbed.

% ----------------------------------------------------
\subsection{Sensor-reaching Radiance}

% ----------------------------------------------------
\subsubsection{Exoatmospheric Irradiance}

From the total energy coming from the sun, only approximately 1390 $\left[\frac{W}{m^2}\right]$ reaches the Earth's atmosphere \citep{Schott}. This integrated value is known as the \emph{exoatmospheric irradiance}\index{irradiance!exoatmospheric}, or $E_S'$, and represents the total energy per unit area just outside the Earth's atmosphere due to the solar energy. Recall that \emph{irradiance} is the rate at which the radiant flux ($\Phi$) is delivered to a surface ($A$), defined as
\begin{equation} \label{eq:irradiance}
E = \frac{d\Phi}{dA}   \indent   \indent  \left[\frac{W}{m^2}\right]  
\end{equation} 

So, the $E_S'$ is calculated assuming that the flux $\Phi$ comes from a point source at the center of the sun such that it would produce an exitance at the sun's surface, producing a flux at the mean Earth-sun distance of 1390 $\left[\frac{W}{m^2}\right]$ . For the present work, it is more convenient to express the irradiance spectrally, or in other words as a function of wavelength, so we can describe the energy at a desired wavelength, or spectral band. \autoref{fig:exoirrad} shows the exoatmospheric irradiance spectrum  along with the transmitted solar irradiance through the atmosphere. Major absorption bands in the atmosphere are clearly apparent. MODIS bands are shown. Note that most of the exoatmospheric irradiance occurs in the visible part of the spectrum of light.

\begin{figure}[htb]
  \centering
  \includegraphics[height=8cm]{/Users/javier/Desktop/Javier/PHD_RIT/Latex/Proposal/Images/MODIS_ATM_solar_irradiance.png}
\caption{Exoatmospheric irradiance as a function of wavelength (green curve) and solar spectral irradiance transmitted through the  atmosphere to the the Earth's surface (brown curve). MODIS bands shown (orange).  (Source: \protect\url{http://commons.wikimedia.org/wiki/File:MODIS_ATM_solar_irradiance.svg}).}
\label{fig:exoirrad} 
\end{figure}

% ----------------------------------------------------
\subsubsection{Solar Energy Paths}
\label{subsubsec:SolarPaths}
The main goal of remote sensing is to extract information about a specific target from all information recorded by an imaging system. In order to only isolate the target information from the rest of the recorded information, we need first to understand what kinds of energy are recorded by the sensor (a satellite is this case). Conceptually, we will separate the information recorded into energy paths, as described by \citet{Schott}. The energy paths, or electromagnetic energy, most influential in the $0.4-2.5\mu m$ spectral region are the solar energy paths (a.k.a. reflective paths), which are followed by radiation that originated from the sun that could reach the satellite camera. \autoref{fig:paths} shows the most significant solar energy paths. The type A photons in the figure originate in the sun, pass through the atmosphere, reflect off the target on Earth's surface, and propagate through the atmosphere in the direction of the sensor. This is the path that carries information about the target of interest. Type B photons in the figure are photons originating in the sun that are scattered by the atmosphere in the direction of the target, and are then reflected by the target in the direction of the sensor. These photons are referred to as {\it skylight}\index{skylight} or {\it sky shine}. Type C is another group of photons that are significant in total signal recorded by the sensor. These photons originate in the sun, and are scattered by the atmosphere into the camera's line of sight, without ever interacting with the target. This path is referred as {\it upwelled radiance}\index{radiance!upwelled}, and it is a function of how ``hazy'' the atmosphere is.

Another photon path that could be added to the total signal is the photons that originate in the sun, propagate to the atmosphere, reflect off background objects in the direction of the target and are then reflected from the target of interest back through the atmosphere to the sensor. These photons are labeled as type G photons in \autoref{fig:paths}. As you can see, this path involves multiple reflection or bounces of the photons. 

The type I photons in the figure are a product of what is called {\it adjacency effect}\index{adjacency effect}. This is also a source of multiple bounce, or bounce and scatter photons. These photons are reflected from surrounding objects and then scattered into the line of sight of the sensor. This path can be included in the path radiance (type C photons). Lastly, if we are considering multiple scattering, we need to take into account the {\it trapping} effect, illustrated by path J in \autoref{fig:paths}. These are photons that are generated in the sun, propagated through the atmosphere, reflected off the target to the air column, but are then reflected back onto the target by the air column, and finally reflected into the line of sight of the sensor. 

When the target is the water column, we can discard the type G photons over open water. For water studies, the most important paths will be type A, B and C photons. The goal of the atmospheric correction is to isolate the type A and B photons that penetrate into the water, interact with the water and are reflected out of the water. To do so, we need to characterize the rest of the paths.

\begin{figure}[htb]
  \centering
  \includegraphics[height=8cm]{/Users/javier/Desktop/Javier/PHD_RIT/Latex/Proposal/Images/PathsVisio.png}
\caption{Photon path contributions to the sensor-reaching radiance in the reflective portion of the spectrum.}
\label{fig:paths} 
\end{figure}

% ----------------------------------------------------
\subsubsection{Governing Equation}

The light from the source, usually the sun, interacts with the target and then reaches the sensor, as described in the previous section. This interaction will help us to extract information about the target, in this case the water body. That is why in order to understand how the water quality parameter are retrieved, first it is necessary to introduce the concept of sensor reaching radiance. The sensor reaching radiance is defined as the accumulation of photons at the front of a sensor that one wishes to collect in an effort to obtain information about the target \citep{GeraceThesis}. 

The total sensor-reaching radiance is the sum of the radiances due to the individual solar and thermal paths.  \citet{Schott} shows that in the VNIR/SWIR region (approximately 0.3-2.5 [$\mu m$]), the solar energy is so many orders of magnitude higher than the self-emitted energy, that the thermal paths are negligible for this study. Also, we consider that radiance from the background ($L_G$) is negligible because the water bodies are typically several kilometers wide, as mentioned previously. However, this contribution needs to be analyzed further for coastal regions where contribution from the surrounding bright targets (e.g. sand) could influence the total signal significantly. The fundamental remote sensing equation that accounts for the most important photon interaction describes how the paths described in the previous section contribute to the signal reaching the imaging system. For the water case, this equation could be written as \citep{Schott}
\begin{equation}
   L = L_A+L_B+L_C+L_I+L_J
\end{equation} 
where $L$ is the total radiance reaching the sensor's aperture and the indexes represent the different reflective paths described in \S\ref{subsubsec:SolarPaths}. Note that $L_G$ is neglected for the water case. Neglecting paths $L_I$ and $L_J$, the simplified total sensor-reaching radiance, $L$, is defined as 
\begin{align} \label{eq:gov1} 
L(\lambda)  &=L_A+L_B+L_C\notag\\
            &= \frac{E'_S(\lambda)cos(\sigma')r_{rF}(\lambda)\tau_1(\lambda)\tau_2(\lambda)}{\pi} + \frac{E_{ds}(\lambda)r_d(\lambda)\tau_2(\lambda)}{\pi} + L_{us}(\lambda)
\end{align} 
where:
\begin{tabbing}
\indent \indent \indent  $L(\lambda)$ \hspace{1mm}\=:  \indent \= total sensor-reaching radiance\\
\indent \indent \indent  $E'_S(\lambda)$\>: \>exoatmospheric spectral irradiance\\
\indent \indent \indent $\sigma'$\>:\>solar-zenith angle\\
\indent \indent \indent $r_{rF}(\lambda)$\>:\>spectral target reflectance factor\\
\indent \indent \indent $r_d(\lambda)$\>:\>spectral diffuse reflectance\\
\indent \indent \indent $\tau_1(\lambda)$\>:\>Sun-target path transmission of atmosphere\\
\indent \indent \indent $\tau_2(\lambda)$\>:\>target-sensor path transmission of atmosphere\\
\indent \indent \indent $E_{ds}(\lambda)$\>:\>solar scattered downwelled irradiance (skylight)\\
\indent \indent \indent $L_{us}(\lambda)$\>:\>solar upwelled radiance (path radiance)\\
\end{tabbing}
\autoref{eq:gov1} is the solar form of the ``big equation'' described by \citet{Schott}. Note that the path $L_I$ (not included in \autoref{eq:gov1}) does not have to be necessarily neglected if the surrounded albedo (background reflectance) is included. If this is the case, these photons can be treated as a constant and lumped in with the path radiance (type C photons). Additionally, path $L_J$ (not included in \autoref{eq:gov1}) can be included with type C photons. Solutions to \autoref{eq:gov1} that use MODTRAN include $L_I$ and $L_J$.

% ----------------------------------------------------
\subsection{Water Contributions to the Total Signal}
Besides the photon paths described in \S\ref{subsubsec:SolarPaths}, a set of different photon paths need to be added to the total signal recorded by the sensor when the target is a water body. \autoref{fig:watercontribution} shows conceptually the different contributions to the radiance $L_s$ measured by a sensor (e.g. satellite or aircraft). These contributions are from the atmosphere ($L_a$), the water surface ($L_r$), and water column ($L_w$). The atmospheric contribution $L_a$ is the same as the type C photons ($L_C$) described in \S\ref{subsubsec:SolarPaths}. The surface-reflected radiance $L_r$ is the portion of the downwelling solar radiance that is reflected by the water surface into the sensor's line of sight. This has two component, one from the sun and one from the skylight, and it is commonly referred to as glint. 

Finally, the water-leaving radiance $L_w$ is the portion of the sun's energy that propagates through the atmosphere, is transmitted through the water surface ($L_t$),interacts with the water column, and is then scattered into the upward direction (subsurface $L_u$ in \autoref{fig:watercontribution}) to eventually be transmitted through the water surface and propagate through the atmosphere into the sensor direction. This is the path of interest because we are particularly interested in how the light is attenuated as it enters the water column. The {\it water column} is defined as a conceptual volume below the water surface, which contains the constituents we will study.

Each contribution previously described could be relevant by itself, depending of the application. $L_a$ gives information about the atmosphere such as aerosol composition, for instance. However, only the water-leaving radiance $L_w$ carries information about the water column, which is relevant to this study. Because the sensor only measures the total upwelling radiance (sensor-reaching radiance) $L_s=L_a+L_r+L_w$ and not each contribution separately, $L_w$ needs to be isolated from the rest of contributions ($L_a$ and $L_r$) through a process called atmospheric correction. 

\begin{figure}[htb]
  \centering
  \includegraphics[height=8cm]{/Users/javier/Desktop/Javier/PHD_RIT/Latex/Proposal/Images/WaterRadianceFixed.png}
\caption{Contribution to the sensor reaching radiance $L_s$ above the water surface. Thick arrows represent single-scattering contributions; thin arrows represent multiple scattering contributions (Source: \protect\url{http://www.oceanopticsbook.info/}).}
\label{fig:watercontribution} 
\end{figure}
% ----------------------------------------------------
\subsubsection{Water Column and Bottom Contributions}
Once the incident energy is transmitted through the water surface, i.e. $L_w$ in the previous section, the incident light could be absorbed or scattered by the different water constituents within the water column. \autoref{fig:WaterColumn} illustrates the different kinds of interactions of the incident light with the water column (or water volume) and the water bottom.

\begin{figure}[htb]
  \centering
      \includegraphics[width=100mm]{/Users/javier/Desktop/Javier/PHD_RIT/Latex/Proposal/Images/WaterColumn.eps}
  \caption{Contributions to sensor-reaching radiance from the water column.}
  \label{fig:WaterColumn}
\end{figure}

Path $I$ represents the bottom effect caused by the incident light that penetrates the water surface, is transmitted through the water column, reflected by the bottom, transmitted through the water column, and leaves the water into the sensor direction. This contribution depends on the depth and the clarity of the water. This signal can be significant in shallow Case 2 waters, i.e. depth $<10-15m$, where there is not much organic and/or inorganic suspended matter within the water column, and it can be used to extract information about the bottom composition or bathymetry, for instance. However, for water quality studies, this is a signal that needs to be avoided or isolated from the total signal. In the present study, this path will be assumed to have zero contribution to the water-leaving signal because the water bodies of interest are deep enough to not allow the incident light to interact with the bottom or the water has a significant concentration of suspended particles (i.e. $>0.1mg/L$ \citep{Pahlevan:2012}). In the case that there is any bottom contribution in the scene, it will be masked out before the processing by manually creating a mask over the areas that the bottom can be visualized.

Path $II$ represents the interaction of the incident light with \gls{cdom}. \gls{cdom} is considered to only be an absorber, and not a scatterer. Therefore, it is an important component in light attenuation in the \gls{uv} and the blue regions of the spectrum of light, therefore it is a important optical constituent of the water that often dominates absorption in the blue. In practice, \gls{cdom} is defined operationally as the material that passes through a filter most often with pore size of $0.2\mu m$. Over the interval $[350nm,700nm]$, the \gls{cdom} absorption coefficient is described by an exponential decreasing function \citep{Jerlov:1976jw,Hojerslev1988}.
\begin{equation}
  a_{CDOM}(\lambda) = a_{CDOM}(\lambda_0)\exp{\left[-S_{CDOM}(\lambda-\lambda_0)\right]}
\end{equation}
where $S_{CDOM}$ is spectral slope and $\lambda_0$ is the reference wavelength. \autoref{fig:compabs} shows examples of CDOM absorption coefficient for Case 1 and Case 2 waters.

\begin{figure}[htb]
\centering
      \includegraphics[height=9cm]{/Users/javier/Desktop/Javier/PHD_RIT/20123_Spring/Modeling/HydroLight/Beamer/AbsCoeff.png}
      \caption{Contribution by the various components to the absorption coefficient (Note: image taken from \citet{Mobley:2001}).}
      \label{fig:compabs}
\end{figure}

Path $III$ illustrates the influence of inorganic suspended particles (a.k.a. \gls{sm} or minerals) on the incident light, but it could also include organic particles. These particles can scatter and/or absorb light, and they vary in size, composition, and distribution, which can influence the optical properties of the water, being different for different particles. For example, the optical properties of clay are different from silt \citep{Pahlevan:2012}. \autoref{fig:compabs} shows examples of minerals absorption coefficient for Case 2 waters (labeled as ``Min'' (minerals) on the right-hand figure). The inorganic suspended particles are often included within the \gls{nap}, which are defined operationally as the particulate material that is not extracted by methanol in the spectrophotometric measurement of particles on filter pads \citep{Kishino1985cj,Mitchell2002}. The \gls{nap} absorption coefficient spectrum is often described by a exponential decreasing function, i.e.
\begin{equation}
  a_{NAP}(\lambda) = a_{NAP}(\lambda_0)\exp{\left[-S_{NAP}(\lambda-\lambda_0)\right]}
\end{equation}
where $S_{NAP}$ is the exponential slope for NAP, which could be estimated from a nonlinear regression from field data. The scattering properties of particles are difficult to measure. Basically, to measure the scattering due to particles, any undissolved material is treated as particle \citep{GeraceThesis}, and therefore this measurement could include the scattering due to chlorophyll as well. An example of a scattering coefficient spectrum for minerals is shown in \autoref{fig:compscat}, for Case 1 and Case 2 waters.

\begin{figure}[htb]
\centering
      \includegraphics[height=9cm]{/Users/javier/Desktop/Javier/PHD_RIT/20123_Spring/Modeling/HydroLight/Beamer/ScatCoeff.png}
      \caption{Contribution by the various components to the scattering coefficient (Note: image taken from \citet{Mobley:2001}).}
      \label{fig:compscat}
\end{figure}

Path $IV$ shows the case of scattering and absorption of light by pure water, which is considered to be composed of only water molecules (i.e. free from particles). \autoref{fig:WaterAbs1} shows the absorption coefficient spectrum for pure water over a wide range of wavelength. Note that pure water has a high absorption in the UV and above NIR, having a window in the visible. \autoref{fig:WaterAbs2} shows the pure water absorption coefficient for the visible ($[400nm,700nm]$), where the absorption is low in the blue and increases in the red and NIR. The scattering coefficient for pure water is illustrated in \autoref{fig:compabs}. \autoref{fig:WaterScat1} shows the absorption and scattering coefficients for pure sea water in the range $[200nm,800nm]$.

\begin{figure}[!ht]
  \centering
      \includegraphics[width=12cm]{/Users/javier/Desktop/Javier/PHD_RIT/Latex/Proposal/Images/Absorption_spectrum_of_liquid_water.png}
  \caption{Pure water absorption coefficient spectrum. Source: \protect\url{http://en.wikipedia.org/wiki/Electromagnetic_absorption_by_water}}
  \label{fig:WaterAbs1}
\end{figure}

\begin{figure}[!ht]
  \centering
      \includegraphics[width=10cm]{/Users/javier/Desktop/Javier/PHD_RIT/Latex/Proposal/Images/WaterAbsorption.png}
  \caption{Pure water absorption coefficients on a semilog scale \citep{Pope1997}.  Source: \protect\url{http://www.oceanopticsbook.info/}}
  \label{fig:WaterAbs2}
\end{figure}

\begin{figure}[!ht]
  \centering
      \includegraphics[width=10cm]{/Users/javier/Desktop/Javier/PHD_RIT/Latex/Proposal/Images/WaterScattAbs.png}
  \caption{Pure sea water absorption (solid line)  and scattering (dotted line) coefficients \citep{Smith1981}. Source: \citet{Mobley1994}}
  \label{fig:WaterScat1}
\end{figure}

Lastly, path $V$ in \autoref{fig:WaterColumn} illustrates the interaction of the incident light with phytoplankton. This interaction can be absorption and/or scattering. Phytoplankton can dramatically affect the optical properties of the water column. Phytoplankton absorption depends on the composition and concentration of pigments. There are different kinds of pigments. The main pigment is chlorophyll-{\it a}, and it can be used as a surrogate or proxy for phytoplankton. \autoref{fig:compabs} shows examples of chlorophyll absorption coefficient spectra (labeled as ``Chl'') for Case 1 and Case 2 water. Chlorophyll typically tends to be a strong absorber of visible light \citep{Mobley1994}, having two absorption peaks, one in the blue ($430nm$) and one in the red ($665nm$). An example scattering coefficient spectrum for chlorophyll-{\it a} is shown in \autoref{fig:compscat}, for Case 1 and Case 2 waters.

It is worth mentioning that \citet{Mobley1994} describes the attenuation of light when interacting with the water column constituents with its complex index of refraction, $m=n-ik$, where $n$ is the real part that governs scattering within the medium, and $k$ is the imaginary part that governs absorption in the medium. The absorption coefficient is related to the imaginary part $k$ of the index of refraction as \citep{Kerker1969}
\begin{equation}
  a(\lambda)=\frac{4\pi k(\lambda)}{\lambda}
\end{equation}

More details about the different water column constituents are presented in the following section.

% ----------------------------------------------------
\subsubsection{Optical Constituents of Water}
The color of water bodies could vary from the deep blue of the open ocean to yellowish-brown in a turbid estuary. Their color depend on different concentration and optical properties of dissolved and particulate matter. Below there is a brief description of the most important optical constituents of natural waters.
\subsubsection*{Pure Water}
\addcontentsline{toc}{subsubsection}{Pure Water}
\index{Water, Pure} Although water itself appears colorless in our everyday life, it displays a blue hue in large volume. This is due to the dominant role of molecular scattering at small wavelengths and the dominant role of molecular absorption at large wavelength values. This blue color is clearly apparent under sunny conditions in oceanic water, for instance.
% ----------------------------------------------------
\subsubsection*{Dissolved Organic Compounds}
\addcontentsline{toc}{subsubsection}{Dissolved Organic Compounds}
The decomposition of phytoplankton cells in the water column (or in the bottom sediments) results in the creation of a variety of complex polymers generally referred to as water humus, or humic substances \citep{Bukata1995}. These humic substances include both water-soluble and water-insoluble fractions. \Gls{doc}\index{DOC} is part of the water-soluble fraction.  The colored portion of \gls{doc} and the only part of \gls{doc} that absorbs light is referred to as \acrfull{cdom}\index{CDOM}. Due to its yellow hue, the dissolved aquatic humus is generally referred to as \gls{ys}, but other terms have been applied to it: gelbstoff, aquatic humic matter, yellow organic acids, humolimnic acid, gilvin, among others. The CDOM absorption is very small in the red, but it increases rapidly at lower wavelengths. CDOM could be the main absorber in the blue region of the spectrum, specially in water influenced by river runoff.
% ----------------------------------------------------
\subsubsection*{Organic Particles}
\addcontentsline{toc}{subsubsection}{Organic Particles}
An organic substance is defined as any substance containing carbon-based compounds, especially produced by or derived from living organisms. The organic particles in water can be bacteria, phytoplankton and detritus, among others.

In clean oceanic waters where the larger phytoplankton are relatively scarce, {\it living bacteria} ($0.2-1.0\mu m$) could scatter and absorb light significantly, especially at the blue region of the light spectrum.

{\it Phytoplankton}\index{Phytoplankton} are microscopic, single cells, free-floating organisms (plants) and have a major effect on the ocean color. Phytoplankton is an important component of the oceanic food web and of the global carbon cycle, which make them the most important primary producers in the ocean, and most importantly for ocean optics, they determine the optical properties of most oceanic waters, which are considered Case 1 waters. Their sizes vary from less than $1\mu m$ to more than $200\mu m$ with different species, shapes and concentrations. Phytoplankton can scatter light strongly because they are in general much larger than the wavelength of visible light. They contain chlorophyll\index{chlorophyll}, which is a pigment that produces energy rich organic material and releases oxygen by absorbing light in a process named photosynthesis. Chlorophyll (and related pigments) have a strong absorption in the blue and red, determining the spectral absorption of water if the concentration is high.

{\it Detritus}\index{Detritus} (a.k.a. {\it tripton}) is non-phytoplankton, non-living organic particles, and it constitutes a large portion of the total organic matter of ecosystems. Detritus is produced when, for example, phytoplankton die and their cells break apart. It can suffer rapidly from photooxidation losing the characteristic absorption spectrum of living phytoplankton and therefore absorbing only in the blue. However, detritus can scatter considerably.  
% ----------------------------------------------------
\subsubsection*{Inorganic Particles}
\addcontentsline{toc}{subsubsection}{Inorganic Particles}

{\it Inorganic particles} are non-living particles created by, for example, weathering of terrestrial rocks that can enter the water as wind-blown dust settles on the sea surface, or they could be eroded soil carried by rivers to the sea or lakes. Their size could vary from less than $0.1\mu m$ to tens of micrometers. When present in high concentrations, inorganic particles could dominate water optical properties.

Note that \acrfull{sm} in natural water bodies comprises both organic and inorganic material. These groups of organic and inorganic material are referred to as seston in limnology. Seston could include mineral particles of terrigenous origin, plankton, detritus (largely residual products of the decomposition of phytoplankton and zooplankton cells as well as macrophytic plants), volcanic ash particles, particulates resulting from {\it in situ} chemical reactions, and particles of anthropogenic origin \citep{Bukata1995}. The major contributor to the water absorption and scattering properties is the particulate matter, being responsible for most of the temporal and spatial variability in these optical properties \citep{Mobley:2001}.

The relevant water constituents for this research are referred to as \acrfull{cpas}\index{CPAs} or optically \gls{oacs} and they are colored dissolved organic matter (CDOM), chlorophyll-{\it a}, and suspended matter (SM) (a.k.a. minerals and organic particulates, or \gls{tss}).

% ----------------------------------------------------
\subsubsection{Atmospheric Effect and Water}
At sea level, the principal constituents of the atmosphere are gases (e.g. nitrogen, oxygen, argon and carbon dioxide), water vapor (significant but variable amount), liquid and solid water (in cloud and in the form of precipitation), dust, and aerosol particles, with variable concentrations for each component. $90\%$ of the atmospheric mass is below a height of about $16km$, therefore a satellite looks through effectively all of the atmosphere.

In order to atmospherically compensate an image, first we need to understand the effect of the atmosphere on the propagating energy generated in the sun. Even with clear sky, the solar energy is significantly reduced when it passes through the atmosphere. The reduction is due mainly to two phenomena: scattering by air molecules and aerosols, and absorption by gases (e.g. water vapor, oxygen, ozone and carbon dioxide). Absorption decreases the amount of energy available in a particular wavelength, while scattering redistributes the energy by changing its direction. In the VIS, atmospheric transmission is mainly affected by ozone absorption and by molecular scattering \citep{Asrar1989}. For example, with the Sun vertically overhead, the total solar irradiance on a horizontal surface at sea level is reduced by about $14\%$ with a dry, clean atmosphere and by about $40\%$ with a moist, dusty atmosphere \citep{Kirk1983}. {\it Attenuation}\index{Attenuation} is the loss of the radiation energy that combines scattering and absorption effects.



A description of the processes involved in the energy interaction in the atmosphere is contained in the following section. % Later on, different techniques for atmospherically correcting satellite images are defined.
% ----------------------------------------------------
% \subsubsection*{Energy Interaction in the Atmosphere }
% ----------------------------------------------------
\subsubsection*{Atmospheric Scattering}
\addcontentsline{toc}{subsubsection}{Atmospheric Scattering}

It is often convenient in visible ocean remote sensing to consider the atmosphere to be made up of two components: Rayleigh scattering of the air molecules and Mie scattering of haze and other aerosols. 

{\it Rayleigh scattering}\index{Rayleigh Scattering}  (a.k.a. molecular scattering\index{molecular scattering|see {Rayleigh scattering}}) occurs where the wavelength of the radiation is much larger than the molecular diameters (e.g. daylight scattering or very pure water), or in other words when the scattering particles are small compared the wavelength. Rayleigh scattering dominates the blue to UV region of the spectrum. Air molecules are much smaller than the wavelength of solar radiation and therefore their scattering obeys Rayleigh scattering. 

Rayleigh's Law states that the amount of scattered energy is proportional to $1/\lambda^4$. Because of this proportionality, blue light is very much more strongly scattered than red light. Therefore, the ``white'' light from the sun suffers selective scattering since much of the blue light is removed from the forward direction and redistributed  sideways. This is the reason why the sky appears blue, and why the rising or setting sun appears red even in the absence of scattering by dust particles \citep{Rees1990}. It is also important to mention that the Rayleigh scattering process generates an increase in the degree of polarization of the scattered radiation. As can be seen in \autoref{fig:ScatPhFn}, in the Rayleigh scattering case there is as much scattering in the forward as in the backward direction. 

{\it Mie scattering}\index{Mie scattering} occurs when the wavelength is of the same order of magnitude as the particle diameter. The smallest non-molecular particles that are responsible for scattering are aerosols. That is why Mie scattering is sometimes referred to as {\it aerosol scattering}\index{aerosol scattering|seealso {Mie scattering}}. Aerosol can be defined as a dispersed systems of particles of small particles, liquid or solid, suspended in a gas, like atmospheric air. Therefore dust, haze, smoke, smog, fog, mist, and clouds can be considered to be specific aerosol types. The typical size and number density for non-molecular particles are shown in \autoref{tab:aerosol_size}.

%--------------------------------------
\begin{table}[htb]
\caption{ Typical sizes and typical number density for non-molecular particles. \label{tab:aerosol_size} } 
\centering
\begin{tabular}{c|c|c}
          \bfseries{Particle}   & \bfseries{Size}  & \bfseries{Number Density} \\ 
  & $[\mu m]$     & $[m^{-3}]$      \\ \hline \hline
    Aerosol (e.g. Haze) & $0.01-1$  &   $10^7-10^9$   \\
    Fog     & 1-10    & $10^7-10^8$   \\
    Cloud     & 1-10    & $10^7-10^9$   \\
    Rain    & $10^2-10^4$   & $10^3-10^4$   \\   
 \end{tabular}
\end{table}

Mie scattering is characterized by an angular distribution predominately in the forward direction, as shown in \autoref{fig:ScatPhFn}. It has a much weaker dependence on wavelength, although scattering is still more intense at shorter wavelengths because it may often be crudely approximated as being proportional to $1/\lambda$ \citep{Rees1990}.

{\it Non-selective scattering}\index{Non-selective scattering} (a.k.a. {\it isotropic scattering}\index{isotropic scattering|see {Non-selective scattering}}) occurs when the particle size is very much larger than the wavelength. Non-selective scattering at visible wavelengths occurs in nature in thick clouds or in fog, and its cross-section is independent of the wavelength. \todo{define cross-section}

\begin{figure}[htb]
  \centering
      \includegraphics[width=14cm]{/Users/javier/Desktop/Javier/PHD_RIT/Latex/Proposal/Images/ScatPhaseFn.png}
  \caption{Polar plot of scattering phase function. Source: \citet{Schott}}
  \label{fig:ScatPhFn}
\end{figure}

% ----------------------------------------------------
\subsubsection*{Atmospheric Absorption}
\addcontentsline{toc}{subsubsection}{Atmospheric Absorption}
{\it Absorption}\index{Absorption} is defined as the process of removal of energy from a beam of light by conversion of energy to another form, which in general is thermal energy \citep{Schott}. The absorptive characteristics of the atmosphere can be described by the {\it absorption coefficient}\index{absorption coefficient} $C_\alpha$, which is defined as the fractional amount of flux lost due to absorption per unit length of transit in a propagating beam. $C_\alpha$ can be expressed as
\begin{equation}
  \beta_\alpha = mC_\alpha
\end{equation}
where $m$ is the number density of the molecules and $C_\alpha$ is the absorption cross-section. The {\it absorption cross-section} $C_\alpha$ is the effective size of a molecule relative to the photon flux \citep{Schott}.

The transmission due to absorption $\tau_a$ is defined as
\begin{equation}
  \tau_a = e^{-\beta_\alpha z} = e^{-\delta_\alpha}
\end{equation}
where $z$ is the path length and the product 
\begin{equation}
  \beta_\alpha z=\delta_\alpha
\end{equation}
is referred to as {\it optical depth}\index{optical depth $\delta$} $\delta$ \citep{Schott}.

% ----------------------------------------------------
% \todo{not forget this section}
% \subsubsection*{Atmospheric Optics}
% \addcontentsline{toc}{subsubsection}{Atmospheric Optics}
% A plane wave traveling through a homogeneous material in the $z$ direction can be perceived as the time-averaged quantity irradiance \index{irradiance}  (power per unit area) on a surface \citep{Eismann2012}. The irradiance after propagation distance $z$ is given by
% \begin{equation}
%   \label{eq:BeerLaw}
%   E(\lambda,z) = E(\lambda,0)e^{-\beta_a z}
% \end{equation}
% where $\beta_a$ is the absorption coefficient \index{absorption coefficient}  defined as
% \begin{equation}
%   \beta_a = \frac{4\pi \kappa}{\lambda}~~[m^{-1}],
% \end{equation}
% where $\kappa$ is the imaginary part of the complex index of refraction and quantifies the absorptive characteristic of the medium. The absorption coefficient $\beta_a$ represents the fractional amount of flux lost to absorption per unit length of transit in a propagating beam \citep{Schott}. Equation \ref{eq:BeerLaw} is known as Beer's law \index{Beer's law} and tell us that the energy of light decreases exponentially with distance.

% The variability in optical characteristics of the atmosphere is caused mainly by variability  in gaseous absorption, and aerosol characteristics and loading. Size distribution of the aerosol, index of refraction of the chemical that composes the aerosol and aerosol density describe the physical characteristics of the aerosol. Sunlight interaction with aerosol can be absorption and/or scattering. The amount of radiation scattered and absorbed by a small volume of aerosol can be different and its properties can be described by three parameters: extinction coefficient $K_e$, single-scattering albedo $\omega_o$ and scattering phase function $P(\theta)$ \citep{Asrar1989}.

% The connection between the physical characteristics of the aerosol (e.g. size distribution and refractive index) and the optical characteristics (extinction coefficient, single-scattering albedo, and scattering phase function) is made by using Mie theory for homogeneous spherical particles. Mie theory states that the extinction cross section $\beta$ ($cm^2$) can be computed as a function of the particle radius $r$ \citep{Asrar1989}. 

% The extinction coefficient $K_e$ describes the fraction of radiation taken from the direct beam by the aerosol (named beam attenuation in the aquatic medium). The $K_e$ of a small volume of air is computed as the contribution of all particles in the volume
% \begin{equation}
%     K_e=\int\beta(r)n(r)dr
% \end{equation}
% where $r$ is each radius, $\beta$ is cross section, and $n(r)$ is density per radius interval $dr$ with $n(r)$ normalized as 
% \begin{equation}
%   \int n(r)dr=N_o
% \end{equation}
% where $N_o$ is the density of the particles in $cm^{-1}$.

% The single-scattering albedo $\omega_o$ is the fraction of scattering from the total extinction ($\omega_o$= scattering coefficient/extinction coefficient). 

% The scattering phase function $P(\theta)$ describes the angular distribution of scattered radiation, where $\theta$ is the scattering angle.

% The {\it optical depth} \index{Optical depth, $\delta$}  $\delta$ (a.k.a. opacity) expresses the extent to which a given layer of material attenuates the intensity of the radiation passing through it \citep{Rees1990}. It is defined as
% \begin{equation}
%     \delta = \beta_{a}z
% \end{equation}

% To be completed...

% \subsubsection*{Atmospheric Effect Removal}
% \addcontentsline{toc}{subsubsection}{Atmospheric Effect Removal}

% \todo{fit content below}
% {\it Atmospheric Correction for Ocean Color Satellites }
% \citet{Gordon:1994} developed an algorithm for atmospherically correcting SeaWiFS data.
% ----------------------------------------------------
\subsubsection{Glint Effect}
Referring to \autoref{fig:watercontribution}, recall that the water surface contribution $L_r$ represents the portion of the downwelling solar radiance that is reflected by the water surface into the sensor's line of sight. Actually, the contribution from the water surface can be caused by two different sources. One is the sun and the other is the skylight, which is solar energy scattered by the atmosphere. The signal reaching the sensor from the water surface is the reflection of these two sources. These contributions are illustrated in \autoref{fig:glint}, where the solar glint is represented by the yellow solid line while the sky glint is represented by the blue dashed line. The phenomenon of glint is undesired signal that is produced by the Fresnel reflection of light at the air-water surface \citep{GeraceThesis}. It is undesired because it does not tell any information about the water column, which is the desired signal in this case. For this reason, Ocean Color satellites are generally designed to avoid the solar glint by tilting away from the incident angle. However, some sensors, such as Landsat 8, are not designed to avoid glint, so at the right illumination conditions, the image can be contaminated by glint.

Because water is a dynamic body, variables such as wind and tidal forces can change its shape. For this reason, the water's surface can be thought of as being made up of many little facets \citep{GeraceThesis}. Some of these facets will illuminate the sensor with solar glint at an appropriate angle. On the other hand, the sensor is always illuminated by sky glint since every facet of the water reflects some portion of the sky. Therefore, sky glint is always present and solar glint can sometimes be avoided by view angle. Below there is a description of both contributions. 

\begin{figure}[htb]
\centering
\includegraphics[height=8cm]{./Images/Glint.png}
\caption{Solar and sky glint. Solid line represents rays due to solar glint dashed line rays due to sun glint.} 
\label{fig:glint}
\end{figure}
% ----------------------------------------------------
\subsubsection*{Sun Glint}
\addcontentsline{toc}{subsubsection}{Sun Glint}
The water-leaving reflectance could be contaminated by glint effects which are a product of sun light reflected off the air-water surface. For a high spatial resolution sensor like Landsat 8, the solar glint will be more dominant in a localized region and it could strongly contaminate one of many pixels, as opposed to lower spatial resolution sensor in which the solar glint will contaminate a larger area but averaging across the complete pixel. It is important to note that the image-derived surface reflectance of the contaminated areas with the sun glint will closely resemble the solar spectrum (or ``white light''). This fact produces \gls{nir} and \gls{swir} pixels that appear brighter than common water pixels. Different algorithms have been developed to correct images for sun glint (glint removal). Most of these algorithms are based upon the concept that water-leaving radiance is zero beyond the NIR, and therefore any contribution is due to sun glint \citep{Pahlevan:2012}. An example of these algorithms is described in \S\ref{subsec:glintremoval}. Another way to detect sun-glint contaminated areas is a simple band ratio between the SWIR bands, which will reveal atmospheric fronts, cloud coverage and/or low fog conditions \citep{Pahlevan:2012}.
% ----------------------------------------------------
\subsubsection*{Sky Glint}
\addcontentsline{toc}{subsubsection}{Sky Glint}
The effect of sky glint is much less than sun glint and wavelength dependent, i.e. higher in the blue region and smaller in longer wavelengths. However, the total sensor reaching radiance is most notable affected by the sky glint in the longer wavelengths due to the low signal in those bands \citep{Pahlevan:2012}. Sky glint effect should be accounted for if an accurate constituent retrieval is needed.

The sky glint is a function of the sky downwelled radiance $L_d$ and it can be expressed as
\begin{equation}
  L_{sg}(\lambda) = \rho_F(\lambda)L_d(\lambda)\tau_2(\lambda)~~~~~\left[\frac{W}{m^2\mu msr}\right]
\end{equation}
where $L_{sg}(\lambda)$ is the TOA radiance due to sky glint, $\rho_F(\lambda)$ is the Fresnel reflection coefficient, and $\tau_2(\lambda)$ is the sensor-target transmission. The Fresnel refraction is a function of imaging geometry, wavelength, and concentration of water constituents \citep{Pahlevan:2012}. It can be considered constant over the entire spectrum, usually $\rho_F=0.002$, for calm water and nadir-viewing geometry. However, $\rho_F$ is a parameter complex to quantify for real world conditions where wave-induced actions yield a non-uniform surface. $\tau_2$ can be either measured or derived from simulations.



% ----------------------------------------------------
\subsection{Optical Classification of Natural Waters}

Several kinds of classification schemes have been suggested for natural waters based on their optical properties, such as the spectral curve of percent transmittance of downward irradiances or vertical attenuation coefficient $K_d$ \citep{Kirk1983}. The classification used in this research was suggested by \citet{Morel:1977rw}, and refined by \citet{GordonMorel1983} based on the role played by phytoplankton. They suggested water be classified in two categories: Case 1 and Case 2 waters\index{Water Classification! Case 1 and Case 2}. {\it Case 1} waters are waters where phytoplankton and their derivative products (e.g. organic detritus and dissolved yellow color) is the main driver, playing a dominant role in determining the optical properties of the ocean \citep{Kirk1983}. On the other hand, in {\it Case 2} waters, phytoplankton and their derivative products may or may not play a dominant role, and the main drivers could also be suspended sediments and/or \gls{cdom}.


% ----------------------------------------------------
\subsection{Water Constituents Retrieval}
% See \citet{Jensen} and \citet{Mustard2001}.
% ----------------------------------------------------
\subsubsection{In-Water Radiative Transfer}
% ------------------------------------------------
\subsubsection*{IOPs}
\addcontentsline{toc}{subsubsection}{IOPs}
The {\it inherent optical properties} (\acrshort{iops})\index{IOPs: inherent optical properties} are defined as those properties of the water that depend only upon the medium, and therefore are independent of the ambient light field \citep{Mobley:2001}. The IOPs mostly used in radiative transfer theory are the absorption and scattering coefficients. As a way to define these concepts, we use the geometry illustrated in \autoref{fig:IOPsdef}. Consider a collimated beam of monochromatic light of wavelength $\lambda$ and spectral radiant power $\Phi_i(\lambda)$ illuminating a small volume $\Delta V$ of water with thickness $\Delta r$. The portion of the incident power $\Phi_i(\lambda)$ that is absorbed by the volume of water is denoted $\Phi_a(\lambda)$, while the part that is scattered out of the beam at an angle $\psi$ is denoted $\Phi_s(\psi,\lambda)$ and total power scattered in all directions $\Phi_s(\lambda)$. The part that is transmitted through the volume with no change in direction is denoted $\Phi_t(\lambda)$. 

\begin{figure}[htb]
\centering
\includegraphics[height=5cm]{/Users/javier/Desktop/Javier/PHD_RIT/20123_Spring/Modeling/HydroLight/Beamer/IOPgeo.png}
\caption{Geometry used to define IOPs (Note: image taken from \citet{Mobley:2001}). \label{fig:IOPsdef} } 
\end{figure}

Using the geometry of \autoref{fig:IOPsdef}, we can define the {\it absorption coefficient} \index{absorption coefficient, $a$} $a(\lambda)$ as the limit of the fraction of the incident power that is absorbed within the volume, as the thickness becomes small \citep{Mobley:2001}, i.e.
\begin{equation}
  a(\lambda)\equiv \lim_{\Delta r\to 0} \frac{1}{\Phi_i(\lambda)}\frac{\Phi_a(\lambda)}{\Delta r}~~\left[m^{-1} \right]
\end{equation}
The absorption coefficient $a(\lambda)$ represents the fraction of the incident power that is absorbed per unit of distance. In the same way, the {\it scattering coefficient} \index{scattering coefficient, $b$} $b(\lambda)$ is defined as
\begin{equation}
  b(\lambda)\equiv \lim_{\Delta r\to 0} \frac{1}{\Phi_i(\lambda)}\frac{\Phi_s(\lambda)}{\Delta r}~~\left[m^{-1} \right],
\end{equation}
and represents the fraction of the incident power that is scattered out of the beam per unit of distance. The {\it beam attenuation coefficient} \index{beam attenuation coefficient, $c$} $c(\lambda)$ is defines as
\begin{equation}
  c(\lambda)=a(\lambda)+b(\lambda)~~\left[m^{-1} \right],
\end{equation}
and represents the fraction of the incident power that is lost or attenuated. As examples of magnitudes and shapes, \autoref{fig:compabs} and \autoref{fig:compscat} show contributions by the various components of waters to the absorption and scattering coefficients, respectively, for Case 1 and Case 2 waters.

The IOP of the different water components are additive. Therefore, the total absorption coefficient, for instance, can be expressed as
\begin{equation}
  a_{total}(z,\lambda) = \sum_{i=1}^{ncomp} a_i(z,\lambda)
\end{equation}
e.g.,
\begin{equation}\label{eq:atotal}
  a_{total}(z,\lambda) =  a_w(\lambda) + a_{Chl}(z,\lambda)+a_{SM}(z,\lambda)+a_{CDOM}(z,\lambda)
\end{equation}
\noindent where $a_w$, $a_{Chl}$, $a_{SM}$ and $a_{CDOM}$ are the spectral absorption coefficients of water, chlorophyll-{\it a}, suspended materials (SM) and CDOM, respectively. This fact is important because it will help to retrieve simultaneously the CPAs in this case. This also implies that we need to know the IOPs for each component, which is not an easy task, especially for the scattering coefficients.

The scattering coefficients does not take into account the angular distribution of the scattered power. The \gls{vsf} is an IOP that takes into account angular information. Consider $\Phi_s(\psi,\lambda)/\Phi_i(\lambda)$ as the fraction of incident power scattered out of the beam through an angle $\psi$ into a solid angle $\Delta\Omega$ centered on $\Psi$, as shown in \autoref{fig:IOPsdef}. Then, the {\it volume scattering function} (VSF) \index{volume scattering function, VSF} is defined as the fraction of scattered power per unit distance and unit solid angle, i.e.
\begin{equation}\label{eq:VSF1}
  \beta(\psi,\lambda)\equiv \lim_{\Delta r\to 0} \lim_{\Delta \Omega\to 0}  \frac{\Phi_s(\psi,\lambda)}{\Phi_i(\lambda)\Delta r\Delta \Omega}~~\left[m^{-1}sr^{-1} \right]
\end{equation}
but $\Phi_s(\psi,\lambda)=I_s(\psi,\lambda)\Delta \Omega$, with $I_s(\psi,\lambda)$ as the spectral radiant intensity scattered into direction $\psi$ and $E_i(\lambda)=\Phi_i(\lambda)/\Delta A$, with $E_i(\lambda)$ as the incident irradiance, therefore \autoref{eq:VSF1} can be written as
\begin{equation} 
  \beta(\psi,\lambda)= \lim_{\Delta V\to 0} \frac{I_s(\psi,\lambda)}{E_i(\lambda)\Delta V}~~\left[m^{-1}sr^{-1} \right]
\end{equation}
with $\Delta V=\Delta r\Delta A$. This last definition is the reason why it is called the volume scattering function \citep{Mobley:2001}. The VSF represents the scattered intensity per unit incident irradiance per unit volume of water. \autoref{fig:VSFex} shows example of VSF for different kind of waters with its respective scattering coefficients.

\begin{figure}[htb]
\centering
      \includegraphics[height=5cm]{/Users/javier/Desktop/Javier/PHD_RIT/20123_Spring/Modeling/HydroLight/Beamer/VSFweb.png}
      \caption{Examples of VSF for different waters: open ocean water (blue curve), harbor (green curve) and very productive coastal water (red curve). $\lambda$ was set equal to $514nm$ (Note: image taken from \citet{Mobley:2001})}
      \label{fig:VSFex}
\end{figure}

If we integrated the VSF over all directions, we obtain
\begin{equation}
  b(\lambda)=\int_{4\pi} \beta(\psi,\lambda)d\Omega=2\pi\int_0^\pi \beta(\psi,\lambda)sin\psi d\psi
\end{equation}
This integration can be divided into forward scattering, $0\leq\psi\leq\pi/2$, and backward scattering, $\pi/2\leq\psi\leq\pi$. Thus the {\it backscatter coefficient} \index{backscatter coefficient, $b_b$} is defined as
\begin{equation}
  b_b(\lambda)\equiv 2\pi\int_{\pi/2}^\pi \beta(\psi,\lambda)sin\psi d\psi
\end{equation}
and the {\it backscattered fraction} \index{backscattered fraction, $b_b/b$} as 
\begin{equation}
  B_b=\frac{b_b}{b},
\end{equation}
which tells how much of the total scattering is due to backscattering.

Again, because the IOPs are additive, the total backscattering coefficients can be expressed as
\begin{equation}\label{eq:b_btotal}
  b_b(z,\lambda) =  b_{bw}(\lambda) + b_{bChl}(z,\lambda)+b_{bSM}(z,\lambda)
\end{equation}

It is important to note that the previous definition assumed that there are no inelastic-scattering processes present. However, fluorescence by dissolved matter or chlorophyll, and Raman scattering by the water molecules themselves, are inelastic-scattering processes that do occur in nature. The power that is lost from $\lambda$ by scattering into $\lambda'\neq\lambda$ results in absorption coefficient $a(\lambda)$ increment. The gain in power at $\lambda'$ is expressed as a source term in the radiative transfer equation (see \autoref{eq:RTEfinal}).

Another IOP commonly used in ocean optics is the {\it single-scattering albedo} \index{single-scattering albedo, $\omega_o$} $\omega_o$, defined as
\begin{equation}
  \omega_o=\frac{b(\lambda)}{c(\lambda)}
\end{equation}
$\omega_o$ is also known as the probability of photon survival because it tells the probability that a photon will be scattered and not absorbed.

Additionally, the {\it volume scattering phase function} $\tilde{\beta}(\psi,\lambda)$ \index{volume scattering phase function, $\tilde{\beta}(\psi,\lambda)$} is defined as
\begin{equation}
  \tilde{\beta}(\psi,\lambda)\equiv \frac{\beta(\psi,\lambda)}{b(\lambda)}~~\left[sr^{-1} \right]
\end{equation}

% ----------------------------------------------------
\subsubsection*{The Radiative Transfer Equation (RTE)}
\addcontentsline{toc}{subsubsection}{The Radiative Transfer Equation}
The connection between the IOPs, boundary conditions, and light sources (e.g. bioluminescence) to the radiances is made through the \gls{rte}. In other words, the \gls{rte} describes the relationship between \gls{iops} and \gls{aops}. It expresses conservation of energy in terms of radiance for a collimated beam of radiance traveling through an absorbing, scattering and emitting medium. All other radiometric variable (irradiances) and AOPs can be derived from the radiance.

\begin{figure}[htb]
\centering
      \includegraphics[height=6cm]{/Users/javier/Desktop/Javier/PHD_RIT/20123_Spring/Modeling/HydroLight/Beamer/RTE1.png}
      \caption{Single beam of radiance and the processes that affect it as it propagates a distance $\Delta r$ (Source: \protect\url{http://www.oceanopticsbook.info/}).}
      \label{fig:RTE1}
\end{figure}

Consider a beam of photons of wavelength $\lambda$ traveling in some direction $(\theta,\phi)$, as shown in \autoref{fig:RTE1}. This beam of photons is accounted in the incident radiance $L(r,\theta,\phi,\lambda)$. This radiance can increase (source) or decrease (lost) in a distance $\Delta r$ along direction  $(\theta,\phi)$, going from depth $z$ to $z+\Delta z$. The losses in radiance can be due to absorption or scattering out of the beam. These losses can be expressed as
\begin{align}\label{eq:RTE1}
  \frac{\Delta L(r+\Delta r,\theta,\phi,\lambda)}{\Delta r}&= -a(r,\lambda)L(r,\theta,\phi,\lambda)-b(r,\lambda)L(r,\theta,\phi,\lambda)\notag\\ 
  &= -c(r,\lambda)L(r,\theta,\phi,\lambda) 
\end{align}
where $a(r,\lambda)$, $b(r,\lambda)$ and $c(r,\lambda)$ are the absorption, scattering and beam attenuation coefficients, respectively. 

The scattering into the beam from all other directions acts as a source increasing the radiance. This source can be expressed as
\begin{equation}\label{eq:RTE2}
\frac{\Delta L(r+\Delta r,\theta,\phi,\lambda)}{\Delta r} = \int_{4\pi} L(r,\theta',\phi',\lambda)\beta(r;\theta',\phi' \to \theta,\phi;\lambda)d\Omega'
\end{equation}
where $L(r,\theta',\phi',\lambda)$ is the radiance coming from direction $(\theta',\phi')$, and $\beta(r;\theta',\phi' \to \theta,\phi;\lambda)$ is the VSF, which tells what amount of the radiance coming from direction $(\theta',\phi')$ scattered into direction $(\theta,\phi)$. The integration is done over all angles (represent by solid angle $\Omega'$) because energy from every direction can be scattered into the direction $(\theta,\phi)$.

There are also internal sources of radiance that could contribute to increase the total radiance, such as bioluminescence or inelastic-scattering processes. This is expressed as
\begin{equation}\label{eq:RTE3}
    \frac{\Delta L(r+\Delta r,\theta,\phi,\lambda)}{\Delta r} = S(r,\theta,\phi,\lambda)
\end{equation}

We can use the conceptual limit of $\Delta r\rightarrow 0$, then
\begin{equation}\label{eq:lim}
  \frac{dL(r,\theta,\phi,\lambda)}{dr} = \lim_{\Delta r \to 0} \frac{\Delta L(r+\Delta r,\theta,\phi,\lambda)}{\Delta r}
\end{equation}. 

Summing up all the different contributions in \autoref{eq:RTE1}-\ref{eq:RTE3} and applying \autoref{eq:lim}, gives
\begin{align}\label{eq:RTEfinal}
  \frac{dL(r,\theta,\phi,\lambda)}{dr}&=-c(r,\lambda)L(r,\theta,\phi,\lambda)\cdots \notag \\
  &+\int_{4\pi} L(r,\theta',\phi',\lambda)\beta(r;\theta',\phi' \to \theta,\phi;\lambda)d\Omega'\cdots \notag  \\
  &+S(r,\theta,\phi,\lambda)~~\left[W~m^{-3}sr^{-1}nm^{-1} \right]
\end{align}
where the angle between the incident direction $(\theta',\phi')$ and the scattered direction $(\theta,\phi)$ is the scattering angle $\psi$ in the VSF. From \autoref{fig:RTE1} $dr=dz/\cos{\theta}$, then \autoref{eq:RTEfinal} becomes
\begin{align}\label{eq:RTEfinal2}
  \cos\theta\frac{dL(z,\theta,\phi,\lambda)}{dz}&=-c(z,\lambda)L(z,\theta,\phi,\lambda)\cdots \notag \\
  &+\int_{4\pi} L(z,\theta',\phi',\lambda)\beta(z;\theta',\phi' \to \theta,\phi;\lambda)d\Omega'\cdots \notag  \\
  &+S(z,\theta,\phi,\lambda)~~\left[W~m^{-3}sr^{-1}nm^{-1} \right],
\end{align}
which is more convenient to use in oceanography because it depends on the depth $z$ and not location $r$ along the beam path. This equation is called the monochromatic, one-dimensional, time-independent RTE, and it expresses location as geometric depth $z$ and the IOPs in terms of the beam attenuation $c$ and the volume scattering function $\beta$. This is the RTE solved by Hydrolight. It needs to be noted that this definition of the RTE does not take into account polarized light occurring in the medium. Therefore, this is called the unpolarized, or scalar RTE (SRTE). However, the SRTE gives sufficiently accurate solutions for many oceanographic applications \citep{MobleyOnline}. If polarization needs to be included, then a polarized or vector RTE (VRTE) could be used.

% \begin{figure}[htb]
% \centering
%       \includegraphics[height=6cm]{/Users/javier/Desktop/Javier/PHD_RIT/20123_Spring/Modeling/HydroLight/Beamer/RTE2.png}
%       \caption{(Note: image taken from \hl{Ocean Optics Book})}
% \end{figure}
% ----------------------------------------------------
\subsubsection*{Apparent Optical Properties (AOPs)}
\addcontentsline{toc}{subsubsection}{Apparent Optical Properties}
The \gls{aops} are those properties that not only depend on the medium (i.e. the IOPs) but also on the directional structure of the ambient light field (radiance distribution). Additionally, \gls{aops} need to display enough stability to be useful descriptors of a water body. Radiances and irradiances are never AOPs themselves. They are always a ratio of two radiometric quantities. Examples of AOPs are the irradiance reflectance, the remote-sensing reflectance, and various diffuse attenuation functions.

The irradiance reflectance\index{reflectance!irradiance,$R(z,\lambda)$} (a.k.a. irradiance ratio\index{ratio!irradiance}) is defined as
\begin{equation}
  R(z,\lambda)\equiv \frac{E_u(z,\lambda)}{E_d(z,\lambda)}
\end{equation}

The fundamental quantity used in ocean color remote sensing is the \acrfull{rrs}\index{reflectance!remote-sensing, $R_{rs}$}, defined as
\begin{equation}
\label{eq:Rrs}
  R_{rs}(\theta,\phi,\lambda)\equiv \frac{L_w(\theta,\varphi,\lambda)}{E_d(\lambda)}~~\left[sr^{-1} \right]
\end{equation}
where $L_w$ is the upwelling water-leaving radiance and $E_d$ is the downwelling plane irradiance. $L_w$ is the total upward radiance minus the sky and solar radiance that is reflected upward by the water surface. $L_w$ and $E_d$ are measured in air, just above the water surface.

Both $R$ and $R_{rs}$ are used to estimate bio-optical variables, such as the chlorophyll concentration, because they are much less affected by the illumination conditions and strongly affected by the water composition.

When the incident light is provided by the sun and the sky, the irradiances and radiances decrease exponentially with depth (but only when they are measured far enough below the surface and far enough above the bottom for shallow water), then
\begin{equation}\label{eq:EdKd}
  E_d(z,\lambda)\equiv E_d(0,\lambda) exp\left[-\int_0^{z}K_d(z',\lambda)dz'\right]
\end{equation}
where $K_d(z,\lambda)$ is the {\it spectral diffuse attenuation coefficient}\index{attenuation coefficient! spectral diffuse, $K_d$} for spectral downwelling plane irradiance. Solving \autoref{eq:EdKd} for $K_d$ gives
\begin{align}
  K_d(z,\lambda)  &=- \frac{d\ln E_d(z,\lambda)}{dz} \notag \\
          &=-\frac{1}{E_d(z,\lambda)}\frac{dE_d(z,\lambda)}{dz} ~~\left[m^{-1} \right].
\end{align}

For example, the diffuse attenuation coefficient for spectral downwelling plane irradiance at $\lambda=490nm$, $K_d(490)$, could be used as an indicator of the turbidity of the water column, among others, since it is directly related to the presence of scattering particles in the water column \citep{Lee2005_Kd}.

% Other $K$ functions such as $K_u$, $K_o$, or $K_L$ can be obtained using similar definition with the corresponding radiometric quantities such $E_u$, $E_o$, or $L$.

It is important to note that AOPs are not additive as the IOPs. Also, AOPs can not be measured in the lab or on a water sample, therefore they must be measured {\it in situ}.

% ------------------------------------------
\subsection{Deriving IOPs from AOPs}
\label{subsec:semianalitic}
The reflectance inversion methods\index{reflectance inversion algorithms} refer to method to derive IOPs (e.g. absorption $a$ and scattering $b$ coefficients) from AOPs (e.g. $R_{rs}$). There are two kinds of inversion algorithms: empirical\index{empirical algorithms|see {reflectance inversion algorithms}} and semi-analytic algorithms\index{semi-analytic algorithms|see {reflectance inversion algorithms}}. The empirical algorithms apply simple or multiple regressions between the property of interest and the ratios of \gls{rrs}. Examples of these algorithms are the \acrfull{oc2} and \acrfull{oc4} algorithms (see \S\ref{subsec:chlempirical}). These kinds of algorithms tend to work only in waters with similar properties to the ones they were based on, resulting in a limited applicability. Some of the advantage of the empirical algorithms are their simplicity and rapidity in data processing. On the other hand, the semianalytical algorithms are based on solving the \gls{rte} and they can be applied to different water types \citep{Lee2002_invQAA}.

It is well known that the AOPs are a function of the ratio between the backscattering and the absorption coefficients, .i.e. $R_{rs}(\lambda)~or~L_{wN} = f[b_b(\lambda)/a(\lambda)]$ with $R_{rs}=L_w/E_d$ the remote-sensing reflectance and $L_{wN}$ the normalized water-leaving radiance \citep{Morel:1977rw,Maritorena:02}. As an example, \citet{Gordon:1988qv} derived this relationship as
\begin{equation}\label{eq:semianal}
  \hat{L}_{wN} = \frac{tF_0(\lambda)}{n_w^2} \sum_{i=1}^2g_i\left[\frac{b_b(\lambda)}{b_b(\lambda)+a(\lambda)}\right]^i
\end{equation}
\noindent where $t$ is the sea-air transmission factor, $F_0(\lambda)$ is the extraterrestrial solar irradiance, $n_w$ is the index of refraction of the water, and the $g_i$ terms are fitting coefficients from Monte Carlo simulations of an idealized ocean by \citet{Gordon:1986fr}. If \autoref{eq:semianal} is used at first order only, the $g_1$ term becomes the $f/Q$ factor, which is dependent on the viewing geometry and not constant \citep{Maritorena:02}. Examples of semianalytical algorithms can be found in \citet{Lee2002_invQAA}, \citet{Lee_Du_Arnone2005} and \citet{Werdell2013_inv}. Note that the IOP spectra, $a(\lambda)$ and $b_b(\lambda)$, can be decomposed in the different water constituents or \gls{cpas} (see \autoref{eq:atotal} and \autoref{eq:b_btotal}). Therefore, if all the \gls{iops} but the $b_b(\lambda)$ are known, then the $b_b(\lambda)$ can be determined from \gls{rrs} using one of these semianalytical algorithms. This can help to determine the unknown information in this research: the backscattering phase function.

This section presented the different phenomena affecting the signal captured by the sensor, such as the atmospheric effect, the glint, and the light that interacts with the water column. Understanding this signal's components allows for a better isolation of the signal of interest, the water-leaving signal, from the rest. Also, the various components that affect the water color, their interaction with light, and their influence in the water-signal was defined. These concepts permit a better separation of each water component, which is the final product of this research.

% ----------------------------------------------------
\section{The OLI Sensor}
The Landsat project, a joint initiative between USGS and NASA, has been monitoring the earth for over four decades, creating the longest uninterrupted data set available. Landsat 8, formally known as the \gls{ldcm}, is the most recent satellite to continue this objective. Carrying two instruments onboard, the \acrfull{oli} and the \acrfull{tirs}, Landsat 8 is the first of a new generation of Landsat satellites with these state-of-the-art technologies. 

Landsat 8 is an optical passive satellite, which means it records the energy reflected from a source (in this case the sun) by a target. It has a temporal resolution of 16 days, which means that it images the same location on Earth every 16 days. Because the area of study (Rochester Embayment) appears in two Landsat 8 paths, there is one image of this area every eight days. OLI is considered to be a multispectral instrument with a total of seven bands: four bands in the \gls{vis}, one band in the \acrfull{nir} and two bands in the \acrfull{swir}, as can be seen in \autoref{fig:olibands}. \autoref{fig:olibands} also shows Landsat 7 bands for comparison. Note the two new bands added to the mission: coastal band and cirrus bands.

\begin{figure}[htb]
\centering
      \includegraphics[height=7cm]{/Users/javier/Desktop/Javier/PHD_RIT/Latex/Proposal/Images/OLIbands.jpg}
      \caption{Landsat 8 bands compared with Landsat 7 bands (Source: \protect\url{http://landsat.gsfc.nasa.gov/}).}
      \label{fig:olibands}
\end{figure}

OLI has a spatial resolution of $30m$ in all seven bands, the same as previous Landsat satellites. Considering its 30-meter resolution, Landsat 8 should be especially useful for studying the nearshore and coastal environment at a much higher spatial resolution, when compared to ocean color satellites (e.g. \gls{modis}, \gls{seawifs}, \gls{meris}). This is illustrated in \autoref{fig:resol} where some features in the nearshore areas of Rochester, NY (such as ponds) can be fully resolved by Landsat 8 ($30m$) and not by Terra-MODIS ($500m$).

\begin{figure}[htb]
  \centering
  \includegraphics[height=4cm]{/Users/javier/Desktop/Javier/PHD_RIT/ConferencesAndApplications/NESSF14/latex/ResolComp.pdf}
  \caption{Spatial resolution comparison between Terra-MODIS (500m) and Landsat 8 (30m). \label{fig:resol} } 
\end{figure}

Although the OLI's spectral bands are not narrow compared with MODIS' spectral bands, the OLI's spectral bands are narrower when compared to Landsat 7 (L7), as seen in \autoref{tab:bandwidth}. OLI also includes a new coastal band that increases the spectral resolution of the instrument, plus a new cirrus band. These two improved features have the potential to more accurately capture signals leaving the water. \citet{Gerace:2013} demonstrated with a simulated dataset that system noise is the main driver of retrieval error, and therefore a higher signal-to-noise ratio (SNR) means a better retrieval. In comparison to its predecessors (e.g. Landsat 5 and Landsat 7), Landsat 8 has an improved SNR because of its 12-bit quantization (4096 levels) and pushbroom sensor design (which allows for more continuous integration on target). This improvement in SNR can be seen in \autoref{fig:L8SNR} which compares the SNR of Landsat 7 and Landsat 8, calculated from actual image data over uniform water regions of the Red Sea that have similar brightness \citep{Hu:2012}. \autoref{fig:L8SNR} also shows the specified SNR from Landsat 8 at typical input signal (radiance L typical) levels (which the instrument significantly exceeds), which were obtained from \citet{Irons:2012}. These improvements are significant drivers behind the hypothesis that the Landsat 8 satellite has superior performance and application in water quality studies than its predecessors.


\begin{table}[!ht]
\caption{ Bandwidth comparison between Landsat 8, Landsat 7 and MODIS. \label{tab:bandwidth}} 
\centering
      \begin{tabular}{c|c|c|c|c}
          \bfseries{Band}& \bfseries{Center}   & \bfseries{L8 Bandwidth} & \bfseries{L7 Bandwidth} & \bfseries{MODIS Bandwidth} \\ 
                  & \bfseries{$[\mu m]$} & $[nm]$   & $[nm]$ & $[nm]$   \\ \hline \hline
          Coastal & 0.44 & 16 & N/A & 10  \\
          Blue    & 0.48 & 60 & 73  & 10  \\
          Green   & 0.56 & 57 & 82  & 10  \\
          Red     & 0.66 & 37 & 61  & 10  \\  
          NIR     & 0.83 & 28 & 126 & 15  \\
          SWIR 1  & 1.65 & 85 & 202 & 24  \\
          SWIR 2  & 2.22 & 18 & 281 & 50  \\ 
       \end{tabular}
\end{table}

\begin{figure}[htb]
\centering
      \includegraphics[height=6.5cm]{/Users/javier/Desktop/Javier/PHD_RIT/Latex/Proposal/Images/L8SNR_2.eps}
      \caption{Comparison between Landsat 7 and Landsat 8 SNR. \label{fig:L8SNR} } 
      \label{fig:olisnr}
\end{figure}

% ----------------------------------------------
\section{Empirical Line Method}
\label{sec:ELM}
The {\it empirical line method}\index{ELM} (ELM) is a method for calibration of image data to reflectance that uses ground truth. The \acrshort{elm} uses a linear regression in each band to relate digital counts (\acrshort{dc}; a.k.a. \gls{dn}) or radiance to reflectance \citep{Schott,Smith:1999}. The ground truth can be in general  either control panels or ad hoc control surfaces of known reflectance. These ground truth objects need to be approximately Lambertian to minimize any errors that could be introduced by sensor view angles effects. Also, these calibration targets are assumed flat and level, with no neighboring obscuration, and homogeneous as well. The ELM method generally assumes that the atmosphere is constant over the complete scene. If that is not the case, corrections must be made for changes in the atmosphere over the scene. The regression to be solved for each band in the ELM method (\autoref{fig:ELMregression}) is given 

\begin{equation}
	\label{eq:ELM} 
	L = m\times R_{rs} + b
\end{equation}

where $L$ is the radiance reaching the sensor, $m$ is the slope of the regression, $R_{rs}$ is the remote-sensing reflectance, and $b=L_u$ is the intercept, with $L_u$ the upwelled radiance or path radiance, which also includes sky glint. Then, the reflectance of the any Lambertian objects can be calculated by rearranging \autoref{eq:ELM}. In order to solve this regression, i.e. determine the value of $m$ and $b$, we need to have at least two targets with known radiance $L$ and reflectance $R_{rs}$. These targets are known as dark and bright targets or objects (\autoref{fig:ELMregression}). After $m$ and $b$ have been determined, the reflectance of each pixel at each wavelength can be calculated from its radiance value from the image.

An ELM target needs to have a size at least three times the ground instantaneous field of view of the sensor that will image it at the time of data collection. Taking this in consideration, the target should be at least 90x90 meters big for the Landsat sensor, which is sometimes difficult to build or even to find in the scene. This traditional \gls{elm} is the base of the \gls{mobelm} atmospheric correction method developed by \citet{Concha2014SPIE} and used in this research (\S\ref{subsec:mobelm}). The \gls{mobelm} tries to overcome the difficulty of finding the appropriate targets to be used in the \gls{elm} algorithm by modeling them.

\begin{figure}[htb]
  \centering
\resizebox{12cm}{!}{%
\begin{tikzpicture}[x=4ex,y=1ex]
  %axis
  \draw (0,0) -- coordinate (x axis mid) (10,0);
  \draw (0,0) -- coordinate (y axis mid) (0,30);
 
    %labels      
  \node[below=0ex] at (8,0) {\small $Band_i~~remote-sensing~reflectance~(R_{rs})$};
  \node[rotate=90] at (-.5,23) {\small $Band_i~~Radiance~(L)$};

  \node[below=.2ex] at (-2.1,4.5) {\scriptsize $b=$offset};
  \node[below=1.4ex] at (-2.1,4.0) {\scriptsize (path radiance)};
  \draw[rotate=90,|<->|] (0,1) -- coordinate (x axis mid) (1,1);

  \node[below=0ex] at (2,15) {\small Dark Object};
  \draw[arrows=-triangle 45] (2,12.5) -- (2,9);

  \node[below=0ex] at (4,20) {\small $m=$ Slope};
  \draw[arrows=-triangle 45] (4,17.5) -- (5,14.5);

  \node[below=0ex] at (7,27) {\small Bright Object};
  \draw[arrows=-triangle 45] (7,24.5) -- (7.9,20.5);

  \node[below=0ex] at (8,9) {\small $R_{rs}=(L-b)/m$};

  %plots
  \draw plot 
    file {linereg.data};
  \draw plot[mark=*] 
    file {linereg2.data};

\end{tikzpicture}

} % resizebox end

\caption{MoB-ELM atmospheric correction method. The MoB-ELM method is based on the traditional empirical line method (ELM). Two pixels from the image, the bright and dark pixel, are used to solve a liner regression with a slope $m$ and offset $b$ in the $R_{rs}$, $L$ space. Once this relationship is established, each $L$ value in the image can be converted to $R_{rs}$ through $R_{rs}=(L-b)/m$. \label{fig:ELMregression}}
\end{figure}

% \begin{tikzpicture}[scale=1.5]
%     % Draw axes
%     \draw [<->,thick] (0,2) node (yaxis) [above] {$r_d$}
%         |- (3,0) node (xaxis) [right] {$L$};
%     % Draw two intersecting lines
%     \draw (0,0) coordinate (a_1) -- (2,1.8) coordinate (a_2);
%     \draw (0,1.5) coordinate (b_1) -- (2.5,0) coordinate (b_2);
%     % Calculate the intersection of the lines a_1 -- a_2 and b_1 -- b_2
%     % and store the coordinate in c.
%     \coordinate (c) at (intersection of a_1--a_2 and b_1--b_2);
%     % Draw lines indicating intersection with y and x axis. Here we use
%     % the perpendicular coordinate system
%     \draw[dashed] (yaxis |- c) node[left] {$y'$}
%         -| (xaxis -| c) node[below] {$x'$};
%     % Draw a dot to indicate intersection point
%     \fill[red] (c) circle (2pt);
% \end{tikzpicture}

% \subsection{Band Ratio}

% -----------------------------------------------------------------------------
\section{Landsat Surface Reflectance CDR}
\label{sec:CDR} 
\subsection{Landsat 4-7 Surface Reflectance Product}
The Landsat \gls{cdr} surface reflectance product for Landsat 4-7 is part of the higher-level Landsat data product to support land surface change studies developed by USGS \citep{LandsatCDR}. This product is generated from specialized software called \gls{ledaps} \citep{Masek:2006}. The LEDAPS software uses MODIS atmospheric correction routines to correct Level-1 Landsat \gls{tm} or \gls{etmplus} data. Atmospheric variables such as water vapor, ozone, aerosol optical thickness along with geometric variables (geopotential height and digital elevation) are input with Landsat data to the \gls{6s} radiative transfer model \citep{Vermote1997_6S}. The 6S model outputs surface reflectance among others parameters. This surface reflectance product is called the Landsat surface reflectance CDR. This Landsat surface reflectance product has comparable uncertainty to the standard MODIS reflectance product \citep{Masek:2006}.

The LEDAPS algorithm works in the following fashion. First, calibrated images from the Landsat satellite are corrected to \gls{toa} reflectance by correcting for solar zenith, Sun-Earth distance, TM or ETM+ bandpass, and solar irradiance. Then, the TOA reflectance is atmospherically corrected with the assumptions that the target is Lambertian and infinite, and the gaseous absorption and particle scattering in the atmosphere can be decoupled. The TOA reflectance \citep{Masek:2006} can be expressed as
\begin{equation}
	\rho_{TOA}=T_g(O_3,O_2,CO_2,NO_2,CH_4)\\	
		\times \left[\rho_{R+A}+T_{R+A}T_g(H_2O)\frac{\rho_s}{1-\rho_s\times S_{R+A}}\right]
		\label{eq:TOAref} 
\end{equation}
\noindent where $\rho_s$ is the surface reflectance, $T_g$ is the gaseous transmission due to the atmospheric gases, $T_{R+A}$ is Rayleigh and aerosol transmission, $\rho_{R+A}$ is the Rayleigh and aerosols atmospheric intrinsic reflectance, and $S_{R+A}$ is the Rayleigh and aerosols spherical albedo. The 6S radiative transfer code is utilized to compute the transmission, intrinsic reflectance, and spherical albedo terms. Ozone concentrations and column water vapor are derived from ancillary data. The \gls{aot} is extracted directly from the imagery by using the dark, dense vegetation (DDV) method developed by \citet{Kaufman_1997}. This method postulates a linear relation between SWIR surface reflectance and reflectance in the visible bands, based on the physical correlation between chlorophyll absorption and bound water absorption. Finally, the derived AOT, ozone, atmospheric pressure, and water vapor are supplied to the 6S radiative transfer algorithm, which then inverts TOA reflectance to surface reflectance using \autoref{eq:TOAref}. 

According to the author in \citet{LandsatCDR}, the Landsat reflectance product has to be used with caution in coastal regions where land area is small relative to adjacent water because the efficacy of the surface correction is likely to be reduced. This product was available only for Landsat 4 TM, Landsat 5 TM and Landsat 7 ETM+ at the time of this publication. A \gls{qa} layer is attached to this product and it can be used for pixel-level conditions and validity production. The surface reflectance product is available in the earthexplorer.com website (\url{http://earthexplorer.usgs.gov/}) in a HDF-EOS package that contains all necessary files and it can be read in ENVI (through an ENVI header file).
% -----------------------------------------------------------------------------
\subsection{Landsat 8 Surface Reflectance Product}
\label{subsec:provisionalCDR} 
With the launch of Landsat 8, a new Landsat surface reflectance product was developed by \gls{usgs}. At the moment of writing this document, the product was named ``Provisional Landsat 8 Surface Reflectance'' \citep{L8SurfProduct2015}. This product is generated from a specialized software called L8SR, which is distinctly different from the \gls{ledaps} algorithm described above \citep{L8SurfProduct2015}. The product has to be used with caution because there are some artifacts reported. These artifacts particularly near feature transitions such as cloud, water-land regions, and high topography variation. The product includes Level 1 data file, and the ``Cloud QA'' band for cloud, cloud shadow, snow and water identification along. Also, the ``cfmask'' and ``cfmask\_conf'' are included as alternative to the Cloud QA band. This product can be downloaded from \url{http://earthexplorer.usgs.gov/}.

The reflectance products presented above were used for the determination of the bright target used for the \gls{mobelm} atmospheric correction algorithm described in \S\ref{subsec:mobelm}. The Landsat 4-7 reflectance product was used at the first part of this research when Landsat 8 imagery was not available. These data helped to developed the first versions of the algorithms described in this work. Once the Landsat 8 reflectance product became available, the Landsat 4-7 reflectance product was replaced by this new product.
% ----------------------------------------------------
% \section{MODTRAN}
% ----------------------------------------------------
\section{HydroLight}
\label{sec:hydrolight}
HydroLight is a radiative transfer numerical model written in Fortran \citep{MobleyHE} (more info: \url{http://www.sequoiasci.com/product/hydrolight/}). It computes radiance distributions and derived quantities (e.g. irradiances, reflectances, K functions, etc.) for natural water bodies. It was developed by Dr. Curtis Mobley for over 20 years (since 1989) and is a commercial software product of Sequoia Scientific, Inc. Hydrolight solves the \gls{rte} based on the \gls{iops} and the boundary conditions  to compute the in-water radiance as a function of depth, direction, and wavelength (\autoref{fig:HLflowchart}). Other quantities of interest for ocean color, such as the remote-sensing reflectance, can be obtained from the computed radiances. It has several models, including the Case 1 water model, whose main input is the chlorophyll IOPs and the Case 2 model, which is a 4-constituent model, pure water, chlorophyll, minerals and CDOM.

\begin{figure}[htb!]
  \begin{minipage}[c]{0.3\linewidth}
      \hspace{0.5cm}
  		\includegraphics[height=3cm]{/Users/javier/Desktop/Javier/PHD_RIT/20123_Spring/Modeling/HydroLight/Beamer/absvsz.png}
  \end{minipage}
  \hfill
  \begin{minipage}[c]{0.3\linewidth}
  		\begin{equation}
  			\cos\theta\frac{dL(z,\theta,\varphi,\lambda)}{dz}=\cdots \notag
  		\end{equation}
	\end{minipage}
  \hfill
  \begin{minipage}[c]{0.3\linewidth}  
		\includegraphics[height=3cm]{/Users/javier/Desktop/Javier/PHD_RIT/20123_Spring/Modeling/HydroLight/Beamer/RadSpec.png}
  \end{minipage}\\
% \end{figure}
% \begin{figure}[htb]
% \resizebox{1.0\textwidth}{!}{%
  \begin{minipage}[c]{1\linewidth}
  \vspace{0.5cm}
	\centering
  \begin{tikzpicture}[node distance=1cm, auto]
          \tikzset{
                  basenode/.style={rectangle,rounded corners,draw=black,very thick, inner sep=1em, minimum size=3em, text centered,text width=3cm},
                  productnode/.style={ellipse,rounded corners,draw=black, very thick, text centered,text width=1.5cm},
                  myarrow/.style={->,>=stealth',thick, double = black},
                  mylabel/.style={text width=7em, text centered}
          }
          %\path[use as bounding box] (0,6) rectangle (4,2);
          \node[basenode] (IOPs) {Inherent Optical Properties};
          \node[basenode, below=of IOPs] (BC) {Boundary Conditions};
          \node[basenode, right=of IOPs] (RTE) {Radiative Transfer Equation};
          \node[basenode, right=of RTE] (rad) {Radiance Distribution};

          \draw[myarrow] (IOPs)--(RTE);
          \draw[myarrow] (BC)-|(RTE);
          \draw[myarrow] (RTE)--(rad);
  \end{tikzpicture}
% } %xobeziser
\end{minipage}
\caption{Hydrolight flow chart. Images taken from \citet{MobleyHE}. \label{fig:HLflowchart} } 
\end{figure}

The HydroLight physical model has the following characteristics \citep{MobleyHE}:

\begin{itemize}
	\item It is time-independent.
	\item Horizontally homogeneous IOPs and boundary conditions $\Rightarrow$ one spatial dimension (depth): no restriction on depth dependence of IOPs.
	\item Wavelength between 300 and 1000 nm.
	\item Finite or infinitely deep (non-Lambertian) water-column bottom.
	\item Arbitrary sky radiance onto sea surface.
	\item Cox-Munk air-water surface (parameterizes gravity and capillary waves via the wind speed)
	\item Various bottom boundary options.
	\item Includes all orders of multiple scattering.
	\item It can optionally include Raman scatter by water.
	\item It can optionally include fluorescence by Chlorophyll-{\it a} and CDOM.
	\item It can optionally include horizontally homogeneous internal sources such as bioluminescing layers.
	\item Polarization not included.
\end{itemize}

The Hydrolight model is of vital importance for this work since the dark pixel of the \gls{mobelm} algorithm and the \gls{lut} used in the developed retrieval are created on it.
% ------------------------------------------
\section{SeaDAS and NASA's Ocean-related Products}
\label{sec:seadas}

The \gls{seadas} software (more info: \url{http://seadas.gsfc.nasa.gov/}) is a comprehensive image analysis package for the processing, display, analysis, and quality control of ocean color data developed by the developers of \gls{esa}'s BEAM software package (\url{http://www.brockmann-consult.de/cms/web/beam/}) and the \gls{obpg} at NASA. Its last version (\gls{seadas} 7.2), recently launched, supports Level 1 Landsat 8 data \citep{Franz:2015}. It can generate Level 2 data from Level 0 and Level 1 through the package named ``l2gen'' tool. The level 2 data is obtained by applying the atmospheric correction method based on \citet{Gordon:1994} to the different heritage ocean color instruments and now Landsat 8, among others. It also can generate Level 3 data, such as temporal binned data. \gls{seadas} includes some  semi-analytical bio-optical models described in \S\ref{subsec:semianalitic}.

A number of standard ocean bio-optical products are supported by the \gls{obpg} at NASA's Goddard Space Flight Center (\url{http://oceancolor.gsfc.nasa.gov/}). The \gls{obpg}'s responsibilities are collection, processing, calibration, validation, archive and distribution of ocean-related products from a large number of operational, satellite-based remote-sensing missions providing ocean color, sea surface temperature and sea surface salinity data to the international research community. Most of the ocean-related products are included in the \gls{seadas} package, supported by the \gls{obpg}. The following section describes some of the products relevant to this research.

% ------------------------------------------
\subsection{Remote-sensing reflectance}
\label{subsec:seadasrrs}
The l2gen tool includes a set of different atmospheric correction methods based on \citet{Gordon:1994}, among others. It also includes the \gls{mumm} algorithm based on \citet{Ruddick:2000bs}. This tool allows one to change the bands used for the atmospheric correction. The user can choose from the NIR or SWIR bands. One of the outputs of the l2gen tool is \gls{rrs} at different wavelengths. These two approaches, the \citet{Gordon:1994} and the \citet{Ruddick:2000bs}, are utilized to compare with the \gls{mobelm} algorithm's results.

% The capability for processing OLI data was recently made fully operational in the SeaDAS version 7.2 \citep{Franz:2015}. 
% ------------------------------------------
\subsection{Chlorophyll-{\it a} Concentration}
\label{subsec:chlempirical}

The chlorophyll-{\it a} concentration retrieval algorithms supported by \gls{seadas} are described in \citet{OReilly1998_Chl} and \citet{OReilly2000}. These include semianalytic models and empirical algorithms. The semianalytic models are based on the $b_b/(a + b_b)$ to $R_{rs}$ relationship \citep{Gordon:1988qv}  (see \S\ref{subsec:semianalitic}). The empirical algorithms are based in band ratios \citep{OReilly2000}. One example of an empirical algorithm is the \gls{oc2} algorithm developed for \gls{seawifs}. The \gls{oc2} algorithm establishes an empirical equation relating \gls{rrs} in the $490$ and $555nm$ bands to chlorophyll-{\it a} concentrations $C_a$ from a large data set of coincident {\it in situ} remote-sensing reflectances and chlorophyll-{\it a} concentrations, $\tilde{R}_{rs}$ and $\tilde{C}_a$. This empirical equation is a \gls{mcp} defined as
\begin{equation}\label{eq:oc2}
  C_a = 10^{\displaystyle a_0+a_1R_2+a_2R_2^2+a_3R_2^3} + a_4
\end{equation}
\noindent where $R_2=\log_{10}\left(R_{555}^{490}\right)$ and $R_{\lambda_j}^{\lambda_i}$ is a compact notation for the $R_{rs}(\lambda_i)/R_{rs}(\lambda_j)$ band ratio \citep{OReilly2000}. The model coefficients were determined from a data set using iterative minimization routines (using IDL). As an example, the \gls{mcp} equation for the \gls{oc2} version 4 (OC2v4; \autoref{fig:chlemp}.a)) is
\begin{equation}
  C_a = 10^{\displaystyle 0.319-2.336R_2+0.879R_2^2-0.135R_2^3} - 0.071
\end{equation}
\noindent where $R_2=\log_{10}\left(R_{555}^{490}\right)$.

The \gls{oc2} algorithm utilized just a green and a blue band. Modifications to this approach have been developed to take advantage of the rest of the blue and green bands. An example of these algorithms is the \gls{oc4} version 4 (OC4v4; \autoref{fig:chlemp}.b) that uses a fourth order polynomial equation defined as
\begin{equation}
  C_a = 10^{\displaystyle 0.366-3.067R_{4S}+1.930R_{4S}^2+0.649R_{4S}^3-1.532R_{4S}^4}
\end{equation}
\noindent where $R_{4S}=\log_{10}\left(R_{555}^{443}>R_{555}^{490}>R_{555}^{510}\right)$, $S$ denotes for \gls{seawifs}.

\begin{figure}[htb]
  \begin{minipage}[c]{0.48\linewidth}
    \centering
      \includegraphics[height=6cm]{/Users/javier/Desktop/Javier/PHD_RIT/Latex/ThesisPHD/Images/OC2v4}  
    % \vspace{1.5cm}
    \centerline{(a)}\medskip
  \end{minipage}
  \hfill
  \begin{minipage}[d]{0.48\linewidth}
    \centering
      \includegraphics[height=6cm]{/Users/javier/Desktop/Javier/PHD_RIT/Latex/ThesisPHD/Images/OC4v4}
    % \vspace{1.5cm}
    \centerline{(b)}\medskip
  \end{minipage}
  \caption{Empirical algorithms for the chlorophyll-{\it a} concentration for \gls{seawifs}. (a) $R_{555}^{490}$ versus $\tilde{C}_a$ and OC2v4 model (solid curve). (b) $R_{555}^{443}>R_{555}^{490}>R_{555}^{510}$ versus $\tilde{C}_a$ and OC4v4 model (solid curve). These empirical algorithms were developed over a data set of $N=2804$ data points (Figures taken from \citet{OReilly2000}). \label{fig:chlemp} } 
\end{figure}

One of the known limitations of the approach is that the data set is mostly representative of Case 1 waters and few Case 2 waters \citep{OReilly2000}. Some modification for low $C_a$ values have been suggested by \citet{Hu:2012fv}.
% ------------------------------------------
\subsection{Total Suspended Matter}

Previous versions of SeaDAS (SeaDAS 5 and earlier) supported the Clark \gls{tsm} product. This product was dropped in SeaDAS 6.2 because the algorithm are not maintained by its original authors, and therefore, it has not been updated. The current version of SeaDAS (SeaDAS 6.2) does not include any \gls{tsm} algorithms. The Clark \gls{tsm} algorithm was an empirical algorithm. For \gls{modis}, the equation is defined as
\begin{align}
  x   &= \log_{10}\left(\frac{L_{wN}[443]+L_{wN}[490]}{L_{wN}[551]}\right)\notag \\
  tsm &= 10^{(\displaystyle a[0] + a[1]x + a[2]x^2 + a[3]x^3 + a[4]x^4 + a[5]x^5)}
\end{align}
\noindent where $L_{wN}$ is the normalized water leaving radiance, and the model coefficients $a[0]\dots a[5]$ are obtained from fitting a $5^{th}$ order polynomial curve to {\it in situ} \gls{tsm} concentrations versus $x$ data. This kind of approach is fundamentally a regional relationship that relates local concentrations with optical signatures. More info about the algorithm can be found at \url{http://oceancolor.gsfc.nasa.gov/forum/oceancolor/forum_show.pl}.
% ------------------------------------------
\subsection{Colored Dissolved Organic Matter}

Previous versions of SeaDAS (SeaDAS 5 and earlier) supported the colored dissolved organic matter (CDOM) index ({\tt cdom\_index}) product described in \citet{Morel2009_CDOM_index}. This {\tt cdom\_index} quantifies the deviation in the relationship between CDOM and chlorophyll concentration. {\tt cdom\_index = 1.0} represents the mean relationship for Morel and Prieur case 1 waters \citep{Morel:1977rw}, and values above or below 1.0 indicate excess or deficit in CDOM relative to that mean relationship, respectively. The {\tt cdom\_index} is based on a unique equation relating the $R(412)/R(443)$ ratio to the $R(490)/R(555)$ ratio, where $R(\lambda)$ is spectral reflectance of the ocean at wavelength $\lambda$ derived from ocean color remote sensing data. This product was dropped from the current SeaDAS version (SeaDAS 6.2).

The current SeaDAS version (SeaDAS 6.2) include semianalytic bio-optical models that can be related to CDOM (see \S\ref{subsec:semianalitic}). Examples of these models are the Garver-Siegel-Maritorena-2001 absorption due to gelbstof and detrital material at sensor wavelength nnn, ``{\tt adg\_nnn\_gsm01}'' \citep{Maritorena:02}, and the Quasi-Analytical Algorithm (QAA) model absorption due to gelbstof and detrital material at sensor wavelength nnn, ``{\tt adg\_nnn\_qaa}'' \citep{Lee_Du_Arnone2005}.

From all the products from the \gls{seadas} tool described above, only the \gls{rrs} and the $C_a$ products are used in this work for comparison with the results of the developed atmospheric correction and retrieval algorithms.
% ----------------------------------------------------
\section{ACOLITE}
\label{subsec:acolite}

The \gls{acolite} package is a binary distribution for processing Lansat 8 imagery developed by the \gls{rbins} (more info \url{http://odnature.naturalsciences.be/remsem/software-and-data/acolite}). \gls{acolite}'s main use is the atmospheric correction for OLI data. It applies the atmospheric correction algorithms based on \citet{Gordon:1994}. The atmospheric correction used is described in \citet{Vanhellemont2014} and \citet{Vanhellemont:2015}. \gls{acolite} generates products as RGB images for TOA, after Rayleigh correction, and atmospherically corrected. It also generates Level 2 output, such as marine reflectance and multiple scattering aerosol reflectance.

The \gls{acolite} tool is used to test the \citet{Gordon:1994} approach on Landsat 8 data and for comparison with the results of the developed \gls{mobelm} atmospheric correction algorithm.

% ----------------------------------------------------
\section{Challenges in the remote sensing of Case 2 waters}
The single-variable models used for Case 1 waters do not work well on Case 2 waters since at least three relevant quantities can vary independently of each other. This implies that we cannot assume a one-to-one relation between optical properties and the pigment concentration \citep{IOCCG2000_Coastal} anymore. Therefore, the algorithms for the retrieval of CPAs for Case 2 waters need to address non-linear, spectrally-varying interactions. For example, at some wavelengths, two or more constituents could affect the water-leaving signal in a similar manner, making it difficult to differentiate among them. Also, the change in signal with change in concentration of CPAs are often very small, and therefore, it is difficult to guarantee the quality and reliability of the extracted information, a fact that becomes even worse with atmospheric effects.

\autoref{fig:OneCPAfixed} shows water true color images simulated in Hydrolight for different concentration of CPAs but one of them fixed. A fixed set of {\it in situ} IOPs was used as input to Hydrolight, for different level of CPAs concentrations. The RGB bands of the remote-sensing reflectance outputs from these Hydrolight runs were utilized to generate the figures (\autoref{fig:OneCPAfixed}). It can be seen that for fixed concentration of CPAs, the variability of even one CPA impacts dramatically the water color. This fact demonstrates that not only the pigment concentration impacts the water color, but also the rest of the components, as opposed to the Case 1 water case, where the rest of the components are correlated with the pigment concentration. \autoref{fig:OCedu_CHLfixed} shows a similar effect, where for fixed chlorophyll-{\it a} concentration, the variability in sediments and CDOM produces different spectra. Also, varying chlorophyll-{\it a} from low, medium to high level, the spectra fluctuate, even for the same set of sediment and CDOM. 
\begin{figure}[htb]
  \begin{minipage}[c]{0.48\linewidth}
    \centering
      \includegraphics[width=7cm]{/Users/javier/Desktop/Javier/PHD_RIT/Latex/ThesisPHD/Images/OCedu_CDOMfixed}\\
      \centerline{(a)}\medskip
  \end{minipage}     
      \hfill
  \begin{minipage}[c]{0.48\linewidth}
    \centering
      \includegraphics[width=7cm]{/Users/javier/Desktop/Javier/PHD_RIT/Latex/ThesisPHD/Images/OCedu_TSSfixed}\\
      \centerline{(b)}\medskip
  \end{minipage}
      \caption{Simulated water color with (a) fixed \gls{cdom} absorption coefficient at $\lambda=440nm$, $a_{CDOM}(440)$ and variable sediments and chlorophyll-{\it a} concentrations, and (b) fixed sediments and variable CDOM and chlorophyll-{\it a} concentrations.\label{fig:OneCPAfixed}}
\end{figure}

\begin{figure}[htb]
  \begin{minipage}[c]{0.48\linewidth}
    \centering
      \includegraphics[width=7cm]{/Users/javier/Desktop/Javier/PHD_RIT/Latex/ThesisPHD/Images/OCedu_CHLfixedlow.png}\\
      \centerline{(a)}\medskip
    \end{minipage} 
   \hfill   
   \begin{minipage}[c]{0.48\linewidth}
    \centering
      \includegraphics[width=7cm]{/Users/javier/Desktop/Javier/PHD_RIT/Latex/ThesisPHD/Images/OCedu_CHLfixedmed.png}\\
      \centerline{(b)}\medskip
  \end{minipage}    
  \begin{minipage}[c]{1.0\linewidth}
    \centering
      \includegraphics[width=7cm]{/Users/javier/Desktop/Javier/PHD_RIT/Latex/ThesisPHD/Images/OCedu_CHLfixedhigh.png}\\
      \centerline{(c)}\medskip
  \end{minipage}    
      \caption{Fixed chlorophyll-{\it a} concentration, $C_a$, and variable sediments and CDOM for (a) low $C_a$, (b) medium $C_a$ and (c) high $C_a$ .\label{fig:OCedu_CHLfixed} }
\end{figure}
\autoref{fig:SmallConcImpact} shows the effect of different levels of chlorophyll-{\it a} in the spectral remote-sensing reflectance for low ($[5-10] \mu g/L$) and high ($[60-65]\mu/L$) levels. It can be seen that the same relative difference has different impacts at low and high levels of chlorophyll-{\it a} concentration. There are more differences at low levels than at high levels.
\begin{figure}[htb!]
\centering
      \includegraphics[width=8cm]{/Users/javier/Desktop/Javier/PHD_RIT/Latex/ThesisPHD/Images/SmallConcImpact.png}
      \caption{Hydrolight simulated remote-sensing reflectance for $C_a$ concentration differences at high level and low levels.\label{fig:SmallConcImpact} } 
\end{figure}

The standard empirical retrieval algorithms described in \citet{OReilly2000} (\S\ref{subsec:chlempirical}) were developed for Case 1 waters. The same approach cannot be used in the Case 2 waters due to the variability of each component. This is illustrated in \autoref{fig:ChlStandardFixedAntNot}, where the OC3 and OC2v4 models are compared with simulated Hydrolight data. \autoref{fig:ChlStandardFixedAntNot}.(a) shows the case when only the chlorophyll-{\it a} concentration is varying, showing a good agreement with the empirical algorithms. On the other hand, \autoref{fig:ChlStandardFixedAntNot}.(b) shows the case when not only the chlorophyll-{\it a} concentration varies, but all the CPAs. It can be concluded that the empirical models do not fit for all cases. These is one of the reason why new approaches need to be developed.

\begin{figure}[htb!]
  \begin{minipage}[c]{0.48\linewidth}
    \centering
      \includegraphics[width=7cm]{/Users/javier/Desktop/Javier/PHD_RIT/Latex/ThesisPHD/Images/ChlStandardFixed.png}\\
      \centerline{(a)}\medskip
  \end{minipage}
  \hfill      
  \begin{minipage}[c]{0.48\linewidth}
    \centering
      \includegraphics[width=7cm]{/Users/javier/Desktop/Javier/PHD_RIT/Latex/ThesisPHD/Images/ChlStandardNoFixed.png}\\
      \centerline{(b)}\medskip
  \end{minipage}
      \caption{$C_a$ versus blue green band ratio for the OC3 and OC2v4 standard model compared with Hydrolight simulated data for (a) fixed and (b) variable sediment and CDOM.\label{fig:ChlStandardFixedAntNot} } 
\end{figure}

Furthermore, higher particulate matter concentration, either phytoplankton or inorganic material, typical of coastal waters, could change the assumption for atmospheric correction based on \citet{Gordon:1994} of negligible water signal in the NIR. In order to apply similar approaches, we are forced to use longer wavelengths, which in general are more noisy bands, affecting the quality of the atmospheric correction, and therefore the retrieval.

Finally, if we want to study areas that have direct influences in human activities, such a coastal areas, rivers, or lakes, the need for higher spatial resolution is a must in order to resolve complex structure in near shore waters. Because these waters are located near to land area, the adjacency effect needs to be taken into consideration.

% \section{State of the Research}

% As shown in \citet{GeraceThesis}
% and \citet{Mobley:2005} and \citet{Lesser}

% ----------------------------------------------------
\section{Concluding Remarks}
An overview of fundamental concepts that are relevant to this research was presented in this chapter. We began by describing the different energy contributions to the signal captured by the sensor followed by a description of the water constituents and their optical properties in the form of absorption and scattering coefficients. We continued with the description of how the atmosphere interacts with photons and its effect on the total signal along with the glint effect due to energy reflected by the water surface from the sun and sky. The in water radiative transfer theory was also treated by describing the variables that form the RTE. A brief description of the Landsat 8 instrument was described. Finally, some tools used in this research were overviewed. These tools are the ELM atmospheric correction method, the Landsat surface Reflectance CDR products and Hydrolight. The next section will present the methodology that will be used to accomplish this research's goal.

% !TEX root=Thesis_PhD.tex  
% the previous is to reference main .bib
%% CHAPTER
\chapter{Methodology and Approach}
\label{ch:method}

% %%%%%%%%%%%%%%%%%%%%%%%%%%%%%%%%%%%%%%%%%%%%%%%%%%%%%%%%%%%%%%%%%%%%%%%
\begin{figure}[!ht]
  \centering
% \resizebox{9cm}{!}{%
% Define block styles

\tikzstyle{startstop} = [rectangle, rounded corners, minimum width=2em, minimum height=2em,text centered, draw=black, fill=red!30]
\tikzstyle{io} = [trapezium, trapezium left angle=70, trapezium right angle=110, minimum width=1em, minimum height=3em, text width=1.0cm, text centered, draw=black, fill=blue!30]
\tikzstyle{process} = [rectangle, minimum width=2em, minimum height=2em, text width=2.0cm, text centered, draw=black, fill=orange!30]
\tikzstyle{decision} = [diamond, minimum width=2em, minimum height=2em, text centered, draw=black, fill=green!30]
\tikzstyle{arrow} = [thick,->,>=stealth]

\begin{tikzpicture}[node distance=3cm]

% \node (start) [startstop] {Start};

\node (L8image) [io] {Landsat \\Image};
\node (prepro) [process, right of=L8image] {Image \\pre-processing};
\node (AtmCorr) [process, right of=prepro] {Atmospheric Correction};
\node (RetProc) [process, right of=AtmCorr] {Retrieval Process};
\node (CPAmap) [io, right of=RetProc] {CPA \\maps};

\node (Val) [process, below of=CPAmap, yshift=1cm] {Validation};

% \node[align=left, below=0.0 of prepro] (List1) {\scriptsize - Cloud/Land/Water Mask\\\scriptsize - Radiometric Calibration};

\draw [arrow] (L8image) -- (prepro);
\draw [arrow] (prepro) -- (AtmCorr);
\draw [arrow] (AtmCorr) -- (RetProc);
\draw [arrow] (RetProc) -- (CPAmap);

\draw [arrow] (CPAmap) -- (Val);

\end{tikzpicture}

% } % resizebox end

\caption{High-level flowchart of the process. \label{fig:highlevelflowchart}}
\end{figure}
\todo{correct space!}

% %%%%%%%%%%%%%%%%%%%%%%%%%%%%%%%%%%%%%%%%%%%%%%%%%%%%%%%%%%%%%%%%%%%%%%%
\begin{figure}[H]
\vspace{1cm}
  \centering
% \resizebox{9cm}{!}{%
% Define block styles

\tikzstyle{startstop} = [rectangle, rounded corners, minimum width=2em, minimum height=2em,text centered, draw=black, fill=red!30]

\tikzstyle{io} = [trapezium, trapezium left angle=70, trapezium right angle=110, minimum width=2em, minimum height=2em, text width=1.5cm, text centered, draw=black, fill=blue!30]

\tikzstyle{iosmall} = [trapezium, trapezium left angle=70, trapezium right angle=110, minimum width=1em, minimum height=1em, text width=1.0cm, text centered, draw=black, fill=blue!30]

\tikzstyle{process} = [rectangle, minimum width=3em, minimum height=2em, text width=2.5cm, text centered, draw=black, fill=orange!30]

\tikzstyle{decision} = [diamond, minimum width=2em, minimum height=2em, text centered, draw=black, fill=green!30]

\tikzstyle{arrow} = [thick,->,>=stealth]
\tikzstyle{arrowdashed} = [thick,dashed,->,>=stealth]

\begin{tikzpicture}[node distance=2cm]


% \node (start) [startstop] {Start};

\node (L8image) [io] {Landsat \\Image (DN)};
\node (radcal) [process, below of=L8image] {Radiometric \\Calibration};
\node (resize) [process, below of=radcal] {Resize};
\node (mask) [process, below of=resize] {Land/Cloud/\\Cloud Shadow \\Mask};
\node (AtmCorr) [process, below of=mask] {Atmospheric \\Correction \\(MoB-ELM)};
\node (rrs) [io, below of=AtmCorr,yshift=-1.5cm] {$R_{rs}$ \\Image};
\node (RetProc) [process, below of=rrs] {Retrieval \\Process \\(RMSE/ \\lsqnonlin)};
\node (CPAmap) [io, below of=RetProc] {CPA \\maps};

\node (CDRImage) [io, left of=L8image, xshift=-1.5cm] {Landsat \\Reflectance \\Product};
\node (PIFmask) [process, below of=CDRImage] {PIF \\Mask};
\node (brightpx) [io, below of=PIFmask] {Bright\\Pixel};

\node (AOPs) [iosmall, right of=L8image, xshift=0.8cm] {{\it in situ} \\AOPs \\($R_{rs}$)};

\node (samples) [io, right of=AOPs, xshift=1.0cm] {Water \\Samples};
\node (labmea) [process, below of=samples] {Lab \\Mea.};
\node (IOPs) [iosmall, below of=labmea, xshift=-1cm] {IOPs};
\node (conc) [iosmall, below of=labmea, xshift=1cm] {Conc.};
\node (hydro) [process, below of=IOPs, xshift=1cm] {Hydrolight};
\node (darkpx) [iosmall, below of=hydro, xshift=-1.5cm] {Dark \\Pixel};
\node (LUT) [iosmall, below of=hydro, xshift=2.5cm] {LUT};

\node (bb_b) [iosmall, right of=hydro, xshift=1cm] {$b_b/b$};


\node (Comp) [process, right of=CPAmap, xshift=1.5cm] {Comparison};
\node (NRMSE) [io, below of=Comp] {NRMSE};

\node (Comprrs) [process, right of=rrs, xshift=0.8cm, yshift=1.5cm] {Comparison};
\node (NRMSE2) [io, right of=Comprrs, xshift=1.4cm] {NRMSE};

% \node[align=left, right=0.0 of prepro] (List1) {\scriptsize - Cloud/Land/Water Mask\\\scriptsize - Radiometric Calibration};

\draw [arrow] (L8image) -- (radcal);
\draw [arrow] (radcal) -- (resize);
\draw [arrow] (resize) -- (mask);
\draw [arrow] (mask) -- (AtmCorr);
\draw [arrow] (AtmCorr) -- (rrs);
\draw [arrow] (rrs) -- (RetProc);
\draw [arrow] (RetProc) -- (CPAmap);

\draw [arrowdashed] (L8image) -- (CDRImage);
\draw [arrow] (CDRImage) -- (PIFmask);
\draw [arrow] (PIFmask) -- (brightpx);
\draw [arrow] (brightpx) |- (AtmCorr);

\draw [arrow] (samples) -- (labmea);
\draw [arrow] (labmea) |- (IOPs);
\draw [arrow] (labmea) |- (conc);
\draw [arrow] (IOPs) -- (hydro);
\draw [arrow] (conc) -- (hydro);
\draw [arrow] (hydro) |- (darkpx);
\draw [arrow] (hydro) |- (LUT);
\draw [arrow] (darkpx) -- (AtmCorr);
\draw [arrow] (LUT) |- (RetProc);

\draw [arrowdashed] (AOPs) |- (hydro);
\draw [arrowdashed] (AOPs) -- (Comprrs);
\draw [arrowdashed] (rrs.east) -| (Comprrs.south);
\draw [arrowdashed] (Comprrs) -- (NRMSE2);

\draw [arrow] (conc.east) |- ([xshift=2.3cm]conc.east)  |- (Comp);
\draw [arrow] (CPAmap) -- (Comp);
\draw [arrow] (Comp) -- (NRMSE);

\draw [arrow] (hydro) -- (bb_b);

\draw [arrow] (bb_b.south) -- ([yshift=-0.8cm]bb_b.south) -| ([xshift=-0.5cm]hydro.south east);

% \draw [arrow] (hydro) |- (AtmCorr);
\end{tikzpicture}
% } % resizebox end
\caption{Detailed flowchart of the process. \label{fig:detailflowchart}}
\end{figure}

This chapter describes the methodology and approach taken to accomplish the specific objectives mentioned in \S\ref{ch:objectives}. As a review, these specific requirements are (\S\ref{sec:objectives}):
\begin{enumerate}
  \item To develop an over-water atmospheric correction algorithm for Landsat 8 reflective imagery.
  \item To design a water constituent concentration retrieval algorithm that can be applied to a specific study area.
  \item To validate results by comparing with {\it in situ} measurements and products from ocean color satellites.
\end{enumerate}

The processes described in this work were first introduced by \citet{Concha2013IGARSS} and are based on previous work done by \citet{Raqueno:2000}, \citet{Gerace:2012}, \citet{Pahlevan:2012b} and \citet{Gerace:2013}. A flowchart of the general processes involved is shown in \autoref{fig:highlevelflowchart}. The inputs are the Landsat 8 images and the final products are the maps of CPAs. The complete process can be divided into four different steps: image pre-processing, atmospheric correction, retrieval process, and validation. This chapter will be divided into these four processes. A more detailed flowchart is illustrated in \autoref{fig:detailflowchart} and a more pictorial diagram is shown in \autoref{fig:retrieval}.

In short, the whole process works as follows. First, the \gls{dn} image from the satellite is radiometrically calibrated, resized to include only the area of study, and masked for cloud and land. Then, the radiance image is atmospherically corrected and transformed to $R_{rs}$ spectra, as shown in \autoref{fig:detailflowchart}. At this point the $R_{rs}$ spectra of water pixels in the scene with unknown CPAs concentrations are compared with a LUT of $R_{rs}$ spectra with known CPAs concentrations. This comparison is made by using a spectrum-matching technique \citep{Raqueno:2000,Mobley:2005} that uses the \gls{rmse} and a non-linear optimization algorithm to find the closest match in the LUT. The end products are maps of CPA concentrations (see \autoref{fig:retrieval}). Finally, the CPA maps are validated by comparison with {\it in situ} data and standard bio-optical products. Each step in this retrieval process is described in more detail below.

% %%%%%%%%%%%%%%%%%%%%%%%%%%%%%%%%%%%%%%%%%%%%%%%%%%%%%%%%%%%%%%%%%%%%%%%
\begin{figure}[htb]
  \centering
    \includegraphics[height=10cm]{/Users/javier/Desktop/Javier/PHD_RIT/Latex/ThesisPhD/Images/UltimateProcess.pdf}
    \caption{Retrieval process flow diagram. The radiance image from the satellite is first corrected for atmospheric effects, having as result $R_{rs}$ spectra for each water pixel in the image. Then, a spectrum-matching methodology is used to find the closest match in the least-squared sense for the water pixels with unknown CPAs concentration in the LUT of water pixels with known CPAs concentration. The final result is a concentration map for each CPA.  \label{fig:retrieval} }
\end{figure}
% %%%%%%%%%%%%%%%%%%%%%%%%%%%%%%%%%%%%%%%%%%%%%%%%%%%%%%%%%%%%%%%%
% @@@@@@@@@@@@@@@@@@@@@@@@@@@@@@@@@@@@@@@@@@@@@@@@@@@@@@@@@@@@@@@@
\section{Image Pre-processing} 
\label{sec:prepro}

The images are downloaded for free from the server \url{http://earthexplorer.usgs.gov/} by \gls{usgs}. EarthExplorer contains data sets from several instruments, including the Landsat satellites (labeled as Landsat Archive). The user needs to create an account in order to download the Level 1 GeoTIFF Data Product (data in \gls{dn}). This data product includes the OLI's nine bands, plus a \gls{qa} band, the TIRS's two bands, and the metadata file (\_MTL). The images are downloaded based on search criteria such as location and time. These images are first ordered on demand for processing. Once the processing is ready, they can be downloaded from the site.

Once the images are downloaded, the next step is to perform a radiometric calibration to convert from \gls{dn} to \gls{toa} radiance. The radiometric calibration can be done in imaging processing tools, such an ENVI, which support Landsat 8 images. One of formulas to convert \gls{dn} to \gls{toa} radiance $L_{\lambda}$ using gain and bias values is
\begin{equation}
  L_{\lambda} = gain_{\lambda}*DN+bias_{\lambda}
\end{equation}
\noindent where $gain_{\lambda}$ and $bias_{\lambda}$ are provided in the MTL file. The image is then resized using the resize tool in the ENVI package in order to include only the area of study and make the process less time consuming.

Then, the following step is to correct the glint in the image, if present. Details of one glint removal method are described in \S\ref{subsec:glintremoval} below.

Finally, only the valid water pixels need to be used in the retrieval process. To do so, the land, cloud and cloud shadow need to be masked. Also, if bottom signal is present in the image, it needs to be masked, or not considered in the retrieval, since the bottom signal was not treated in this study. The water mask is created by thresholding the Landsat 8's SWIR 2 band, since the water pixels will have small values compared with the rest of the pixels, including land. The cloud and cloud shadow mask are obtained from the Landsat 8 Surface Reflectance Product's \citep{L8SurfProduct2015} cloud mask described in \S\ref{subsec:provisionalCDR}. Once all the masks have been created, they are combined using an $.AND.$ logic operation.

% @@@@@@@@@@@@@@@@@@@@@@@@@@@@@@@@@@@@@@@@@@@@@@@@@@@@@@@@@@@@@@@@
\section{Over-Water Atmospheric Correction} 
\label{sec:atmcorr}
The first objective in this research is to identify a suitable approach to atmospherically correct the type of dataset provided by the OLI sensor. This is a complex task to perform over water because the signal leaving the water that reaches the sensor is very small compared to the signal reaching the sensor from atmospheric scattering. In ocean color, the end goal of the atmospheric correction is to convert from radiance units to the variable of interest for ocean color: \acrfull{rhow} or \acrfull{rrs}.

Most of the atmospheric correction algorithms used for the heritage Ocean Color instruments (e.g. CZCS, MODIS and SeaWiFS) are based on the work done by \citet{Gordon:1994} (also described in \citet{Gordon:1997}). The NASA's standard algorithms for atmospheric correction based on \citet{Gordon:1994} have proved to work well in Case 1 waters where there are at least two wavelengths with a water-leaving signal negligible or known. Therefore, these algorithms could be considered a global solution for Case 1 waters, i.e. they could be applied in most cases. The wavelengths used for clear waters (Case 1) are in general in the near infrared (NIR) part of the spectrum. For highly turbid water (Case 2) or highly productive Case 1 waters, a combination of wavelengths in the NIR and shortwave infrared (SWIR), or two SWIR wavelengths can be used \citep{Wang:2007,Wang:2007dz,Wang2009}. Some efforts have been made to demonstrate the feasibility of using Landsat 8 for ocean color measurements in coastal waters that apply the Gordon and Wang approach for the atmospheric correction using the Landsat 8's NIR and SWIR 1 bands, or the SWIR 1 and SWIR 2 bands \citep{Vanhellemont2014a,Vanhellemont2014,Vanhellemont:2015,Franz:2015}. One problem with using the Gordon and Wang approach on Landsat 8 imagery is that it requires a sufficient \gls{snr} over water in order to discriminate the water-leaving signal from the instrument noise. This could be particular important for OLI since it has noisy SWIR bands (relative to ocean color instruments), and so derived ocean color products that use these bands for the aerosol determination can be noisy \citep{Vanhellemont2014a}.

However, some of the atmospheric correction algorithms applied to the heritage Ocean Color instruments are not suitable for highly turbid coastal waters because the {\it black pixel assumption} cannot be applied to these types of waters \citep{Patt2003}. \citet{Concha2014SPIE} developed the \gls{mobelm} as an alternative for the standard atmospheric correction algorithms that does not rely on a negligible water-leaving signal assumption, and therefore it is not as sensitive to the SNR. The MoB-ELM algorithm uses a bright and dark pixel from the image to perform the atmospheric correction. % One of the goals of this work is to compare the MoB-ELM with the standard algorithms as well as with {\it in situ} data. These comparisons are made in remote-sensing reflectance $R_{rs}$ units. 

Two approaches for atmospheric compensation are investigated in this research: the \citet{Gordon:1994} approach, and the \citet{Concha2014SPIE} approach.

Some atmospheric correction methods described in this section will use the notation used by \citet{Gordon:1994} and \citet{Ruddick:2000bs} who define the \gls{rho} as 

\begin{equation}\label{eq:rho}
  \rho = \frac{\pi L}{F_o \cos{\theta}}
\end{equation}
where $L$ is upward radiance in the given viewing direction, $F_o$ is exoatmospheric irradiance, and $\theta$ is the solar-zenith angle. \autoref{eq:rho} allows a direct transformation from radiance to reflectance and vice versa. Therefore, taking into account all its contributors, the governing equation for \gls{rhot} can be expressed as

\begin{equation}\label{eq:rho_t}
  \rho_t(\lambda) = \rho_r(\lambda) + \rho_a(\lambda) + \rho_{ra}(\lambda) + T_v[\rho_w(\lambda) + \rho_{wc}(\lambda)]
\end{equation}
where:\\
\indent $\rho_t(\lambda)$ is the reflectance at the top of the atmosphere \\
\indent $\rho_r(\lambda)$ is the reflectance due to multiple scatter by air molecules only (Rayleigh scattering)\\
\indent $\rho_a(\lambda)$ is the reflectance due to multiple scatter by aerosols only\\
\indent $\rho_{ra}(\lambda)$ is the reflectance due to the interaction between Rayleigh and aerosol scattering\\
\indent $T_v(\lambda)$ is the diffuse atmospheric transmittance from the water to the sensor\\
\indent $\rho_{wc}(\lambda)$ is the reflectance due to solar photons reflecting off the air-water interface (from whitecaps and glint or glitter)\\
\indent $\rho_w(\lambda)$ is the water-leaving reflectance

The \acrfull{rhow}\index{marine reflectance $\rho_w$} (a.k.a. marine reflectance) is related to the \acrfull{rrs} as
\begin{equation}\label{eq:Rrs_And_rhow}
 R_{rs} = \frac{\rho_w}{\pi}~~~\left[\frac{1}{sr}\right]
\end{equation}

In order to perform the water constituent retrieval, we need to solve for only the $\rho_w(\lambda)$ term in \autoref{eq:rho_t}, which would be an easy task if all the rest of the terms were known. Unfortunately, the only term known {\it a priori} is the $\rho_t(\lambda)$, which is precisely the image of the scene itself. The main difference among the methods based on \citet{Gordon:1994} is in the approach used to estimate $\rho_a(\lambda) + \rho_{ra}(\lambda)$ in the visible (VIS) using an estimation of $\rho_a(\lambda) + \rho_{ra}(\lambda)$ in the near infrared (NIR).

This section describes the different alternatives to obtain the rest of the terms in \autoref{eq:rho_t}.

% ---------------------------------------------------------------------------
\subsection{Solar-Glint Removal Algorithm}
\label{subsec:glintremoval}
The first step to solve \autoref{eq:rho_t} is to find the term due to glint, $\rho_{wc}(\lambda)$ and remove it from the total signal. This term $\rho_{wc}(\lambda)$ can be ignored in some cases since ocean-color sensors are designed to be tilted to avoid the specular image of the sun. However, Landsat 8 is not cataloged as an ocean-color satellite and it may need to be corrected for the glint effect depending on the location of the water in the scene, the sun location, and the wind conditions. Sometimes, the influence of glint can be strong, making it difficult to process the images, as can be seen in the Landsat 8 image over the study area acquired on 07-11-2014 (LC80170302014192LGN00), shown in \autoref{fig:L8glint}, with a sun elevation equal to $62^\circ$. Note the structure underneath the glint.
% %%%%%%%%%%%%%%%%%%%%%%%%%%%%%%%%%%%%%%%%%%%%%%%%%%%%%%%%%%%%%%%%%%%%%%%
\begin{figure}[htb]
  \centering
    \includegraphics[height=10cm]{/Users/javier/Desktop/Javier/PHD_RIT/Latex/ThesisPhD/Images/L8Glint}
    \caption{Landsat 8 image acquired on 07-11-2014 (LC80170302014192LGN00) with a strong glint over the study area.  \label{fig:L8glint} }
\end{figure}
% %%%%%%%%%%%%%%%%%%%%%%%%%%%%%%%%%%%%%%%%%%%%%%%%%%%%%%%%%%%%%%%%

An example of a glint removal algorithm is the method suggested by \citet{Hedley:2005}, which is a revised version of the method suggested by \citet{Hochberg:2003}. The method described by Hochberg was developed for high spatial resolutions where glint effects occur at physical scales comparable to image pixels ($<10m$) as opposed to methods developed for ocean-color sensor, which tend to have large physical scales ($>1km$).
The method presented by \citet{Hochberg:2003} is sensitive to outlier pixels and it needs to mask the land and cloud areas before deglinting. The method suggested by \citet{Hedley:2005} overcomes these inconveniences. This method assumes that (a) the brightness in the NIR is composed only of sun glint and a spatially constant ambient NIR component associated with NIR backscatter in the atmosphere (if the image is not atmospherically corrected), and (b) that the amount of sun glint in the visible bands is linearly related to the brightness (glint) in the NIR band. 

The first assumption is true for waters that are not highly turbid since water is relatively opaque to NIR wavelength ($700-1000nm$). The second assumption of a linear relationship between NIR brightness and the amount of sun glint in the visible bands relies on the fact that the real index of refraction, which is associated with the reflection in the water surface, is nearly equal for NIR and visible wavelengths \citep{Mobley1994}. As a result, the amount of light reflected in the visible bands is proportional to the amount of light reflected in the NIR, and therefore, a linear relationship can be established among them. The first step is to establish this relationship. This is accomplished by solving a linear regression between NIR and visible bands using one or more ROIs (\gls{roi}) from the water pixels in the image where some sun glint is noticeable, and their pixels values would be of similar values otherwise. An example of a ROI could be a ROI over deep water in the lake. A slope is determined from solving this linear regression with NIR pixel values in the ROI as the independent variable ($x$-axis) and a particular visible band as the dependent variable ($y$-axis), as shown in \autoref{fig:regressiohedley}. If the regression slope for band $i$ is $b_i$, then sun-glint corrected pixel brightness in band $i$ can be obtained by applying the following formula
\begin{equation}\label{eq:deglint}
  R_i' = R_i - b_i(R_{NIR}-min_{NIR})
\end{equation}
where $R_i'$ is the sun-glint corrected pixel value in band $i$, $R_i$ is the pixel value in the visible band $i$, $R_{NIR}$ is the NIR pixel value, and $min_{NIR}$ is the ambient NIR level and it represents the NIR brightness of a pixel with no sun glint. $min_{NIR}$ can be the minimum NIR value in the ROI used in the linear regression or as the minimum NIR in the water pixels. This method has the advantage that it operates purely on the relative magnitudes of values, and therefore the absolutes magnitudes are not important. The author suggests use of ROIs from different regions in the image, including ROIs where there is not glint at all. An example of an image before and after applying this method is shown in \autoref{fig:HedlyGlint}.
% %%%%%%%%%%%%%%%%%%%%%%%%%%%%%%%%%%%%%%%%%%%%%%%%%%%%%%%%%%%%%%%%%%%%%%%
\begin{figure}[htb]
  \centering
  \includegraphics[width=11cm,clip=true]{/Users/javier/Desktop/Javier/PHD_RIT/Latex/Proposal/Images/RegressionHedley.png}
  \caption{Linear regression used in the deglinting process (Note: image taken from \citet{Hedley:2005}). \label{fig:regressiohedley} } 
\end{figure}
% %%%%%%%%%%%%%%%%%%%%%%%%%%%%%%%%%%%%%%%%%%%%%%%%%%%%%%%%%%%%%%%%%%%%%%%
\begin{figure}[htb]
  \centering
    \includegraphics[height=8cm]{/Users/javier/Desktop/Javier/PHD_RIT/Latex/ThesisPhD/Images/HedleyGlint}
    \caption{Example of the glint removal algorithm (a) before and (b) after correction (image taken from \citet{Hedley:2005}).  \label{fig:HedlyGlint} }
\end{figure}
% %%%%%%%%%%%%%%%%%%%%%%%%%%%%%%%%%%%%%%%%%%%%%%%%%%%%%%%%%%%%%%%%

The different steps are summarized as follows:\\
\noindent{\bf Solar-Deglinting Algorithm Summary}\\
{\bf Step 1:} Select a ROI in the image where there is a range of sun glint, but where the brightness values would be uniform otherwise.\\
{\bf Step 2:} Determine $min_{NIR}$ by selecting the minimum NIR value of ROI.\\
{\bf Step 3:} Determine the slope $b_i$ from a linear regression between the NIR values ($x$-axis) and the visible band $i$ to be deglinted ($y$-axis).\\
{\bf Step 4:} Deglint all pixels in the image using \autoref{eq:deglint}.\\
{\bf Step 5:} Repeat step 1-4 for each band to be deglinted.

Some considerations need to be taken into account in order to apply this method. The first assumption is only valid when there is no water with high concentration of sediment present in the image, so precautions have to be taken to avoid highly turbid water in the image. As a way to overcome this problem, the OLI's SWIR 1 band (band 6) could be used instead of the NIR band since the water-leaving signal is negligible at $1600nm$, as suggested by \citet{GeraceThesis}. The real part of the index of refraction is still nearly equal for all wavelengths in the VNIR/SWIR, and therefore the second assumption of a linear relationship between NIR values and the amount of sun glint in the visible band is still valid. Another consideration is that if the ROIs selected include non-submerged objects (i.e. buoys, boats, land), this algorithm could output negative values. Therefore, it is recommended to mask all non-submerged objects before applying this method. 

The purpose of the deglinting process is to find the term $T_v\rho_{wc}$ in \autoref{eq:rho_t} and subtract it from the total TOA reflectance $\rho_t$,

\begin{equation}\label{eq:rhodeglint}
  \rho_t(\lambda)-T_v(\lambda)\rho_{wc}(\lambda) = \rho_r(\lambda)+\rho_a(\lambda)+\rho_{ra}(\lambda)+T_v(\lambda)\rho_{w}(\lambda)
\end{equation}

Once the water pixels in the image are deglinted, the next step is to atmospherically correct the image in order to isolate $\rho_w$ from \autoref{eq:rhodeglint}. There was not visual evidence of glint in the scenes used in this work, and therefore, they were not corrected for glint.


% ------------------------------------------------------------------
\subsection{Model-Based Empirical Line Method (MoB-ELM) Atmospheric Correction Method}
\label{subsec:mobelm}

The first atmospheric correction method will be the \gls{mobelm}\index{atmospheric correction!MoB-ELM} algorithm based on previous work done by \citet{Gerace:2013} and \citet{Gerace:2012}  for simulated OLI data and adapted by \citet{Concha2014SPIE} for actual OLI data. While this new method is based on the traditional ELM method (see \S\ref{sec:ELM}), this MoB-ELM method tries to avoid the measurement of ground truth at every sensor overpass over the scene by using \glspl{pif} in the scene as one target along with an estimation of water reflectivity for an open lake region for the other target. In the MoB-ELM atmospheric correction method, the dark pixel is obtained from a run of the radiative transfer model Hydrolight \citep{MobleyHEtech} simulating a water pixel in the scene with known CPAs concentrations and inherent optical properties (IOPs). The bright pixel is obtained from the Provisional Landsat 8 Surface Reflectance product \citep{L8SurfProduct2015} from USGS over a bright object in the scene or an average of bright pixels in the scene. The two targets used in this MoB-ELM to solve the regression in \autoref{eq:ELM} are referred to in this documents as the {\it bright pixel} \index{bright pixel} and the {\it dark pixel}\index{dark pixel}.

\subsubsection{Pseudo-Invariant Feature Extraction}

This method employs a \gls{pif}\index{pseudo-invariant features (PIF)} pixel extraction to mask urban landscape from both the reflectance product and the Landsat 8 image for the bright pixel determination. Pseudo-invariant targets are defined as targets whose reflectivity properties do not change rapidly between different times of collection  \citep{Schott:1988}. Examples of pseudo-invariant target are urban features in the scene.  The PIF extraction isolates the pseudoinvariant features from the digital imagery. In our case, the PIF are the man-made urban features in a scene. A flowchart of the process is shown in \autoref{fig:PIFflowchart}. 

\begin{figure}[htb]
	\centering
  \begin{tikzpicture}[node distance=0.75cm, auto]
          \tikzset{
                  basenode/.style={rectangle,rounded corners,draw=black,very thick, inner sep=1em, minimum size=3em, text centered,text width=2cm},
                  productnode/.style={ellipse,rounded corners,draw=black, very thick, text centered,text width=1.5cm},
                  myarrow/.style={->,>=stealth',thick, double = black},
                  mylabel/.style={text width=7em, text centered}
          }
          % SWIR branch
          \node[basenode] (SWIR) {SWIR 2\\ Band};
          \node[basenode, below=of SWIR] (TS1) {Mask by Threshold (upward)};
          \node[align=left, right=0.0 of TS1] (C1) {Urban\\Veget.\\Water};
          \node[align=left, right=-0.15 of C1] (C2) {ON\\ON\\OFF};

          % Ratio branch
          \node[basenode, right=2.5cm of SWIR] (Ratio) {Ratio\\ NIR Band/ Red Band};
          \node[basenode, below=of Ratio] (TS2) {Mask by Threshold (downward)};
          \node[align=left, right=0.0 of TS2] (C3) {Urban\\Veget.\\Water};
          \node[align=left, right=-0.15 of C3] (C4) {ON\\OFF\\ON};

          % AND
          \path (TS1.south)--(TS2.south) node[pos=.5,below=2cm] (AND) {.AND.};


          % PIF Mask
          \node[basenode, below=of AND] (PIFMask){PIF Mask};
          \node[align=left, left=0.85 of PIFMask] (C5) {Urban\\Veget.\\Water};
          \node[align=left, right=-0.15 of C5] (C6) {ON\\OFF\\OFF};

          \node[basenode, below=of TS2,right=2.0cm of AND] (Image) {Image};
          \path (Image.south)--(PIFMask.east) node[below=of Image,right=2cm of PIFMask] (AND2) {.AND.};
          \node[basenode, right=2cm of AND2] (PIFIm){PIF Image};

          \draw[myarrow] (SWIR)--(TS1);
          \draw[myarrow] (Ratio)--(TS2);
          \draw[myarrow] (TS1)--(AND);
          \draw[myarrow] (TS2)--(AND);
          \draw[myarrow] (AND)--(PIFMask);
          \draw[myarrow] (Image)--(AND2);
          \draw[myarrow] (PIFMask)--(AND2);
          \draw[myarrow] (AND2)--(PIFIm);
  \end{tikzpicture}
\caption{Illustration of the logic used to segment PIF features in a satellite image. \label{fig:PIFflowchart}}
\end{figure}

The PIF extraction from digital imagery proceeds in the following fashion (\autoref{fig:PIFflowchart}). An infrared-to-red ratio image is very effective in the classification of water, vegetation, and urban features. The vegetation in this ratio image will tend to have a high brightness when compared to the urban features and water brightness. This infrared-to-red ratio image can be derived from the quotient of the NIR band (band 4 for Landsat-5; band 5 for Landsat 8) and the red band (band 3 for Landsat-5; band 4 for Landsat 8), as seen in \autoref{fig:PIFflowchart}. This ratio image is thresholded from the high digital count values downward to create a mask so the high brightness pixels are eliminated (vegetation pixels) from the image, that is, these pixels are set to a value of zero and the rest (water and urban pixels) to a value of one. The SWIR 2 band (band 7 in Landsat-5 and Landsat 8) is used to eliminate the water pixels from the previous mask since water has nearly zero reflectance in this spectral region. This SWIR 2 band is thresholded from the low brightness values upward. Water pixels will exhibit a low value when compare to the rest of the pixels. A mask is created by assigning a value of zero to the low brightness pixels (water pixels) and a value of one to the rest (urban features and vegetation). Finally, the two masks created are combined using a logical .AND., resulting in a mask that will have a value of one only in the urban feature pixels, i.e. the PIFs, as shown in \autoref{fig:PIFflowchart}. This mask will be named ``PIF mask'' for the rest of this document. An example of a PIF mask is illustrated in \autoref{fig:PIFmask}. A false color image of Downtown Rochester, NY is shown on the left (vegetation in red) and a RGB image of the same area with the PIF mask applied is shown on the right (urban features in bright color while masked pixels in black).

% \vspace{-.3cm}
\begin{figure}[htb]
  \begin{minipage}[c]{0.48\linewidth}
    \centering
      \includegraphics[trim=30 0 30 0,clip,height=6cm]{/Users/javier/Desktop/Javier/PHD_RIT/Latex/Proposal/Images/DTROCL8falsecolor.jpg}  
    % \vspace{1.5cm}
    \centerline{(a)}\medskip
  \end{minipage}
  \hfill
  \begin{minipage}[d]{0.48\linewidth}
    \centering
      \includegraphics[trim=30 0 30 0,clip,height=6cm]{/Users/javier/Desktop/Javier/PHD_RIT/Latex/Proposal/Images/PIFmaskApplied.jpg}
    % \vspace{1.5cm}
    \centerline{(b)}\medskip
  \end{minipage}
  \caption{PIF mask determination. (a) False color image, with vegetation in red and (b) PIF mask over downtown Rochester with PIF features in gray. \label{fig:PIFmask} } 
\end{figure}

\subsubsection{Bright Pixel Determination}
The PIF mask is used to determine the bright pixel spectra in both radiance (from the Landsat 8 image) and reflectance values (from the Landsat Surface Reflectance CDR image \citep{LandsatCDR,L8SurfProduct2015}). See \S\ref{sec:CDR} for more details about the Landsat surface reflectance products. At the moment of starting this research, the Landsat reflectance product was available for a total of 9 Landsat 5 scenes where clear sky conditions were acceptable. Reflectance products for Landsat 8 were not available yet. One PIF mask for each of these 9 Landsat reflectance product scenes was created using ENVI. In addition, one PIF mask was created from the Landsat 8 radiance image. Finally, these 10 PIF masks were combined using a logical .AND. to create a ``master'' PIF mask \index{master PIF mask} in order to only include the PIF pixels coincident in all images. This is necessary because there is a non perfect geometry registration among all images. Then, the statistics were calculated in ENVI for each scene using this master PIF mask. An example of the statistical results obtained from ENVI are shown in \autoref{fig:PIFstats}.(a) and \autoref{fig:PIFstats}.(b) for one scene of the Landsat reflectance product (in reflectance units) and for the Landsat 8 image (in radiance units), respectively. The mean value is shown in black solid line, the green solid lines are the mean plus standard deviation and the mean minus standard deviation, and the red solid lines are the maximum and minimum values for each band. The mean values for each one of the 9 scenes are shown in \autoref{fig:ZenithCorr}. 

\begin{figure}[!ht]
  \begin{minipage}[c]{0.48\linewidth}
    \centering
      \includegraphics[height=9cm]{/Users/javier/Desktop/Javier/PHD_RIT/Latex/Proposal/Images/PIFstatCDR.png}  
    % \vspace{1.5cm}
    \centerline{(a)}\medskip
  \end{minipage}
  \hfill
  \begin{minipage}[d]{0.48\linewidth}
    \centering
      \includegraphics[height=9cm]{/Users/javier/Desktop/Javier/PHD_RIT/Latex/Proposal/Images/PIFstatL8.png}
    % \vspace{1.5cm} 
    \centerline{(b)}\medskip
  \end{minipage}
  \caption{Bright pixel determination using the PIF mask in ENVI. Statistics with the PIF mask applied for (a) Landsat reflectance product (in reflectance units) and (b) statistics for Landsat 8 image (in radiances units). \label{fig:PIFstats} } 
\end{figure}

\subsubsection{Solar Zenith Correction}

As seen in \autoref{fig:ZenithCorr}, the PIF reflectance values for each scene are not the same, but a high correlation between the reflectance values and the solar zenith angle for each band was found. A linear relationship was determined for each band by applying a linear regression in MATLAB and the $R^2$ and root mean square error (RMSE) values were calculated as a way to measure this correlation. This linear relationship has the form 
\begin{equation}
	y = m*x + y_0
	\label{eq:linear}
\end{equation}
where $x$ represents the solar zenith angle and $m$ the reflectance value. \autoref{fig:Band1Corr} shows the reflectance values versus the solar zenith angle for band 1 for the 9 Landsat 5 reflectance scenes and the calculated linear relationship. The values $m$ and $y_0$ found for all the bands are shown in \autoref{tab:ZenithCorr} along with the $R^2$ and RMSE values for each band. Note that the RMSE values in the visible are small. It can been seen in \autoref{tab:ZenithCorr} that the $R^2$ values are bigger than $0.9$ for all bands, which suggests there is a high correlation between the reflectance values and the solar zenith angle. As a conclusion, these results show that the reflectance values remain constant over time and the apparent reflectance (i.e. including shadow effects) depends on the solar zenith angle of the sensor. This is an expected behavior since intuitively the zenith angle influences the length of shadows in the scene, and the amount of shadow in the scene affects the brightness of the pixels, therefore the reflectance values. This is illustrated in \autoref{fig:shadow}, where two different solar zenith angles ($\theta_1$ and $\theta_2$) are shown, with $\theta_1<\theta_2$. A smaller zenith angle ($\theta_1$ in \autoref{fig:shadow}) means that the sun is positioned almost straight overhead, therefore there are smaller shadows from buildings in the scene. On the other hand, if the sun is closer to the horizon ($\theta_2$ in \autoref{fig:shadow}), i.e. larger zenith angle, the shadows produced by buildings will be bigger. From \autoref{fig:shadow}, one can intuitively conclude at least that the length of shadow is proportional to the zenith angle $\theta$, and consequently inversely proportional to $\cos{\theta}$.

\begin{figure}[!ht]
    \centering
    \includegraphics[height=13cm]{/Users/javier/Desktop/Javier/PHD_RIT/Latex/Proposal/Images/Shadow.png}
  \caption{Shadow size as function of zenith angle. The shadow size are impacted by the zenith angle. \label{fig:shadow} } 
\end{figure}

The Landsat 8 image has associated a particular solar zenith angle. The previous linear relationships calculated will help to estimate the values for the reflectance value of the bright pixel for that particular solar zenith angle. For example, the solar zenith angle for the 09-19-13 Landsat 8 scene is equal to $45^\circ$, and therefore $x=45^\circ$ in \autoref{eq:linear}. The reflectance values for $x=45^\circ$ are shown in the last column of \autoref{tab:ZenithCorr} and plotted in \autoref{fig:ZenithCorr} as red asterisks.
%--------------------------------------
% \vspace{.5cm}
\begin{table}[htb]
\caption{ Zenith angle correction parameters. \label{tab:ZenithCorr} } 
\centering
\begin{tabular}{c|c|c|c|c|c} 
 \bfseries{Band n} & \bfseries{$m$}      & \bfseries{$y_0$}    & \bfseries{$R^2$}     & \bfseries{$RMSE$} & $y(x=45^\circ)$   \\ \hline \hline
 Band 1 & -0.000412 & 0.122631 & 0.937155 & 0.001705 &  0.1041\\
 Band 2 & -0.000634 & 0.147424 & 0.934344 & 0.002685 &  0.1189\\
 Band 3 & -0.000756 & 0.161421 & 0.976599 & 0.001869 &  0.1274\\
 Band 4 & -0.001316 & 0.220031 & 0.906946 & 0.006733 &  0.1608\\
 Band 5 & -0.001148 & 0.217231 & 0.903702 & 0.005984 &  0.1656\\
 Band 6 & -0.001159 & 0.206725 & 0.929626 & 0.005096 &  0.1546\\  
 \end{tabular}
\end{table}

\begin{figure}[htb]
  	\centering
  	\includegraphics[height=7cm]{/Users/javier/Desktop/Javier/PHD_RIT/Latex/Proposal/Images/ZenithCorrelation.eps}
  \caption{Correlation for band 1. \label{fig:Band1Corr} } 
\end{figure}

\begin{figure}[htb]
  	\centering
  	\includegraphics[height=7cm]{/Users/javier/Desktop/Javier/PHD_RIT/Latex/Proposal/Images/ZenithCorrection.eps}
  \caption{Bright pixel for 9 different scenes. \label{fig:ZenithCorr} } 
\end{figure}
Because the Landsat reflectance products was not available for Landsat 8 at the moment of starting this research, it was necessary to estimate a theoretical reflectance value for the coastal band for Landsat 5 in order to match with the Landsat 8 bands. To accomplish this, it was assumed that the coastal band would exhibit a similar trend as the blue and green bands. Hence, a straight-line that passes through the blue and green band values was used to extrapolate the value of the coastal band, as seen in \autoref{fig:Extrapol}, where the estimation of this reflectance value for the coastal band is shown at $443 [nm]$ and the straight-line is shown as a black solid line. At the moment of writing this document, the surface reflectance product was available for Landsat 8. Therefore, the Landsat 8 reflectance could be used directly, and the previous step was not necessary. It was found that this assumption was pretty close to the real values in the Landsat 8's coastal band. Finally, the reflectance spectra for the bright pixel is shown in \autoref{tab:brightref}. As was mentioned previously, the radiance spectra for the bright pixel is obtained by applying the master PIF mask to the Landsat 8 image (see \autoref{fig:PIFstats}.(b)).

\begin{table}[htb]
\caption{ Reflectance spectra for the bright pixel. \label{tab:brightref} } 
\centering
\begin{tabular}{l|c} 
 \bfseries{Band} & \bfseries{Reflectance values}\\ \hline \hline
 Band 1 (Coastal Band) &  0.0965 \\
 Band 2 (Blue Band) &  0.1039 \\
 Band 3 (Green Band) &  0.1186 \\
 Band 4 (Red Band) &  0.1270 \\
 Band 5 (NIR Band) &  0.1601 \\
 Band 6 (SWIR 1 Band) &  0.1650 \\ 
 Band 7 (SWIR 2 Band) &  0.1539 \\ 
 \end{tabular}
\end{table}

\begin{figure}[htb]
  	\centering
  	\includegraphics[height=7cm]{/Users/javier/Desktop/Javier/PHD_RIT/Latex/Proposal/Images/Extrapolation.eps}
  \caption{Extrapolation for the coastal band. \label{fig:Extrapol} } 
\end{figure}

\subsubsection{Black Pixel Determination}
\label{subsubsec:blackpixel}
The reflectance spectra for the dark pixel is obtained from Ecolight, which is a version of Hydrolight that runs faster because it only calculates the radiance for the nadir angle and not in all directions \citep{MobleyHEtech}. This Ecolight run represents a ROI present in the Landsat 8 radiance image. IOPs and concentrations measurements taken in the field from the same ROI are input to Ecolight. The Case 2 model in Ecolight is used to generate a remote sensing reflectance ($R_{rs}$). This model is a generic four-component (pure water, chlorophyll-bearing particles, CDOM, and mineral particles) IOP model \citep{MobleyHEtech}. Note that for this research the mineral particles component defined in Hydrolight are replaced by the suspended materials (SM) defined in previous sections. The terms mineral particles and \gls{sm} will be used interchangeable in this research. The difference between both terms is that suspended materials include both organic and inorganic particles, and not only minerals. The Case 2 model in Ecolight requires specification of the IOPs of each component one at a time. This includes concentration, absorption and scattering coefficient spectra and phase function for each component. It is worthy to mention that in this research, the concentrations and absorption coefficients for each CPA are measured, scattering coefficients were available from previous work, but phase function for each CPA was not measured, or available. Therefore, the phase function information needs to be determined (page \pageref{pag:phasefn}). The IOPs for each CPA provided to Ecolight that are used to generate the reflectance spectra for the dark pixel are explained below.

\autoref{tab:ONTNSconc} shows the constituent concentrations for two different water samples from the data collection on September, 19th, 2013. These concentrations were obtained from lab measurement at RIT (see Appendix~\ref{ch:labmea} for details about lab measurements). These water samples were collected from the nearshore of Lake Ontario (labeled as ONTNS) and from the southern part of Long Pond (labeled as LONGS), and they represent two scenarios with totally different characteristics. The water sample ONTNS was used to generate the reflectance spectra for the dark pixel in Ecolight. The concentrations for each component were set to be constant with depth with the values shown in \autoref{tab:ONTNSconc}. 
\vspace{.5cm}
\begin{table}[!ht]
\caption{ Water samples concentration for the September, 19th, 2013 collections. \label{tab:ONTNSconc} } 
\centering
\begin{tabular}{c|c|c|c} 
 \bfseries{Sample} & \bfseries{$X_{Chl}$} & \bfseries{$a(\lambda_0=440)$}& \bfseries{$X_{SM}$}\\
 & $[\mu g/L]$ & $[1/m]$ & $[mg/L]$ \\ \hline \hline
ONTNS & 0.48 & 0.1151 & 1.6\\ 
LONGS & 112.76 & 1.1953 & 46.0\\ 
 \end{tabular}
\end{table}

The absorption properties for the component chlorophyll were input by user-supplied data files containing mass-specific absorption coefficients as a function of wavelength. These mass-specific absorption spectra are shown in \autoref{fig:CHLabast}.(a). These data were obtained from lab measurements of absorption coefficient spectra for the water sample ONTNS with a spectrophotometric method (see Appendix~\ref{ch:labmea} for method details). The spectrophotometric method yields absorption coefficients, which are converted to mass-specific absorption coefficients by dividing the absorption coefficient spectra by the concentration. The chlorophyll concentration was determined in the lab by a spectrophotometric method as well. For the chlorophyll scattering properties, the same mass-specific scattering coefficient data used in \citet{Raqueno:2000} and \citet{Raqueno:2003} were utilized. These data are shown in \autoref{fig:CHLabast}.(b). A Fournier-Forand (FF) phase function with backscatter fraction 0.010 was selected as the phase function for the chlorophyll. The details about the selection of this phase function will be described below (page \pageref{pag:phasefn}).
\begin{figure}[htb]
  \begin{minipage}[c]{0.48\linewidth}
    \centering
  	\includegraphics[width=7cm]{/Users/javier/Desktop/Javier/PHD_RIT/Latex/Proposal/Images/CHLaastJavier.eps}
    \centerline{(a)}\medskip
  \end{minipage}
  \hfill
  \begin{minipage}[c]{0.48\linewidth}
    \centering
  	\includegraphics[width=7cm]{/Users/javier/Desktop/Javier/PHD_RIT/Latex/Proposal/Images/CHLbastRolo.eps}
    \centerline{(b)}\medskip
  \end{minipage}
  \caption{Chlorophyll IOPs. Chlorophyll mass-specific (a) absorption and (b) scattering spectra used for the determination of the reflectance spectra of the dark pixel in Hydrolight. \label{fig:CHLabast} }   
  % \caption{Chlorophyll mass-specific scattering spectra used for the determination of the reflectance spectra of the dark pixel in Hydrolight. \label{fig:CHLabast}} 
\end{figure}

For the CDOM component, the absorption specification was the following. First, the absorption coefficients were determined by spectrophotometric measurements in the lab for the ONTNS water sample (see Appendix~\ref{ch:labmea} for details about lab measurements). These data are shown in \autoref{fig:CDOM_IOP}.(a). Then, the data were normalized by the absorption value $a(\lambda_0)=0.1151[1/m]$ at the reference wavelength $\lambda_0=440nm$, so that $a^*(\lambda_0)=1$. These normalized data are shown in \autoref{fig:CDOM_IOP}.(b) as purple dots. An exponential curve with the following equation
\begin{equation}
	\label{eq:CDOMabs}
	a^*(\lambda)=a^*(\lambda_0)\exp{\left[-\gamma(\lambda-\lambda_0)\right]}
\end{equation}
was fitted to the normalized data. It was determined that the decay constant $\gamma=0.0126$. The fitted curve is illustrated in \autoref{fig:CDOM_IOP}.(b) in solid line. The parameters of this fitted curve are input in Ecolight to specify the CDOM specific absorption $a^*$, and $a(\lambda_0)=0.1151[1/m]$ to specify the dependence of the CDOM absorption at a reference wavelength.

\begin{figure}[!ht]
  \begin{minipage}[c]{0.48\linewidth}
  	\centering
  	\includegraphics[width=7cm]{/Users/javier/Desktop/Javier/PHD_RIT/Latex/Proposal/Images/ONTNS_CDOMabs.eps}
    \centerline{(a)}\medskip
  \end{minipage} 
  \hfill 
  \begin{minipage}[c]{0.48\linewidth}
  	\centering
  	\includegraphics[width=7cm]{/Users/javier/Desktop/Javier/PHD_RIT/Latex/Proposal/Images/ONTNS_CDOMfitting.eps}
    \centerline{(b)}\medskip
  \end{minipage}    
  \caption{CDOM IOPs. CDOM (a) absorption coefficient spectra and (b) mass-specific absorption spectra used for the determination of the reflectance spectra of the dark pixel in Hydrolight. \label{fig:CDOM_IOP} } 
  % \caption{CDOM mass-specific absorption spectra used for the determination of the reflectance spectra of the dark pixel in Hydrolight. \label{fig:CDOMaast} }
\end{figure}

The SM mass-specific absorption coefficient was determined by dividing the absorption coefficients measured in the lab from the water ONTNS water sample by the \gls{tss}, also measured in the lab (see Appendix~\ref{ch:labmea} for details about lab measurements). The SM mass-specific absorption coefficients are shown in \autoref{fig:SMabast}.(a). The SM mass-specific scattering coefficients were the same used by \citet{Raqueno:2000} and \citet{Raqueno:2003} and are shown in \autoref{fig:SMabast}.(b).

\begin{figure}[!ht]
  \begin{minipage}[c]{0.48\linewidth}
  	\centering
  	\includegraphics[width=7cm]{/Users/javier/Desktop/Javier/PHD_RIT/Latex/Proposal/Images/SMaastJavier.eps}
    \centerline{(a)}\medskip
  \end{minipage}     
  \hfill
  \begin{minipage}[c]{0.48\linewidth}
  	\centering
  	\includegraphics[width=7cm]{/Users/javier/Desktop/Javier/PHD_RIT/Latex/Proposal/Images/SMbastRolo.eps}
    \centerline{(b)}\medskip
  \end{minipage} 
  \caption{Mineral mass-specific (a) absorption and (b) scattering spectra used for the determination of the reflectance spectra of the dark pixel in Hydrolight. \label{fig:SMabast} }     
  % \caption{Mineral mass-specific scattering spectra used for the determination of the reflectance spectra of the dark pixel in Hydrolight. \label{fig:SMbast} } 
\end{figure}

The following approach was used to determine phase function\label{pag:phasefn} for both the chlorophyll and mineral particle components. Ecolight was run several times with the different phase functions from the library of discretized Fournier-Forand phase functions files supplied with Ecolight 5.2 to create a LUT of reflectance spectra, but maintaining the rest of the parameters the same. These parameters corresponded to the ONTNS water sample. The different reflectance spectra generated as output were compared with the reflectance measured {\it in situ}. The best match was determining by choosing the lowest root mean squared error (RMSE) between the reflectance measured {\it in situ} and the simulated reflectance spectra.

It was determined that the best matched corresponds to the discretized Fournier-Forand phase function with a backscatter equal to $0.010$, i.e. $1\%$ of backscatter fraction (file FFbb010.dpf in Ecolight)\todo{correct alignment}. Therefore, this discretized Fournier-Forand phase function is used for both the chlorophyll and SM. \autoref{fig:BestMatchONTNS} illustrates the reflectance measured {\it in situ} (red solid line) and the best match reflectance from the LUT (blue solid line), generated with a discretized Fournier-Forand phase function of 0.010 backscatter fraction. It can be seen in \autoref{fig:BestMatchONTNS} that both spectra agree in values above $550nm$, but not below this wavelength. This suggests that there is still room for improvement in the determination of the phase function and rest of the parameters in the Ecolight model.

The following parameters were input to Ecolight in order to simulate the Landsat acquisition conditions. The illumination conditions were input to Ecolight by specifying the solar zenith angle and day of the year that matched the Landsat 8 image. Internal sources and inelastic scatter were not included in the simulations. The wavelength range was $[400nm,1000nm]$\todo{correct alignment!}, with a $1nm$ step. Default values for the air-water surface conditions were used, with a windspeed equal to $5m/s$, a real index of refraction equal to $n=1.34$, and the semi-empirical sky model (based on RADTRAN-X). Recall that Hydrolight uses a Cox-Munk air-water surface model that parameterizes gravity and capillary waves via the wind speed \citep{Cox1954_mea_sea,Cox1954_stats_sea_surf}. Finally, the bottom boundary condition used was ``the water column is infinitely deep.''

The best matched spectra in the LUT is used as the reflectance spectra of the dark pixel in the MoB-ELM method. This reflectance is further spectrally sampled to the Landsat 8 response using the OLI's sensor response obtained from ENVI. 

\begin{figure}[htb]
  	\centering
  	\includegraphics[height=7cm]{/Users/javier/Desktop/Javier/PHD_RIT/Latex/Proposal/Images/RefWithFFbbONTNS.eps}
  \caption{Reflectance for ONTNS sample and best matching from Hydrolight. \label{fig:BestMatchONTNS} } 
\end{figure}

The radiance spectra for the dark pixel is obtained from the corresponding ROI in the water present in the Landsat 8 image, from where the {\it in situ} IOP data used in Hydrolight were taken. Statistics are computed in this dark region, and the mean values in each band are used as the radiance spectrum for the dark pixel. An example of a ROI over water is shown in \autoref{fig:ENVIROI_darkpx}, along with its statistics calculated in ENVI.

As an example, \autoref{fig:ELMpxsENVI} shows the different spectra used to perform the MoB-ELM, where four different spectra can be seen: one reflectance and one radiance spectra for the bright pixel (obtained using the PIF extraction over the Landsat reflectance product and Landsat 8 image, respectively), one reflectance spectra for the dark pixel (obtained from Ecolight/Hydrolight), and one radiance spectra for bright pixel (obtained from the statistics of a ROI over water in the Landsat 8 image). These spectra are used to atmospherically correct the Landsat 8 image using the ENVI Classic software \citep{ENVIUserGuide}. This is performed using the ``Empirical Line'' algorithm of the ``Calibration Utilities'' in ENVI classic, where the Landsat 8 image is used as the input image, and the reflectance spectra are labeled as ``field spectra'' and the radiance spectra are labeled as ``data spectra.'' The product of this process is an image in reflectance values, which will be used to perform the retrieval of water constituents described in \S\ref{sec:retrieval} below. Note that the Landsat 8 image used was not glint corrected.

One of the great values of this approach is that it can correct for slight mismatches between the model and observations such as shown in \autoref{fig:BestMatchONTNS}. Because the dark reflectance uses a model derived spectrum, it will pull all the atmospherically corrected data up slightly at the shorter wavelengths, which will be a better match to the model based LUT that will be used to generate the CPAs.

\noindent {\bf MoB-ELM Correction Summary:}
\begin{enumerate}[itemsep=2pt,parsep=2pt]
  \item Select reflectance of the bright pixel from the Landsat 8 Surface Reflectance \gls{cdr} product by using a PIF mask or a particular target
  \item Obtain radiance of the bright pixel from corresponding target in the Landsat 8 image
  \item Select dark pixel from a Hydrolight run using {\it in situ} IOPs
  \item Obtain radiance of the dark pixel from corresponding target in the Landsat 8 image
  \item Use ENVI's Empirical Line tool with the bright and dark pixel obtained in previous steps
\end{enumerate}

\begin{figure}[htb]
    \centering
    \includegraphics[width=12cm]{/Users/javier/Desktop/Javier/PHD_RIT/Latex/ThesisPhD/Images/ENVI_ROIdarkpx}
  \caption{The radiance of the dark pixel is obtained over a ROI over water (red square) of the radiance Landsat 8 image. The mean value of this ROI is used as dark pixel.\label{fig:ENVIROI_darkpx} } 
\end{figure}
\begin{figure}[htb]
  \centering
  \includegraphics[width=14cm,clip=true]{/Users/javier/Desktop/Javier/PHD_RIT/Latex/Proposal/Images/ELMpixelsENVI.pdf}
  \caption{Bright and Dark pixels used in ENVI to apply the MoB-ELM. \label{fig:ELMpxsENVI} } 
  % \vspace{0.5cm}
\end{figure}

% ----------------------------
\subsection{SeaWiFS Algorithm for Case 1 Waters}
\label{subsec:gordon}
The following algorithm is based on the method developed by \citet{Gordon:1994}\index{atmospheric correction!Gordon and Wang method} for retrieval of water-leaving radiance and aerosol optical thickness over the oceans with \gls{seawifs}. The method developed by \citet{Gordon:1994} is still applied to the basic SeaWiFS and \gls{modis} atmospheric correction algorithms for Case 1 water \citep{IOCCG:2010}.

The first step in this algorithm is performing a Rayleigh scattering correction, which means to subtract the reflectance due to Rayleigh scatter $\rho_r$ from the total TOA reflectance $\rho_t$ in \autoref{eq:rhodeglint}. In the SeaWiFS/MODIS algorithm, the Rayleigh scatter component is computed from the Rayleigh LUTs, which were calculated using the vector radiative transfer theory \citep{Wang:1991,IOCCG:2010}. \todo{Dr. Schott: check these two sentences}For this work, one alternative could have been to calculate this Rayleigh scattering component $\rho_r$ directly from the \gls{modtran} software for the particular illumination and viewing geometry of the sun and the sensor in multiple-scatter mode but without aerosol. \gls{modtran} is a computer program designed to model atmospheric propagation of electromagnetic radiation \citep{Berk:1989fk}\index{MODTRAN} (more info: \url{http://www.modtran5.com/}). It was decided to use the \gls{seadas} tool instead, which has the atmospheric LUTs built-in. In the SeaWiFS/MODIS atmospheric correction algorithm, the whitecap reflectance $\rho_{wc}(\lambda)$ (including glint) is modeled using input of the sea surface wind speed, and the TOA sun glint component is mostly masked out \citep{IOCCG:2010}. After calculating this Rayleigh scatter and whitecap component, \autoref{eq:rhodeglint} becomes 
\begin{equation}\label{eq:rhodeRayliegh}
 \rho_c(\lambda) = \rho_t(\lambda)-\rho_r(\lambda)-T_v(\lambda)\rho_{wc}(\lambda) = \rho_a(\lambda)+\rho_{ra}(\lambda)+T_v(\lambda)\rho_{w}(\lambda)
\end{equation}
where $\rho_c(\lambda)$ is the Rayleigh-corrected reflectance. 

If we define the total \gls{rhoam}\index{reflectance!multiple-scattering aerosol, $\rho_{am}$} as
\begin{equation}\label{eq:rhoam1}
  \rho_{am}(\lambda) = \rho_a(\lambda)+\rho_{ra}(\lambda)
\end{equation}
then \autoref{eq:rhodeRayliegh} becomes
\begin{equation}\label{eq:rhoam}
 \rho_c(\lambda) = \rho_{am}(\lambda) + T_v(\lambda)\rho_{w}(\lambda)
\end{equation}

The following approach is taken in the SeaWiFS/MODIS atmospheric correction algorithm to retrieve $\rho_w$. For Case 1 water (e.g. open ocean), the contribution of the water $\rho_{w}$ to the total reflectance $\rho_t$ in the NIR is negligible. This fact is used to estimate $\rho_{am}$ in the NIR bands, and then these results are extrapolated to the visible bands using aerosol modeling (\autoref{fig:epsilonvslambda}). Finally, if $T_v$ is estimated, the $\rho_w$ in the visible bands can be computed.

There are two different approaches to determine the $\rho_{am}(\lambda)$ term at this point. The first one is to use a single-scattering approximation for $\rho_{am}(\lambda)$. The second one is to determine the multiple-scattering term $\rho_{am}(\lambda)$ based on the single scattering approximation, assuming that there exist a linear relationship between them. Both approaches are described below.

% -------------------
\subsubsection{Single Scattering approach to determine \texorpdfstring{$\rho_{am}(\lambda)$}{aerosol contribution}}
\label{subsubsec:singlescat}
The aerosol is highly variable, and unlike the Rayleigh scattering component $\rho_r$, its effect in the total signal cannot be known {\it a priori} \citep{Gordon:1994}. One of the first efforts to overcome this problem was developed for the \gls{czcs} atmospheric correction algorithm using a single-scattering approximation for calculating the aerosol effect in the total signal. Its logic is as follow. If the optical thickness of the atmospheric is considered $<<1$, then the term $\rho_a$ can be replaced by is single-scattering value $\rho_{as}$ (the $\rho_{ra}$ term is ignored since it is a term related to multiple scattering). Using the \gls{rhoas}\index{reflectance!single-scattering aerosol, $\rho_{as}$} in \autoref{eq:rhoam1}, then
\begin{equation}\label{eq:singleapprox}
  \rho_{am}(\lambda) = \rho_a(\lambda)+\cancel{\rho_{ra}(\lambda)} \approx \rho_{as}(\lambda)
\end{equation}
where
\begin{equation}\label{eq:rhoas}
  \begin{gathered}
    \rho_{as}(\lambda) = \frac{\omega_a(\lambda)\tau_a(\lambda)p_a(\theta,\theta_0,\lambda)}{4\cos(\theta)\cos(\theta_0)},\\  
    p_a(\theta,\theta_0,\lambda) = P_a(\theta_{-},\lambda) + [r(\theta)+r(\theta_0)]P_a(\theta_{+},\lambda),\\
    \cos(\theta_{\pm}) = \pm \cos(\theta_0)\cos(\theta)-\sin(\theta_0)\sin(\theta)\cos(\phi-\phi_0)
  \end{gathered}
\end{equation}
and $r(\lambda)$ is the Fresnel reflectance of the interface for an incident angle $\theta$, $\tau_a(\lambda)$ is the aerosol optical thickness, $\omega_a(\lambda)$ is the aerosol single-scattering albedo, $P_a(\alpha,\lambda)$ is the aerosol scattering phase function for a scattering angle $\alpha$, $\theta_0$ and $\phi_0$ are the zenith and azimuth angles from the target to the sun, respectively, and $\theta_0$ and $\phi_0$ are the zenith and azimuth angles from the target to the sensor, respectively. 


For Case 1 waters, the term $\rho_{w}$ is assumed to be zero for NIR bands. If we take the SeaWiFS case for band 7 ($\lambda_7=765nm$) and band 8 ($\lambda_8=865nm$), from \autoref{eq:rhoam}
\begin{equation}\label{eq:seawifsam7}
    \rho_{am}^{(7)} = \rho_{c}^{(7)} = \rho_{as}^{(7)},
\end{equation}
\begin{equation}\label{eq:seawifsam8}
    \rho_{am}^{(8)} = \rho_{c}^{(8)} = \rho_{as}^{(8)},
\end{equation}
where $\rho_{x}^{(i)}$ denotes a reflectance at band $i$ with wavelength $\lambda_i$. \autoref{eq:seawifsam7} and \autoref{eq:seawifsam8} mean that the aerosol reflectance term for band 7 and band 8 in SeaWiFS is equal to only the Rayleigh-corrected reflectance, which is known from the image. Now, we need to find a way to use this result to calculate the atmospheric reflectance for the rest of the bands.

\citet{Gordon:1994} define the atmospheric correction parameter named \gls{sse}\index{single scattering epsilon (SSE), $\varepsilon$} $\varepsilon(\lambda_s,\lambda_l)$ \citep{IOCCG:2010} as
\begin{equation}\label{eq:espilon}
  \varepsilon(\lambda_s,\lambda_l) \equiv \frac{\rho_{as}(\lambda_s)}{\rho_{as}(\lambda_l)} = \\
  \frac{\omega_a(\lambda_s)\tau_a(\lambda_s)p_a(\theta_v,\phi_v;\theta_0,\phi_0;\lambda_s)}{\omega_a(\lambda_l)\tau_a(\lambda_l)p_a(\theta_v,\phi_v;\theta_0,\phi_0;\lambda_l)}
\end{equation}
where the indexes ``$s$'' and ``$l$'' stand for short and long wavelength, associated with the NIR bands, i.e. $\lambda_s=756nm$ and $\lambda_l=865nm$ for SeaWiFS. If the value of $\varepsilon(\lambda_i,\lambda_l)$ for the \gls{vis} band at $\lambda_i$ can be computed from $\varepsilon(\lambda_s,\lambda_l)$, then $\rho_{as}(\lambda_i)$ can be determined as
\begin{equation}\label{eq:rholambda_i}
  \rho_{as}(\lambda_i) = \varepsilon(\lambda_i,\lambda_l)\rho_{as}(\lambda_l),
\end{equation}

The next step is find a way to relate $\varepsilon(\lambda_i,\lambda_l)$ from $\varepsilon(\lambda_s,\lambda_l)$. \citet{Gordon:1994} tried to find a relationship by computing  $\varepsilon(\lambda_i,\lambda_l)$ for several aerosol models that were developed by \citet{Shettle:1979} for the LOWTRAN-6 model. \autoref{fig:epsilonvslambda} shows sample results for $\varepsilon(\lambda_i,\lambda_l)$ for SeaWiFS for these different aerosol models, where $\lambda_l=865nm$. It can be seen that over the range $412-865nm$, $\varepsilon(\lambda_i,\lambda_l)$ can be described as
\begin{equation}\label{eq:epsilonexp}
  \varepsilon(\lambda_i,\lambda_l) = \frac{\rho_{as}(\lambda_i)}{\rho_{as}(\lambda_l)} \approx exp[c(\lambda_i-\lambda_l)]
\end{equation}
where $c$ is a constant that depends on the viewing geometry and the aerosol model. 

\begin{figure}[htb]
  \centering
  \includegraphics[width=11cm,clip=true]{/Users/javier/Desktop/Javier/PHD_RIT/Latex/Proposal/Images/epsilonvslambdaGordon.png}
  \caption{$\varepsilon(\lambda,\lambda_l)$ values in natural logaritmic scale for different aerosol models and relative humidity (Note: image taken from \citet{Gordon:1997}). \label{fig:epsilonvslambda} } 
  % \vspace{0.5cm}
\end{figure}

The constant $c$ can be calculated using the known value $\varepsilon(\lambda_s,\lambda_l)$ for the NIR bands using \autoref{eq:seawifsam7} and \autoref{eq:seawifsam8}. If $c$ is known then \autoref{eq:rholambda_i} becomes
\begin{equation}\label{eq:rholambda_ifinal}
  \rho_{as}(\lambda_i) = exp[c(\lambda_i-\lambda_l)]\rho_{as}(\lambda_l),
\end{equation}

Once the aerosol single-scattering contribution is determined for the visible bands, $\rho_w$ can be calculated using \autoref{eq:rhoam}, but assuming $\rho_{am}(\lambda)\approx\rho_{as}(\lambda)$. 

The single-approximation is no longer an adequate approximation in cases where the aerosols are not at least moderately absorbing (i.e. strong continental influence) or $\tau_a(\lambda)$ is sufficiently large. Therefore, a full multiple-scattering approach is needed for a more general application.



%where $\rho_{as}^{(i)}$ for band $i$ is the single-scattering aerosol reflectance and $I=1..N$ denominates a set of $N$ simulated aerosol models obtained from, for example, an atmospheric radiative transfer model such as MODTRAN. This set of simulated aerosol models could be MODTRAN runs for different combination of particle size distribution (e.g. maritime, tropospheric, coastal, or urban), relative humidity, visibility, water vapor, etc., to simulate different atmosphere conditions that can occur in the scene.
% -------------------
\subsubsection{Multiple Scattering approach to determine \texorpdfstring{$\rho_{am}(\lambda)$}{aerosol contribution}}
As mentioned  previously, we need to calculate the total multiple-scattering aerosol reflectance $\rho_{am}$ in order to obtain the desired water-leaving reflectance $\rho_w$. The multiple-scattering depends significantly on the aerosol model \citep{Gordon:1997}. In the single-scattering approach previously described, the multiple-scattering was ignored and specific aerosol properties were not needed for the atmospheric correction. On the other hand, if we want to include the multiple-scattering effects in the atmospheric correction algorithm in order to obtain more accurate water reflectance $\rho_w$, it is necessary to utilize specific aerosol models. As a way to use the same reasoning used in the single-scattering algorithm, we can define multiple-scattering term as
\begin{equation}\label{eq:multscat}
  \rho_a(\lambda) + \rho_{ra}(\lambda) = K[\lambda,\rho_{as}(\lambda)]\rho_{as}(\lambda)
\end{equation}
where $K$ represents the relationship between the multiple-scattering and single-scattering. \citet{Wang:1991} has shown that a monotonic near-linear relation exists between $\rho_a(\lambda)+\rho_{ra}(\lambda)$ for the multiple-scattering model and $\rho_{as}(\lambda)$ for the single-scattering model. \citet{Gordon:1994} provided a LUT for $K[\lambda,\rho_{as}(\lambda)]$ by solving the \gls{rte} of a set of $N$ candidate aerosol models. These aerosol models were from, or derived from, the work of \citet{Shettle:1979}. The aerosol models are Oceanic, Maritime, Coastal and Tropospheric for different values of \gls{rh}. The LUT was created for different sensor-sun geometry, single-scattering albedo and \AA ngstr\"{o}m exponent. The RTE is solved by using vector radiative transfer \citep{IOCCG:2010}. For the SeaWiFS/MODIS algorithm, the $\rho_a(\lambda)+\rho_{ra}$ value for a given geometry is fit to a fourth order polynomial in the single-scattering aerosol reflectance $\rho_{as}(\lambda)$, i.e.,
\begin{equation}\label{eq:polynomial}
  \rho_a+\rho_{ra} = a\rho_{as}+b\rho_{as}^2+c\rho_{as}^3+d\rho_{as}^4
\end{equation}
where $a$, $b$, $c$ and $d$ are the constants contained in the LUT for a large number of viewing-sun geometries and for values of $\tau_a(\lambda)$ up to $0.8$.

Similar to the single-scattering approach, the assumption of negligible $\rho_w$ in the NIR allows us to determine the quantities $\rho_a(\lambda_s)+\rho_{ra}(\lambda_s)$ and $\rho_a(\lambda_l)+\rho_{ra}(\lambda_l)$ in the NIR. Once these quantities are determined from the sensor-measured values, the $\rho_{as}(\lambda)$ values for the NIR bands are estimated from these quantities using the LUT. Furthermore, because $\rho_{as}(\lambda)$ depends on aerosol phase function, single-scattering albedo, and optical thickness, this value can be computed for each aerosol model. Then, the \gls{sse}  values (described in \S\ref{subsubsec:singlescat}) for the NIR bands (i.e. $\varepsilon(\lambda_s,\lambda_l)$) can be calculated from the single-scattering reflectance values for each aerosol model and the measured values using \autoref{eq:espilon}. 

In order to determine multiple-scattering values in the VIS, the most appropriate aerosol models are selected by comparing the SSE computed from the sensor-measured values with the ones computed from each aerosol model. This is accomplished in the following fashion. After deriving $\varepsilon(\lambda_s,\lambda_l)$, the next step is to estimate $\varepsilon(\lambda_i,\lambda_l)$. $\varepsilon(\lambda_s,\lambda_l)$ falls between those for two of the $N$ aerosol models. Therefore, $\varepsilon(\lambda_i,\lambda_l)$ is assumed to fall between the same two aerosol models proportionately in the same manner as $\varepsilon(\lambda_s,\lambda_l)$. Then, \autoref{eq:rholambda_i} is used to estimate single-scattering value in the rest of the bands from $\rho_{as}(\lambda_l)$. 

Finally, the LUT is used to transform single-scattering to multiple-scattering values to obtain $\rho_{am}(\lambda)$ and using \autoref{eq:rhoam} to obtain $\rho_w(\lambda)$. 

A summary of the method developed by \citet{Gordon:1994} is described below.

\noindent {\bf SeaWiFS/MODIS Atmospheric Correction Summary:}
\begin{enumerate}[itemsep=2pt,parsep=2pt]
  \item Enter the atmospheric correction routine with Rayleigh-corrected reflectances $\rho_c(\lambda_s)$ and $\rho_c(\lambda_l)$.
  \item Assuming water-leaving reflectances for NIR bands are equal to zero, set multiple-scattering aerosol reflectances $\rho_{am}(\lambda_s)$ and $\rho_{am}(\lambda_l)$ equal to Rayleigh-corrected reflectances.
  \item Using the aerosol LUTs, calculate the corresponding single-scattering aerosol value at the two NIR bands, i.e. $\rho_{as}(\lambda_s)$ and $\rho_{as}(\lambda_l)$. Then, calculate the corresponding SSE.
  \item For each $N$ aerosol, compute the single-scattering value using \autoref{eq:rhoas}. Calculate corresponding SSE.
  \item Select the best two aerosol models by comparing the retrieved SSE with the theoretical SSE and determine the interpolation ratio between them.
  \item For the optimal aerosol model use the tabulated $\varepsilon(\lambda)$ in the VIS and $\varepsilon(\lambda_l)$ to obtain $\rho_{as}(\lambda)$ and then $\rho_{am}(\lambda)$ in the VIS using \autoref{eq:rholambda_i} and the LUT.
  \item Remove $\rho_{am}(\lambda)$ in the VIS from $\rho_c(\lambda)$ and divide by the corresponding atmospheric transmittance that corresponds to the best aerosol model to return $\rho_w(\lambda)$ in the VIS using \autoref{eq:rhoam}.
\end{enumerate}

% ------------------------------
\subsection{SeaWiFS Algorithm for Case 2 Water}
\label{subsec:ruddick}

The next atmospheric correction method is based on the method developed by \citet{Gordon:1994} for ocean color satellites (\S\ref{subsec:gordon}), and extended by \citet{Ruddick:2000bs} for use over turbid coastal and inland waters (Case 2) or high productive Case 1 waters. As stated previously, the methods developed by \citet{Gordon:1994} assumes a zero water-leaving radiance for the NIR bands, which is not valid for highly turbid coastal and inland waters. Backscatter from particles in these waters could contribute to signal in the NIR bands, causing an over-estimation of the aerosol contribution and therefore an under-estimation of the water-leaving reflectances, even leading to negative values in the visible, which is not possible. 

In order to overcome this problem, \citet{Ruddick:2000bs} replaced the black water assumption with two assumptions: the assumption of spatial homogeneity of the SeaWiFS's NIR bands ratio ($765:865-nm$) for aerosol reflectance and for water-leaving reflectance. These two ratios are used as calibration parameters after inspection of the Rayleigh-corrected reflectance scatterplot. The algorithm is described in more details below.

In order to extend the standard atmospheric correction procedure to turbid water, two assumptions are used:

\begin{enumerate}[itemsep=2pt,parsep=2pt]
  \item At least over the ROI, the calibration parameter $\varepsilon_m(\lambda_s,\lambda_l)$, defined as the ratio of multiple-scattering aerosols and aerosol-Rayleigh reflectances at the NIR bands, is assumed to be spatially homogeneous, i.e.
  \begin{equation}\label{eq:rhohomo}
    \varepsilon_m(\lambda_s,\lambda_l)\equiv \frac{\rho_{am}(\lambda_s)}{\rho_{am}(\lambda_l)},
  \end{equation}
  and fixed for each image.

  \item The calibration parameter $\alpha$, defined as the ratio of water-leaving reflectances normalized by the sun-sea atmospheric transmittance at the NIR bands $T_0(\lambda)$, is assumed to be spatially homogeneous, i.e.
  \begin{equation}\label{eq:alpha}
    \alpha \equiv \frac{\rho_w(\lambda_s)/T_0(\lambda_s)}{\rho_w(\lambda_l)/T_0(\lambda_l)},
  \end{equation}
  and fixed for each image.
\end{enumerate}

By using these two assumption, the multiple-scattering aerosol and the aerosol-Rayleigh reflectance for the NIR bands can be derived as

\begin{equation}\label{eq:rhoams}
  \rho_{am}(\lambda_s) = \frac{\alpha\rho_c(\lambda_l)-\rho_c(\lambda_s)}{\alpha-\varepsilon_m(\lambda_s,\lambda_l)},
\end{equation}
and
\begin{equation}\label{eq:rhoaml}
  \rho_{am}(\lambda_l) = \varepsilon_m(\lambda_s,\lambda_l)\left[\frac{\alpha\rho_c(\lambda_l)-\rho_c(\lambda_s)}{\alpha-\varepsilon_m(\lambda_s,\lambda_l)}\right],
\end{equation}

Once these aerosol reflectances $\rho_{am}(\lambda_s)$ and $\rho_{am}(\lambda_l)$ are determined, they can be used in the standard algorithm, described in the previous section (instead of $\rho_c(\lambda_s)$ and $\rho_c(\lambda_l)$) to derive $\rho_w(\lambda)$ in the VIS.

\citet{Ruddick:2000bs} suggested a method to determine the calibration parameter $\varepsilon_m(\lambda_s,\lambda_l)$ by inspection of the scatterplot between the Rayleigh-corrected reflectance $\rho_c$ in the NIR bands, as shown in \autoref{fig:ruddickplot}. This assumption is only valid for a small scale space, where the aerosol type varies only weakly in space. As for the calibration parameter $\alpha$, \citet{Ruddick:2000bs} set it to a default value of $1.72$, which is a first-order estimate from a greatly simplified ocean color model. The derivation can be seen in \citet{Ruddick:2000bs} and will not be explained here.

The process is summarized as follows:

\noindent {\bf SeaWiFS/MODIS Atmospheric Correction for Turbid Water Summary:}(\citep{Ruddick:2000bs})
\begin{enumerate}[itemsep=2pt,parsep=2pt]
  \item Enter the atmospheric correction routine to produce a scatter plot of Rayleigh-corrected reflectances $\rho_c(\lambda_s)$ and $\rho_c(\lambda_l)$) for the ROI of study, and select the calibration parameter $\varepsilon_m(\lambda_s,\lambda_l)$. Set calibration parameter $\alpha$ equal to 1.72.

  \item Reenter the atmospheric correction routine with $\rho_c(\lambda_s)$ and $\rho_c(\lambda_l)$ and use \autoref{eq:rhoams} and \autoref{eq:rhoaml} to obtain $\rho_{am}(\lambda_s)$ and $\rho_{am}(\lambda_l)$, taking account of nonzero water-leaving reflectances.

  {\bf Note:} The following steps are the same as the standard algorithm described in \S\ref{subsec:gordon}.

  \item Using the aerosol LUTs, calculate the corresponding single-scattering aerosol value at the two NIR bands, i.e. $\rho_{as}(\lambda_s)$ and $\rho_{as}(\lambda_l)$. Then, calculate corresponding SSE.

  \item For each of the $N$ aerosol types, compute the single-scattering value using \autoref{eq:rhoas}. Calculate corresponding SSE.
  \item Select the best two aerosol models by comparing the retrieved SSE with the theoretical SSE and determine the interpolation ratio between them.
  \item For the optimal aerosol model use the tabulated $\varepsilon(\lambda)$ in the VIS and $\varepsilon(\lambda_l)$ to obtain $\rho_{as}(\lambda)$ and then $\rho_{am}(\lambda)$ in the VIS using \autoref{eq:rholambda_i} and LUT.
  \item Remove $\rho_{am}(\lambda)$ in the VIS from $\rho_c(\lambda)$ and divide by the corresponding atmospheric transmittance that corresponds to the best aerosol model to return $\rho_w(\lambda)$ in the VIS using \autoref{eq:rhoam}.
\end{enumerate}

\begin{figure}[htb]
  \centering
  \includegraphics[width=10cm]{/Users/javier/Desktop/Javier/PHD_RIT/Latex/Proposal/Images/RuddickPlot.png}
  \caption{Scatterplot of Rayleigh-corrected reflectances at 765 and 865 nm for a subregion in a SeaWiFS image taken 28 October 1997, 12:15 UTC.  (Note: image taken from \citet{Ruddick:2000bs}). \label{fig:ruddickplot} } 
\end{figure}
% The second atmospheric correction method will be an extension of a method developed for SeaWiFS over turbid coastal and inland waters \citep{Ruddick:2000bs}. This method is a modified version of the methods developed by Gordon \citet{Gordon:1997} for ocean color satellites, but when the signal leaving the water does contribute to the overall signal beyond the NIR part of the spectrum. By using longer wavelengths and restricting the input pixels to open waters, these methods can be  applied to many fresh and coastal regions. The water-leaving reflectance values obtained after atmospheric correction will be validated through comparison to water-leaving reflectance measured in situ. 

% ------------------------------
\subsection{SWIR bands Atmospheric Correction}
\label{subsec:wang}

For Case 2 or highly productive Case 1 waters, especially turbid waters, there is a significant contribution from the water-leaving radiance and hence the zero water-leaving assumption in the NIR bands is not valid anymore. This is often the case in inland and coastal waters and therefore the NIR bands cannot be used to atmospherically correct this kind of imagery. However, these kinds of waters are indeed black in the shortwave infrared (SWIR) bands ($\geq 1000nm$) due to stronger water absorption. So, we can use the same atmospheric correction procedure used for SeaWiFS and MODIS but replacing the NIR bands with the two OLI SWIR bands ($1690$ and $2200nm$) for the data processing, as suggested by \citep{Wang:2007,Wang:2005}. \citet{Wang:2007} evaluated different combinations of the MODIS' SWIR bands to atmospherically correct a specific study image and compared the results with the traditional algorithm (NIR bands). Actually, the latest MODIS atmospheric algorithm uses a NIR-SWIR combined atmospheric correction approach that uses the NIR bands for non-highly productive Case 1 water and the SWIR bands for Case 2 or highly productive Case 1 waters, and a turbidity index to decide when to use them \citep{Wang:2007dz}. 

% ------------------------------
\subsection{OLI Algorithm for Case 2 Waters (Blue Band)}
\label{subsec:blueband}
\index{atmospheric correction!Gerace's blue band}
\citet{GeraceThesis} proposed the use of a combination of spectral matching and band ratio techniques (method developed by \citet{Gordon:1994}) to atmospherically correct OLI data over Case 2 waters. A requirement to use this technique for atmospherically correcting OLI data is $\varepsilon^{(1,6)}\cong constant$, or in other words the ratio of reflectance from OLI's band 1 (coastal band) and band 6 (SWIR 1 band) should be approximately constant for all water pixels in the region of interest. Due to variability in band 1, the previous requirement is not always true. However, this variability can be accounted using spectral matching. This method needs to be applied with caution as it will not work in highly turbid water due to high variability in band 1. Therefore, an {\it a priori} analysis of the band histogram is needed, and only the bands whose histogram has little spread and resemble a normal distribution should be used \citep{GeraceThesis}.

This method first uses forwarding modeling to create a three dimensional (3-D) LUT in Hydrolight. Then, it adds atmospheric visibility as a fourth dimension (4-D). The dimensions of the LUT are concentration of chlorophyll-{\it a}, suspended particles and CDOM. The range of the LUT is made up of spectral water-leaving reflectances. Then, this 3-D reflectance LUT is propagated to the TOA using MODTRAN for a range of visibilities. Therefore, the 4-D LUT will be made up of the independent variables chlorophyll-{\it a}, suspended particles, CDOM and visibility, while the range will be made up of spectral sensor-reaching radiances. A diagram showing the concept of this 4-D LUT appears in \autoref{fig:4DLUT}. Note that this approach requires some knowledge of the aerosol type (e.g. rural, urban, maritime, etc.). Furthermore, the sensor-reaching radiances should be spectrally sampled at the OLI's sensor response and the data from the image should be corrected for the effect of glint before comparing with the 4-D LUT since it does not include this effect.

\begin{figure}[htb]
  \centering
  \includegraphics[width=14cm]{/Users/javier/Desktop/Javier/PHD_RIT/Latex/Proposal/Images/4DLUT.png}
  \caption{Four dimensional LUT. The dimension are concentrations of chlorophyll-{\it a}, suspended particles, CDOM and visibility and the range is spectral sensor-reaching radiance. Source: \citet{GeraceThesis}}
  \label{fig:4DLUT} 
\end{figure}
% \todo{change figure for a better quality}
Once the 4-D LUT has been created, an iterative search of the closest match to an imaged water pixel is performed. The first step in this iteration is to obtain an initial guess of the visibility. To do so an imaged spectrum is compared to the spectra contained in the 4-D LUT. The parameters (concentration of CPAs and visibility) associated with the closest non-interpolated spectrum in the 4-D LUT in a RMS sense are associated with the imaged spectrum. Then, these associated parameters are fixed and the observed $\varepsilon^{(1,6)}$ is compared with the 4-D LUT $\varepsilon^{(1,6)}$ values for different visibilities but the same concentration, and the visibility of the closest match is selected as the initial visibility estimate.

After estimating the initial visibility, an optimization routine can be used to estimate the four parameters from the 4-D LUT by interpolation. The final results of this process are interpolated CPA's concentration and visibility associated with the imaged pixel.

\citet{GeraceThesis} suggested using an average of all visibility solutions obtained previously as a fixed visibility and repeating the estimation described above with this fixed visibility. This is because the atmosphere should not have a high variability in the scene and therefore be approximately constant over the study region.

A summary of this process is described below.

\noindent{\bf Summary:}
Enter the algorithm with glint corrected data:\\
{\bf Step 1:} Using the three dimensions of chlorophyll-{\it a}, particles and CDOM concentrations, create a water-leaving reflectance 3-D LUT using Hydrolight.\\
\noindent{\bf Step 2:} Propagate the 4-D LUT to the TOA using MODTRAN to develop a 4-D LUT of sensor-reaching radiances. Use a best-estimate aerosol type and a range of visibility.\\
\noindent{\bf Step 3:} Search best match of imaged water pixel in the 4-D LUT.\\
\noindent{\bf (a)} Obtain initial guess of the visibility by using spectral matching and epsilon ratios.\\
\noindent{\bf (b)} Search for best match in the 4-D LUT using initial guess of the visibility.\\
\noindent{\bf Step 4:} Obtain average visibility from all the water pixels and repeat search for best match.\\

% ------------------------------
\subsection{OLI Algorithm for Case 2 Waters (Band Ratios)}
\label{subsec:bandratios}
\index{atmospheric correction!Gerace's band ratios}
Another algorithm suggested by \citet{GeraceThesis} uses the concept adopted by \citet{Ruddick:2000bs}, but incorporating OLI's NIR (band 5) and SWIR 1 (band 6) bands instead of the two NIR bands used for SeaWiFS. This algorithm utilizes the concept of band ratios (using the epsilon ratio values) in its implementation for calculating the reflectance due to aerosol in the scene. Recall from \citet{Ruddick:2000bs} that we would like $\varepsilon^{(i,j)}$ to be constant over the region of interest, for some band $i$ and band $j$. 

For this algorithm specifically, the requirement should be $\varepsilon^{(5,6)}\cong constant$. If this is the case for any two bands in the scene, then a band ratio technique (\S\ref{subsec:gordon}, \S\ref{subsec:ruddick} and \S\ref{subsec:wang}) can be used to solve for $\rho_w$ in \autoref{eq:rhoam}. However, the requirement of $\varepsilon^{(5,6)}\cong constant$ could not be true for turbid waters because there is some signal coming from the water in the NIR, not only from the atmosphere, i.e. $\rho_w^{(5)}\neq0$. This fact produces variability in the water reflectance in band 5. Therefore, simply calculating $\varepsilon^{(5,6)}$ will result in a misrepresentation of the atmosphere and the water signal. A solution to this problem is to select the signal in band 5 (NIR band; $862nm$) from a region of dark waters, and make the black pixel assumption in band 6. This will allow one to calculate $\varepsilon^{(5,6)}$ over dark water and therefore determine its atmosphere signal, i.e. the \citet{Ruddick:2000bs} approach adapted to the SWIR region. Then, the whole water scene is assumed to have the chosen atmosphere, the image is atmospherically corrected for that atmosphere. The details of this algorithm are similar to the ones described in \citet{Ruddick:2000bs} (see \S\ref{subsec:ruddick}), and summarized below.

\noindent{\bf Summary:}
Enter the algorithm with glint corrected data:\\
{\bf Step 1:} Create a LUT of aerosol reflectances for different atmospheric models using MODTRAN and Hydrolight to determine $\rho_w$ for a dark water pixel. Then, calculate $\varepsilon^{(5,6)}$ for each atmospheric model in the LUT. \\
\noindent{\bf Step 2:} Calculate an averaged $\varepsilon^{(5,6)}$ from a region of interest over dark water by averaging the reflectance at the TOA ($\rho$) values for that region.\\ 
\noindent{\bf Step 3:} Use the averaged $\varepsilon^{(5,6)}$ to find the closest two matches for the modeled atmosphere in the LUT from Step 1 and calculate the interpolation ratio between these two matched and the averaged $\varepsilon^{(5,6)}$. \\
\noindent{\bf Step 4:} Extrapolate the determined model to all wavelengths using interpolation ratio. \\
\noindent{\bf Step 5:} Globally correct the scene for the atmospheric effect. \\

% ------------------------------
\subsection{Concluding Remarks}
The developed \gls{mobelm} atmospheric correction described in \S\ref{subsec:mobelm} is used to test the methodology. The atmospheric correction algorithms based on \citet{Gordon:1994} are used to compare with the developed algorithm, but they are applied directly from the \gls{seadas} and \gls{acolite} tools. Finally, the algorithm proposed by \citet{GeraceThesis} (\S\ref{subsec:blueband} and \S\ref{subsec:bandratios}) are not used. They were described here for reference.
% @@@@@@@@@@@@@@@@@@@@@@@@@@@@@@@@@@@@@@@@@@@@@@@@@@@@@@@@@@@@@@@@
\section{In-Water Constituent Retrieval Process}
\label{sec:retrieval}
The retrieval algorithm will be based on previous work done by \citet{Raqueno:2000}, \citet{Gerace:2013} and \citet{Concha2013IGARSS}. The water-leaving reflectance product (or \acrshort{rrs} product) obtained after atmospheric correction from the previous stage is used as input to the retrieval algorithm (\autoref{fig:detailflowchart}). Each water pixel in the \acrfull{rhow} product has an unknown concentration. A spectral matching technique is applied to predict this concentration by comparing the spectral shape of each pixel with the elements in a \gls{lut}. The complete retrieval process will be explained in the following sections. 
% -------------------------------------
\subsection{LUT generation}
\label{subsec:LUTgen}
The \gls{lut} is generated in Ecolight \citep{MobleyHE} for different triplets of water constituent concentrations (CPAs). Ecolight was used in the same fashion as in the black pixel determination for the MoB-ELM algorithm (\S\ref{subsubsec:blackpixel}). As an example, \autoref{tab:LUTconc2} shows the different parameters used to create a LUT in Ecolight. Two different sets of {\it in situ} IOPs (mass-specific absorption and scattering coefficient spectra) were used, one set for modeling the open lake conditions (low concentration of CPAs) and one set for modeling the pond condition (high concentration of CPAs). The set of IOPs are specific and unique for each image. These IOPs are labeled in \autoref{tab:LUTconc2} as ``ONTNS'' for the open lake conditions and as ``LONGS'' for the pond conditions. \autoref{tab:LUTconc2} also shows the CPAs concentration used to create the LUT in Ecolight, where $C_a$ is the chlorophyll-{\it a} concentration, \acrfull{tss}, $a_{CDOM}(440)$ is the CDOM absorption coefficient at the wavelength $\lambda=440nm$. Furthermore, discretized Fournier-Forand phase functions with four different \acrfull{b_boverb} values ($0.5$, $1.0$, $1.5$ and $2.0\%$) were used to account for the backscattering variability in the scene.


\begin{table}[htb]
\caption{ Input parameters for the LUT generation in Ecolight. \label{tab:LUTconc2} } 
\centering
		\begin{tabular}{ccccc}
    \hline \hline
    \multirow{2}{*}{IOPs Input} & \bfseries{$C_a$}  	& \bfseries{$TSS$}  & \bfseries{$a_{CDOM}(440)$} & \bfseries{$b_b/b$}\\
		           & $[mg~m^{-3}]$   		& $[g~m^{-3}]$      & 	$[1/m]$                  &	$[\%]$	         \\ \hline \hline
\multirow{7}{*}{ONTNS} & 0.1   & 1.0  &  0.11 &  0.5 \\
                       & 0.5   & 2.0  &  0.15 &  1.0 \\
                       & 1.0   & 5.0  &  0.21 &  1.4 \\
                       & 3.0   & 10.0 &  0.6  &  2.0 \\
                       & 10.0  & --   &  --   &  --  \\
                       & 20.0  & --   &  --   &  --  \\
                       & 40.0  & --   &  --   &  --  \\ \hline

\multirow{5}{*}{LONGS} & 60.0  & 25.0 & 1.0   &  0.5 \\  
                       & 90.0  & 45.0 & 1.2   &  1.0 \\  
                       & 110.0 & 50.0 & --    &  1.4 \\  
                       & 135.0 & --   & --    &  2.0 \\  
                       & 150.0 & --   & --    &  --  \\  \hline \hline    
	 	\end{tabular}
	\end{table}


After obtaining the LUT from Ecolight, the spectral curves in the LUT are spectrally sampled to the OLI's spectral response. An example of a LUT created in Ecolight is shown in \autoref{fig:LUT} with 2000 spectral curves. 

\begin{figure}[htb]
    \centering
      \includegraphics[height=7cm]{/Users/javier/Desktop/Javier/PHD_RIT/ConferencesAndApplications/2014_ASPRS_SOY/Images/LUT.eps}
      \caption{LUT created in Hydrolight}
      \label{fig:LUT}
\end{figure}

% -----------------------------------------
\subsection{Retrieval}
The retrieval is divided in two stages (\autoref{fig:detailflowchart}). The first one uses the \acrfull{rmse} to fix the IOP set (ponds or lake) and the phase function. The \gls{rmse} is calculated between the water pixel and each element in the LUT, and the one with the smallest \gls{rmse} is selected as the first candidate. The IOP set and the phase function associated with this candidate is chosen as the IOP set and the phase function for the water pixel.

The second stage of the retrieval uses a non-linear optimization to estimate the CPA concentrations of the water pixel. Only the elements in the LUT with the same IOP set and phase function selected from the first stage are utilized in this stage. The water pixel is input to a non-linear optimization routine to estimate the CPA concentration through a trilinear interpolation within the LUT's elements. This method gives continuous concentration values. The \gls{rmse} and the non-linear optimization are described below.
% -  -  -  -  -  -  -  -  -  -  -  -  -  -  -  -  -  -  -  -  
\subsubsection{Root Mean Square Error}
In this work, the RMSE is defined as
\begin{equation}
  RMSE(i) = \sqrt{\frac{1}{m}\sum_1^m\left[\widetilde{R}_{rs}(i,\lambda_m)-R_{rs}(\lambda_m)\right]^2}
\end{equation}
where $\widetilde{R}_{rs}(i,\lambda_m)$ is the $i$th database spectrum from the LUT at wavelength band $m$ and $R_{rs}(\lambda_m)$ is the spectrum for a particular water pixel from the image.

% -----------------------------------------
\subsubsection{Non-linear Optimization}
The spectral matching is made by a least square error minimization algorithm using the ``\texttt{lsqnonlin}'' package of the MATLAB's Optimization Toolbox. \texttt{lsqnonlin} solves least-squares problems, including nonlinear data-fitting problems \citep{MatlabHelp}. If $f(x)$ is a user-defined vector function defined as
\begin{equation}
  f(x)=
  \left[
    \begin{array}{c}
      f_1(x) \\
      f_2(x) \\
      \vdots \\
      f_m(x) \\
    \end{array}
  \right],
\end{equation}
with $x$ a vector. \texttt{lsqnonlin} tries to minimize the function
\begin{equation}
  \underset{x}{min}\parallel f(x) \parallel^2_2=\underset{x}{min}(f_1(x)^2+f_2(x)^2+f_3(x)^2+...+f_m(x)^2)
\end{equation}
In this case, $x$ includes the three CPA concentrations and the function $f(x)$ is the difference between the water spectra $R_{rs}$ for each pixel and an estimated curve $F$ from the LUT, this is
\begin{equation}
  f = R_{rs} - F
\end{equation}
where $F$ is obtained from a trilinear interpolation based on the CPA concentrations of the LUT. The dimension $m$ is the number of bands. In other words, for each pixel in the image, \texttt{lsqnonlin} tries to find a function that minimizes the error between the measured value and an interpolated spectra from the LUT. \texttt{lsqnonlin} stops the search after reaching a certain threshold. 

The output of this process is a concentration mapping for each water constituent that spans the range of constituents levels in the scene. 


% @@@@@@@@@@@@@@@@@@@@@@@@@@@@@@@@@@@@@@@@@@@@@@@@@@@@@@@@@@@@@@@@
\section{Validation} 
\label{sec:val}

The validation of the retrieval is performed through comparison with both {\it in situ} data and standard bio-optical products. This comparison is done for the results from the atmospheric correction (\gls{rrs}) and from the retrieval process (CPA concentrations). The following section will describe the area of study, the comparison with {\it in situ} data and the comparison with standard products.
% -------------------------------------------------------------
\subsection{Area of Study}
\label{subsec:areaofstudy}
The area of study for this research is the Lake Ontario Rochester Embayment (latitude: 43°15'32.53"N and longitude: 77°36'13.10"W), which includes some nearby ponds (Long and Cranberry Ponds), the Genesee River plume, the Irondequoit Bay and the southern end of Lake Ontario, as shown in Figure~\ref{fig:areaofstudy1} and Figure~\ref{fig:areaofstudy2}. This area was selected because it exhibits a wide range of variability in concentration of water constituents, including some eutrophic water bodies (ponds) with high concentration of CPAs, and oligomesotrophic water bodies (Lake Ontario) with low concentration of CPAs, so the retrieval algorithm can be tested with different scenarios. Landsat 8 images from this area of study and corresponding water samples collected at the time of the satellite's overpass will be used to test the retrieval algorithm. So far, there are only three satisfactory images available from the 2013, 2014 and 2015 seasons. Therefore, these images will be used to test the methodology. Note that a difficult challenge of this research is to obtain images with relatively clear weather conditions (i.e. cloud free) over the area of study when {\it in situ} data are also available.
\begin{figure}[htb]
  \centering
  \includegraphics[height=9cm]{/Users/javier/Desktop/Javier/PHD_RIT/Latex/Proposal/Images/AreaOfStudy1.pdf}
  \caption{Location of the area of study. The chosen area of study is located in Rochester, NY, USA. \label{fig:areaofstudy1} } 
\end{figure}
\begin{figure}[htb]
  \centering
  \includegraphics[height=7cm]{/Users/javier/Desktop/Javier/PHD_RIT/Latex/Proposal/Images/AreaOfStudy2.pdf}
  \caption{Rochester Embayment. This area of study exhibit a variety of water types, from oligotrophic to eutrophic waters. \label{fig:areaofstudy2} } 
\end{figure}
\todo{check site labels in color printing!}

% -------------------------------------------------------------
\subsection{Comparison with {\it in situ} data}
A field collection that includes water samples and \gls{rrs} measurements was conducted at the same time as the sensor overpass. The $R_{rs}$ measurements were performed using either the spectroradiometers SVC HR-1024i  \citep{SVCHR1024i} or ASD FieldSpec 4 \citep{ASDManual2012} following the method described by \citet{Mobley:1999} for measuring \gls{rrs} from three consecutive measurements: the spectra of the downwelling irradiance $E_d$, the surface reflected sky radiance $L_s$, and the water-leaving radiance $L_w$ for each site (see Appendix~\ref{ch:fieldmea} for more details about the water samples collection and the $R_{rs}$ measurements). Then, the water samples were analyzed in the lab, following SeaWiFS protocols described in \citet{Mueller1995} for obtaining chlorophyll-{\it a} concentration ($C_a$) and total suspended solid concentration (TSS) (see Appendix~\ref{ch:labmea} for more details about the lab measurements). \autoref{tab:Sites} shows the different site names, location and the $C_a$ and $TSS$ for each site for the collection on 09-19-2013, as example. Note the difference in concentration levels between the ponds (i.e. LONGN, LONGS and CRANB) and the lake (i.e. ONTNS, ONTOS and ONTEX) samples.

% Site  & $C_a$  &    Latitude  & Longitude
% ONTNS &   ~~0.48 &  43.272159 & -77.538274  
% ONTOS &   ~~0.96 &  43.308923 & -77.540085  
% ONTEX &   ~~1.68 &  43.244892 & -77.536671  
% RVRPI &   ~~2.88 &  43.259925 & -77.601587  
% RVRPL &   ~~0.48 &  43.270990 & -77.592282  
% LONGN &   123.85 &  43.290836 & -77.690662  
% LONGS &   112.76 &  43.289182 & -77.696458  
% CRANB &   ~64.08 &  43.299938 & -77.692915  
% BRADI &   ~19.22 &  43.313675 & -77.717531  
% BRADO &   ~~1.44 &  43.325780 & -77.706432 

\begin{table}[!ht]
\caption{ Different sites for the collection on 09-19-2013. This collection included sampling in the ponds and the lake, which exhibit a wide range of concentrations. \label{tab:Sites} } 
\vspace{0.2cm}
\centering
\begin{tabular}{lccccl} 
 % \bfseries{Band n} & \bfseries{$m$}      & \bfseries{$y_0$}    & \bfseries{$R^2$}     & \bfseries{$RMSE$} & $y(x=45^\circ)$   \\ \hline \hline
 \hline
Site  &     Latitude  & Longitude  &  $C_a$      &  $TSS$   & Description \\ 
      &               &        &  $[mg/m^3]$ & $[g/m^3]$  &   \\ \hline \hline
ONTNS &     43.272159 & -77.538274 &  ~~0.48 & ~1.60      & Lake Ontario near-shore \\    
ONTOS &     43.308923 & -77.540085 &  ~~0.96 & ~1.00      & Lake Ontario off-shore  \\    
ONTEX &     43.244892 & -77.536671 &  ~~1.68 & ~0.70      & Lake Ontario extra  \\    
RVRPI &     43.259925 & -77.601587 &  ~~2.88 & ~2.10      & Genese River pier \\    
RVRPL &     43.270990 & -77.592282 &  ~~0.48 & ~1.00      & Genese River plume  \\    
LONGN &     43.290836 & -77.690662 &  123.85 & 48.00      & Long Pond north \\    
LONGS &     43.289182 & -77.696458 &  112.76 & 46.00      & Long Pond south \\    
CRANB &     43.299938 & -77.692915 &  ~64.08 & 26.70      & Cranberry Pond  \\    
BRADIN&     43.313675 & -77.717531 &  ~19.22 & 13.10      & inside Braddock bay \\    
BRADONT&  43.325780 & -77.706432 &  ~~1.44 & ~2.00        & Braddock Bay, Lake Ontario side \\  \hline
 \end{tabular}  
\end{table} 

\begin{figure}[htbp!]
  \centering
  \includegraphics[height=8.0cm]{/Users/javier/Desktop/Javier/PHD_RIT/ConferencesAndApplications/2015_SPIE_SanDiego/Images/ROI_RocEmbayment130919-eps-converted-to.pdf}
  \caption{Landsat 8 image acquired on 09-19-2015 (scene LC80160302013262LGN00) showing the study area, the Rochester Embayment. The labels indicate the sites of the field collection at the same time as the satellite overpass (\autoref{tab:Sites}).\label{fig:RrsROIs130919} } 
\end{figure}

Additionally, in order to have outputs in Hydrolight that are representative of the water bodies that are being studied, inherent optical properties (IOPs) of those specific waters have to be defined as input to the Hydrolight/Ecolight model. After collection, these water samples are analyzed in the lab to obtain IOPs for the main water constituents (see Appendix~\ref{ch:labmea} for more info about these lab measurements). Some IOPs and concentration measurements are used for the MoB-ELM algorithm and for the LUT creation described previously (\S\ref{subsec:mobelm} and \S\ref{subsec:LUTgen}, respectively). 

The measured \gls{rrs}s and CPA concentrations are used to validate the developed MoB-ELM and retrieval algorithms by comparison (\autoref{fig:detailflowchart})). The \acrfull{rmse} and \acrfull{nrmse} are utilized to quantify these comparisons.

% Furthermore, apparent optical properties (AOPs) (i.e. water-leaving reflectance) and backscattering measurements will be also collected for further comparison and to pursue closure between the Hydrolight AOPs results and in-situ AOPs measurements.

% ----------------------------------------------------
\subsection{Comparison with standard bio-optical product}
The \gls{rrs} spectra obtained from the \gls{mobelm} method are compared with the results from the \citet{Gordon:1994} approach obtained from both the ``l2gen'' tool in \gls{seadas} (\S\ref{subsec:seadasrrs}) and \gls{acolite} (\S\ref{subsec:acolite}), along with {\it in situ} data.

Additionally, this study presents a comparison of chlorophyll-{\it a} concentrations ($C_a$) retrieved from the standard bio-optical products described in \S\ref{subsec:chlempirical} (using \gls{seadas}) with the $C_a$ retrieved from the developed retrieval algorithm. These results are further compared with {\it in situ} data.

% The results from the retrieval process will be validated by comparison with the concentration of water samples taken during field campaigns in the spring and summer of 2013 and 2014\todo{2015?}. These concentrations will be obtained from lab measurements made at the Rochester Institute of Technology. %For further validation, the results will be compared with products derived from ocean color satellites such as MODIS (e.g. MODIS Chl-{\it a} product), in regions where it is possible.

% ----------------------------------------------------
\section{Concluding Remarks}
The purpose of this chapter was to introduce the methodology required to achieve the objectives defined in Chapter \ref{ch:objectives}. First, we began by explaining the different atmospheric correction techniques that will be investigated in this research along with the solar-glint removal algorithm. These atmospheric correction techniques include two different approaches. The first approach includes methods applied to ocean color satellites. The second one is the MoB-ELM method. The in-water constituent retrieval process was presented. The LUT generation and the metrics used to perform the retrieval were described. Additionally, the ground truth data collection was briefly explained. Finally, how the results will be validated was described. 

The next Chapter (\S\ref{ch:results}) will present the data and laboratory measurements available to date, along with the final results. These results include the \gls{mobelm} atmospheric correction applied to Landsat 8 imagery, and concentration of CPA maps obtained from the developed retrieval.

%% CHAPTER
\chapter{Results}
\section{Synthetic Data}
\subsection{Deep Water}

  \begin{figure}[H]
	\centering
    	\includegraphics[width=100mm]{/Users/javier/Desktop/Javier/MASTER_RIT/2011_THESIS/LUT/LUT_1/Images/LUT1_120re.eps}
 	\caption{Error  \label{fig:errorLUT1}}
  \end{figure}

  \begin{figure}[H]
	\centering
    	\includegraphics[width=100mm]{/Users/javier/Desktop/Javier/MASTER_RIT/2011_THESIS/LUT/LUT_1/Images/TS1_120re.eps}
 	\caption{Error  \label{fig:errorLUT1}}
  \end{figure}
  
    \begin{figure}[H]
	\centering
    	\includegraphics[width=100mm]{/Users/javier/Desktop/Javier/MASTER_RIT/2011_THESIS/LUT/LUT_1/Images/TS1TOA120.eps}
 	\caption{Error  \label{fig:errorLUT1}}
  \end{figure}
  
      \begin{figure}[H]
	\centering
    	\includegraphics[width=100mm]{/Users/javier/Desktop/Javier/MASTER_RIT/2011_THESIS/LUT/LUT_1/Images/TS1TOA8.eps}
 	\caption{Error  \label{fig:errorLUT1}}
  \end{figure}
  
        \begin{figure}[H]
	\centering
    	\includegraphics[width=100mm]{/Users/javier/Desktop/Javier/MASTER_RIT/2011_THESIS/LUT/LUT_1/Images/TS1TOA8NQ.eps}
 	\caption{Error  \label{fig:errorLUT1}}
  \end{figure}
  
          \begin{figure}[H]
	\centering
    	\includegraphics[width=100mm]{/Users/javier/Desktop/Javier/MASTER_RIT/2011_THESIS/LUT/LUT_1/Images/TS1ELM.eps}
 	\caption{Error  \label{fig:errorLUT1}}
  \end{figure}
  
          \begin{figure}[H]
	\centering
    	\includegraphics[width=100mm]{/Users/javier/Desktop/Javier/MASTER_RIT/2011_THESIS/LUT/LUT_1/Images/DarkBrightpxre.eps}
 	\caption{Error  \label{fig:errorLUT1}}
  \end{figure}
  
  After ELM...

  \begin{figure}[H]
	\centering
    	\includegraphics[width=100mm]{/Users/javier/Desktop/Javier/MASTER_RIT/2011_THESIS/LUT/LUT_1/Images/SMrealretrievedELM.eps}
 	\caption{Error  \label{fig:errorLUT1}}
  \end{figure}



  \begin{figure}[H]
	\centering
    	\includegraphics[width=100mm]{/Users/javier/Desktop/Javier/MASTER_RIT/2011_THESIS/LUT/LUT_1/Images/CDOMrealretrievedELM.eps}
 	\caption{Error  \label{fig:errorLUT1}}
  \end{figure}
  
  
  
  \begin{figure}[H]
	\centering
    	\includegraphics[width=100mm]{/Users/javier/Desktop/Javier/MASTER_RIT/2011_THESIS/LUT/LUT_1/Images/CHLrealretrievedELM.eps}
 	\caption{Error  \label{fig:errorLUT1}}
  \end{figure}

As you can see in Figure \ref{fig:errorLUT1}...

  \begin{figure}[H]
	\centering
    	\includegraphics[width=100mm]{/Users/javier/Desktop/Javier/MASTER_RIT/2011_THESIS/LUT/LUT_1/Images/errorCHLSMCDOM.eps}
 	\caption{Error  \label{fig:errorLUT1}}
  \end{figure}

\subsection{Shallow Water}

As you can see in Figure \ref{fig:errorLUT2}...

  \begin{figure}[H]
	\centering
    	\includegraphics[width=100mm]{/Users/javier/Desktop/Javier/MASTER_RIT/2011_THESIS/LUT/LUT_2/figures/error_CDOMDepthBRELM.eps}
 	\caption{Error  \label{fig:errorLUT2}}
  \end{figure}

\subsection{Sensitivity Analysis}
\subsection{Adding a new band to WV-2}

\section{Real Data}
% !TEX root=Thesis_PhD.tex  
% the previous is to reference main .bib
%% CHAPTER
\chapter{Conclusions/Summary and Recommendations}
\section{Conclusions}

% The NASA's standard bio-optical algorithms OC2, OC3, OC4\cite{OReilly2000} and OCI\cite{Hu:2012} for the retrieval of chlorophyll-{\it a} were developed mainly for Case 1 waters, where the main driver is chlorophyll-{\it a}. The {\it in situ} data used to develop these algorithms does not appear to contain enough data for high concentration of color producing agents (CPAs; chlorophyll-{\it a}, minerals (or total suspended matter or TSS), and colored dissolved organic matter (CDOM)), and therefore these data are not fully representative of Case 2 waters. 

% Gerace {\it et al.} (2013)\cite{Gerace:2013} developed an algorithm for the simultaneous retrieval of color producing agents (CPAs) based on spectral matching and a look-up table (LUT) using simulated OLI data. Concha and Schott (2014)\cite{Concha2013IGARSS} extended Gerace's approach to actual OLI data. This approach does not depend on a global {\it in situ} dataset. Instead it builds a LUT specific to the inherent optical properties (IOPs) and observation conditions for the site.

% ----- added From SPIE SD
\label{sec:conc}  % \label{} allows reference to this section
\subsection{Spectral-Matching and LUT retrieval}
% copied from RS of Env. paper
The retrieval results shown in this study are promising for the use of Landsat 8 for monitoring of coastal and inland waters. The retrieval process was applied to two Landsat 8 scenes over the same study area and compared with field measurements. Maps of CPA concentrations show the expected trends of low concentration in the lake and higher concentration in the ponds. The retrieval was validated with ground-truth data taken at the same time as the satellite overpass, as opposed to comparison with historical field measurements. The comparison with field measurements exhibit error comparable with previous performance predictions for Landsat 8. An advantage of this retrieval algorithm is that it retrieves simultaneously all three CPAs.

The MoB-ELM atmospheric correction algorithm presented here tries to avoid the use of field reflectance ground-truth as commonly used in the traditional ELM method. In the MoB-ELM, the bright pixel is obtained from either the Landsat reflectance product over a bright target in the scene or from an Ecolight run simulating a water body with high concentration of CPAs in the scene. The dark pixel is obtained from an Ecolight run simulating a water body with low concentration of CPAs present in the scene. This algorithm does not require zero water signal in the NIR bands, so it could be applied to highly turbid waters. This algorithm assumes that the atmosphere is the same over the area of study.

 % and that the water signal in the SWIR bands is zero, which is commonly the case since water has a high absorption in the SWIR wavelengths, even in highly turbid waters
% Practical applications  
% Disadvantages and Advantages
% Limitations

One of the limitation of the developed retrieval algorithm is that the MoB-ELM needs some knowledge of the water body (e.g. IOPs and concentration of constituents at at least one point), which is often available but not always, and therefore, it will not work in every case because of the need for this knowledge. However, it is still a good answer for many cases where this knowledge is indeed available. Future work is aiming for an approach with good atmospheric correction without the need for ground-truth. For example, IOPs are often stable and could be estimated from previous studies (perhaps seasonally in some water bodies).

% Inclusion of a new red edge band in the future Landsat 9 will improve the results. 
Some pixels from the lake shoreline include signal from the bottom causing outliers in the retrieval results since the bottom reflectance was not accounted for in the process. The next version of this retrieval algorithm should address this issue. Glint and adjacency effects were also not addressed in this work, and they could affect the atmospheric correction.

For further validation, this method needs to be applied to more scenes over the same area of study or to a different area of study where sufficient ground-truth data are available to increase the number of samples to be compared. Additionally, the results from the MoB-ELM and the retrieval algorithm will be compared with standard products derived from ocean color satellites.

To date, there are no other sources of free access, open to the international science community, satellite imagery with similar spatial resolution or similar standard product (e.g. MODIS chl-a product) to compare with. Therefore, a direct comparison of results from our approach with typical algorithms over water bodies smaller than one kilometer is not possible. This is a challenge that needs to be addressed since there is a particular interest from local communities for monitoring water bodies that are not resolvable by current ocean color satellites. This is the case of the ponds included in this study, which are less than one kilometer in size. This fact makes Landsat 8 a pioneer in the retrieval of water quality parameters over medium to small water bodies. This also opens a need for more field measurement collection (i.e. IOPs, $R_{rs}$ and concentrations) on a regular basis where water quality needs to be assessed for the validation of products derived from moderate spatial resolution sensors such a Landsat 8 and the upcoming Sentinel 2. 
% ----- End copy


\subsection{Comparison with Standard Algorithms}
There were four different atmospheric correction algorithms analyzed in this study. Three of them are based in the NASA's standard algorithms based on the methods developed by Gordon and Wang\cite{Gordon:1994}. The fourth algorithm is based on the MoB-ELM algorithm developed by Concha and Schott\cite{Concha2014SPIE}. The remote-sensing reflectances retrieved from these atmospheric correction algorithms were compared against each other. There are better agreements in band 3 and 4 than in bands 1 and 2 for all algorithms, which can be concluded from the RMSE (\autoref{tab:Sites}). When compared with {\it in situ} data, the MoB-ELM and Gordon and Wang's algorithm from SeaDAS (SeaDAS-SWIR) show similar results for all bands (\autoref{fig:NRMSE130919_RRS}). The results from the Gordon and Wang's algorithm from Acolite (Acolite-SWIR) shows the largest disagreement (\autoref{fig:NRMSE130919_RRS}).

When retrieved chlorophyll-{\it a} concentrations ($C_a$) are compared with {\it in situ}, the results from the Concha and Schott's approach performed better than the rest of the algorithms(\autoref{fig:NRMSE130919CHL}). This was expected since the ocean color algorithms were developed with the lack of {\it in situ} data representative of Case 2 waters with high concentration of color producing agents (CPAs).

These results demonstrate that for Case 2 waters, the solution for the ocean color measurements likely needs to be local and not global, as opposed to Case 1 waters. This local solution as implemented here, the combined MoB-ELM algorithm with Concha and Schott's $C_a$ retrieval algorithm, requires some knowledge of the waters to be studied, such as inherent optical properties (IOPs) and CPA concentration in at least one site.

The presented work is a very limited study using only a single data set. The results need to be tested on a much larger set of data to see if the can be generalized. This implies that more {\it in situ} data need to be collected.
% --------- end SPIE SD
% -------------------------------------
\section{Future Work}
\subsection{Hydrodynamics models} 
The next step will be to use the validated results from the retrieval process for training hydrodynamics models to predict the future behavior of the water bodies. This would be based on previous work done by Pahlevan~{\it et al.} \cite{Pahlevan:2012b}, who used concentration maps obtained from the retrieval process using satellite imagery to train the ALGE hydrodynamic model. For example, the hydrodynamic model would allow us to monitor the dynamics of coastal and inland waters near river discharges. The maps of water constituent concentrations on the surface can be used to calibrate the hydrodynamic models.

\subsection{Investigate New Sensor Enhancements for Future Missions}
Water pixel spectra from a hyperspectral image (e.g. Hyperspectral Imager for the Coastal Ocean (HICO),  Airborne Visible/InfraRed Imaging Spectrometer (AVIRIS)) will be modified to simulate data similar to Landsat 8 but with the addition of a new NIR band. The retrieval process will be performed with these simulated data with and without the new NIR band in order to evaluate performance improvement. A similar analysis will be done to evaluate narrower spectral bandwidths available in Landsat 8 compared to those found in the MEdium Resolution Imaging Spectrometer (MERIS) and MODIS, for instance. 

\section{Concluding Remarks}
% !TEX root=Thesis_PhD.tex 
% the previous is to reference main.bib
%% CHAPTER
\begin{appendices}
% @@@@@@@@@@@@@@@@@@@@@@@@@@@@@@@@@@@@@@@@@@@@@@@@@@@@@@@@@@@@@@@@@@@@@@@@@@@@@
\chapter{Field Measurements}
% \addcontentsline{toc}{chapter}{Appendix}
% \renewcommand{\thesection}{\Alph{section}}
% @@@@@@@@@@@@@@@@@@@@@@@@@@@@@@@@@@@@@@@@@@@@@@@@@@@@@@@@@@@@@@@@@@@@@@@@@@@@@
\label{ch:fieldmea}
% -----------------------------------------------------------------------------
\section{Water Samples}


\subsection{Equipment}

\begin{multicols}{3}
\begin{itemize}
  \item Dark Nalgene bottles
  \item Monroe County Environmental Lab bottles
  \item Cooler
  \item Marker
  \item Bottle label
  \item Ice packs
  \item GPS
  \item Extra batteries GPS
  \item Data sheet
  \item Pen
  \item Back pen
  \item Canoe 
  \item Transport straps for canoe
  \item Paddles
  \item Life jacket
  \item Suncream 
  \item Drinking water
  \item Wipes to clean extra suncream from hands
  \item Bucket with rope (in case is not possible to take water samples due to bad condition weather, for example)
\end{itemize}
\end{multicols}


\subsection{Procedure}
\begin{enumerate}
  \item Throughly clean the bottles prior to collection by brushing them inside with tap water a couple of times and rinse with DIW water a couple of times
  \item Once in the site, press GPS button to save location
  \item Fill the log sheet with the ``Location Description'', ``GPS WAYPOINT'' and ``Time''
  \item Take a bottle from the cooler and write the bottle label down in the ``Bottle Number'' section on the data sheet along
  \item Rinse the Nalgene Bottle and cap at least 3 times with water before filling
  \item Submerge bottle with the cap on it in an undisturbed location.
  \item Uncap the submerged bottle  to take subsurface water sample (avoid to take water from the surface) \cite{Montana08} 
  \item Cap the bottle with the bottle still submerged
  \item Store bottle up-right immediately in the cooler in order to avoid direct sun light \cite{Mueller1995}
  \item Once off the water, text to Nina or person in charge of the collection for example: ``Safe, Long Pond team''
  \item Take water some water samples to the Monroe County Environmental Lab, if applicable.
  \item Place the sample bottles in the refrigerator as soon as possible
  \item Filter water samples right after collection to preserve the chlorophyll and storage filters in the freezer as soon as possible
\end{enumerate}
Notes: 
\begin{itemize}
  \item Do not take any personal electronic device with you (recommended) to avoid dropping it on the water
  \item Storage car keys in zipped bag in you packet
  \item In case of bad weather conditions that do not allow paddle the canoes, take at least water samples from the Charlotte pier with the bucket
\end{itemize}

% -----------------------------------------------------------------------------
\section{\texorpdfstring{$R_{rs}$}{Rrs}}

% method described by \cite{Mobley:1999} for measuring the spectra of the downwelling irradiance $E_d$, the surface reflected sky radiance $L_s$, and the water-leaving radiance $L_w$ for each site 

This section describes how to take the remote-sensing measurement using a single instrument that measures radiance (spectroradiometer or spectrometer) such as a SVC \cite{SVCHR1024i} or an ASD \cite{ASDManual2012} instrument. This procedure is taken from \cite{Mobley:1999} and \cite{Mueller1995}. 

Recall that the \gls{rrs} is defined as

\begin{equation}\label{eq:Rrs}
	R_{rs}(\theta,\phi,\lambda)=\frac{L_w(\theta,\phi,\lambda)}{E_d(\lambda)}
\end{equation}
where $L_w$ is the water-leaving radiance in the polar and azimuthal directions $\theta$ and $\phi$, respectively, and $E_d$ is the downwelling spectral plane irradiance incident onto the water surface. A radiometer pointing down toward the water surface in direction $(\pi-\theta,\phi)$ does not directly measure $L_w$. Instead, it measures $L_w$  plus any incident sky radiance reflected $L_r$ by the water surface into the field of view of the sensor. This total radiance at the sensor $L_t$ is define as

\begin{equation}\label{eq:Lt}
	L_t(\theta,\phi) = L_r(\theta,\phi)+L_w(\theta,\phi)\Rightarrow L_w(\theta,\phi)=L_t(\theta,\phi) - L_r(\theta,\phi)
\end{equation}

The term $L_r$ can be replaced by

\begin{equation}\label{eq:Lsky}
	L_r = \rho L_{sky}
\end{equation}
where $\rho$ is the proportionality factor that relates the radiance measured when the sensor views the sky to the reflected sky radiance measured when the sensor views the water surface. \cite{Mobley:1999} suggests to use $\rho \approx 0.028$ for a sensor view angle $\theta_v \approx 40^\circ$ from the nadir and  $\phi_v \approx 135^\circ$ from the Sun with the constraints of a clear sky and wind speed less than $5m/s$.

Although $E_d$ could be measured directly with an appropriate sensor, it will be estimated from the radiance measured from a Lambertian surface (Spectralon) because both instruments in this case (SVC and ASD) are set to measure radiance. When an irradiance $E_d$ falls in a Lambertian surface with a known irradiance reflectance $R_g$, the uniform radiance $L_g$ leaving the surface is given by 

\begin{equation}\label{eq:Lg}
	L_g = (R_g/\pi)E_d\Rightarrow E_d = L_g*\pi/R_g
\end{equation}

Applying \autoref{eq:Lt} ,\autoref{eq:Lsky} and \autoref{eq:Lg} in \autoref{eq:Rrs} yields

\begin{equation}
	R_{rs} = \frac{L_t-\rho L_{sky}}{\frac{\displaystyle \pi}{\displaystyle R_g}L_g}
\end{equation}

\subsection{ASD}

The ASD should be in ``radiance mode''. Three different radiance measurements need to be taken:
\begin{itemize}
	\item $L_g$: pointing the Spectralon

Description: $L_g$ is measured with the sensor pointing downward in the same direction as is used in viewing the water surface (see \autoref{fig:Ltmea}), while the Spectralon is inserted into the sensor FOV. The Spectralon should be normal to the water surface.

	\item $L_t$: pointing the water surface

Description: $L_t$ is measured with the sensor pointing downward toward the water surface in the direction $\approx \pi-\theta_v = 140^\circ$ from nadir with $\theta_v = 40^\circ$ and $\phi_v \approx 135^\circ$ or $\phi_v \approx -135^\circ$ from the Sun as illustrated in \autoref{fig:Ltmea}.

\begin{figure}[htb]
\centering
    \includegraphics[width=10cm]{/Users/javier/Desktop/Javier/PHD_RIT/LDCM/WaterQualityProtocols/Latex/Images/Lgmea.png}
    \vspace{0.5cm}
   \caption[]{\label{fig:Ltmea} $L_t$ measurement.}
\end{figure}

	\item $L_{sky}$: pointing the sky

Description: $L_{sky}$ is measured with the sensor pointing upward toward the sky in the direction $\approx \theta_v = 40^\circ$ from nadir and $\phi_v \approx 135^\circ$ or $\phi_v \approx -135^\circ$ from the Sun as illustrated in \autoref{fig:Lskymea}.

\begin{figure}[htb]
\centering
    \includegraphics[width=10cm]{/Users/javier/Desktop/Javier/PHD_RIT/LDCM/WaterQualityProtocols/Latex/Images/Lskymea.png}
    \vspace{0.5cm}
   \caption[]{\label{fig:Lskymea} $L_{sky}$ measurement.}
\end{figure}

\end{itemize}

\subsection{SVC}
The same three radiance measurements described above need to be taken:

\begin{itemize}
	\item $L_g$: pointing the Spectralon

Description: The measurement is taken in the same fashion described in the previous section and it is taken only once per site. When the SVC instrument is used in ``reflectance mode'', it is necessary to measure first an standard measurement (Spectralon measurement). This standard measurement is the $L_r$ and is recorded internally in the ``sig'' file. Therefore, $L_r$ needs to be extracted from the later from the ``sig'' file. 

	\item $L_t$: pointing the water surface

Description: $L_t$ is measured in the same fashion described in the previous section. However, this measurement is saved internally in the ``sig'' file after the standard measurement column ($L_g$).

	\item $L_{sky}$: pointing the sky	

Description: $L_{sky}$ is measured in the same fashion described in the previous section. However, this measurement is saved internally in the ``sig'' file after the standard measurement column ($L_g$).

\end{itemize}
{\bf Notes:}
\begin{itemize}
	\item Wear dark clothes, preferable black, to avoid contamination from adjacent objects.
	\item Avoid any reflection from nearby objects in the boat or ship by covering the ship's side with a black turf. 
\end{itemize}



% @@@@@@@@@@@@@@@@@@@@@@@@@@@@@@@@@@@@@@@@@@@@@@@@@@@@@@@@@@@@@@@@@@@@@@@@@@@@@
\chapter{Lab Measurements}
\label{ch:labmea} 


\begin{figure}[htb]
% \subfloat[]{
\centering
    \includegraphics[width=14cm]{/Users/javier/Desktop/Javier/PHD_RIT/LDCM/WaterQualityProtocols/Images/WaterQualityProtocolDiagram.png}%}\hspace{0.5cm}
% \subfloat[]{   
%     \includegraphics[width=8cm]{/Users/javier/Desktop/Javier/PHD_RIT/20122_Winter/Instrumentation/report3/Images/SideFluoSpec.jpg}}
    \vspace{0.5cm}
   \caption[]{\label{fig:ProtocolsDiagram} Lab measurement protocols diagram.}
\end{figure}
 %------------- 

% $a_{YS}$ cannot be determined directly. An approximation of $a_{YS}$ may be obtained by a spectrophotometer scan of a filtered sample (\cite{Bukata1995}, p.125). Spectrophotometer used in normal mode do not measure true absorbance but {\color{red} attenuance} because all the scattered light is measured. To overcome this, the cells can be placed close to a wide photomultiplier (\cite{Kirk1983}, p.51).
% -----------------------------------------------------------------------------
\section{IOPs}
%*******************************
\subsection{Chlorophyll absorption coefficients}
The spectrophotometric methods are described by \cite{Mitchell2002} and \cite{Cleveland1993}.
%*******************************
\subsection{Minerals absorption coefficients}

%*******************************
\subsubsection{Equipment}
%*******************************
\subsubsection*{Filtration}
\begin{itemize}
  \item Vacuum pump
  \item Filter tower (filter funnel stem, filter base, funnel, filter cup)
  \item Whatman Binder-Free Glass Microfiber Filters: Type GF/F - Diameter: 2.5cm
  \item Forceps
\end{itemize}
%*******************************
\subsubsection*{Measurement}
\begin{itemize}
  \item Spectrophotometer
  \item {\color{red} Two lenses support}
  \item Squirt bottle with {\color{red} DIW} or small pipette with {\color{red} DIW} 
  \item Methanol
\end{itemize}
%*******************************
\subsubsection{Procedure}
%*******************************
\begin{enumerate}
  \item Turn the spectrophotometer on at least 30 minutes before measuring
  \item Set the spectrophotometer parameters in the UV-2101PC software menu: Configure > Parameters...
  \item Select the Serial Port to be use for communication with the instrument. Go to: Configure > PC Configuration... In the PC Configuration Parameters, select Photometer Serial Port and click OK
  \item From the menu, go to Configure > Utilities. In the System Utilities window, Turn Photometer On and press OK
  \item Pour DIW water to two GF/F Whatman filters and stick them in the two lenses support. Both filter should have the same amount of water. Add water with the pipette or the squirt bottle
  \item Place the two lenses support in the spectrophotometer
  \item Press the Baseline button in the UV-2101PC software
  \item Perform an scanner to see the baseline level of the instrument by pressing the "Start" button in the software
  \item Press the "Go To WL" button of the software and type $850 [nm]$. Press the "Auto Zero" button of the software (optional)
  \item Invert water sample bottle a couple of times to mix by turbulence and ensure large particles that settle are re-suspended (\cite{Mitchell2002})
  \item \label{item:place_filter} Using the forceps, place the filter on the filter base and place the filter cup on the base. \textbf{Record volume filtered}. 
  \item \label{item:filtration} Turn the vacuum pump on and turn the knob $90^\circ$ to allow filtration. Once all the water pass through the filter, turn the know $90^\circ$ back and the turn the vacuum pump off
  \item Using the forceps, take the filter with just water from the two lenses support and storage it for future baselines. Do not remove the blank filter ({\color{red} OR reference filter}) for the whole measurement session
  \item \label{item:place_filter_spec} Using the forceps, take the sample filter from the filtering tower. Add one or a few water drops to the sample filter and stick in the two lenses support. 
  \item  Measure absorbance in the spectrophotometer by pressing the "Start" button of the software and save data. This will be the $OD_{filt}$ measurement
  \item Using the forceps, remove carefully the sample filter from two lenses support avoiding to break it and place in the filter tower as in step \ref{item:place_filter}
  \item Pour enough solvent {\color{red} to sumerge} the filter in the filter cup. {\color{red} Wait 5 min} and then filter as in step \ref{item:filtration}
  \item Repeat step \ref{item:place_filter_spec}
  \item Measure absorbance in the spectrophotometer by pressing the "Start" button of the software and save data. This will be the $OD_{no~pig}$ measurement 
  \item Record the area of filtration in the sample filter
  \item[]Note: The instrument only allows to save four measurement at the time. To save measurement, go to File > Data Translation > ASCII Export... {\color{red} Select channels to be saved, name files and press OK}
\end{enumerate}
%*******************************
\subsubsection{Calculations}


%*******************************
\subsection{CDOM absorption coefficients}
%*******************************
\subsubsection{Equipment}
%*******************************
\subsubsection*{Filtration}
\begin{itemize}
  \item Whatman GD/X 13 and 25mm Disposable Syringe Filters - Nylon $0.2[\mu m]$ Nylon
  \item {\color{red}Syringe}
\end{itemize}
\subsubsection*{Measurement}
\begin{itemize}
  \item Spectrophotometer (Shimadzu UV2100V - Dual beam spectrophotometer)
  \item Blank cell
  \item Sample cell
  \item Purified water
  \item Ethanol
\end{itemize}
%*******************************
\subsubsection{Procedure}
%*******************************
\begin{enumerate}
  \item \textbf{Turn the Spectrophotometer on at least one hour before measuring}. It needs to be warmed up for optimal measurements.
  \item Wash the syrenge filter out 3 times with purified water
  \item Rinse cells a couple of times with a small amount of ethanol by shaking it (optional, if the cells seem dirty)
  \item Use cotton sweep to clean internal face (optional, if face seems dirty)
  \item Rinse cell with purified water
  \item Clean and dry the external surface of the cells with optics paper. Be careful with scratching the surface, specially the front and bottom faces
  \item Select a Slit Width equal to $5.0~[nm]$ in the photometer software
  \item Select a Sampling Interval of $1~[nm]$ or $2~[nm]$ in the photometer software
  \item Fill both cells with purified water and extract bubbles
  \item Place both blank and sample cells filled with purified water in the sample compartment of the spectrophotometer
  \item Press ``Auto Zero'' button in the spectrophotometer software
  \item Press ``Baseline'' button in the spectrophotometer software
  \item Fill sample cell with filtered water from the syringe filter
  \item Press start button
  \item Save Channel in the spectrophotometer software
  \item Go to Data Translation > ASCII Export in spectrophotometer software
\end{enumerate}
\textbf{Important:} the samples should be at room temperature. The absorbance measurement is sensible to temperature changes.
%*******************************
\subsubsection{Data treatment}
%*******************************
\begin{itemize}
  \item Absorbance: $A=-\ln{\displaystyle\frac{1}{T}}$ 
  \item Substract bias before convert to coefficients
  \item $a_{CDOM}=2.303~A(\lambda)/L~~[m^{-1}]$ where $A(\lambda)$ is the absorbance and $L$ the pathlength of the absorbance cell in meters.
\end{itemize}
% -----------------------------------------------------------------------------
\section{Concentrations}
\subsection{Chlorophyll-{\it a} concentration}

\todo{Show comparison of Monroe county to RIT Lab}Methods described by \cite{Lorenzen:1967fk} and \cite{Ritchie:2008eu}.

\subsubsection{Calculations}

The calculations used \cite{Lorenzen:1967fk} are:

\begin{equation}
  C_a = \frac{26.7(655_o - 665_a)\times v}{V\times l}
\end{equation}

\begin{equation}
  Pheo = \frac{26.7([1.7\times 665_a]-665_o)\times v}{V\times l}
\end{equation}

\noindent where: \\
$665_o = 665 - (750-blank~value)~before~acidification$\\
$665_a = 665 - (750-blank~value)~after~acidification$  \\
$v = $ volume of extract in mililiters $[ml]$ \\
$V = $ volume of water filtered in liters $[L]$ \\
$l = $ pathlength of cuvette ($1cm$ for the cuvette used) \\

Note: concentrations are in $[mg/m^3]$ or $[\mu g/L]$.


\subsection{Total suspended solids (TSS)}

\subsubsection{Equipment}
%*******************************
% \subsubsection*{Filtration}
\begin{itemize}

  \item TCLP filters ($47 mm$, $0.7\mu m$)
  \item Vacuum
  \item Balance
  \item Graduate cylinder
  \item Forceps

\end{itemize}

%*******************************
\subsubsection{Procedure}
%*******************************
\begin{enumerate}
  \item Weight filters before filtering
  \item Record volume to filter. Use graduated cylinder
  \item Use vacuum to filter water with the TCLP filters to filter the particles
  \item Weight filter in balance
  \item Put in aluminium foil with weight
  \item Dry sample at $75^\circ C$ for a couple of hours
\end{enumerate}


\begin{equation}
SPM_{\displaystyle concentration} = \frac{[final~filter~weight~(mg) - tare~filter~weight~(mg)]}{volume~filtered~(L)}~~~\left[\frac{mg}{L}\right]
\end{equation}


% &&&&&&&&&&&&&&&&&&

% @@@@@@@@@@@@@@@@@@@@@@@@@@@@@@@@@@@@@@@@@@@@@@@@@@@@@@@@@@@@@@@@@@@@@@@@@@@@@
\chapter{Main Codes}


\singlespacing
\lstset{language=bash,caption={Example of an input file used in Ecolight.},label=code:EcolightInput}
\renewcommand{\lstlistingname}{Code}
\begin{lstlisting}
0,400,2500,.02,488,.00026,1,5.3
FFbb determination for ONTOS
OutputEL
0,1,0,0,0,1
2,1,0,2,3
4,4
0,flaCH,flaCD,flaSM
0,2,440,0.1,0.014
0,0,440,0.1,0.014
0,4,440,1,0.01712
0,0,440,0.1,0.014
/home/jxc4005/hydrolight52Javier_install/data/H2OabDefaults_FRESHwater.txt
/home/jxc4005/HYDROLIGHT/EL5.2/user_inputs/astar_CH_ONTOS140929_CountyUncorr.txt
dummyastar.txt
/home/jxc4005/HYDROLIGHT/EL5.2/user_inputs/astar_SM_ONTOS140929_County.txt
4, 660, 0.189, 1, 0.751, -999
0,-999,-999,-999,-999,-999
-1,-999,0,-999,-999,-999
0,-999,-999,-999,-999,-999
bstarDummy.txt
/home/jxc4005/HYDROLIGHT/EL5.2/user_inputs/ChloroSct.txt
dummybstar.txt
/home/jxc4005/HYDROLIGHT/EL5.2/user_inputs/susmin.sct
0, 0, 550, 0.01, 0
0, 0, 0, 0, 0
-1, 0, 0, 0, 0
0, 0, 550, 0.01, 0
pureh2o.dpf
user_dpfCHL
isotrop.dpf
user_dpfTSS_b
 120
400, 405, 410, 415, 420, 425, 430, 435, 440, 445,
450, 455, 460, 465, 470, 475, 480, 485, 490, 495,
500, 505, 510, 515, 520, 525, 530, 535, 540, 545,
550, 555, 560, 565, 570, 575, 580, 585, 590, 595,
600, 605, 610, 615, 620, 625, 630, 635, 640, 645,
650, 655, 660, 665, 670, 675, 680, 685, 690, 695,
700, 705, 710, 715, 720, 725, 730, 735, 740, 745,
750, 755, 760, 765, 770, 775, 780, 785, 790, 795,
800, 805, 810, 815, 820, 825, 830, 835, 840, 845,
850, 855, 860, 865, 870, 875, 880, 885, 890, 895,
900, 905, 910, 915, 920, 925, 930, 935, 940, 945,
950, 955, 960, 965, 970, 975, 980, 985, 990, 995,
1000,
0,0,0,0,2
2, 3, 48, 0, 0
272, 43.28085,-77.61919, 29.92, 1, 80, 2.5, 15, 4.99746, 300
4.99746, 1.34, 20, 35
0, 0
0, 5, 0, 5, 10, 15, 20, 
/home/jxc4005/hydrolight52Javier_install/data/H2OabDefaults_FRESHwater.txt
1
/home/jxc4005/hydrolight5Aaron_install/data/user/mascot_ac9.txt
dummyFilteredAc9.txt
dummyHscat.txt
/home/jxc4005/hydrolight5Aaron_install/data/user/Chlzdata_10m.txt
dummyComp.txt
dummyR.bot
dummydata.txt
/home/jxc4005/hydrolight5Aaron_install/data/user/Chlzdata_10m.txt
dummyComp.txt
dummyComp.txt
/home/jxc4005/hydrolight5Aaron_install/data/user/Ed_total.txt
/home/jxc4005/hydrolight5Aaron_install/data/MyBiolumData.txt
\end{lstlisting}

\end{appendices}

\listoftodos

%\bibliographystyle{ieeetr}
%\bibliographystyle{unsrtnat}
\bibliographystyle{apalike}

\bibliography{/Users/javier/Desktop/Javier/PHD_RIT/Latex/javier_bib}

\printindex

\end{document}  