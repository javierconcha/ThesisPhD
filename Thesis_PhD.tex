% \documentclass[draft]{book}
\documentclass[draft]{book}

\usepackage{graphicx}

\usepackage{epstopdf}
\usepackage{epsfig}

\usepackage{float}
\usepackage{amssymb,amsmath}
\newcommand{\bm}[1]{\boldsymbol{#1}}

\usepackage[makeroom]{cancel} %for cancel out in eq.

\usepackage{multirow}
\usepackage{fullpage}
\usepackage{appendix}
\usepackage{setspace}

\usepackage[bf, small, center]{caption}
\setlength{\belowcaptionskip}{10pt}

\usepackage{longtable}

\usepackage{geometry}
\geometry{
%  top=1.0in,  
%  inner=1.5in,
%  outer=1.5in,
  bottom=1.2in,
%  headheight=3ex, 
  headsep=3ex,         
}

\usepackage{fancyhdr}
\pagestyle{fancy}
\fancyhead[LO, RE]{\rightmark}
\fancyhead[LE, RO]{\thepage}
\fancyfoot[]{}
%\renewcommand{\headrulewidth}{0.3pt}

\usepackage{array}
\newcolumntype{L}[1]{>{\raggedright\let\newline\\\arraybackslash\hspace{0pt}}m{#1}}
\newcolumntype{C}[1]{>{\centering\let\newline\\\arraybackslash\hspace{0pt}}m{#1}}
\newcolumntype{R}[1]{>{\raggedleft\let\newline\\\arraybackslash\hspace{0pt}}m{#1}}



\usepackage[phd]{thesisfrontmatter}

% *** NOTE WHICH PROGRAM GENERATED IMAGES OR WHICH POWER POINT THEY ARE FROM
%compare methodology with script for final process
%how much detail in methodology?
%some sort of processing flowchart?
%how much do we care about processing?

\usepackage[parfill]{parskip}    % Activate to begin paragraphs with an empty line rather than an indent


\usepackage{mathtools}


\usepackage{color}
\usepackage{soul}



\DeclareGraphicsRule{.tif}{png}{.png}{`convert #1 `dirname #1`/`basename #1 .tif`.png}

\let\stdsection\chapter  
\renewcommand\chapter{\newpage\stdsection}  

%%%%%%%%%%% for Appendix
\makeatletter
\newcommand\appendix@section[1]{%
  \refstepcounter{section}%
  \orig@section*{Appendix \@Alph\c@section: #1}%
  \addcontentsline{toc}{section}{Appendix \@Alph\c@section: #1}%
}
\let\orig@section\section
\g@addto@macro\appendix{\let\section\appendix@section}
\makeatother

% Added 12-03-13 ----------------------------------------
\usepackage{imakeidx} % to create index
\indexsetup{othercode=\small}
\makeindex[program=makeindex,columns=2,intoc=true,options={-s MyIndex.ist}]
% -------------------------------------------------------
%%%%%%%%%%%%%%

\usepackage{hyperref}
\hypersetup{
    % bookmarks=true,         % show bookmarks bar?
    unicode=false,          % non-Latin characters in AcrobatÕs bookmarks
    pdftoolbar=true,        % show AcrobatÕs toolbar?
    pdfmenubar=true,        % show AcrobatÕs menu?
    pdffitwindow=false,     % window fit to page when opened
    pdfstartview={FitH},    % fits the width of the page to the window
    pdftitle={L8's potential for water constituents retrieval },    % title
    pdfauthor={Javier Concha},     % author
    pdfsubject={Subject},   % subject of the document
    pdfcreator={Creator},   % creator of the document
    pdfproducer={Producer}, % producer of the document
    pdfkeywords={keyword1} {key2} {key3}, % list of keywords
    pdfnewwindow=true,      % links in new window
    colorlinks=true,       % false: boxed links; true: colored links
    linkcolor=blue,          % color of internal links
    citecolor=cyan,        % color of links to bibliography
    filecolor=magenta,      % color of file links
    urlcolor=blue           % color of external links
}

\usepackage[all]{hypcap} % to see figure with hyper ref

\setcounter{secnumdepth}{5}
\setcounter{tocdepth}{5}

% Added 11-12-13 ----------------------------------------

\usepackage{titlesec} % For the spacing after and before chapter title

\titleformat{\chapter}[display]
    {\normalfont\huge\bfseries}{\chaptertitlename\ \thechapter}{10pt}{\Huge}
\titlespacing*{\chapter}{0pt}{40pt}{20pt}

\titlespacing*{\section}{0ex}{2.5ex}{2ex}
\titlespacing*{\subsection}{0ex}{1.5ex}{1ex}

\usepackage{indentfirst}
\setlength{\parindent}{20pt}
\doublespacing

% Added 11-17-13 ----------------------------------------
% Select what to do with todonotes: 
\usepackage[disable]{todonotes} % notes not showed
% \usepackage[draft]{todonotes}   % notes showed
% \setlength{\marginparwidth}{2cm}
% \usepackage[textwidth=3.7cm]{todonotes}

\setlength{\marginparwidth}{3.7cm}

% Added 11-15-13 ----------------------------------------
% \includeonly{Introduction,Objectives,Background_and_Theory,Methodology,Results,Summary,Appendix} 
% \includeonly{Methodology} 
\includeonly{Introduction,Objectives,Background_and_Theory,Methodology,Results} 
\usepackage{tikz} % for flow charts
  \usetikzlibrary{shapes,arrows,positioning,shadows,calc}
\usepackage{colortbl}
\usepackage{caption}
\usepackage{subcaption}
\usepackage{multirow}

\listfiles

\setlength{\headheight}{15pt} % to avoid "Package Fancyhdr Warning: \headheight is too small (0.0pt): Make it at least 12.0pt."

% Added 02-18-14 ----------------------------------------
% \setlength{\abovecaptionskip}{-2ex}
% \setlength{\belowcaptionskip}{-4ex} % space after caption
%******************************************************************************************************


% Added 08-10-14 
\usepackage{enumitem}
%******************************************************************************************************
%******************************************************************************************************
%******************************************************************************************************
\begin{document}

\degreetitle{The Use of Landsat-8 for Monitoring of Fresh and Coastal Water}
\degreeauthor{Javier A. Concha}
\degreedate{October 7, 2014}
% \degreedate{\today{}}
\prevdegreeA{M.S. Rochester Institute of Technology, 2012}
\advisor{Dr. John R. Schott}
\memberA{Dr. Anthony Vodacek}
\memberB{Dr. Charles Bachmann}
\memberC{Dr. Christy Tyler}
% \memberD{Dr. Christy Tyler}

\makeproposaldeclaration
\makePHDproposalapproval
% \makecopyright


% \maketitle

\pagenumbering{roman}

% \chapter*{Abstract}
% \addcontentsline{toc}{chapter}{Abstract}

\begin{abstract}
\setlength{\parindent}{20pt}
The Landsat Data Continuity Mission (LDCM; a.k.a. Landsat-8), recently launched (February 2013), is the next generation of Landsat satellite and continues more than 40 years of uninterrupted imaging acquisition, playing a critical role in monitoring, understanding and managing natural resources such as water. Landsat-8, with its improved spectral bands and radiometric resolution, has the potential to dramatically improve our ability to simultaneously retrieve the three primary coloring agents, chlorophyll (Chl), colored dissolved organic material (CDOM) and suspended material (SM) from water bodies. This work presents an approach to obtain these coloring agents in coastal and fresh waters.

In the Case 2 water problem, the sensor-reaching signal due to water is very small when compared to the signal due to the atmospheric effects. Therefore, adequate atmospheric correction becomes an important first step to accurately retrieving water parameters. As a first approach, a model based empirical line method (ELM) atmospheric correction method converts sensor-reaching radiance to water leaving reflectance. This model employs pseudo invariant feature (PIF) pixels extracted from Landsat images along with an in water radiative transfer model (HydroLight) to obtain the field spectra. Further atmospheric compensation technique based in algorithms currently used in Ocean Color sensors will be investigated.

A look-up-table (LUT) methodology is implemented to retrieve the water parameters. The LUT is created using HydroLight.

Collections of water samples when the satellite passes over the Rochester area are planned for summer 2013 and 2014. Concentration and inherent optical properties (IOPs) measurement will help to validate the methods.

\end{abstract}

% \chapter*{Acknowledgements}
% %*****************************************************************************************************
% \begin{acknowledgements}
% \setlength{\parindent}{20pt}
% To my parents, Osvaldo Concha and Nelly Sepulveda, sister Loretto Concha and brother Alvaro Concha. USGS. Monroe County Environment: Scott, Providencia and Gary. Nina and Rolo. Dr. Christy Taylor. Paul. Aaron Gerace. Alan Wiedmman. Ocean Color community: Emmanuel Boss and attendants of the 2013 Ocean color and instrumentation summer course at the Darling Marine Center. Fulbright commission. My advisor Dr. John Schott. Amanda and Cindy. Classmates: Aly, Bikash, Viraj, Madurima, Peter Sun. Professor Dr. John Kerekes. Collection help: Harold Valdivia. My dear friends: Juan Saldana, Kader, Patrick, Sarada, Susan, Melisa, Sasha, Jeremy. The people at Crossfit Rochester: Joe, Andrew. The Latin Rhythm Dance Club at RIT. My best support here Aixa de Jesus. My adopted Chilean family and dancers in Rochester: Marcia, Rosa and Doña Rosita, Raul, Willy and Marisol, Harold and Nancy. My Chilean friends in U. of R. Felipe, Maria Eugenia, Emilio, Brenda and Pauly. Kimberley Thoms.
% \end{acknowledgements}
%*****************************************************************************************************

\tableofcontents

\listoffigures
\addcontentsline{toc}{chapter}{List of Figures}

\listoftables
\addcontentsline{toc}{chapter}{List of Tables}

% % !TEX root=Thesis_PhD.tex  
% the previous is to reference main .bib
%% CHAPTER
\chapter{Introduction}
\label{ch:introduction} 
\pagenumbering{arabic} 
Ocean color studies at a global scale, such as chlorophyll-{\it a} level trends in oceans, can be performed by the heritage Ocean Color satellites (e.g. SeaWiFS, MODIS). These satellites satisfied the spatial requirement for these kinds of studies. However, when the region of interests include coastal or inland waters, which could be considered Case 2 water, their spatial resolution of roughly a thousand meters are not enough to resolve smaller water bodies. These kinds of waters are important for us, because it is these kinds of waters that we have the most interaction with, like for drinking or recreation. 

% Landsat
The Landsat project has been monitoring the earth for over four decades, being the longest uninterrupted data set available. The Landsat satellites' main mission is to image the land areas of the earth and therefore there are typically no open ocean (case 1 water) images available. This is the reason why Landsat satellites have been underestimated by the ocean color community for the study of water bodies. In addition, the Landsat instruments have generally had broad bands and low SNR when compared to heritage ocean color satellites such as SeaWiFS and MODIS. Carrying two instruments onboard, the Operational Land Imager (OLI) and the Thermal InfraRed Scanner (TIRS), Landsat 8 is the first of a new generation of Landsat satellite with state-of-the-art technology. With its 12-bit quantization and improved signal-to-noise ratio (SNR), OLI is a big improvement to the Landsat mission. In addition, OLI includes a new coastal band that increased the spectral resolution of the instrument. These improvements are the main drivers to hypothesize that the Landsat 8 satellite will definitely have a better performance in water quality studies than its predecessors. Since its launch in 2013, the Operational Land Imager (OLI) instrument onboard Landsat 8 has created high expectations in the ocean color community. Its spatial resolution of $30m$ and its improved signal-to-noise ratio (SNR) compared with its predecessors make Landsat 8 a perfect candidate to be used in coastal and inland water studies. Therefore, the overall objective of this research is to demonstrate that the new generation of Landsat satellites are capable of accurately retrieving water constituents.

The retrieval of water components is in general performed in the reflectance domain, so the very first step in this work is to perform a high quality atmospheric correction to the radiance image from Landsat 8. This is a complex task to perform over water because the signal leaving the water that reaches the sensor is very small when compared to the signal reaching the sensor produced by atmospheric scattering. Most of the atmospheric correction algorithms applied to Ocean Color satellites (e.g. SeaWiFS and MODIS) are not suitable for highly turbid coastal water \cite{Patt2003}. In this work, different approaches for atmospheric correction will be investigated. The first atmospheric correction algorithm investigated is the model-based empirical line method (MOB-ELM) that uses a combination of an in-water radiative transfer model over water and a Landsat reflectance product to determine the bright and dark pixels in the image. The second one is the standard SeaWiFS/MODIS atmospheric correction algorithm that uses SWIR bands (\cite{Wang:2007}). The water-leaving reflectance values obtained after atmospheric correction are validated by comparison with water surface reflectance measured in situ. 

After having corrected the image, the next step is to apply a retrieval algorithm that outputs water component retrieval maps of the main water components (chlorophyll, {\todo{or SM?} sediment} and colored dissolved organic matter (CDOM)). A spectral matching and look-up table (LUT) approach is utilized. It uses a least square error minimization algorithm to find the best match for a specific reflectance signal in a LUT of spectral water-leaving reflectance curves. The LUT is created using HydroLight 5, an in-water radiative transfer model (\cite{Mobley:2005}). Each curve in the LUT has a specific set of water component concentrations. This is performed on a pixel-by-pixel basis. The concentration values obtained from the retrieval algorithm are validated by comparison with concentration measured in the lab from water bodies present in the Landsat 8 image.

In order to have outputs that are representative of the water bodies that are being studied, Inherent Optical Properties (IOPs) of those specific waters have to be input to the HydroLight model. To accomplish this, collections of water samples were conducted at the same time that the Landsat 8 satellite passed over the area of study (Rochester, NY). After collection, these water samples were analyzed in the lab to obtain IOPs for the main water constituents. Furthermore, apparent optical properties (AOPs) and backscattering measurements were also collected for further comparison and to pursue closure between HydroLight AOPs results and in-situ AOPs measurements.
% !TEX root=Thesis_PhD.tex  
% the previous is to reference main .bib
%% CHAPTER
\chapter{Introduction}
\label{ch:introduction} 
\pagenumbering{arabic} 
Ocean color studies at a global scale, such as chlorophyll-{\it a} level trends in oceans, can be performed by the heritage Ocean Color satellites (e.g. SeaWiFS, MODIS). These satellites satisfied the spatial requirement for these kinds of studies. However, when the region of interests include coastal or inland waters, which could be considered Case 2 water, their spatial resolution of roughly a thousand meters are not enough to resolve smaller water bodies. These kinds of waters are important for us, because it is these kinds of waters that we have the most interaction with, like for drinking or recreation. 

% Landsat
The Landsat project has been monitoring the earth for over four decades, being the longest uninterrupted data set available. The Landsat satellites' main mission is to image the land areas of the earth and therefore there are typically no open ocean (case 1 water) images available. This is the reason why Landsat satellites have been underestimated by the ocean color community for the study of water bodies. In addition, the Landsat instruments have generally had broad bands and low SNR when compared to heritage ocean color satellites such as SeaWiFS and MODIS. Carrying two instruments onboard, the Operational Land Imager (OLI) and the Thermal InfraRed Scanner (TIRS), Landsat 8 is the first of a new generation of Landsat satellite with state-of-the-art technology. With its 12-bit quantization and improved signal-to-noise ratio (SNR), OLI is a big improvement to the Landsat mission. In addition, OLI includes a new coastal band that increased the spectral resolution of the instrument. These improvements are the main drivers to hypothesize that the Landsat 8 satellite will definitely have a better performance in water quality studies than its predecessors. Since its launch in 2013, the Operational Land Imager (OLI) instrument onboard Landsat 8 has created high expectations in the ocean color community. Its spatial resolution of $30m$ and its improved signal-to-noise ratio (SNR) compared with its predecessors make Landsat 8 a perfect candidate to be used in coastal and inland water studies. Therefore, the overall objective of this research is to demonstrate that the new generation of Landsat satellites are capable of accurately retrieving water constituents.

The retrieval of water components is in general performed in the reflectance domain, so the very first step in this work is to perform a high quality atmospheric correction to the radiance image from Landsat 8. This is a complex task to perform over water because the signal leaving the water that reaches the sensor is very small when compared to the signal reaching the sensor produced by atmospheric scattering. Most of the atmospheric correction algorithms applied to Ocean Color satellites (e.g. SeaWiFS and MODIS) are not suitable for highly turbid coastal water \cite{Patt2003}. In this work, different approaches for atmospheric correction will be investigated. The first atmospheric correction algorithm investigated is the model-based empirical line method (MOB-ELM) that uses a combination of an in-water radiative transfer model over water and a Landsat reflectance product to determine the bright and dark pixels in the image. The second one is the standard SeaWiFS/MODIS atmospheric correction algorithm that uses SWIR bands (\cite{Wang:2007}). The water-leaving reflectance values obtained after atmospheric correction are validated by comparison with water surface reflectance measured in situ. 

After having corrected the image, the next step is to apply a retrieval algorithm that outputs water component retrieval maps of the main water components (chlorophyll, {\todo{or SM?} sediment} and colored dissolved organic matter (CDOM)). A spectral matching and look-up table (LUT) approach is utilized. It uses a least square error minimization algorithm to find the best match for a specific reflectance signal in a LUT of spectral water-leaving reflectance curves. The LUT is created using HydroLight 5, an in-water radiative transfer model (\cite{Mobley:2005}). Each curve in the LUT has a specific set of water component concentrations. This is performed on a pixel-by-pixel basis. The concentration values obtained from the retrieval algorithm are validated by comparison with concentration measured in the lab from water bodies present in the Landsat 8 image.

In order to have outputs that are representative of the water bodies that are being studied, Inherent Optical Properties (IOPs) of those specific waters have to be input to the HydroLight model. To accomplish this, collections of water samples were conducted at the same time that the Landsat 8 satellite passed over the area of study (Rochester, NY). After collection, these water samples were analyzed in the lab to obtain IOPs for the main water constituents. Furthermore, apparent optical properties (AOPs) and backscattering measurements were also collected for further comparison and to pursue closure between HydroLight AOPs results and in-situ AOPs measurements.
% !TEX root=ProposalJavier.tex 
% the previous is to reference main .bib
\chapter{Objectives}
\label{ch:objectives}

As discussed in Chapter \ref{ch:introduction},  the retrieval of water constituent concentration using multispectral satellite imagery in a effort to monitor fresh and coastal water (referred to as Case 2 waters) is a complex problem because there is not a direct relationship between pixel values and water constituent concentration. However, since this problem possesses different links that depend on each other, it can be addressed in smaller tasks to make it easier to solve. 

The purpose of this chapter is precisely to define these tasks as an outline that will help to make the problem manageable. This chapter is divided in four sections in an effort to describe each of these tasks. Section \ref{sec:problemstatement} details the problem being approached. In Section \ref{sec:objectives}, the problem is outlined in {\color{red} three} separate objectives as well as some future objectives. Section \ref{sec:tasks} describes the tasks needed to accomplish these objectives. Finally, this chapter closes with Section \ref{sec:contributiontofield} that delineates this work's original contribution to the field of remote sensing, imaging science and ocean optics. 
% -----------------------------------------------------------------
\section{Problem Statement}
\label{sec:problemstatement}
The hypothesis addressed in this thesis is the following: 

{ \bf ``The Landsat-8 sensor can be utilized to simultaneously quantify the concentration of the water color producing agents (CPAs) (specifically chlorophyll-{\it a}, {\todo{or minerals} sediment}, and colored dissolved organic matter) in fresh and coastal waters.''} 

This leads to the goal of our work: to develop a process to retrieve water constituents (CPAs) from Landsat-8 imagery to evaluate this satellite performance. Specifically, the algorithm will be used over Case 2 water, which includes fresh and coastal water. The retrieval algorithm compares a water leaving reflectance with unknown concentrations to water leaving reflectance whose concentrations are known. Because the comparison is made in the reflectance domain, the process first requires atmospherically correcting the Landsat-8 image and two approaches are investigated to do so. The first one is an ELM-based algorithm while the second one is the standard SeaWiFS/MODIS algorithm using the shortwave infrared (SWIR) bands.

% -----------------------------------------------------------------
\section{Statement of Objectives}
\label{sec:objectives}
The successful completion of this research effort will be marked by completion of the following primary requirements. Future objectives will be addressed if time permits.

\subsection{Primary Requirements:}
\begin{enumerate}
	\item Develop over-water atmospheric correction algorithms for Landsat-8 reflective imagery.
	\item Design a water constituent concentration retrieval algorithm that can be applied to a specific study area.
	\item {\todo{Check with Dr. Schott} Include a glint correction.} 
	\item Validate results by comparing with in-situ measurements and products from ocean color satellites.
	\item Demo this process to a second study site.
\end{enumerate}

\subsection{Future Objectives:}
\begin{enumerate}
	\item Integration with Hydrodynamics models \todo{continue numeration} .
	\item Make the processes and algorithms more user friendly 
\end{enumerate}
% -----------------------------------------------------------------
\section{Description of Tasks}
\label{sec:tasks}

\subsection{Primary Requirements:}
\begin{enumerate} 
	{\bf \item Develop over-water atmospheric correction algorithms for Landsat-8 reflective imagery.} 

The first objective in this research is to identify the best approach to atmospherically correct the type of dataset provided by the OLI sensor. Two methods will be investigated. The first method will be based on previous work done on simulated OLI data \cite{Gerace:2013,Gerace:2012,GeraceThesis,Pahlevan:2012} that consists of an ELM-based method that combines the Landsat reflectance product (Landsat Surface Reflectance CDR;\cite{LandsatCDR}) and a physics-based numerical model for water (HydroLight) to determine both the bright and dark pixel reflectance. The second method will be based on methods developed for ocean color satellite such as SeaWiFS, MODIS, and MERIS \cite{Gordon:1997}. These methods are based on the fact that the signal leaving the water does not contribute to the overall signal beyond the NIR part of the spectrum of light (black pixel assumption), so the signal reaching the sensor is caused only by atmospheric scattering. This concept can be expanded to the SWIR bands when the black pixel assumption is not valid in the NIR bands, which is the case for Case 2 and high productive Case 1 waters (\cite{Wang:2007}).

	{\bf \item Design a water constituent concentration retrieval algorithm that can be applied to a specific study area.}

The retrieval algorithm is based on previous work done by \cite{Raqueno:2003} and \cite{GeraceThesis}. The water-leaving reflectance product obtained after atmospheric correction from the previous stage is used as input to the retrieval algorithm. Each pixel of reflectance product has an unknown concentration. A spectral matching technique is applied to predict this concentration by comparing the spectral shape of each pixel with the elements in a LUT. The spectral matching is made by a least square error minimization along with a trilinear interpolation. This utilizes a non-linear optimization code provided in the Optimization Toolbox of the MATLAB software \todo{Name Matlab package}. The output of this process is a concentration mapping for each water constituent.

The LUT is generated using the ``Case 2'' \todo{check this word} algorithm in HydroLight for different triplets of water constituent concentrations. In order to generate congruent result from HydroLight, the user needs to input IOPs characteristic of the water bodies to be studied. Consequently, IOPs measured spectrophotometrically in the lab from water samples from the field are used as input to HydroLight along with backscattering measurements in the field. These measurements were collected when the Landsat-8 sensor passed over the area of study.

The area of study is the Lake Ontario Rochester Embayment that includes some nearby ponds (Long and Cranberry ponds), the Genesee River plume, the Irondequoit bay and part of Lake Ontario. This area was selected because it exhibits a wide range of variability in concentration of water constituents, so the retrieval algorithm can be tested against a wide range of water conditions.
 
	{\bf \item Validate results by comparing with in-situ measurements and products from ocean color satellites.}

The results from the retrieval process are validated by comparison with measurements taken from the water bodies being studied. Before this process, the measurements needed to be validated with measurements analyzed by a credible lab (Monroe County Environmental Laboratory). This comparison with this lab shows agreement between the measurements. 

For further validation, the results will be compared with products derived from ocean color satellites such as MODIS (e.g. MODIS Chl{\it a} product) if possible.

	{\bf \item {\todo{Check with Dr. Schott} Demo this process to a second study site}.}

After validation of the retrieval algorithm over the study area, the next step would be to make it applicable to a second study site. To do so, a more general LUT would be created with elements representative of the different water bodies present in both study sites.


\end{enumerate}


% \subsection{Primary Requirements}
\subsection{Future Objectives}
	\begin{enumerate}
			{\bf \item Integration with hydrodynamics models. \todo{continue numeration} } 

The next step would be to use the validated results from the retrieval process for training hydrodynamics models to predicts future behavior of the water bodies. This would be based on previous work made by \cite{Pahlevan:2012} and \cite{GeraceThesis}, who used concentration maps obtained from the retrieval process using satellite imagery to train hydrodynamic models. For example, the hydrodynamic model would allow us to monitor the dynamics of coastal/inland waters near river discharges. The maps of water constituent concentrations on the surface can be used to feed into the hydrodynamic models in order to calibrate them. 

			{\bf \item Make the processes and algorithms more user friendly} 

The retrieval process described here requires integration of different modules from different software and use of different programming languages. The next step would be to create a graphical user interface (GUI) in Python to make the process more user friendly, so that anyone with basic remote sensing knowledge could use the methods describe in this thesis. We suggest Python because it is a free platform.

	\end{enumerate}	

		

% -----------------------------------------------------------------
\section{Contribution to Field}
\label{sec:contributiontofield}
This research will make several contributions to the field of remote sensing.

First, one important contribution is to demonstrate that Landsat satellites, which have been historically underestimated for the use of water quality measurements, could have a good performance in the estimation of water constituent concentrations.

Second, Landsat-8 was just launched in February 2013 and therefore there are few studies done about its performance so far, specially in its applications related to water assessments. Hence, this is the perfect time to investigate how its new upgrades will improve/impact our capability of retrieving water parameters. Therefore, this research will present one of the first results of Landsat-8 performance over water studies.  

Third, while there are other global water constituent concentrations products, Landsat provides a unique combination of temporal (16 days repeat cycle) and spatial resolution (30 m pixel size). Most of the retrieval algorithms available in the literature use ocean color satellites (e.g. SeaWiFS), which have spatial resolution of about $1 km$ to $250 m$, but with products with \todo{check resolution}$4km$ spatial resolution (e.g. MODIS Chlorophyll-{\it a} product). Even though this resolution is suitable for large scale studies, they fail to cover small scale studies (less than 100 m). On the other hand, high spatial resolution sensors carried on aircraft (e.g. AVIRIS) or even satellite (e.g. WorldView-2) although they can be used for small scale studies, their imagery tends to be expensive or not frequently available. Here is where Landsat-8 has the potential of filling that gap because its spatial resolution (30 m) could allow study of medium size targets, a river plume, for instance, and it is free to the international scientific community.

This research also contributes to the field of remote sensing by developing a novel approach to correct the atmospheric effect in Landsat-8 images over Case 2 waters via two different approaches. In spite of the fact that the ELM method is widely used to correct satellite image, it needs measurements in the field that are not always available. We developed an algorithm that overcomes this issue by estimating these measurements. Additionally, the {\todo{correct name, is it band ratio?} \color{red} second method} tries to benefit from the concepts behind the methods largely developed and used to atmospherically correct ocean optics sensors.

Finally, a large dataset is made available for potential water quality studies through this research. Landsat-8 collects images all around the world where there is land including fresh and coastal waters. Such wide-reaching temporal and spatial coverage is not being broadly exploited for water quality studies.

The background material necessary to attain these goals is described in the following chapter.

%% CHAPTER
\chapter{Background and Theory}
\section{Water Quality Parameters Retrieval}
\cite{Jensen}
\cite{Mustard2001}
\subsection{Governing Equation}

\hl{The comparison is performed in the reflectance domain.}

\hl{CHL, SM and CDOM explanation}

From the total energy coming from the sun, only approximately 1390 $\left[\frac{W}{m^2}\right]$ reaches the Earth's atmosphere \cite{Schott}. This integrated value is known as the \emph{exoatmospheric irradiance}, or $E_S'$, and represents the total energy per unit area just outside the Earth's atmosphere due to the solar energy. Recall that \emph{irradiance} is the rate at which the radiant flux ($\Phi$) is delivered to a surface ($A$), defined as

\begin{equation} \label{eq:irradiance}
E = \frac{d\Phi}{dA}   \indent   \indent  \left[\frac{W}{m^2}\right]  
\end{equation} 

So, the $E_S'$ is calculated assuming that the flux $\Phi$ comes from a point source at the center of the sun such that it would produce an existence at the sun's surface, producing a flux at the mean Earth-sun distance of 1390 $\left[\frac{W}{m^2}\right]$ . For the present work, it is more convenient to express the irradiance spectrally, or other words as a function of wavelength, so we can describe the energy at desired wavelength, or spectral band.

The light source, usually the sun, interact with the target and then reach the sensor. This interaction will help us to extract about the target, in this case the water body. That is why in order to understand how the water quality parameter are retrieved, first it is necessary to introduce the concept of sensor reaching radiance. The sensor reaching radiance is defined as the accumulation of photons at the front of a sensor that one wishes to collect in an effort to obtain information about the target \cite{Gerace}. The total sensor-reaching radiance is the sum of the radiances due to the individual solar and thermal paths.  \cite{Schott} shows that in the VNIR region (approximately 0.3-2.5 [$\mu m$]), the solar energy is so many orders of the magnitude higher than the self-emitted energy, so the thermal paths are negligible for this study. Also, we consider that radiance from the background is negligible because the water bodies are typically several kilometers wide. Assuming that the target is approximately Lambertian (radiance is equal in all directions), the total sensor-reaching radiance, $L$, is defined as

\begin{equation} \label{eq:gov1}
L(\lambda) = \frac{E'_S(\lambda)cos(\sigma')r(\lambda)\tau_1(\lambda)\tau_2(\lambda)}{\pi} +
                        \frac{E_{ds}(\lambda)r(\lambda)\tau_2(\lambda)}{\pi} + L_{us}(\lambda)
\end{equation} 
where:
\begin{tabbing}
\indent \indent \indent  $L(\lambda)$ \hspace{1mm}\=:  \indent \= total sensor-reaching radiance\\
\indent \indent \indent  $E'_S(\lambda)$\>: \>exoatmospheric spectral irradiance\\
\indent \indent \indent $\sigma'$\>:\>solar-zenith angle\\
\indent \indent \indent $r(\lambda)$\>:\>spectral reflectance target\\
\indent \indent \indent $\tau_1(\lambda)$\>:\>Sun-target path transmission of atmosphere\\
\indent \indent \indent $\tau_2(\lambda)$\>:\>target-sensor path transmission of atmosphere\\
\indent \indent \indent $E_{ds}(\lambda)$\>:\>solar downwelling irradiance\\
\indent \indent \indent $L_{u}(\lambda)$\>:\>solar upwelling irradiance\\
\end{tabbing}

Equation \eqref{eq:gov1} is the solar term of the "big equation" described by \cite{Schott}.

\subsection{Constituent Retrieval}

When imaging water body, a set of new paths need to be taken in account to determine the sensor-reaching radiance.
 \begin{figure}[H]
	\centering
    	\includegraphics[width=100mm]{/Users/javier/Desktop/Javier/MASTER_RIT/2011_THESIS/LaTeX/Thesis/Images/WaterColumn.eps}
 	\caption{Contributions to sensor-reaching radiance from the water column \label{fig:WaterColumn}}
  \end{figure}


Each sensor-reaching radiance curve is associated with a specific combination of water components (CHL, SM and CDOM). HidroLight provides us the \hl{remote sensing reflectance}, $R_{RS}$, which describes how much of the total incident downwelling irradiance is ultimately returned from a water column in a given direction, defined as

\begin{equation} \label{eq:Rrs}
R_{RS}(\theta,\phi,\lambda,z=a) = \frac{L(\theta,\phi,\lambda,z=a)}{E_d(\lambda,z=a)}   \indent   \indent  \left[\frac{1}{sr}\right]  
\end{equation} 
where:
\begin{tabbing}
\indent \indent \indent  $\theta$ \hspace{1.5mm}\=:  \indent \= sensor-zenith angle\\
\indent \indent \indent  $\phi$\>: \>sensor-azimuth angle\\
\indent \indent \indent $L$\>:\>water-leaving radiance\\
\indent \indent \indent $E_d$\>:\>total downwelling irradiance\\
\indent \indent \indent $\lambda$\>:\>wavelength dependent\\
\indent \indent \indent $a$\>:\>height just above the water's surface\\
\end{tabbing}



\section{Shallow Water Retrieval}
\section{LUT Method}
\subsection{Population}
\subsection{Optimization Algorithm  - lsqnonlin}

\section{State of the Research}

As shown in \cite{Gerace}
and \cite{Mobley} and \cite{Lesser}

\section{Atmospheric Compensation}
% !TEX root=ProposalJavier.tex 
% the previous is to reference main .bib
%% CHAPTER
\chapter{Methodology and Approach}
\label{ch:method}

The methodology is separated into the specific objectives mentioned in \S\ref{ch:objectives}. In this work, a look-up-table (LUT) methodology was implemented to retrieve concentration of water constituents using Landsat-8 imagery. Figure~\ref{fig:retrieval} shows a diagram of this retrieval process. First, the Landsat-8 image data (shown at the top of the figure) needs to be atmospherically corrected. Then, a non-linear optimization routine uses the water pixels (reflectance values) and a LUT of reflectance spectra to estimate concentrations for each water pixels in the scene. The outputs from the process are concentration maps for each water constituents, as shown in Figure~\ref{fig:retrieval}. This process is explained in detail in \S\ref{sec:atmcorr} and \S\ref{sec:retrieval} below.
\begin{figure}[htb]
  \centering
  \includegraphics[height=7cm]{/Users/javier/Desktop/Javier/PHD_RIT/ConferencesAndApplications/NESSF14/latex/Retrieval.pdf}
  \caption{Retrieval process diagram. \label{fig:retrieval} } 
\end{figure}

\section{Over-Water Atmospheric Correction} 
\label{sec:atmcorr}
The first objective in this research is to identify a suitable approach to atmospherically correct the type of dataset provided by the OLI sensor. This is a complex task to perform over water because the signal leaving the water that reaches the sensor is very small compared to the signal reaching the sensor from atmospheric scattering. Most of the atmospheric correction algorithms used for Ocean Color Satellites (i.e. CZCS, MODIS, SeaWiFS) are based in the work done by \cite{Gordon:1994} (also described in \cite{Gordon:1997}).

However, some of the atmospheric correction algorithms applied to Ocean Color Satellites are not suitable for highly turbid coastal waters because the {\it black pixel assumption} cannot be applied to these types of waters~\cite{Patt2003}. Several methods for atmospheric compensation are be investigated in this research.

Some atmospheric correction methods described in this section will use the notation used by \cite{Gordon:1994} and \cite{Ruddick:2000bs} who define the apparent reflectance as 

\begin{equation}\label{eq:rho}
  \rho = \frac{\pi L}{F_o \cos{\theta}}
\end{equation}
where $L$ is upward radiance in the given viewing direction, $F_o$ is exoatmospheric irradiance, and $\theta$ is the solar-zenith angle. \autoref{eq:rho} allows a direct transformation from radiance to reflectance and vice versa. Therefore, taking in account all its contributors, the governing equation for sensor-reaching reflectance can be expressed as

\begin{equation}\label{eq:rho_t}
  \rho_t(\lambda) = \rho_r(\lambda) + \rho_a(\lambda) + \rho_{ra}(\lambda) + T_v[\rho_w(\lambda) + \rho_{wc}(\lambda)]
\end{equation}
where:\\
\indent $\rho_t(\lambda)$ is the reflectance at the top of the atmosphere \\
\indent $\rho_r(\lambda)$ is the reflectance due to multiple scatter by air molecules only (Rayleigh scattering)\\
\indent $\rho_a(\lambda)$ is the reflectance due to multiple scatter by aerosols only\\
\indent $\rho_{ra}(\lambda)$ is the reflectance due to the interaction between Rayleigh and aerosol scattering\\
\indent $T_v(\lambda)$ is the diffuse atmospheric transmittance from the water to the sensor\\
\indent $\rho_{wc}(\lambda)$ is the reflectance due to solar photons reflecting off the air-water interface (from whitecaps and glint or glitter)\\
\indent $\rho_w(\lambda)$ is the water-leaving reflectance

In order to perform the water constituent retrieval, we need to solve for only the $\rho_w(\lambda)$ term in \autoref{eq:rho_t}, which would be an easy task if all the rest of the terms were known. Unfortunately, the only term known {\it a priori} is the $\rho_t(\lambda)$, which is precisely the image of the scene itself. The main difference among the methods based on \cite{Gordon:1994} is in the approach used to estimate $\rho_a(\lambda) + \rho_{ra}(\lambda)$ in the visible (VIS) using an estimation of $\rho_a(\lambda) + \rho_{ra}(\lambda)$ in the near infrared (NIR).

This section describes the different alternatives to obtain the rest of the terms in \autoref{eq:rho_t}.

\subsection{Solar-Glint Removal Algorithm}
\label{subsec:glintremoval}
The first step to solve \autoref{eq:rho_t} is to find the term due to glint, $\rho_{wc}(\lambda)$ and remove it from the total signal. This term $\rho_{wc}(\lambda)$ can be ignored in some cases since ocean-color sensors are designed to be tilted to avoid the specular image of the sun. However, Landsat-8 is not cataloged as an ocean-color satellite and it may need to be corrected for the glint effect depending on the location of the water in the scene.

The method to remove the sun glint effect chosen for this study is the method suggested by \cite{Hedley:2005}, which is a revised version of the method suggested by \cite{Hochberg:2003}. The method described by Hochberg was developed for high spatial resolutions where glint effects occur at physical scales comparable to image pixels ($<10m$) as opposed to methods developed for ocean-color sensor, which tend to have large physical scales ($>1km$).
The method presented by Hochberg is sensitive to outlier pixels and it needs to mask the land and cloud areas before deglinting. The method suggested by Hedley overcomes these inconveniences. This method assumes that (a) the brightness in the NIR is composed only of sun glint and a spatially constant ambient NIR component associated with NIR backscatter in the atmosphere (if the image is not atmospherically corrected), and (b) that the amount of sun glint in the visible bands is linearly related to the brightness (glint) in the NIR band. 

The first assumption is true for waters that are not highly turbid since water is relatively opaque to NIR wavelength ($700-1000nm$). The second assumption of a linear relationship between NIR brightness and the amount of sun glint in the visible bands relies on the fact that the real index of refraction, which is associated with the reflection in the water surface, is nearly equal for NIR and visible wavelengths (\cite{Mobley1994}). As a result, the amount of light reflected in the visible bands is proportional to the amount of light reflected in the NIR, and therefore, a linear relationship can be established among them. The first step is to establish this relationship. This is accomplished by solving a linear regression between NIR and visible bands using one or more ROIs from the water pixels in the image where some sun glint is noticeable, and their pixels values would be of similar values otherwise. An example of a ROI could be a ROI over deep water in the lake. A slope is determined from solving this linear regression with NIR pixel values in the ROI as the independent variable ($x$-axis) and a particular visible band as the dependent variable ($y$-axis), as shown in \autoref{fig:regressiohedley}. If the regression slope for band $i$ is $b_i$, then sun-glint corrected pixel brightness in band $i$ can be obtained by applying the following formula
\begin{equation}\label{eq:deglint}
  R_i' = R_i - b_i(R_{NIR}-min_{NIR})
\end{equation}
where $R_i'$ is the sun-glint corrected pixel value in band $i$, $R_i$ is the pixel value in the visible band $i$, $R_{NIR}$ is the NIR pixel value, and $min_{NIR}$ is the ambient NIR level and it represents the NIR brightness of a pixel with no sun glint. $min_{NIR}$ can be the minimum NIR value in the ROI used in the linear regression or as the minimum NIR in the water pixels. This method has the advantage that it operates purely on the relative magnitudes of values, and therefore the absolutes magnitudes are not important. The author suggests use of ROIs from different regions in the image, including ROIs where there is not glint at all. The different steps are summarized as follows.

\noindent{\bf Solar-Deglinting Algorithm Summary}\\
{\bf Step 1:} Select a ROI in the image where there is a range of sun glint, but where the brightness values would be uniform otherwise.\\
{\bf Step 2:} Determine $min_{NIR}$ by selecting the minimum NIR value of ROI.\\
{\bf Step 3:} Determine the slope $b_i$ from a linear regression between the NIR values ($x$-axis) and the visible band $i$ to be deglinted ($y$-axis).\\
{\bf Step 4:} Deglint all pixels in the image using \autoref{eq:deglint}.\\
{\bf Step 5:} Repeat step 1-4 for each band to be deglinted.

\begin{figure}[!ht]
  \centering
  \includegraphics[width=11cm,clip=true]{/Users/javier/Desktop/Javier/PHD_RIT/Latex/Proposal/Images/RegressionHedley.png}
  \caption{Linear regression used in the deglinting process (Note: image taken from \cite{Hedley:2005}). \label{fig:regressiohedley} } 
\end{figure}

Some considerations need to be taken into account in order to apply this method. The first assumption is only valid when there is no water with high concentration of sediment present in the image, so precautions have to be taken to avoid highly turbid water in the image. As a way to overcome this problem, the OLI's SWIR band (band 6) will be used instead of the NIR band since the water-leaving signal is negligible at $1600nm$, as suggested by \cite{GeraceThesis}. The real part of the index of refraction is still nearly equal for all wavelengths in the VNIR/SWIR, and therefore the second assumption of a linear relationship between NIR values and the amount of sun glint in the visible band is still valid. Other consideration is that if the ROIs selected include non-submerged objects (i.e. buoys, boats, land), this algorithm could output negative values. Therefore, it is recommended to mask all non-submerged objects before applying this method.

\missingfigure{Example of deglinting}

The purpose of the deglinting process is to find the term $T_v\rho_{wc}$ in \autoref{eq:rho_t} and subtract it from the total TOA reflectance $\rho_t$,

\begin{equation}\label{eq:rhodeglint}
  \rho_t(\lambda)-T_v(\lambda)\rho_{wc}(\lambda) = \rho_r(\lambda)+\rho_a(\lambda)+\rho_{ra}(\lambda)+T_v(\lambda)\rho_{w}(\lambda)
\end{equation}

Once water pixels in the image are deglinted, the next step is to atmospherically correct the image in order to isolate $\rho_w$ from \autoref{eq:rhodeglint}.



\subsection{Model-based ELM}
The first method will be a model-based empirical line method (ELM) based on previous work done by \cite{Gerace:2013} and \cite{Gerace:2012}  for simulated OLI data. While this new method is based on the traditional ELM method (see \S\ref{subsec:ELM}), this model-based ELM method tries to avoid the measurement of ground truth at every sensor overpass over the scene by using pseudo-invariant features (PIF) \index{pseudo-invariant features (PIF)} in the scene as one target along with an estimation of water reflectivity for an open lake region for the other target. The two targets used in this model-based ELM to solve the regression in \autoref{eq:ELM} are referred to in this documents as the {\it bright pixel} \index{bright pixel} and the {\it dark pixel}\index{dark pixel}.

\subsubsection{Pseudo-Invariant Feature Extraction}

This method employs a PIF pixel extraction \cite{Schott:1988} to mask urban landscape from both the reflectance product and the Landsat-8 image for the bright pixel determination. Pseudo-invariant targets are defined as targets whose reflectivity properties do not change rapidly between different times of collection. Examples of pseudo-invariant target are urban features in the scene.  The PIF extraction isolates the pseudoinvariant features from the digital imagery. In our case, the PIF are the man-made urban features in a scene. A flowchart of the process is shown in \autoref{fig:PIFflowchart}. 

\begin{figure}[htb]
	\centering
  \begin{tikzpicture}[node distance=0.75cm, auto]
          \tikzset{
                  basenode/.style={rectangle,rounded corners,draw=black,very thick, inner sep=1em, minimum size=3em, text centered,text width=2cm},
                  productnode/.style={ellipse,rounded corners,draw=black, very thick, text centered,text width=1.5cm},
                  myarrow/.style={->,>=stealth',thick, double = black},
                  mylabel/.style={text width=7em, text centered}
          }
          % SWIR branch
          \node[basenode] (SWIR) {SWIR 2\\ Band};
          \node[basenode, below=of SWIR] (TS1) {Mask by Threshold (upward)};
          \node[align=left, right=0.0 of TS1] (C1) {Urban\\Veget.\\Water};
          \node[align=left, right=-0.15 of C1] (C2) {ON\\ON\\OFF};

          % Ratio branch
          \node[basenode, right=2.5cm of SWIR] (Ratio) {Ratio\\ NIR Band/ Red Band};
          \node[basenode, below=of Ratio] (TS2) {Mask by Threshold (downward)};
          \node[align=left, right=0.0 of TS2] (C3) {Urban\\Veget.\\Water};
          \node[align=left, right=-0.15 of C3] (C4) {ON\\OFF\\ON};

          % AND
          \path (TS1.south)--(TS2.south) node[pos=.5,below=2cm] (AND) {.AND.};


          % PIF Mask
          \node[basenode, below=of AND] (PIFMask){PIF Mask};
          \node[align=left, left=0.85 of PIFMask] (C5) {Urban\\Veget.\\Water};
          \node[align=left, right=-0.15 of C5] (C6) {ON\\OFF\\OFF};

          \node[basenode, below=of TS2,right=2.0cm of AND] (Image) {Image};
          \path (Image.south)--(PIFMask.east) node[below=of Image,right=2cm of PIFMask] (AND2) {.AND.};
          \node[basenode, right=2cm of AND2] (PIFIm){PIF Image};

          \draw[myarrow] (SWIR)--(TS1);
          \draw[myarrow] (Ratio)--(TS2);
          \draw[myarrow] (TS1)--(AND);
          \draw[myarrow] (TS2)--(AND);
          \draw[myarrow] (AND)--(PIFMask);
          \draw[myarrow] (Image)--(AND2);
          \draw[myarrow] (PIFMask)--(AND2);
          \draw[myarrow] (AND2)--(PIFIm);
  \end{tikzpicture}
\caption{Illustration of the logic used to segment PIF features. \label{fig:PIFflowchart}}
\end{figure}

The PIF extraction process is illustrated in \autoref{fig:PIFflowchart}. The PIF extraction from digital imagery proceeds in the following fashion. An infrared-to-red ratio image is very effective in the classification of water, vegetation, and urban features. The vegetation in this ratio image will tend to have a high brightness when compared to the urban features and water brightness. This infrared-to-red ratio image can be derived from the quotient of the NIR band (band 4 for Landsat-5; band 5 for Landsat-8) and the red band (band 3 for Landsat-5,band 4 for Landsat-8), as seen in \autoref{fig:PIFflowchart}. This ratio image is thresholded from the high digital count values downward to create a mask so the high brightness pixels are eliminated (vegetation pixels) from the image, that is, these pixels are set to a value of zero and the rest (water and urban pixels) to a value of one. The SWIR 2 band (band 7 in Landsat-5 and Landsat-8) is used to eliminate the water pixels from the previous mask since water has nearly zero reflectance in this spectral region. This SWIR 2 band is thresholded from the low brightness values upward. Water pixels will exhibit a low value when compare to the rest of the pixels. A mask is created by assigning a value of zero to the low brightness pixels (water pixels) and a value of one to the rest (urban features and vegetation). Finally, the two masks created are combined using a logical .AND., resulting in a mask that will have a value of one only in the urban feature pixels, i.e. the PIFs, as shown in \autoref{fig:PIFflowchart}. This mask will be named ``PIF mask'' for the rest of this document. An example of a PIF mask is illustrated in \autoref{fig:PIFmask}. A false color image of Downtown Rochester, NY is shown on the left (vegetation in red) and a RGB image of the same area with the PIF mask applied is shown on the right (urban features in bright color while masked pixels in black).

% \vspace{-.3cm}
\begin{figure}[htb]
  \begin{minipage}[c]{0.48\linewidth}
    \centering
      \includegraphics[trim=30 0 30 0,clip,height=6cm]{/Users/javier/Desktop/Javier/PHD_RIT/Latex/Proposal/Images/DTROCL8falsecolor.jpg}  
    % \vspace{1.5cm}
    \centerline{(a)}\medskip
  \end{minipage}
  \hfill
  \begin{minipage}[d]{0.48\linewidth}
    \centering
      \includegraphics[trim=30 0 30 0,clip,height=6cm]{/Users/javier/Desktop/Javier/PHD_RIT/Latex/Proposal/Images/PIFmaskApplied.jpg}
    % \vspace{1.5cm}
    \centerline{(b)}\medskip
  \end{minipage}
  \caption{PIF mask determination. (a) False color image, with vegetation in red and (b) PIF mask over downtown Rochester. \label{fig:PIFmask} } 
\end{figure}

\subsubsection{Bright Pixel Determination}

The PIF mask is used to determine the bright pixel spectra in both radiance (from the Landsat-8 image) and reflectance values (from the Landsat Surface Reflectance CDR \cite{LandsatCDR} image). See \S\ref{sec:CDR} for more details about the Landsat reflectance product. The Landsat reflectance product was available for a total of 9 Landsat 5 scenes where clear sky conditions were acceptable. One PIF mask for each of these 9 Landsat reflectance product scenes was created using ENVI. In addition, one PIF mask was created from the Landsat-8 radiance image. Finally, these 10 PIF masks were combined using a logical .AND. to create a ``master'' PIF mask \index{master PIF mask} in order to only include the PIF pixels coincident in all images. This is necessary because there is a no perfect geometry registration among all images. Then, the statistics were calculated in ENVI for each scene using this master PIF mask. An example of the statistical results obtained from ENVI are shown in \autoref{fig:PIFstats}.(a) and \autoref{fig:PIFstats}.(b) for one scene of the Landsat reflectance product (in reflectance units) and for the Landsat-8 image (in radiance units), respectively. The mean value is shown in black solid line, the green solid lines are the mean plus standard deviation and the mean minus standard deviation, and the red solid lines are the maximum and minimum values for each band. The mean values for each one of the 9 scenes are shown in \autoref{fig:ZenithCorr}. 

\begin{figure}[!ht]
  \begin{minipage}[c]{0.48\linewidth}
    \centering
      \includegraphics[height=9cm]{/Users/javier/Desktop/Javier/PHD_RIT/Latex/Proposal/Images/PIFstatCDR.png}  
    % \vspace{1.5cm}
    \centerline{(a)}\medskip
  \end{minipage}
  \hfill
  \begin{minipage}[d]{0.48\linewidth}
    \centering
      \includegraphics[height=9cm]{/Users/javier/Desktop/Javier/PHD_RIT/Latex/Proposal/Images/PIFstatL8.png}
    % \vspace{1.5cm} 
    \centerline{(b)}\medskip
  \end{minipage}
  \caption{Bright pixel determination using the PIF mask in ENVI. Statistics with the PIF mask applied for (a) Landsat reflectance product (in reflectance units) and (b) statistics for Landsat-8 image (in radiances units). \label{fig:PIFstats} } 
\end{figure}

\subsubsection{Solar Zenith Correction}

As seen in \autoref{fig:ZenithCorr}, the PIF reflectance values for each scene are not the same, but a high correlation between the reflectance values and the solar zenith angle for each band was found. A linear relationship was determined for each band by applying a linear regression in MATLAB and the $R^2$ and root mean square error (RMSE) values were calculated as a way to measure this correlation. This linear relationship has the form 
\begin{equation}
	y = m*x + y_0
	\label{eq:linear}
\end{equation}
where $x$ represents the solar zenith angle and $m$ the reflectance value. \autoref{fig:Band1Corr} shows the reflectance values versus the solar zenith angle for band 1 for the 9 Landsat reflectance scenes and the calculated linear relationship. The values $m$ and $y_0$ found for all the bands are shown in \autoref{tab:ZenithCorr} along with the $R^2$ and RMSE values for each band. Note that the RMSE values in the visible are small. It can been seen in \autoref{tab:ZenithCorr} that the $R^2$ values are bigger than $0.9$ for all bands, which suggests there is a high correlation between the reflectance values and the solar zenith angle. As a conclusion, these results show that the reflectance values remain constant over time and depend of the solar zenith angle of the sensor. This is an expected behavior since intuitively the zenith angle influences the length of shadows in the scene, and the amount of shadow in the scene affects the brightness of the pixels, therefore the reflectance values. This is illustrated in \autoref{fig:shadow}, where two different solar zenith angles ($\theta_1$ and $\theta_2$) are shown, with $\theta_1<\theta_2$. A smaller zenith angle ($\theta_1$ in \autoref{fig:shadow}) means that the sun is positioned almost straight overhead, therefore there are smaller shadows from buildings in the scene. On the other hand, if the sun is closer to the horizon ($\theta_2$ in \autoref{fig:shadow}), i.e. larger zenith angle, the shadows produced by buildings will be bigger. From \autoref{fig:shadow}, one can intuitively conclude at least that the length of shadow is proportional to the zenith angle $\theta$, and consequently inversely proportional to $\cos{\theta}$.

\begin{figure}[!ht]
    \centering
    \includegraphics[height=13cm]{/Users/javier/Desktop/Javier/PHD_RIT/Latex/Proposal/Images/Shadow.png}
  \caption{Shadow size as function of zenith angle. \label{fig:shadow} } 
\end{figure}

The Landsat-8 image has associated a particular solar zenith angle. The previous linear relationships calculated will help to estimate the values for the reflectance value of the bright pixel for that particular solar zenith angle. For example, the solar zenith angle for the 09-19-13 Landsat-8 scene is equal to $45^\circ$, and therefore $x=45^\circ$ in \autoref{eq:linear}. The reflectance values for $x=45^\circ$ are shown in the last column of \autoref{tab:ZenithCorr} and plotted in \autoref{fig:ZenithCorr} as red asterisks.
%--------------------------------------
% \vspace{.5cm}
\begin{table}[htb]
\caption{ Zenith angle correction parameters. \label{tab:ZenithCorr} } 
\centering
\begin{tabular}{c|c|c|c|c|c} 
 \bfseries{Band n} & \bfseries{$m$}      & \bfseries{$y_0$}    & \bfseries{$R^2$}     & \bfseries{$RMSE$} & $y(x=45^\circ)$   \\ \hline \hline
 Band 1 & -0.000412 & 0.122631 & 0.937155 & 0.001705 &  0.1041\\
 Band 2 & -0.000634 & 0.147424 & 0.934344 & 0.002685 &  0.1189\\
 Band 3 & -0.000756 & 0.161421 & 0.976599 & 0.001869 &  0.1274\\
 Band 4 & -0.001316 & 0.220031 & 0.906946 & 0.006733 &  0.1608\\
 Band 5 & -0.001148 & 0.217231 & 0.903702 & 0.005984 &  0.1656\\
 Band 6 & -0.001159 & 0.206725 & 0.929626 & 0.005096 &  0.1546\\  
 \end{tabular}
\end{table}

\begin{figure}[htb]
  	\centering
  	\includegraphics[height=7cm]{/Users/javier/Desktop/Javier/PHD_RIT/Latex/Proposal/Images/ZenithCorrelation.eps}
  \caption{Correlation for band 1. \label{fig:Band1Corr} } 
\end{figure}

\begin{figure}[htb]
  	\centering
  	\includegraphics[height=7cm]{/Users/javier/Desktop/Javier/PHD_RIT/Latex/Proposal/Images/ZenithCorrection.eps}
  \caption{Bright pixel for 9 different scenes. \label{fig:ZenithCorr} } 
\end{figure}
Because the Landsat reflectance products was not available for Landsat-8 at the moment of writing this documents, it was necessary to estimate a theoretical reflectance value for the coastal band for Landsat 5 in order to match with the Landsat-8 bands. To accomplish this, it was assumed that the coastal band would exhibit a similar trend as the blue and green bands. Hence, a straight-line that passes through the blue and green band values was used to extrapolate the value of the coastal band, as seen in \autoref{fig:Extrapol}, where the estimation of this reflectance value for the coastal band is shown at $443 [nm]$ and the straight-line is shown as a black solid line. It is expected the Landsat reflectance product will be available for Landsat-8 in the future. Therefore, the Landsat-8 reflectance could be used directly, and the previous step would not be necessary. Finally, the reflectance spectra for the bright pixel is shown in \autoref{tab:brightref}. As was mentioned previously, the radiance spectra for the bright pixel is obtained by applying the master PIF mask to the Landsat-8 image (see \autoref{fig:PIFstats}.(b)).

\begin{table}[htb]
\caption{ Reflectance spectra for the bright pixel. \label{tab:brightref} } 
\centering
\begin{tabular}{l|c} 
 \bfseries{Band} & \bfseries{Reflectance values}\\ \hline \hline
 Band 1 (Coastal Band) &  0.0965 \\
 Band 2 (Blue Band) &  0.1039 \\
 Band 3 (Green Band) &  0.1186 \\
 Band 4 (Red Band) &  0.1270 \\
 Band 5 (NIR Band) &  0.1601 \\
 Band 6 (SWIR 1 Band) &  0.1650 \\ 
 Band 7 (SWIR 2 Band) &  0.1539 \\ 
 \end{tabular}
\end{table}

\begin{figure}[htb]
  	\centering
  	\includegraphics[height=7cm]{/Users/javier/Desktop/Javier/PHD_RIT/Latex/Proposal/Images/Extrapolation.eps}
  \caption{Extrapolation for the coastal band. \label{fig:Extrapol} } 
\end{figure}

\subsubsection{Black Pixel Determination}
\label{subsubsec:blackpixel}

The reflectance spectra for the dark pixel is obtained from Ecolight, which is a version of Hydrolight that runs faster because it only calculates the radiance for the nadir angle and not in all directions (\cite{MobleyHEtech}). This Ecolight run represents a ROI present in the Landsat-8 radiance image. IOPs and concentrations measurements taken in the field from the same ROI are input to Ecolight. The Case 2 model in Ecolight is used to generate a remote sensing reflectance ($R_{rs}$). This model is a generic four-component (pure water, chlorophyll-bearing particles, CDOM, and mineral particles) IOP model \cite{MobleyHEtech}. The Case 2 model in Ecolight requires us to specify the IOPs of each component one at a time. This includes concentration, absorption and scattering coefficient spectra and phase function for each component. The IOPs for each component provided to Ecolight are used to generate the reflectance spectra for the dark pixel are explained below.

\autoref{tab:ONTNSconc} shows the constituent concentration for two different water samples from the data collection on September, 19th, 2013. These water samples were collected from the nearshore of Lake Ontario (labeled as ONTNS) and from the southern part of Long Pond (labeled as LONGS), and they represent two scenarios with totally different characteristics. The water sample ONTNS was used to generate the reflectance spectra for the dark pixel in Ecolight. The concentrations for each component were set to be constant with depth with the values shown in \autoref{tab:ONTNSconc}. 
\vspace{.5cm}
\begin{table}[!ht]
\caption{ Water samples concentration for the September, 19th, 2013 collections. \label{tab:ONTNSconc} } 
\centering
\begin{tabular}{c|c|c|c} 
 \bfseries{Sample} & \bfseries{$X_{Chl}$} & \bfseries{$a(\lambda_0=440)$}& \bfseries{$X_{SM}$}\\
 & $[\mu g/L]$ & $[1/m]$ & $[mg/L]$ \\ \hline \hline
ONTNS & 0.48 & 0.1151 & 1.6\\ 
LONGS & 112.76 & 1.1953 & 46.0\\ 
 \end{tabular}
\end{table}

The absorption properties for the component chlorophyll were input by user-supplied data files containing mass-specific absorption coefficient as a function of wavelength. This mass-specific absorption spectra is shown in \autoref{fig:CHLaast}. This data was obtained from lab measurements of absorption coefficient spectra for the water sample ONTNS with a {\todo{describe method} spectrophotometric method}. The spectrophotometric method yields absorption coefficients, which are converted to mass-specific absorption coefficient by dividing the absorption coefficient spectra by the concentration. This chlorophyll concentration was determined in lab by a {\todo{describe method} spectrophotometric method} as well. For the chlorophyll scattering properties, the same mass-specific scattering coefficient data used in \cite{Raqueno:2000} and \cite{Raqueno:2003} were utilized. This data is shown in \autoref{fig:CHLbast}. A Fournier-Forand (FF) phase function with backscatter fraction 0.010 was selected as the phase function for the chlorophyll. The details about the selection of this phase function will be described {\todo{correct?!} below}.
\begin{figure}[htb]
  	\centering
  	\includegraphics[height=7cm]{/Users/javier/Desktop/Javier/PHD_RIT/Latex/Proposal/Images/CHLaastJavier.eps}
  \caption{Chlorophyll mass-specific absorption spectra used for the determination of the reflectance spectra of the dark pixel in HydroLight. \label{fig:CHLaast} } 
\end{figure}


\begin{figure}[htb]
  	\centering
  	\includegraphics[height=7cm]{/Users/javier/Desktop/Javier/PHD_RIT/Latex/Proposal/Images/CHLbastRolo.eps}
  \caption{Chlorophyll mass-specific scattering spectra used for the determination of the reflectance spectra of the dark pixel in HydroLight. \label{fig:CHLbast} } 
\end{figure}

For the CDOM component, the absorption specification was the following. First, the absorption coefficients were determined by spectrophotometric measurements in lab for the ONTNS water sample. This data is shown in \autoref{fig:CDOMa}. Then, the data were normalized by the absorption value $a(\lambda_0)=0.1151[1/m]$ at the reference wavelength $\lambda_0=440nm$, so that $a^*(\lambda_0)=1$. This normalized data is shown in \autoref{fig:CDOMaast} as purple dots. An exponential curve with the following equation
\begin{equation}
	\label{eq:CDOMabs}
	a^*(\lambda)=a^*(\lambda_0)\exp{\left[-\gamma(\lambda-\lambda_0)\right]}
\end{equation}
was fitted to the normalized data. It was determined that the decay constant $\gamma=0.0126$. The fitted curve is illustrated in \autoref{fig:CDOMaast} in solid line. The parameters of this fitted curve are input in Ecolight to specify the CDOM specific absorption $a^*$, and $a(\lambda_0)=0.1151[1/m]$ to specify the dependence of the CDOM absorption at a reference wavelength.

\begin{figure}[!ht]
  	\centering
  	\includegraphics[height=7cm]{/Users/javier/Desktop/Javier/PHD_RIT/Latex/Proposal/Images/ONTNS_CDOMabs.eps}
  \caption{CDOM absorption coefficient spectra used for the determination of the reflectance spectra of the dark pixel in HydroLight. \label{fig:CDOMa} } 
\end{figure}

\begin{figure}[!ht]
  	\centering
  	\includegraphics[height=7cm]{/Users/javier/Desktop/Javier/PHD_RIT/Latex/Proposal/Images/ONTNS_CDOMfitting.eps}
  \caption{CDOM mass-specific absorption spectra used for the determination of the reflectance spectra of the dark pixel in HydroLight. \label{fig:CDOMaast} }
\end{figure}

The mineral mass-specific absorption coefficient was determined in the same fashion as for the chlorophyll mass-specific absorption coefficient from lab measurements of the ONTNS water sample. The mineral mass-specific absorption coefficient are shown in \autoref{fig:SMaast}. The mineral mass-specific scattering coefficient was the same used by \cite{Raqueno:2000} and \cite{Raqueno:2003} and are shown in \autoref{fig:SMbast}.

\begin{figure}[!ht]
  	\centering
  	\includegraphics[height=7cm]{/Users/javier/Desktop/Javier/PHD_RIT/Latex/Proposal/Images/SMaastJavier.eps}
  \caption{Mineral mass-specific absorption spectra used for the determination of the reflectance spectra of the dark pixel in HydroLight. \label{fig:SMaast} } 
  % \vspace{0.5cm}
\end{figure}

\begin{figure}[!ht]
  	\centering
  	\includegraphics[height=7cm]{/Users/javier/Desktop/Javier/PHD_RIT/Latex/Proposal/Images/SMbastRolo.eps}
  \caption{Mineral mass-specific scattering spectra used for the determination of the reflectance spectra of the dark pixel in HydroLight. \label{fig:SMbast} } 
\end{figure}

\todo{update with new Rrs measurement}The following approach was used to determine phase function for both the chlorophyll and mineral particle components. Ecolight was run several times with the different phase function from the library of discretized Fournier-Forand phase functions files supplied with Ecolight 5.2 to created a LUT of reflectance spectra, but maintaining the rest of the parameters the same. These parameters correspond to the ONTNS water sample. The different reflectance spectra generated as output were compared with the reflectance measured in situ. The best match was determining by choosing the lowest root mean squared error (RMSE) between the reflectance measured in situ and the simulated reflectance spectra.

It was determined that the best matched corresponds to the discretized Fournier-Forand phase function with a backscatter equal to $0.010$, i.e. $1\%$ of backscatter fraction (FFbb010.dpf). Therefore, this discretized Fournier-Forand phase function is used for both the chlorophyll and mineral particle. \autoref{fig:BestMatchONTNS} illustrated the reflectance measured in situ (red solid line) and the best match reflectance from the LUT (blue solid line), generated with a discretized Fournier-Forand phase function of 0.010 backscatter fraction. It can be seen in \autoref{fig:BestMatchONTNS} that both spectra agree in values above $550nm$, but not below this wavelength. This suggests that there is still need for improvement in the determination of the phase function and rest of the parameters in the Ecolight model.

The following parameters were input to Ecolight in order to simulate the Landsat acquisition conditions. The illumination conditions were input to Ecolight by specifying the solar zenith angle and day of the year that matched the Landsat-8 image. Internal sources and inelastic scatter were not included in the simulations. The wavelength range was $[400nm,1000nm]$, with a $1nm$ step. Default values for the air-water surface conditions were used, with a windspeed equal to $5m/s$, a real index of refraction equal to $n=1.34$, and the semi-empirical sky model (based on RADTRAN-X). Recall that Hydrolight uses a Cox-Munk air-water surface model that parameterizes gravity and capillary waves via the wind speed \todo{reference to Cox-Munk}. Finally, The bottom boundary condition used was ``the water column is infinitely deep.''

This best matched is used as the reflectance spectra of the dark pixel in the model-based ELM method. This reflectance is further spectrally sampled to the Landsat-8 response. 

\begin{figure}[htb]
  	\centering
  	\includegraphics[height=7cm]{/Users/javier/Desktop/Javier/PHD_RIT/Latex/Proposal/Images/RefWithFFbbONTNS.eps}
  \caption{Reflectance for ONTNS sample and best matching from HydroLight. \label{fig:BestMatchONTNS} } 
\end{figure}

The radiance spectra for the dark pixel is obtained from a ROI in the water present in the Landsat-8 image that could be considered a dark region (i.e. open lake). Statistics are computed in this dark region, and the mean values in each band is used as radiance spectra for the dark pixel.

\missingfigure{figure showing and ROI in the Landsat-8 image and statistic in ENVI}

As a review, \autoref{fig:ELMpxsENVI} shows the different spectra used to perform the model-based ELM, where four different spectra can be seen: one reflectance and one radiance spectra for the bright pixel (obtained using the PIF extraction over the Landsat reflectance product and Landsat-8 image, respectively), one reflectance spectra for the dark pixel (obtained from HydroLight), and one radiance spectra for bright pixel (obtained from the statistics of a ROI over water in the Landsat-8 image). These spectra are used to atmospherically correct the Landsat-8 image using the ENVI Classic software \cite{ENVIUserGuide}. This is performed using the ``Empirical Line'' algorithm of the ``Calibration Utilities'' in ENVI classic, where the Landsat-8 image is used as the input image, and the reflectance spectra are labeled as ``field spectra'' and the radiance spectra are labeled as ``data spectra.'' The product of this process is an image in reflectance values, which will be used to perform the retrieval of water constituents described in \S\ref{sec:retrieval} below. Note that the Landsat-8 image used was not glint corrected.

\begin{figure}[htb]
  \centering
  \includegraphics[width=14cm,clip=true]{/Users/javier/Desktop/Javier/PHD_RIT/Latex/Proposal/Images/ELMpixelsENVI.pdf}
  \caption{Bright and Dark pixels used in ENVI to apply ELM. \label{fig:ELMpxsENVI} } 
  % \vspace{0.5cm}
\end{figure}
\todo{include summary of model-based ELM}

% ----------------------------
\subsection{SeaWiFS Algorithm for Case 1 Waters}
\label{subsec:gordon}
The following algorithm is based in the method developed by \cite{Gordon:1994} for retrieval of water-leaving radiance and aerosol optical thickness over the oceans with SeaWiFS. The method develop by \cite{Gordon:1994} is still applied to the basic SeaWiFS and MODIS atmospheric correction algorithms for Case 1 water (\cite{IOCCG:2010}).

The first step in this algorithm is performing a Rayleigh scattering correction, which means to subtract the reflectance due to Rayleigh scatter $\rho_r$ from the total TOA reflectance $\rho_t$ in \autoref{eq:rhodeglint}. In the SeaWiFS/MODIS algorithm, the Rayleigh scatter component is computed from the Rayleigh LUTs, which were calculated using the vector radiative transfer theory (\cite{Wang:1991,IOCCG:2010}). For this work, this Rayleigh scattering component $\rho_r$ can be calculated directly from MODTRAN for the particular illumination and viewing geometry of the sun and the sensor in multiple-scatter mode but without aerosol. In the SeaWiFS/MODIS atmospheric correction algorithm, the whitecap reflectance $\rho_{wc}(\lambda)$ is modelled using input of the sea surface wind speed, and the TOA sun glint component is mostly masked out (\cite{IOCCG:2010}). After calculating this Rayleigh scatter and whitecap component, \autoref{eq:rhodeglint} becomes 
\begin{equation}\label{eq:rhodeRayliegh}
 \rho_c(\lambda) = \rho_t(\lambda)-\rho_r(\lambda)-T_v(\lambda)\rho_{wc}(\lambda) = \rho_a(\lambda)+\rho_{ra}(\lambda)+T_v(\lambda)\rho_{w}(\lambda)
\end{equation}
where $\rho_c(\lambda)$ is the Rayleigh-corrected reflectance. 

If we define the total multiple-scattering aerosol reflectance $\rho_{am}$ as
\begin{equation}\label{eq:rhoam1}
  \rho_{am}(\lambda) = \rho_a(\lambda)+\rho_{ra}(\lambda)
\end{equation}
then \autoref{eq:rhodeRayliegh} becomes
\begin{equation}\label{eq:rhoam}
 \rho_c(\lambda) = \rho_{am}(\lambda) + T_v(\lambda)\rho_{w}(\lambda)
\end{equation}

The following approach is taken in the SeaWiFS/MODIS atmospheric correction algorithm to retrieve $\rho_w$. For Case 1 water (e.g. open ocean), the contribution of the water $\rho_{w}$ to the total reflectance $\rho_t$ in the NIR is negligible. This fact is used to estimate $\rho_{am}$ in the NIR bands, and then these results are extrapolated to the visible bands using aerosol modeling. Finally, if $T_v$ is estimated, the $\rho_w$ in the visible bands can be computed.

There are two different approaches to determine the $\rho_{am}(\lambda)$ term at this point. The first one is to use a single-scattering approximation for $\rho_{am}(\lambda)$. The second one is to determine the multiple-scattering term $\rho_{am}(\lambda)$ based in the single scattering approximation, assuming that there exist a linear relationship between them. Both approaches are described below.

% -------------------
\subsubsection{Single Scattering}
\label{subsubsec:singlescat}
The aerosol is highly variable, and unlike the Rayleigh scattering component $\rho_r$, its effect in the total signal cannot be known {\it a priori} (\cite{Gordon:1994}). One of the first efforts to overcome this problem was developed for the CZCS atmospheric correction algorithm using a single-scattering approximation for calculating the aerosol effect in the total signal. Its logic is as follow. If the optical thickness of the atmospheric is considered $<<1$, then the term $\rho_a$ can be replaced by is single-scattering value $\rho_{as}$ (the $\rho_{ra}$ is ignored since it is a term related to multiple scattering). \autoref{eq:rhoam1} then becomes
\begin{equation}\label{eq:singleapprox}
  \rho_{am}(\lambda) = \rho_a(\lambda)+\cancel{\rho_{ra}(\lambda)} \approx \rho_{as}(\lambda)
\end{equation}
where
\begin{equation}\label{eq:rhoas}
  \begin{gathered}
    \rho_{as}(\lambda) = \frac{\omega_a(\lambda)\tau_a(\lambda)p_a(\theta,\theta_0,\lambda)}{4\cos(\theta)\cos(\theta_0)},\\  
    p_a(\theta,\theta_0,\lambda) = P_a(\theta_{-},\lambda) + [r(\theta)+r(\theta_0)]P_a(\theta_{+},\lambda),\\
    \cos(\theta_{\pm}) = \pm \cos(\theta_0)\cos(\theta)-\sin(\theta_0)\sin(\theta)\cos(\phi-\phi_0)
  \end{gathered}
\end{equation}
and $r(\lambda)$ is the Fresnel reflectance of the interface for an incident angle $\theta$, $\tau_a(\lambda)$ is the aerosol optical thickness, $\omega_a(\lambda)$ is the aerosol single-scattering albedo, $P_a(\alpha,\lambda)$ is the aerosol scattering phase function for a scattering angle $\alpha$, $\theta_0$ and $\phi_0$ are the zenith and azimuth angles from the target to the sun, respectively, and $\theta_0$ and $\phi_0$ are the zenith and azimuth angles from the target to the sensor, respectively. 


For Case 1 waters, the term $\rho_{w}$ is assumed to be zero for NIR bands. If we take the SeaWiFS case for band 7 ($\lambda_7=765nm$) and band 8 ($\lambda_8=865nm$), from \autoref{eq:rhoam}
\begin{equation}\label{eq:seawifsam7}
    \rho_{am}^{(7)} = \rho_{c}^{(7)} = \rho_{as}^{(7)},
\end{equation}
\begin{equation}\label{eq:seawifsam8}
    \rho_{am}^{(8)} = \rho_{c}^{(8)} = \rho_{as}^{(8)},
\end{equation}
where $\rho_{x}^{(i)}$ denotes a reflectance at band $i$ with wavelength $\lambda_i$. \autoref{eq:seawifsam7} and \autoref{eq:seawifsam8} mean that the aerosol reflectance term for band 7 and band 8 in SeaWiFS is equal to only the Rayleigh-corrected reflectance, which is known from the image. Now, we need to find a way to use this result to calculate the atmospheric reflectance for the rest of the bands.

\cite{Gordon:1994} define the atmospheric-correction parameter named single scattering epsilon (SSE)(\cite{IOCCG:2010}) $\varepsilon(\lambda_s,\lambda_l)$ as
\begin{equation}\label{eq:espilon}
  \varepsilon(\lambda_s,\lambda_l) \equiv \frac{\rho_{as}(\lambda_s)}{\rho_{as}(\lambda_l)} = \\
  \frac{\omega_a(\lambda_s)\tau_a(\lambda_s)p_a(\theta_v,\phi_v;\theta_0,\phi_0;\lambda_s)}{\omega_a(\lambda_l)\tau_a(\lambda_l)p_a(\theta_v,\phi_v;\theta_0,\phi_0;\lambda_l)}
\end{equation}
where the indexes ``$s$'' and ``$l$'' stand for short and long wavelength, associated with the NIR bands, i.e. $\lambda_s=756nm$ and $\lambda_l=865nm$ for SeaWiFS. If the value of $\varepsilon(\lambda_i,\lambda_l)$ for the band at $\lambda_i$ (visible bands) can be computed from $\varepsilon(\lambda_s,\lambda_l)$, then $\rho_{as}(\lambda_i)$ can be determined as
\begin{equation}\label{eq:rholambda_i}
  \rho_{as}(\lambda_i) = \varepsilon(\lambda_i,\lambda_l)\rho_{as}(\lambda_l),
\end{equation}

The next step is find a way to relate $\varepsilon(\lambda_i,\lambda_l)$ from $\varepsilon(\lambda_s,\lambda_l)$. \cite{Gordon:1994} tried to find a relationship by computing  $\varepsilon(\lambda_i,\lambda_l)$ for several aerosol models that were developed by \cite{Shettle:1979} for the LOWTRAN-6 model. \autoref{fig:epsilonvslambda} shows sample results for $\varepsilon(\lambda_i,\lambda_l)$ for SeaWiFS for these different aerosol models, where $\lambda_l=865nm$. It can be seen that over the range $412-865nm$, $\varepsilon(\lambda_i,\lambda_l)$ can be described as
\begin{equation}\label{eq:epsilonexp}
  \varepsilon(\lambda_i,\lambda_l) = \frac{\rho_{as}(\lambda_i)}{\rho_{as}(\lambda_l)} \approx exp[c(\lambda_i-\lambda_l)]
\end{equation}
where $c$ is a constant that depends on the viewing geometry and the aerosol model. 

\begin{figure}[htb]
  \centering
  \includegraphics[width=11cm,clip=true]{/Users/javier/Desktop/Javier/PHD_RIT/Latex/Proposal/Images/epsilonvslambdaGordon.png}
  \caption{$\varepsilon(\lambda,\lambda_l)$ values in natural logaritmic scale for different aerosol models and relative humidity (Note: image taken from \cite{Gordon:1997}). \label{fig:epsilonvslambda} } 
  % \vspace{0.5cm}
\end{figure}

The constant $c$ can be calculated using the known value $\varepsilon(\lambda_s,\lambda_l)$ for the NIR bands using \autoref{eq:seawifsam7} and \autoref{eq:seawifsam8}. If $c$ is known then \autoref{eq:rholambda_i} becomes
\begin{equation}\label{eq:rholambda_ifinal}
  \rho_{as}(\lambda_i) = exp[c(\lambda_i-\lambda_l)]\rho_{as}(\lambda_l),
\end{equation}

Once the aerosol single-scattering contribution is determined for the visible bands, $\rho_w$ can be calculated using \autoref{eq:rhoam}, but assuming $\rho_{am}(\lambda)\approx\rho_{as}(\lambda)$. 

The single-approximation is no longer an adequate approximation in cases where the aerosols are not at least moderately absorbing (i.e. strong continental influence) or $\tau_a(\lambda)$ is sufficiently large. Therefore, a full multiple-scattering approach is needed for a more general application.



%where $\rho_{as}^{(i)}$ for band $i$ is the single-scattering aerosol reflectance and $I=1..N$ denominates a set of $N$ simulated aerosol models obtained from, for example, an atmospheric radiative transfer model such as MODTRAN. This set of simulated aerosol models could be MODTRAN runs for different combination of particle size distribution (e.g. maritime, tropospheric, coastal, or urban), relative humidity, visibility, water vapor, etc., to simulate different atmosphere conditions that can occur in the scene.
% -------------------
\subsubsection{Multiple Scattering}
As mentioned  previously, we need to calculate this total multiple-scattering aerosol reflectance $\rho_{am}$ in order to obtain the desired water-leaving reflectance $\rho_w$. The multiple-scattering depends significantly on the aerosol model (\cite{Gordon:1997}). In the single-scattering approach previously described, the multiple-scattering was ignored and specific aerosol properties were not needed for the atmospheric correction. On the other hand, if we want to include the multiple-scattering effects in the atmospheric correction algorithm in order to obtain more accurate water reflectance $\rho_w$, it is necessary to utilize specific aerosol models. As a way to use the same reasoning used in the single-scattering algorithm, we can define multiple-scattering term as
\begin{equation}\label{eq:multscat}
  \rho_a(\lambda) + \rho_{ra}(\lambda) = K[\lambda,\rho_{as}(\lambda)]\rho_{as}(\lambda)
\end{equation}
where $K$ represents the relationship between the multiple-scattering and single-scattering. \cite{Wang:1991} has shown that a monotonic near-linear relation exists between $\rho_a(\lambda)+\rho_{ra}(\lambda)$ for the multiple-scattering model and $\rho_{as}(\lambda)$ for the single-scattering model. \cite{Gordon:1994} provided a LUT for $K[\lambda,\rho_{as}(\lambda)]$ by solving the radiative transfer equation (RTE) of a set of $N$ candidate aerosol models. These aerosol models were from, or derived from, the work of \cite{Shettle:1979}. The aerosol models are Oceanic, Maritime, Coastal and Tropospheric for different values of relative humidity (RH). The LUT was created for different sensor-sun geometry, single-scattering albedo and \AA ngstr\"{o}m exponent. The RTE is solved by using vector radiative transfer (\cite{IOCCG:2010}). For the SeaWiFS/MODIS algorithm, the $\rho_a(\lambda)+\rho_{ra}$ value for a given geometry is fit to a fourth order polynomial in the single-scattering aerosol reflectance $\rho_{as}(\lambda)$, i.e.,
\begin{equation}\label{eq:polynomial}
  \rho_a+\rho_{ra} = a\rho_{as}+b\rho_{as}^2+c\rho_{as}^3+d\rho_{as}^4
\end{equation}
where $a$, $b$, $c$ and $d$ are the constants contained in the LUT for a large number of viewing-sun geometries and for values of $\tau_a(\lambda)$ up to $0.8$.

Similar to the single-scattering approach, the assumption of negligible $\rho_w$ in the NIR allows us to determine the quantities $\rho_a(\lambda_s)+\rho_{ra}(\lambda_s)$ and $\rho_a(\lambda_l)+\rho_{ra}(\lambda_l)$ in the NIR. Once these quantities are determined from the sensor-measured values, the $\rho_{as}(\lambda)$ values for the NIR bands are estimated from these quantities using the LUT. Furthermore, because $\rho_{as}(\lambda)$ depends on aerosol phase function, single-scattering albedo, and optical thickness, this value can be computed for each aerosol model. Then, the single scattering epsilon (SSE)  values (described in \S\ref{subsubsec:singlescat}) for the NIR bands (i.e. $\varepsilon(\lambda_s,\lambda_l)$) can be calculated from the single-scattering reflectance values for each aerosol model and the measured values using \autoref{eq:espilon}. 

In order to determine multiple-scattering values in the VIS, the most appropriate aerosol models are selected by comparing the SSE computed from the sensor-measured values with the ones computed from each aerosol model. This is accomplished in the following fashion. After deriving $\varepsilon(\lambda_s,\lambda_l)$, the next step is to estimate $\varepsilon(\lambda_i,\lambda_l)$. $\varepsilon(\lambda_s,\lambda_l)$ mostly falls between those for two of the $N$ aerosol models. Therefore, $\varepsilon(\lambda_i,\lambda_l)$ is assumed to fall between the same two aerosol models proportionately in the same manner as $\varepsilon(\lambda_s,\lambda_l)$. Then, \autoref{eq:rholambda_i} is used to estimate single-scattering value in the rest of the bands from $\rho_{as}(\lambda_l)$. 

Finally, the LUT is used to transform single-scattering to multiple-scattering values to obtain $\rho_{am}(\lambda)$ and using \autoref{eq:rhoam} to obtain $\rho_w(\lambda)$. 

\missingfigure{Gordon and Wang method diagram}

A summary of the method developed by \cite{Gordon:1994} is described below.

\noindent {\bf SeaWiFS/MODIS Atmospheric Correction Summary:}
\begin{enumerate}[itemsep=2pt,parsep=2pt]
  \item Enter the atmospheric correction routine with Rayleigh-corrected reflectances $\rho_c(\lambda_s)$ and $\rho_c(\lambda_l)$.
  \item Assuming water-leaving reflectances for NIR bands are equal to zero, set multiple-scattering aerosol reflectances $\rho_{am}(\lambda_s)$ and $\rho_{am}(\lambda_l)$ equal to Rayleigh-corrected reflectances.
  \item Using the aerosol LUTs, calculate the corresponding single-scattering aerosol value at the two NIR bands, i.e. $\rho_{as}(\lambda_s)$ and $\rho_{as}(\lambda_l)$. Then, calculate the corresponding SSE.
  \item For each $N$ aerosol, compute the single-scattering value using \autoref{eq:rhoas}. Calculate corresponding SSE.
  \item Select the best two aerosol model by comparing the retrieved SSE with the theoretical SSE and determine the interpolation ratio between them.
  \item For the optimal aerosol model use the tabulated $\varepsilon(\lambda)$ in the VIS and $\varepsilon(\lambda_l)$ to obtain $\rho_{as}(\lambda)$ and then $\rho_{am}(\lambda)$ in the VIS using \autoref{eq:rholambda_i} and LUT.
  \item Remove $\rho_{am}(\lambda)$ in the VIS from $\rho_c(\lambda)$ and divide by the corresponding atmospheric transmittance that corresponds to the best aerosol model to return $\rho_w(\lambda)$ in the VIS using \autoref{eq:rhoam}.
\end{enumerate}

% ------------------------------
\subsection{SeaWiFS Algorithm for Case 2 Water}\label{subsec:ruddick}

The next atmospheric correction method is based in the method developed by Gordon~\cite{Gordon:1997} for ocean color satellites (i.e. SeaWiFS, MODIS), and extended by \cite{Ruddick:2000bs} for use over turbid coastal and inland waters (Case 2) or high productive Case 1 waters. As stated previously, the methods developed by \cite{Gordon:1994} assumes a zero water-leaving radiance for the NIR bands, which is not valid for highly turbid coastal and inland waters. Backscatter from particles in this waters could contribute to signal in the NIR bands, causing an over-estimation of the aerosol contribution and therefore an under-estimation of the water-leaving reflectances, even leading to negative values in the visible, which is not possible. 

In order to overcome this problem, \cite{Ruddick:2000bs} replaced the black water assumption with two assumptions: the assumption of spatial homogeneity of the SeaWiFS's NIR bands ratio ($765:865-nm$) for aerosol reflectance and for water-leaving reflectance. These two ratios are used as calibration parameters after inspection of the Rayleigh-corrected reflectance scatterplot. The algorithm is described in more details below.

In order to extend the standard atmospheric correction procedure to turbid water, two assumptions are used:

\begin{enumerate}[itemsep=2pt,parsep=2pt]
  \item At least over the ROI, the calibration parameter $\varepsilon_m(\lambda_s,\lambda_l)$, defined as the ratio of multiple-scattering aerosols and aerosol-Rayleigh reflectances at the NIR bands, is assumed to be spatially homogeneous, i.e.
  \begin{equation}\label{eq:rhohomo}
    \varepsilon_m(\lambda_s,\lambda_l)\equiv \frac{\rho_{am}(\lambda_s)}{\rho_{am}(\lambda_l)},
  \end{equation}
  and fixed for each image.

  \item The calibration parameter $\alpha$, defined as the ratio of water-leaving reflectances normalized by the sun-sea atmospheric transmittance at the NIR bands $T_0(\lambda)$, is assumed to be spatially homogeneous, i.e.
  \begin{equation}\label{eq:alpha}
    \alpha \equiv \frac{\rho_w(\lambda_s)/T_0(\lambda_s)}{\rho_w(\lambda_l)/T_0(\lambda_l)},
  \end{equation}
  and fixed for each image.
\end{enumerate}

By using these two assumption, the multiple-scattering aerosol and the aerosol-Rayleigh reflectance for the NIR bands can be derived as

\begin{equation}\label{eq:rhoams}
  \rho_{am}(\lambda_s) = \frac{\alpha\rho_c(\lambda_l)-\rho_c(\lambda_s)}{\alpha-\varepsilon_m(\lambda_s,\lambda_l)},
\end{equation}
and
\begin{equation}\label{eq:rhoaml}
  \rho_{am}(\lambda_l) = \varepsilon_m(\lambda_s,\lambda_l)\left[\frac{\alpha\rho_c(\lambda_l)-\rho_c(\lambda_s)}{\alpha-\varepsilon_m(\lambda_s,\lambda_l)}\right],
\end{equation}

Once these aerosol reflectances $\rho_{am}(\lambda_s)$ and $\rho_{am}(\lambda_l)$ are determined, they can be used in the standard algorithm, described in the previous section (instead of $\rho_c(\lambda_s)$ and $\rho_c(\lambda_l)$) to derive $\rho_w(\lambda)$ in the VIS.

\cite{Ruddick:2000bs} suggested to determine the calibration parameter $\varepsilon_m(\lambda_s,\lambda_l)$ by inspection of the scatterplot between the Rayleigh-corrected reflectance $\rho_c$ in the NIR bands, as shown in \autoref{fig:ruddickplot}. This assumption is only valid for small scale space, where the aerosol type varies only weakly in space. As for the calibration parameter $\alpha$, \cite{Ruddick:2000bs} set it to a default value of $1.72$, which is a first-order estimate from a greatly simplified ocean color model. The derivation can be seen in \cite{Ruddick:2000bs} and will not be explained here.

The process is summarized as follows:

\noindent {\bf SeaWiFS/MODIS Atmospheric Correction for Turbid Water Summary:}(\cite{Ruddick:2000bs})
\begin{enumerate}[itemsep=2pt,parsep=2pt]
  \item Enter the atmospheric correction routine to produce a scatter plot of Rayleigh-corrected reflectances $\rho_c(\lambda_s)$ and $\rho_c(\lambda_l)$) for the ROI of study, and select the calibration parameter $\varepsilon_m(\lambda_s,\lambda_l)$. Set calibration parameter $\alpha$ equal to 1.72.

  \item Reenter the atmospheric correction routine with $\rho_c(\lambda_s)$ and $\rho_c(\lambda_l)$ and use \autoref{eq:rhoams} and \autoref{eq:rhoams} to obtain $\rho_{am}(\lambda_s)$ and $\rho_{am}(\lambda_l)$, taking account of nonzero water-leaving reflectances.

  {\bf Note:} The following steps are the same as the standard algorithm described in \S\ref{subsec:gordon}.

  \item Using the aerosol LUTs, calculate the corresponding single-scattering aerosol value at the two NIR bands, i.e. $\rho_{as}(\lambda_s)$ and $\rho_{as}(\lambda_l)$. Then, calculate corresponding SSE.

  \item For each $N$ aerosol, compute the single-scattering value using \autoref{eq:rhoas}. Calculate corresponding SSE.
  \item Select the best two aerosol models by comparing the retrieved SSE with the theoretical SSE and determine the interpolation ratio between them.
  \item For the optimal aerosol model use the tabulated $\varepsilon(\lambda)$ in the VIS and $\varepsilon(\lambda_l)$ to obtain $\rho_{as}(\lambda)$ and then $\rho_{am}(\lambda)$ in the VIS using \autoref{eq:rholambda_i} and LUT.
  \item Remove $\rho_{am}(\lambda)$ in the VIS from $\rho_c(\lambda)$ and divide by the corresponding atmospheric transmittance that corresponds to the best aerosol model to return $\rho_w(\lambda)$ in the VIS using \autoref{eq:rhoam}.
\end{enumerate}

\begin{figure}[htb]
  \centering
  \includegraphics[width=10cm]{/Users/javier/Desktop/Javier/PHD_RIT/Latex/Proposal/Images/RuddickPlot.png}
  \caption{Scatterplot of Rayleigh-corrected reflectances at 765 and 865 nm for an subregion in a SeaWiFS image taken 28 October 1997, 12:15 UTC.  (Note: image taken from \cite{Ruddick:2000bs}). \label{fig:ruddickplot} } 
\end{figure}
% The second atmospheric correction method will be an extension of a method developed for SeaWiFS over turbid coastal and inland waters \cite{Ruddick:2000bs}. This method is a modified version of the methods developed by Gordon \cite{Gordon:1997} for ocean color satellites, but when the signal leaving the water does contribute to the overall signal beyond the NIR part of the spectrum. By using longer wavelengths and restricting the input pixels to open waters, these methods can be  applied to many fresh and coastal regions. The water surface reflectance values obtained after atmospheric correction will be validated through comparison to water surface reflectance measured in situ. 

% ------------------------------
\subsection{SWIR bands Atmospheric Correction}\label{subsec:wang}

For Case 2 or high productive Case 1 waters, especially turbid waters, there is a significant contribution from the water-leaving radiance and hence the zero water-leaving assumption in the NIR bands is not valid anymore. This is often the case in inland and coastal waters and therefore the NIR bands cannot be used to atmospherically correct this kind of imagery. However, these kinds of waters are indeed black in the shortwave infrared (SWIR) bands ($\geq 1000nm$) due to stronger water absorption. So, we can use the same atmospheric correction procedure used for SeaWiFS and MODIS but replacing the NIR bands with the two OLI's SWIR band ($1690$ and $2200nm$) for the data processing, as suggested by \cite{Wang:2007,Wang:2005}. \cite{Wang:2007} evaluated different combinations of the MODIS' SWIR bands to atmospherically correct a specific study image and compared the results with the traditional algorithm (NIR bands). Actually, the latest MODIS atmospheric algorithm uses a NIR-SWIR combined atmospheric correction approach that uses the NIR bands for Case 1 non-high productive water and the SWIR bands for Case 2 or high productive Case 1 waters, and a turbidity index to decide when to use them (\cite{Wang:2007dz}). 

% ------------------------------
\subsection{OLI Algorithm for Case 2 Waters (Blue Band)}
\cite{GeraceThesis} proposed to use a combination of spectral matching and band ratio techniques (method developed by \cite{Gordon:1997}) to atmospherically correct OLI data over Case 2 waters. A requirement to use band ratio technique for atmospherically correct OLI data is $\varepsilon^{(1,6)}\cong constant$, or in other words the ratio of reflectance from OLI's band 1 (coastal band) and band 6 (SWIR 1 band) should be approximately constant for all water pixels in the region of interest. Due to variability in band 1, the previous requirement is not always true. However, this variability can be accounted using spectral matching. This method needs to be applied with caution as it will not work in highly turbid water due to high variability in band 1. Therefore, an {\it a priori} analysis of the band histogram is needed, and only the bands whose histogram has little spread and resemble a normal distribution should be used.

This method first uses a forwarding modeling to create a three dimensional (3-D) LUT. Then, it adds atmospheric visibility as a fourth dimension (4-D). First, a 3-D LUT is created using Hydrolight. The dimensions are concentration of chlorophyll-{\it a}, suspended particles and CDOM. The range of the LUT is made up of spectral water-leaving reflectances. Then, this 3-D reflectance LUT is propagated to the TOA using MODTRAN for a range of visibilities. Therefore, the 4-D LUT will be made up of the independent variables chlorophyll-{\it a}, suspended particles, CDOM and visibility, while the range will be made up of spectral sensor-reaching radiances. A diagram showing the concept of this 4-D LUT is shown in \autoref{fig:4DLUT}. Note that this approach requires some knowledge of the aerosol type (e.g. rural, urban, maritime, etc.). Furthermore, the sensor-reaching radiances should be spectrally sampled by the OLI's sensor response and the data from the image should corrected for glint effect before comparing with the 4-D LUT since it does not include this effect.

\begin{figure}[htb]
  \centering
  \includegraphics[width=14cm]{/Users/javier/Desktop/Javier/PHD_RIT/Latex/Proposal/Images/4DLUT.png}
  \caption{Four dimensional LUT. The dimension are concentrations of chlorophyll-{\it a}, suspended particles, CDOM and visibility and the range is spectral sensor-reaching radiance. Source: \cite{GeraceThesis}}
  \label{fig:4DLUT} 
\end{figure}\todo{change figure for a better quality}

Once the 4-D LUT has been created, an iterative search of the closest match to an imaged water pixel is performed. The first step in this iteration is to obtain an initial guess of the visibility. To do so an imaged spectrum is compared to the spectra contained in the 4-D LUT. The parameters (concentration of CPAs and visibility) associated to the closest non-interpolated spectrum in the 4-D LUT in a RMS sense are associated with the imaged spectrum. Then, these associated parameters are fixed and the observed $\varepsilon^{(1,6)}$ is compared with the 4-D LUT $\varepsilon^{(1,6)}$ values for different visibility but same concentration, and the visibility of the closest match is selected as initial visibility estimate.

After estimating the initial visibility, an optimization routine can be used to estimate the four parameters from the 4-D LUT by interpolation. The final result of this process are interpolated CPA's concentration and visibility associated with the imaged pixel.

\cite{GeraceThesis} suggests using an average of all visibility solutions obtained previously as a fixed visibility and repeat the estimation described above with this fixed visibility. This is because the atmosphere should not have a high variability in the scene and therefore be approximately constant over the study region.

A summary of this process is described below.

\noindent{\bf Summary:}
Enter the algorithm with glint corrected data:\\
{\bf Step 1:} Using the three dimensions of chlorophyll-{\it a}, particles and CDOM concentrations, create a water-leaving reflectance 3-D LUT using Hydrolight.\\
\noindent{\bf Step 2:} Propagate the 4-D LUT to the TOA using MODTRAN to develop a 4-D LUT of sensor-reaching radiances. Use a best-estimate aerosol type and a range of visibility.\\
\noindent{\bf Step 3:} Search best match of imaged water pixel in the 4-D LUT.\\
\noindent{\bf (a)} Obtain initial guess of the visibility by using spectral matching and epsilon ratios.\\
\noindent{\bf (b)} Search for best match in the 4-D LUT using initial guess of the visibility.\\
\noindent{\bf Step 4:} Obtain average visibility from all the water pixel and repeat search for best match.\\

% ------------------------------
\subsection{OLI Algorithm for Case 2 Waters (Band Ratios)}
Another algorithm suggested by \cite{GeraceThesis} uses the concept adopted by \cite{Ruddick:2000bs}, but using OLI's NIR (band 5) and SWIR 1 (band 6) bands instead of the two NIR bands used for SeaWiFS. This algorithm utilizes the concept of band ratios (using the epsilon ratio values) in its implementation for calculating the reflectance due to aerosol in the scene. Recall from \cite{Ruddick:2000bs} that we would like $\varepsilon^{(i,j)}$ to be constant over the region of interest, for some band $i$ and band $j$. 

For this algorithm specifically, the requirement should be $\varepsilon^{(5,6)}\cong constant$. If this is the case for any two band in the scene, then a band ratio technique (\S\ref{subsec:gordon}, \S\ref{subsec:ruddick} and \S\ref{subsec:wang}) can be used to solve for $\rho_w$ in \autoref{eq:rhoam}. However, the requirement of $\varepsilon^{(5,6)}\cong constant$ could not be true for turbid waters because there are some signal coming from the water in the NIR and not only from the atmosphere, i.e. $\rho_w^{(5)}\neq0$. This fact produces variability in the water reflectance in band 5. Therefore, simply calculating $\varepsilon^{(5,6)}$ will result in a misrepresentation of the atmosphere and the water signal. A solution to this problem is to select the signal in band 5 (NIR band; $862nm$) from a region of dark waters, and make the black pixel assumption in the band 6. This will allow to calculate $\varepsilon^{(5,6)}$ over a dark water and therefore determine its atmosphere signal, i.e. the \cite{Ruddick:2000bs} approach adapted to the SWIR region. Then, the whole water scene is assumed to have the chosen atmosphere, the image is atmospherically correct for that atmosphere. The details of this algorithm are similar to the ones described in \cite{Ruddick:2000bs} (see \S\ref{subsec:ruddick}), and summarized below.

\noindent{\bf Summary:}
Enter the algorithm with glint corrected data:\\
{\bf Step 1:} Create a LUT of aerosol reflectances for different atmospheric models using MODTRAN and Hydrolight to determine $\rho_w$ for a dark water pixel. Then, calculate $\varepsilon^{(5,6)}$ for each atmospheric model in the LUT. \\
\noindent{\bf Step 2:} Calculate an averaged $\varepsilon^{(5,6)}$ from a region of interest over dark water by averaging the reflectance at the TOA ($\rho$) values for that region.\\ 
\noindent{\bf Step 3:} Use the averaged $\varepsilon^{(5,6)}$ to find the closest two matches for the modeled atmosphere in the LUT from Step 1 and calculate the interpolation ratio between these two matched and the averaged $\varepsilon^{(5,6)}$. \\
\noindent{\bf Step 4:} Extrapolate the determined model to all wavelengths using interpolation ratio. \\
\noindent{\bf Step 5:} Globally correct the scene for the atmospheric effect. \\
% %%%%%%%%%%%%%%%%%%%%%%%%%%%%%%%%%%%%%%%%%%%%%%%%
% -----------------------------------------------
\section{In-Water Constituent Retrieval Process}
\label{sec:retrieval}
The retrieval algorithm will be based on previous work done by Gerace {\it et al.} \cite{Gerace:2013} and Raqueno {\it et al.} \cite{Raqueno:2000}. The water surface reflectance product obtained after atmospheric correction from the previous stage is used as input to the retrieval algorithm. Each pixel in the reflectance product has an unknown concentration. A spectral matching technique is applied to predict this concentration by comparing the spectral shape of each pixel with the elements in a look-up table (LUT). The complete retrieval process will be explained in the following sections. 
% -------------------------------------
\subsection{LUT generation}
The LUT is generated in Ecolight~\cite{MobleyHE} for different triplets of water constituent concentrations (CPAs). Ecolight was used in the same fashion as in the black pixel determination for the model-based ELM (\S\ref{subsubsec:blackpixel}). \autoref{tab:LUTconc2} shows the different parameters used to create the LUT in Ecolight. Two different sets of IOPs (mass-specific absorption and scattering spectra) were used, one set for modeling the open lake conditions (low concentration of CPAs) and one set for modeling the pond condition (high concentration of CPAs)\todo{show IOPs figures}. These IOPs are labeled in \autoref{tab:LUTconc2} as ``ONTNS'' for the open lake conditions and as ``LONGS'' for the pond conditions. \autoref{tab:LUTconc2} also shows the CPAs concentration used to create the LUT in Ecolight. Furthermore, discretized Fournier-Forand phase function with four different backscatter fraction values ($0.5$, $1.0$, $1.5$ and $2.0\%$) were used to account for the backscattering variability in the scene.


\begin{table}[htb]
\caption{ Input parameters for the LUT generation in Ecolight. \label{tab:LUTconc2} } 
\centering
		\begin{tabular}{c|c|c|c|c}
        		\bfseries{$<CHL>$}  	& \bfseries{$<SM>$}  & \bfseries{$a_{CDOM}(440)$} & \bfseries{$Backscatter$} & IOPs Input\\
		$[ug/L]$  		& $[mg/L]$ & 	$[1/m]$ &	\bfseries{$Fraction$}, $[\%]$	\\ \hline \hline
0.1   & 1.0  &  0.11 &  0.5 & ONTNS\\
0.5   & 2.0  &  0.15 &  1.0 & LONGS\\
1.0   & 5.0  &  0.21 &  1.5 & --\\
3.0   & 10.0 &  0.6  &  2.0 & --\\
10.0  & 25.0 &  1.0  &  --  & --\\
20.0  & 45.0 &  1.2  &  --  & --\\
40.0  & 50.0 &  --   &  --  & --\\
60.0  & --   &  --   &  --  & --\\  
90.0  & --   &  --   &  --  & --\\  
110.0 & --   &  --   &  --  & --\\  
135.0 & --   &  --   &  --  & --\\  
150.0 & --   &  --   &  --  & --\\     
	 	\end{tabular}
	\end{table}

After obtaining the LUT from Ecolight, the spectral curves in the LUT are spectrally sampled to the OLI's spectral response. An example of a LUT created in Ecolight is shown in \autoref{fig:LUT} with 2000 spectral curves. 

\begin{figure}[htb]
    \centering
      \includegraphics[height=7cm]{/Users/javier/Desktop/Javier/PHD_RIT/ConferencesAndApplications/2014_ASPRS_SOY/Images/LUT.eps}
      \caption{LUT created in HydroLight}
      \label{fig:LUT}
\end{figure}

% \begin{figure}[htb]
%   	\centering
%   	\includegraphics[height=7cm]{/Users/javier/Desktop/Javier/PHD_RIT/Latex/Proposal/Images/CHLaastLONGSJavier.eps}
%   \caption{Chlorophyll mass-specific absorption spectra used to create the LUT in HydroLight. \label{fig:CHLaastLONGS} } 
% \end{figure}


% \begin{figure}[htb]
%   	\centering
%   	\includegraphics[height=7cm]{/Users/javier/Desktop/Javier/PHD_RIT/Latex/Proposal/Images/astar_CDOM_LONGS091913.eps}
%   \caption{CDOM mass-specific absorption spectra used to create the LUT in HydroLight. \label{fig:CDOMaastLONGS} } 
% \end{figure}

% \begin{figure}[htb]
%   	\centering
%   	\includegraphics[height=7cm]{/Users/javier/Desktop/Javier/PHD_RIT/Latex/Proposal/Images/astar_SM_LONGS091913.eps}
%   \caption{SM mass-specific absorption spectra used to create the LUT in HydroLight. \label{fig:SMaastLONGS} } 
% \end{figure}
\subsection{Retrieval}
Two methods are investigated for performing the retrieval. The first one is using the root mean squared error (RMSE) to determine the concentrations based in the closest element in the LUT. This method gives discretized concentration values corresponding to the LUT concentration values. The second one uses a non-linear optimization to estimate these values. This method gives continuous concentration values based in a trilinear interpolation that interpolated between the closest matches in the LUT.
% -----------------------------------------
\subsubsection{Root Mean Square Error}
In this work, the RMSE is defined as
\begin{equation}
  RMSE(i) = \sqrt{\frac{1}{m}\sum_1^m\left[\widetilde{R}_{rs}(i,\lambda_m)-R_{rs}(\lambda_m)\right]^2}
\end{equation}
where $\widetilde{R}_{rs}(i,\lambda_m)$ is the $i$th database spectrum from the LUT at wavelength band $m$ and $R_{rs}(\lambda_m)$ is the measured spectrum for a particular image pixel. For each pixel in the image, the RMSE error is calculated with every $i$th element in the LUT. Then, the element of the LUT that results in the lowest RMSE is chosen as the concentration for that particular pixel. This method gives discretized values for the concentrations equal to the values used to create the LUT.

% -----------------------------------------
\subsubsection{Non-linear Optimization}
\todo{Mention: water mask and RMSE} 
The spectral matching is made by a least square error minimization algorithm using the ``\texttt{lsqnonlin}'' package of the MATLAB's Optimization Toolbox. \texttt{lsqnonlin} solves least-squares problems, including nonlinear data-fitting problems \todo{cite Matlab Help}. If $f(x)$ is a user-defined vector function defined as
\begin{equation}
  f(x)=
  \left[
    \begin{array}{c}
      f_1(x) \\
      f_2(x) \\
      \vdots \\
      f_m(x) \\
    \end{array}
  \right],
\end{equation}
with $x$ a vector. \texttt{lsqnonlin} tries to minimize the function
\begin{equation}
  \underset{x}{min}\parallel f(x) \parallel^2_2=\underset{x}{min}(f_1(x)^2+f_2(x)^2+f_3(x)^2+...+f_m(x)^2)
\end{equation}
In this case, $x$ is the three CPA concentrations and the function $f(x)$ is the difference between the water spectra $R_{rs}$ for each pixel and an estimated curve $F$ from the LUT, this is
\begin{equation}
  f = R_{rs} - F
\end{equation}
where $F$ is obtained from a trilinear interpolation based in the CPA concentrations of the LUT. The dimension $m$ is the number of bands. In other words, for each pixel in the image, \texttt{lsqnonlin} tries to find a function that minimizes the error between the measured value and an interpolated spectra from the LUT. \texttt{lsqnonlin} stops the search after reaching certain threshold. 

The output of this process is a concentration mapping for each water constituent that spans the range of constituents levels in the scene. 

\section{Ground Truth Data Collection}
The area of study for this research is the Lake Ontario Rochester Embayment (latitude: 43°15'32.53"N and longitude: 77°36'13.10"W), which includes some nearby ponds (Long and Cranberry Ponds), the Genesee River plume, the Irondequoit Bay and the southern end of Lake Ontario, as shown in Figure~\ref{fig:areaofstudy1} and Figure~\ref{fig:areaofstudy2}. This area was selected because it exhibits a wide range of variability in concentration of water constituents, so the retrieval algorithm can be tested with different scenarios. Landsat-8 images from this area of study and corresponding water samples collected at the time of the satellite's overpass will be used to test the retrieval algorithm. So far, there are only three satisfactory images available from the summer 2013. This project contemplates performing one new ground truth data collection during 2014. Therefore, images from the 2013-2014 spring and summer collection seasons will be used to test the methodology. Note that a difficult challenge of this research is to obtain images with relatively clear weather conditions (i.e. cloud free) over the area of study.
\begin{figure}[htb]
  \centering
  \includegraphics[height=9cm]{/Users/javier/Desktop/Javier/PHD_RIT/Latex/Proposal/Images/AreaOfStudy1.pdf}
  \caption{Area of Study. \label{fig:areaofstudy1} } 
\end{figure}
\begin{figure}[htb]
  \centering
  \includegraphics[height=7cm]{/Users/javier/Desktop/Javier/PHD_RIT/Latex/Proposal/Images/AreaOfStudy2.pdf}
  \caption{Area of Study. \label{fig:areaofstudy2} } 
\end{figure}

In order to have outputs in HydroLight that are representative of the water bodies that are being studied, inherent optical properties (IOPs) of those specific waters have to be defined as input to the HydroLight model. After collection, these water samples need to be analyzed in the lab to obtain IOPs for the main water constituents. Furthermore, apparent optical properties (AOPs) (i.e. water surface reflectance) and backscattering measurements will be also collected for further comparison and to pursue closure between the HydroLight AOPs results and in-situ AOPs measurements.

\section{Validation}
The results from the retrieval process will be validated by comparison with the concentration of water samples taken during field campaigns in the spring and summer of 2013 and 2014. These concentrations will be obtained from lab measurements made at the Rochester Institute of Technology. For further validation, the results will be compared with products derived from ocean color satellites such as MODIS (e.g. MODIS Chl-{\it a} product), in regions where it is possible. 

Additionally, the results will be compared with the products derived in the SeaWiFS Data Analysis System (SeaDAS) software for Landsat-8, which is contemplated to be launched soon (more info: \url{http://seadas.gsfc.nasa.gov/}). The latest version (SeaDAS 7.0.2) is a comprehensive image analysis package for the processing, display, analysis, and quality control of ocean color data developed by the developers of ESA's BEAM software package and the Ocean Biology Processing Group (OBPG) at NASA. If this capability is launched in the next version of SeaDAS, Landsat-8 level L1 and L2 data could potentially be  atmospherically corrected using the method used for MODIS \cite{Gordon:1994,Ruddick:2000bs,Wang:2007dz}, and therefore L3 data (e.g. Chl-{\it a} product) could be obtained from Landsat-8 images.
% \section{OLI Sensor}
% \subsection{Sensor Response}
% Noise level. Quantization. SNR.

% \section{Retrieval Process}
% \subsection{The Look-Up Table}


% \subsubsection{Real Atmosphere Conditions}

% \section{Over-Water Atmospheric Compensation}
% \subsection{Model Based Empirical Line Method}

% \subsubsection{Pseudo Invariant Features}

% This model employs pseudo-invariant feature (PIF) pixels extraction from the Landsat Climate Data Record (CDR) Surface Reflectance product along with an in-water radiative transfer model (HydroLight) to obtain the field spectra to be used in the ELM method. 
%  A mask is created applying a threshold a the ratio between the NIR band and the red band. 
%  A mask is created applying a threshold to the SWIR 2 band. 

% The model based empirical line method (MBELM) atmospheric correction method is based in the well known empirical line method (ELM).

% \subsection{LUT Method}


\section{Concluding Remarks}
The purpose of this chapter was to introduce the methodology required to achieve the objectives defined in the Chapter \S\ref{ch:objectives}. First, we began by explaining the different atmospheric correction techniques that will be investigated in this research along with the solar-glint removal algorithm. These atmospheric correction techniques include two different approaches. The first approach are methods applied to ocean color satellites. The second one is the model-based ELM method. The in-water constituent retrieval process was presented. The LUT generation and the metrics used to perform the retrieval were described. Additionally, the ground truth data collection was briefly explained. Finally, how the results will be validated was described. 

The next Chapter (\S\ref{ch:results}) will present the data and laboratory measurements available to date, along with preliminary results. These results include the atmospheric correction applied to Landsat-8 imagery, concentration of CPAs maps.



% !TEX root=Thesis_PhD.tex 
% the previous is to reference main .bib
%% CHAPTER
\chapter{Results}
\label{ch:results}

This chapter is separated in four sections. The first section explains the data collection process, giving some details in the kind of measurements taken in the field. Also, it includes a summary of the data collected at the moment of writing this document. The second section gives a brief description of the lab measurements. The atmospheric section shows preliminary results from the model-based ELM atmospheric correction method. Finally, the last section shows preliminary retrieval results over the area of study.
% ------------------------------
\section{Data Collection}

The first area of study used in this research is the Rochester Embayment in the city of Rochester, NY. This area of study includes some nearby ponds (Long and Cranberry ponds), the Genesee River plume, the Irondequoit bay and part of Lake Ontario. This area was selected because it exhibits a wide range of variability in concentration of water constituents, so the retrieval algorithm can be tested with different scenarios. \autoref{fig:0910913Sites} shows an image over this area of study for the data collection done on September, $19^{th}$, 2013  with the different sites as an example. The data collections are divided in two crews. One crew, named ``Lake crew'', is in charge of the Irondequoit Bay, Ontario near shore, Ontario off shore, Genesee River plume, Genesee River pier sites (labeled in \autoref{fig:0910913Sites} as IBayN, OntNS, OntOS, RvrPLM and RvrPIER, respectively). The other crew, named ``Ponds crew'', is in charge of the Long Pond north and south, Cranberry Pond sites (labeled in \autoref{fig:0910913Sites} as LongN, LongS and Cranb, respectively).

\begin{figure}[htb]
  \centering
  \includegraphics[width=12cm]{/Users/javier/Desktop/Javier/PHD_RIT/Latex/Proposal/Images/groundtruth-sitenames-no-ends.jpg}
  \caption{Sites in the Rochester Embayment for the water sample collection on September, $19^{th}$, 2013.\label{fig:0910913Sites} } 
  % \vspace{1cm}
\end{figure}

Water samples are collected for each site in dark Nalgene bottles. Additionally, remote-sensing reflectances are measured using an ASD and a SVC instrument (one for each crew). Backscattering measurements are taken using a HydroScat-2. This measurement is taken by both crew only if the logistic of the particular day allows it. Otherwise priority is given to the Lake crew. For each site, a location is recorded using a GPS. \autoref{tab:collect} shows a summary of the data collections done in 2013 and 2014 seasons with the respective available data.

\begin{table}[htb]
  \caption{Summary of 2013 and 2014 data collections.}
  \centering
  \includegraphics[width=13cm]{/Users/javier/Desktop/Javier/PHD_RIT/Latex/Proposal/Images/Collect1314.png}
  \label{tab:collect}
\end{table}

% ------------------------------
\section{Laboratory Measurements}

After collection, the water samples are analyzed in laboratory. These lab measurements include concentration of CHL and TSS, absorption coefficients of CHL, TSS and CDOM. The Chlorophyll-{\it a} concentration measurements needed to be validated with measurements analyzed by a credible lab (Monroe County Environmental Laboratory). This comparison with this lab shows agreement between the measurements. \autoref{tab:collect} shows the measurements available. \todo{include references to procedures}

% ------------------------------
\section{Atmospheric Correction}

Preliminary results of the model-based ELM atmospheric correction methods for different water bodies in the Rochester Embayment area are shown in \autoref{fig:waterpxs}. This figure shows the spectrum of the water pixels in reflectance values after applying the model-based ELM atmospheric correction. These curves exhibit shapes that correspond with the shapes of typical water pixels. \autoref{fig:refcomp} shows water reflectance spectra as preliminary results from the model-based ELM method (solid lines) compared with results from a traditional ELM method (dashed lines) for four different ROIs in the Rochester Embayment area (Cranberry Pond, Long Pond, and nearshore and offshore of the Lake Ontario, labeled as Cranb, LongS, OntNS and OntOS in \autoref{fig:0910913Sites}, respectively). 

The traditional ELM method was performed with reflectance measurements taken in the field. A reflectance measurement taken of the sand of Charlotte Beach, Rochester, NY (labeled as SandDry in \autoref{fig:0910913Sites}) was used for the bright pixel while a reflectance measurement taken at the site OntNS was used for the dark pixel. The radiance values were taken from the corresponding ROIs in the Landsat 8 image. 

As can be seen in \autoref{fig:refcomp}, the atmospheric correction algorithm proposed in this study exhibits less than one percent reflectance unit ($<0.01\Rightarrow <1\%$) of difference in comparison to the results from the traditional ELM algorithm.

\begin{figure}[htb]
  \begin{minipage}[c]{0.48\linewidth}
    \centering
      \includegraphics[width=7cm]{/Users/javier/Desktop/Javier/PHD_RIT/ConferencesAndApplications/NESSF14/latex/WaterPixels_2.eps}
      \caption{Water pixel spectra after applying the model-based ELM atmospheric correction method.}
      \label{fig:waterpxs}
    % \vspace{1.5cm}
    % \centerline{(a)}\medskip
  \end{minipage}
  \hfill
  \begin{minipage}[d]{0.48\linewidth}
    \centering
      \includegraphics[width=7cm]{/Users/javier/Desktop/Javier/PHD_RIT/ConferencesAndApplications/NESSF14/latex/WaterPixelComparisonELMELMbased}
      \caption{Comparison between traditional ELM (dashed lines) and model-based ELM (solid lines).}
      \label{fig:refcomp}
    % \vspace{1.5cm}
    % \centerline{(b)}\medskip
  \end{minipage}
  % \caption{Water pixel spectra: (a) after atmospheric correction and (b) comparison with traditional ELM method.}
\end{figure}

% ------------------------------
\subsection{$R_{rs}$ Comparison}

The four atmospheric correction methods (MoB-ELM, SeDAS-SWIR, Acolite-SWIR and SeaDAS-MUMM) described in Section~\ref{sec:atmcorr} \todo{Check section!} are compared in this section. These algorithm were applied to the 09-19-2013 Landsat 8 image. A mask was created to mask out all pixels but water pixels. This mask was created by thresholding the Landsat 8's SWIR 2 band. \autoref{fig:Rrs443} and \autoref{fig:Rrs561} show the $R_{rs}$ for band 1 and 3 for these methods. When analyzed visually, the similarities among the methods are higher in band 3 than band 1. For band 1, SeaDAS-SWIR and SeaDAS-MUMM look similar in both the lake and ponds. Also, the SeaDAS-SWIR method exhibits some noise in band 1. This is caused due to the low SNR in SWIR bands used for the atmospheric correction. Note that there are some bottom effect in the lake's shoreline causing all methods to retrieve high $R_{rs}$ in those areas, which is not the case. This should be taken in account for future processing by masking the areas where the water is clear enough that the bottom can be seen.

Another comparison of $R_{rs}$ at $443nm$ and $561nm$ among all four atmospheric correction methods is shown in \autoref{fig:13262Rrs443} and \autoref{fig:13262Rrs561} as scatter plots with a regression line in red solid lines. Again, all the methods show more similarities in band 3 than band 1, which is corroborated by the smaller offset values for band 3 than band 1, which translates to more closeness to the $1:1$ line. This indicates that there is a higher correlation for band 3 than band 1 for all methods. \autoref{tab:RrsCompMethod} shows the slope and offset for the regression lines and the goodness of fit $R^2$ values for the comparison among all methods and for all bands. The $R^2$ values show a good correlation for all cases in the range from $0.7090$ to $0.9302$ (underlined in \autoref{tab:RrsCompMethod}). \autoref{tab:RrsCompMethod} also shows the number of valid pixels retrieved $N$ (non negative) and the root mean squared error (RMSE) between the methods compared. 

All methods produced negative $R_{r}$ values. These negative values indicates that the atmosphere contribution has been over estimated. These negative values are shown in \autoref{tab:RrsCompMethod} as percentage of the total of water pixels for each method compared. These negative values were not used in the comparison. Finally, \autoref{tab:RrsCompMethod} shows the percentage of total number of water pixels used in the comparison (labeled as Used $[\%]$). It can be seen that the Gordon and Wang's based methods generate more negative values compared with the MoB-ELM method overall. 

\autoref{fig:13262RrsCompField} shows a comparison between $R_{rs}$ spectra measured in the field (blue dashed line) with retrieved $R_{rs}$ spectra from the four atmospheric correction methods for different sites within the study area. The spectra for each method tend to differ more from the field spectra in band 1 and 2 than in band 3 and 4, for all sites. We calculated the normalized root mean squared error (NRMSE) for $R_{rs}(\lambda)$ to evaluate the differences between field measurements and spectra retrieved from the atmospheric correction methods. The NRMSE is defined as

\begin{equation}
\label{eq:NRMSE}
  NRMSE =\frac{\sqrt{\frac{1}{N}\sum_{n=1}^N{\left[R_{rs}(\lambda)_{ret}(n) - R_{rs}(\lambda)_{mea}(n)\right]^2}}}{max\{R_{rs}(\lambda)_{mea}(n)\} - min\{R_{rs}(\lambda)_{mea}(n)\}}\times100 ~[\%]
\end{equation}

\noindent where $R_{rs}(\lambda)_{ret}$ is the retrieved $R_{rs}(\lambda)$, $R_{rs}(\lambda)_{mea}$ is the measured $R_{rs}(\lambda)$, and $n=1\dots N$ is the number of measured concentrations. \autoref{fig:NRMSE130919_RRS} shows the NRMSE for all the atmospheric correction methods for all bands. It can be seen that the MoB-ELM perform the best for all bands, followed by SeaDAS-SWIR. 
% The biggest errors are obtained when applying the Acolite-SWIR method, specially in band 1.



% \begin{figure}[htbp!]
%   \centering
%   \includegraphics[height=8.0cm]{/Users/javier/Desktop/Javier/PHD_RIT/ConferencesAndApplications/2015_SPIE_SanDiego/Images/Collated2013262_2_band_1_D.png}
%   \caption{$R_{rs}$ for the sites for the 09-19-2013 collection after applying the MoB-ELM atmospheric correction (Labels: LONGS: Long Pond south, CRANB: Cranberry Pond, ONTOS: Lake Ontario off-shore, and ONTNS: Lake Ontario near-shore).\label{fig:RrsROIs130919} } 
% \end{figure}

% \begin{overpic}[width=0.5\textwidth,grid,tics=10]{pictures/baum}
%  \put (20,85) {\huge$\displaystyle\gamma$}
% \end{overpic}
%^^^^^^^^^^^^^^^^^^^  FIGURE ^^^^^^^^^^^^^^^^^^^^^^^^^^^^^^^^^^^^^^^^^^^^
\begin{figure}[htbp!]
  \begin{minipage}[c]{0.48\linewidth}
      \centering
      \begin{overpic}[trim=0 200 0 0,clip,width=7.0cm]{/Users/javier/Desktop/Javier/PHD_RIT/ConferencesAndApplications/2015_SPIE_SanDiego/Images/subset_0_of_Collocated13262_ACOSWIR_MOB_SEA_Rrs_443MOBdivpi}
      \put (5,5) {MOB-ELM}
      \end{overpic}
    \end{minipage}
    \hfill
  \begin{minipage}[c]{0.48\linewidth}
      \centering
      \begin{overpic}[trim=0 200 0 0,clip,width=7.0cm]{/Users/javier/Desktop/Javier/PHD_RIT/ConferencesAndApplications/2015_SPIE_SanDiego/Images/subset_0_of_Collocated13262_ACOSWIR_MOB_SEA_Rrs_443ACO}
      \put (5,5) {Acolite-SWIR}
      \end{overpic}
    \end{minipage}

    \vspace{0.7cm}

  \begin{minipage}[c]{0.48\linewidth}
      \centering
      \begin{overpic}[trim=0 200 0 0,clip,width=7.0cm]{/Users/javier/Desktop/Javier/PHD_RIT/ConferencesAndApplications/2015_SPIE_SanDiego/Images/subset_0_of_Collocated13262_ACOSWIR_MOB_SEA_Rrs_443SEA}
      \put (5,5) {SeaDAS-SWIR}
      \end{overpic}
    \end{minipage}
    \hfill
  \begin{minipage}[c]{0.48\linewidth}
      \centering
      \begin{overpic}[trim=30 170 40 150,clip,width=7.0cm]{/Users/javier/Desktop/Javier/PHD_RIT/ConferencesAndApplications/2015_SPIE_SanDiego/Images/Collocated13262_ACOSWIR_MOB_SEA5x5_MUMM45_Rrs_443_MUMM45}
      \put (5,5) {SeaDAS-MUMM}
      \end{overpic}
    \end{minipage}

    \begin{minipage}[c]{1.0\linewidth}
      \centering
      \vspace{0.5cm}
      \begin{overpic}[trim=0 0 0 0,clip,height=1.2cm]{/Users/javier/Desktop/Javier/PHD_RIT/ConferencesAndApplications/2015_SPIE_SanDiego/Images/Collocated13262_ACOSWIR_MOB_SEA5x5_MUMM45_colorbar}
      \put (28,16) {$R_{rs}(443nm) [1/sr]$}
      \end{overpic}
    \end{minipage}

  \caption{Remote-sensing reflectance ($R_{rs}$) at $443nm$ from the 09-19-2013 image over the Rochester Embayment (scene LC80160302013262LGN00) processed using the MoB-ELM, SeaDAS-SWIR, Acolite-SWIR and SeaDAS-MUMM.\label{fig:Rrs443} } 
\end{figure}
% %^^^^^^^^^^^^^^^^^^^  FIGURE ^^^^^^^^^^^^^^^^^^^^^^^^^^^^^^^^^^^^^^^^^^^^
\begin{figure}[htbp!]
  \begin{minipage}[c]{0.48\linewidth}
      \centering
      \begin{overpic}[trim=0 150 40 150,clip,width=7.0cm]{/Users/javier/Desktop/Javier/PHD_RIT/ConferencesAndApplications/2015_SPIE_SanDiego/Images/Collocated13262_ACOSWIR_MOB_SEA5x5_MUMM45_Rrs_483_MOB}
      \put (5,6) {MOB-ELM}
      \end{overpic}
    \end{minipage}
    \hfill
  \begin{minipage}[c]{0.48\linewidth}
      \centering
      \begin{overpic}[trim=0 0 40 0,clip,width=7.0cm]{/Users/javier/Desktop/Javier/PHD_RIT/ConferencesAndApplications/2015_SPIE_SanDiego/Images/Collocated13262_ACOSWIR_MOB_SEA5x5_MUMM45_Rrs_483_ACO_R_R}
      \put (5,5) {Acolite-SWIR}
      \end{overpic}
    \end{minipage}

    \vspace{0.7cm}

  \begin{minipage}[c]{0.48\linewidth}
      \centering
      \begin{overpic}[trim=0 0 40 0,clip,width=7.0cm]{/Users/javier/Desktop/Javier/PHD_RIT/ConferencesAndApplications/2015_SPIE_SanDiego/Images/Collocated13262_ACOSWIR_MOB_SEA5x5_MUMM45_Rrs_482_SEA5x5_R}
      \put (5,5) {SeaDAS-SWIR}
      \end{overpic}
    \end{minipage}
    \hfill
  \begin{minipage}[c]{0.48\linewidth}
      \centering
      \begin{overpic}[trim=0 150 40 150,clip,width=7.0cm]{/Users/javier/Desktop/Javier/PHD_RIT/ConferencesAndApplications/2015_SPIE_SanDiego/Images/Collocated13262_ACOSWIR_MOB_SEA5x5_MUMM45_Rrs_482_MUMM45}
      \put (5,5) {SeaDAS-MUMM}
      \end{overpic}
    \end{minipage}
    

    \begin{minipage}[c]{1.0\linewidth}
      \centering
      \vspace{0.5cm}
      \begin{overpic}[trim=0 0 0 0,clip,height=1.2cm]{/Users/javier/Desktop/Javier/PHD_RIT/ConferencesAndApplications/2015_SPIE_SanDiego/Images/Collocated13262_ACOSWIR_MOB_SEA5x5_MUMM45_colorbar}
      \put (28,16) {$R_{rs}(482nm) [1/sr]$}
      \end{overpic}
    \end{minipage}

  \caption{Remote-sensing reflectance ($R_{rs}$) at $482nm$ from the 09-19-2013 image over the Rochester Embayment (scene LC80160302013262LGN00) processed using the MoB-ELM, SeaDAS-SWIR, Acolite-SWIR and SeaDAS-MUMM.\label{fig:Rrs482} } 
\end{figure}
%^^^^^^^^^^^^^^^^^^^  FIGURE ^^^^^^^^^^^^^^^^^^^^^^^^^^^^^^^^^^^^^^^^^^^^
\begin{figure}[htbp!]
  \begin{minipage}[c]{0.48\linewidth}
      \centering
      \begin{overpic}[trim=0 155 40 150,clip,width=7.0cm]{/Users/javier/Desktop/Javier/PHD_RIT/ConferencesAndApplications/2015_SPIE_SanDiego/Images/Collocated13262_ACOSWIR_MOB_SEA5x5_MUMM45_Rrs_561_MOB}
      \put (5,5) {MOB-ELM}
      \end{overpic}
    \end{minipage}
    \hfill
  \begin{minipage}[c]{0.48\linewidth}
      \centering
      \begin{overpic}[trim=0 150 40 150,clip,width=7.0cm]{/Users/javier/Desktop/Javier/PHD_RIT/ConferencesAndApplications/2015_SPIE_SanDiego/Images/Collocated13262_ACOSWIR_MOB_SEA5x5_MUMM45_Rrs_561_ACO_R_R}
      \put (5,5) {Acolite-SWIR}
      \end{overpic}
    \end{minipage}

    \vspace{0.7cm}

  \begin{minipage}[c]{0.48\linewidth}
      \centering
      \begin{overpic}[trim=0 150 40 150,clip,width=7.0cm]{/Users/javier/Desktop/Javier/PHD_RIT/ConferencesAndApplications/2015_SPIE_SanDiego/Images/Collocated13262_ACOSWIR_MOB_SEA5x5_MUMM45_Rrs_561_SEA5x5_R}
      \put (5,5) {SeaDAS-SWIR}
      \end{overpic}
    \end{minipage}
    \hfill
  \begin{minipage}[c]{0.48\linewidth}
      \centering
      \begin{overpic}[trim=0 150 40 150,clip,width=7.0cm]{/Users/javier/Desktop/Javier/PHD_RIT/ConferencesAndApplications/2015_SPIE_SanDiego/Images/Collocated13262_ACOSWIR_MOB_SEA5x5_MUMM45_Rrs_561_MUMM45}
      \put (5,5) {SeaDAS-MUMM}
      \end{overpic}
    \end{minipage}
    

    \begin{minipage}[c]{1.0\linewidth}
      \centering
      \vspace{0.5cm}
      \begin{overpic}[trim=0 0 0 0,clip,height=1.2cm]{/Users/javier/Desktop/Javier/PHD_RIT/ConferencesAndApplications/2015_SPIE_SanDiego/Images/Collocated13262_ACOSWIR_MOB_SEA5x5_MUMM45_colorbar}
      \put (28,16) {$R_{rs}(561nm) [1/sr]$}
      \end{overpic}
    \end{minipage}

  \caption{Remote-sensing reflectance ($R_{rs}$) at $561nm$ from the 09-19-2013 image over the Rochester Embayment (scene LC80160302013262LGN00) processed using the MoB-ELM, SeaDAS-SWIR, Acolite-SWIR and SeaDAS-MUMM..\label{fig:Rrs561} } 
\end{figure}
%^^^^^^^^^^^^^^^^^^^  FIGURE ^^^^^^^^^^^^^^^^^^^^^^^^^^^^^^^^^^^^^^^^^^^^
\begin{figure}[htbp!]
  \begin{minipage}[c]{0.48\linewidth}
      \centering
      \begin{overpic}[trim=0 0 40 0,clip,width=7.0cm]{/Users/javier/Desktop/Javier/PHD_RIT/ConferencesAndApplications/2015_SPIE_SanDiego/Images/Collocated13262_ACOSWIR_MOB_SEA5x5_MUMM45_Rrs_655_MOB}
      \put (5,5) {MOB-ELM}
      \end{overpic}
    \end{minipage}
    \hfill
  \begin{minipage}[c]{0.48\linewidth}
      \centering
      \begin{overpic}[trim=0 0 40 0,clip,width=7.0cm]{/Users/javier/Desktop/Javier/PHD_RIT/ConferencesAndApplications/2015_SPIE_SanDiego/Images/Collocated13262_ACOSWIR_MOB_SEA5x5_MUMM45_Rrs_655_ACO_R_R}
      \put (5,5) {Acolite-SWIR}
      \end{overpic}
    \end{minipage}

    \vspace{0.7cm}

  \begin{minipage}[c]{0.48\linewidth}
      \centering
      \begin{overpic}[trim=0 0 40 0,clip,width=7.0cm]{/Users/javier/Desktop/Javier/PHD_RIT/ConferencesAndApplications/2015_SPIE_SanDiego/Images/Collocated13262_ACOSWIR_MOB_SEA5x5_MUMM45_Rrs_655_SEA5x5_R}
      \put (5,5) {SEADAS-SWIR}
      \end{overpic}
    \end{minipage}
    \hfill
  \begin{minipage}[c]{0.48\linewidth}
      \centering
      \begin{overpic}[trim=0 0 40 0,clip,width=7.0cm]{/Users/javier/Desktop/Javier/PHD_RIT/ConferencesAndApplications/2015_SPIE_SanDiego/Images/Collocated13262_ACOSWIR_MOB_SEA5x5_MUMM45_Rrs_655_MUMM45}
      \put (5,5) {SeaDAS-MUMM}
      \end{overpic}
    \end{minipage}
    

    \begin{minipage}[c]{1.0\linewidth}
      \centering
      \vspace{0.5cm}
      \begin{overpic}[trim=0 0 0 0,clip,height=1.2cm]{/Users/javier/Desktop/Javier/PHD_RIT/ConferencesAndApplications/2015_SPIE_SanDiego/Images/Collocated13262_ACOSWIR_MOB_SEA5x5_MUMM45_colorbar}
      \put (28,16) {$R_{rs}(655nm) [1/sr]$}
      \end{overpic}
    \end{minipage}

  \caption{Remote-sensing reflectance ($R_{rs}$) at $655nm$ from the 09-19-2013 image over the Rochester Embayment (scene LC80160302013262LGN00) processed using the MoB-ELM, SeaDAS-SWIR, Acolite-SWIR and SeaDAS-MUMM.\label{fig:Rrs655} } 
\end{figure}

% Method 1    Method 2    wl    NegTool1  NegTool2    usable  rsq_SS      rsq_corr    slope   offset      R^2         N           RMSE
% Acolite     MoB-ELM     443   0.53      0.09        97      .0.8791     0.8828      0.9444  -0.0047     0.8791      144047     0.0054
% Acolite     SeaDAS      443   0.53      76.74       98      .0.7804     0.7924      1.1437  -0.0053     0.7804      145186     0.0038
% Acolite     MUMM        443   0.53      74.41       99      .0.8554     0.8606      1.0600  -0.0032     0.8554      147730     0.0026
% SeaDAS      MoB-ELM     443   43.98     0.05        55      .0.7637     0.7776      0.8664  -0.0007     0.7637      141563     0.0019
% SeaDAS      MUMM        443   43.98     42.64       56      .0.8456     0.8516      0.9175  +0.0018     0.8456      145315     0.0014
% MUMM        MoB-ELM     443   42.64     0.05        56      .0.7474     0.7634      0.8811  -0.0018     0.7474      144947     0.0029
% Acolite     MoB-ELM     483   0.51      0.00        97      .0.9302     0.9314      0.9353  -0.0030     0.9302      144077     0.0037
% Acolite     SeaDAS      483   0.51      76.36       98      .0.8739     0.8779      1.1269  -0.0041     0.8739      145692     0.0028
% Acolite     MUMM        483   0.51      74.29       99      .0.9175     0.9192      1.0693  -0.0024     0.9175      147837     0.0018
% SeaDAS      MoB-ELM     483   43.76     0.00        55      .0.8454     0.8514      0.8547  +0.0002     0.8454      142116     0.0014
% SeaDAS      MUMM        483   43.76     42.57       56      .0.9122     0.9141      0.9441  +0.0015     0.9122      145888     0.0012
% MUMM        MoB-ELM     483   42.57     0.00        56      .0.8408     0.8471      0.8639  -0.0007     0.8408      145138     0.0022
% Acolite     MoB-ELM     561   0.44      0.00        97      .0.9044     0.9067      1.0183  -0.0011     0.9044      144192     0.0011
% Acolite     SeaDAS      561   0.44      75.56       99      .0.8233     0.8311      1.0307  -0.0013     0.8233      146761     0.0013
% Acolite     MUMM        561   0.44      74.15       100     .0.8968     0.8995      0.9663  -0.0001     0.8968      148040     0.0007
% SeaDAS      MoB-ELM     561   43.30     0.00        55      .0.7828     0.7946      1.0043  +0.0001     0.7828      143288     0.0009
% SeaDAS      MUMM        561   43.30     42.49       57      .0.8728     0.8768      0.9374  +0.0011     0.8728      147116     0.0010
% MUMM        MoB-ELM     561   42.49     0.00        56      .0.8468     0.8527      1.0611  -0.0010     0.8468      145339     0.0009
% Acolite     MoB-ELM     655   0.49      0.00        97      .0.8592     0.8641      1.1703  -0.0017     0.8592      144108     0.0013
% Acolite     SeaDAS      655   0.49      76.22       98      .0.7090     0.7301      0.9628  -0.0011     0.7090      145798     0.0014
% Acolite     MUMM        655   0.49      74.15       99      .0.7880     0.7992      0.9700  -0.0008     0.7880      147956     0.0010
% SeaDAS      MoB-ELM     655   43.68     0.00        55      .0.7708     0.7840      1.2158  -0.0004     0.7708      142375     0.0007
% SeaDAS      MUMM        655   43.68     42.49       56      .0.7964     0.8067      1.0063  +0.0004     0.7964      146144     0.0007
% MUMM        MoB-ELM     655   42.49     0.00        56      .0.6841     0.7091      1.2403  -0.0008     0.6841      145340     0.0009   
 
\begin{table}[!ht]
\vspace{.3cm}
\caption{ Comparison of the different atmospheric correction methods for retrieving $R_{rs}$ with the slope and offset for the regression lines and their respective goodness of fit values. \label{tab:RrsCompMethod} } 
\centering
\vspace{.2cm}
\tiny
\begin{tabular}{cllcccccccc} 
 % \bfseries{Band n} & \bfseries{$m$}      & \bfseries{$y_0$}    & \bfseries{$R^2$}     & \bfseries{$RMSE$} & $y(x=45^\circ)$   \\ \hline \hline
Band    &   Method 1      &  Method 2   & Slope   & Offset  & $R^2 $  & N       & RMSE    &\multicolumn{2}{c}{$R_{rs}<0~[\%]$}   &   Used    \\ 
$[nm]$    &             &         &       &     &       &       &$[mg/m^3]$ & Method 1    & Method 2         & $[\%]$    \\ \hline \hline
\multirow{6}{*}{443}&Acolite-SWIR&MoB-ELM     & 0.9444  & -0.0047 & 0.8791  & 144047  & 0.0054  &  ~0.53      & ~0.09            &   97      \\
      &   SeaDAS-SWIR   &  MoB-ELM      & 0.8664  & -0.0007 & 0.7637  & 141563  & 0.0019  &  43.98      & ~0.05            &   55      \\
      &   SeaDAS-MUMM   &  MoB-ELM      & 0.8811  & -0.0018 & 0.7474  & 144947  & 0.0029  &  42.64      & ~0.05            &   56      \\
      &   Acolite-SWIR  &  SeaDAS-SWIR  & 1.1437  & -0.0053 & 0.7804  & 145186  & 0.0038  &  ~0.53      & 76.74            &   98      \\
      &   Acolite-SWIR  &  SeaDAS-MUMM  & 1.0600  & -0.0032 & 0.8554  & 147730  & 0.0026  &  ~0.53      & 74.41            &   99      \\
      &   SeaDAS-SWIR   &  SeaDAS-MUMM  & 0.9175  & ~0.0018 & 0.8456  & 145315  & 0.0014  &  43.98      & 42.64            &   56      \\  \hline
\multirow{6}{*}{448}&Acolite-SWIR&MoB-ELM     & 0.9353  & -0.0030 & \underline{0.9302}  & 144077  & 0.0037  &  ~0.51      & ~0.00            &   97      \\
      &   SeaDAS-SWIR   &  MoB-ELM      & 0.8547  & ~0.0002 & 0.8454  & 142116  & 0.0014  &  43.76      & ~0.00            &   55      \\
      &   SeaDAS-MUMM   &  MoB-ELM      & 0.8639  & -0.0007 & 0.8408  & 145138  & 0.0022  &  42.57      & ~0.00            &   56      \\ 
      &   Acolite-SWIR  &  SeaDAS-SWIR  & 1.1269  & -0.0041 & 0.8739  & 145692  & 0.0028  &  ~0.51      & 76.36            &   98      \\
      &   Acolite-SWIR  &  SeaDAS-MUMM  & 1.0693  & -0.0024 & 0.9175  & 147837  & 0.0018  &  ~0.51      & 74.29            &   99      \\
      &   SeaDAS-SWIR   &  SeaDAS-MUMM  & 0.9441  & ~0.0015 & 0.9122  & 145888  & 0.0012  &  43.76      & 42.57            &   56      \\ \hline
\multirow{6}{*}{561}&Acolite-SWIR&  MoB-ELM   & 1.0183  & -0.0011 & 0.9044  & 144192  & 0.0011  &  ~0.44      & ~0.00            &   97      \\
      &   SeaDAS-SWIR   &  MoB-ELM      & 1.0043  & ~0.0001 & 0.7828  & 143288  & 0.0009  &  43.30      & ~0.00            &   55      \\
      &   SeaDAS-MUMM   &  MoB-ELM      & 1.0611  & -0.0010 & 0.8468  & 145339  & 0.0009  &  42.49      & ~0.00            &   56      \\
      &   Acolite-SWIR  &  SeaDAS-SWIR  & 1.0307  & -0.0013 & 0.8233  & 146761  & 0.0013  &  ~0.44      & 75.56            &   99      \\
      &   Acolite-SWIR  &  SeaDAS-MUMM  & 0.9663  & -0.0001 & 0.8968  & 148040  & 0.0007  &  ~0.44      & 74.15            &   100     \\
        &   SeaDAS-SWIR   &  SeaDAS-MUMM  & 0.9374  & ~0.0011 & 0.8728  & 147116  & 0.0010  &  43.30      & 42.49            &   57      \\ \hline
\multirow{6}{*}{655}&Acolite-SWIR&MoB-ELM     & 1.1703  & -0.0017 & 0.8592  & 144108  & 0.0013  &  ~0.49      & ~0.00            &   97      \\
      &   SeaDAS-SWIR   &  MoB-ELM      & 1.2158  & -0.0004 & 0.7708  & 142375  & 0.0007  &  43.68      & ~0.00            &   55      \\
      &   SeaDAS-MUMM   &  MoB-ELM      & 1.2403  & -0.0008 & 0.6841  & 145340  & 0.0009  &  42.49      & ~0.00            &   56      \\ 
      &   Acolite-SWIR  &  SeaDAS-SWIR  & 0.9628  & -0.0011 & \underline{0.7090}  & 145798  & 0.0014  &  ~0.49      & 76.22            &   98      \\
      &   Acolite-SWIR  &  SeaDAS-MUMM  & 0.9700  & -0.0008 & 0.7880  & 147956  & 0.0010  &  ~0.49      & 74.15            &   99      \\
      &   SeaDAS-SWIR   &  SeaDAS-MUMM  & 1.0063  & ~0.0004 & 0.7964  & 146144  & 0.0007  &  43.68      & 42.49            &   56      \\
 \end{tabular}
\end{table}
%^^^^^^^^^^^^^^^^^^^  FIGURE ^^^^^^^^^^^^^^^^^^^^^^^^^^^^^^^^^^^^^^^^^^^^
\begin{figure}[htbp!]
  \begin{minipage}[c]{0.48\linewidth}
      \centering
      \begin{overpic}[trim=0 280 0 0,clip,width=7.0cm]{/Users/javier/Desktop/Javier/PHD_RIT/ConferencesAndApplications/2015_SPIE_SanDiego/Images/2013262_ACOMOBSEAMUM_443_Acolite-SWIR_SeaDAS-MUMM.png}
      % \put (65,17) {\large A) $443nm$}
      \end{overpic}  
  \end{minipage}
  \hfill
  \begin{minipage}[d]{0.48\linewidth}
    \centering
      \begin{overpic}[trim=0 280 0 0,clip,width=7.0cm]{/Users/javier/Desktop/Javier/PHD_RIT/ConferencesAndApplications/2015_SPIE_SanDiego/Images/2013262_ACOMOBSEAMUM_443_Acolite-SWIR_MoB-ELM.png}
      % \put (65,17) {\large B) $483nm$}
      \end{overpic}
  \end{minipage}

  \begin{minipage}[c]{0.48\linewidth}
      \centering
      \begin{overpic}[trim=0 280 0 0,clip,width=7.0cm]{/Users/javier/Desktop/Javier/PHD_RIT/ConferencesAndApplications/2015_SPIE_SanDiego/Images/2013262_ACOMOBSEAMUM_443_Acolite-SWIR_SeaDAS-SWIR.png}
      % \put (65,17) {\large C) $561nm$}
      \end{overpic}  
  \end{minipage}
  \hfill
  \begin{minipage}[d]{0.48\linewidth}
    \centering
      \begin{overpic}[trim=0 280 0 0,clip,width=7.0cm]{/Users/javier/Desktop/Javier/PHD_RIT/ConferencesAndApplications/2015_SPIE_SanDiego/Images/2013262_ACOMOBSEAMUM_443_SeaDAS-MUMM_MoB-ELM.png}
      % \put (65,17) {\large D) $655nm$}
      \end{overpic}
  \end{minipage}

  \begin{minipage}[c]{0.48\linewidth}
      \centering
      \begin{overpic}[trim=0 280 0 0,clip,width=7.0cm]{/Users/javier/Desktop/Javier/PHD_RIT/ConferencesAndApplications/2015_SPIE_SanDiego/Images/2013262_ACOMOBSEAMUM_443_SeaDAS-SWIR_SeaDAS-MUMM.png}
      % \put (65,17) {\large C) $561nm$}
      \end{overpic}  
  \end{minipage}
  \hfill
  \begin{minipage}[d]{0.48\linewidth}
    \centering
      \begin{overpic}[trim=0 280 0 0,clip,width=7.0cm]{/Users/javier/Desktop/Javier/PHD_RIT/ConferencesAndApplications/2015_SPIE_SanDiego/Images/2013262_ACOMOBSEAMUM_443_SeaDAS-SWIR_MoB-ELM.png}
      % \put (65,17) {\large D) $655nm$}
      \end{overpic}
  \end{minipage}

  \begin{minipage}[d]{1.0\linewidth}
    \centering
      \begin{overpic}[trim=70 0 0 1470,clip,width=7.0cm]{/Users/javier/Desktop/Javier/PHD_RIT/ConferencesAndApplications/2015_SPIE_SanDiego/Images/2013262_ACOMOBSEAMUM_655_Acolite-SWIR_SeaDAS-MUMM.png}
      \end{overpic}
  \end{minipage}    

% 
  \caption{Scatter plots showing the comparison of remote-sensing reflectance ($R_{rs}$) at 443 nm, derived from the 09-29-2013 image over the Rochester Embayment (scene LC80160302013262LGN00) using the different methods. Colors denote pixel densities, the dashed black line is the 1:1 line, and the Reduced Major Axis (RMA) regression line is drawn in red. \label{fig:13262Rrs443} } 
\end{figure}

%^^^^^^^^^^^^^^^^^^^  FIGURE ^^^^^^^^^^^^^^^^^^^^^^^^^^^^^^^^^^^^^^^^^^^^
\begin{figure}[htbp!]
  \begin{minipage}[c]{0.48\linewidth}
      \centering
      \begin{overpic}[trim=0 280 0 0,clip,width=7.0cm]{/Users/javier/Desktop/Javier/PHD_RIT/ConferencesAndApplications/2015_SPIE_SanDiego/Images/2013262_ACOMOBSEAMUM_483_Acolite-SWIR_SeaDAS-MUMM.png}
      % \put (65,17) {\large A) $443nm$}
      \end{overpic}  
  \end{minipage}
  \hfill
  \begin{minipage}[d]{0.48\linewidth}
    \centering
      \begin{overpic}[trim=0 280 0 0,clip,width=7.0cm]{/Users/javier/Desktop/Javier/PHD_RIT/ConferencesAndApplications/2015_SPIE_SanDiego/Images/2013262_ACOMOBSEAMUM_483_Acolite-SWIR_MoB-ELM.png}
      % \put (65,17) {\large B) $483nm$}
      \end{overpic}
  \end{minipage}

  \begin{minipage}[c]{0.48\linewidth}
      \centering
      \begin{overpic}[trim=0 280 0 0,clip,width=7.0cm]{/Users/javier/Desktop/Javier/PHD_RIT/ConferencesAndApplications/2015_SPIE_SanDiego/Images/2013262_ACOMOBSEAMUM_483_Acolite-SWIR_SeaDAS-SWIR.png}
      % \put (65,17) {\large C) $483nm$}
      \end{overpic}  
  \end{minipage}
  \hfill
  \begin{minipage}[d]{0.48\linewidth}
    \centering
      \begin{overpic}[trim=0 280 0 0,clip,width=7.0cm]{/Users/javier/Desktop/Javier/PHD_RIT/ConferencesAndApplications/2015_SPIE_SanDiego/Images/2013262_ACOMOBSEAMUM_483_SeaDAS-MUMM_MoB-ELM.png}
      % \put (65,17) {\large D) $655nm$}
      \end{overpic}
  \end{minipage}

  \begin{minipage}[c]{0.48\linewidth}
      \centering
      \begin{overpic}[trim=0 280 0 0,clip,width=7.0cm]{/Users/javier/Desktop/Javier/PHD_RIT/ConferencesAndApplications/2015_SPIE_SanDiego/Images/2013262_ACOMOBSEAMUM_483_SeaDAS-SWIR_SeaDAS-MUMM.png}
      % \put (65,17) {\large C) $483nm$}
      \end{overpic}  
  \end{minipage}
  \hfill
  \begin{minipage}[d]{0.48\linewidth}
    \centering
      \begin{overpic}[trim=0 280 0 0,clip,width=7.0cm]{/Users/javier/Desktop/Javier/PHD_RIT/ConferencesAndApplications/2015_SPIE_SanDiego/Images/2013262_ACOMOBSEAMUM_483_SeaDAS-SWIR_MoB-ELM.png}
      % \put (65,17) {\large D) $655nm$}
      \end{overpic}
  \end{minipage}

  \begin{minipage}[d]{1.0\linewidth}
    \centering
      \begin{overpic}[trim=70 0 0 1470,clip,width=8.0cm]{/Users/javier/Desktop/Javier/PHD_RIT/ConferencesAndApplications/2015_SPIE_SanDiego/Images/2013262_ACOMOBSEAMUM_655_Acolite-SWIR_SeaDAS-MUMM.png}
      \end{overpic}
  \end{minipage}    

% 
  \caption{Scatter plots showing the comparison of remote-sensing reflectance ($R_{rs}$) at 483 nm, derived from the 09-29-2013 image over the Rochester Embayment (scene LC80160302013262LGN00) using the the different methods. Colors denote pixel densities, the dashed black line is the 1:1 line, and the Reduced Major Axis (RMA) regression line is drawn in red. \label{fig:13262Rrs483} } 
\end{figure}

%^^^^^^^^^^^^^^^^^^^  FIGURE ^^^^^^^^^^^^^^^^^^^^^^^^^^^^^^^^^^^^^^^^^^^^
\begin{figure}[htbp!]
  \begin{minipage}[c]{0.48\linewidth}
      \centering
      \begin{overpic}[trim=0 280 0 0,clip,width=7.0cm]{/Users/javier/Desktop/Javier/PHD_RIT/ConferencesAndApplications/2015_SPIE_SanDiego/Images/2013262_ACOMOBSEAMUM_561_Acolite-SWIR_SeaDAS-MUMM.png}
      % \put (65,17) {\large A) $443nm$}
      \end{overpic}  
  \end{minipage}
  \hfill
  \begin{minipage}[d]{0.48\linewidth}
    \centering
      \begin{overpic}[trim=0 280 0 0,clip,width=7.0cm]{/Users/javier/Desktop/Javier/PHD_RIT/ConferencesAndApplications/2015_SPIE_SanDiego/Images/2013262_ACOMOBSEAMUM_561_Acolite-SWIR_MoB-ELM.png}
      % \put (65,17) {\large B) $483nm$}
      \end{overpic}
  \end{minipage}

  \begin{minipage}[c]{0.48\linewidth}
      \centering
      \begin{overpic}[trim=0 280 0 0,clip,width=7.0cm]{/Users/javier/Desktop/Javier/PHD_RIT/ConferencesAndApplications/2015_SPIE_SanDiego/Images/2013262_ACOMOBSEAMUM_561_Acolite-SWIR_SeaDAS-SWIR.png}
      % \put (65,17) {\large C) $561nm$}
      \end{overpic}  
  \end{minipage}
  \hfill
  \begin{minipage}[d]{0.48\linewidth}
    \centering
      \begin{overpic}[trim=0 280 0 0,clip,width=7.0cm]{/Users/javier/Desktop/Javier/PHD_RIT/ConferencesAndApplications/2015_SPIE_SanDiego/Images/2013262_ACOMOBSEAMUM_561_SeaDAS-MUMM_MoB-ELM.png}
      % \put (65,17) {\large D) $655nm$}
      \end{overpic}
  \end{minipage}

  \begin{minipage}[c]{0.48\linewidth}
      \centering
      \begin{overpic}[trim=0 280 0 0,clip,width=7.0cm]{/Users/javier/Desktop/Javier/PHD_RIT/ConferencesAndApplications/2015_SPIE_SanDiego/Images/2013262_ACOMOBSEAMUM_561_SeaDAS-SWIR_SeaDAS-MUMM.png}
      % \put (65,17) {\large C) $561nm$}
      \end{overpic}  
  \end{minipage}
  \hfill
  \begin{minipage}[d]{0.48\linewidth}
    \centering
      \begin{overpic}[trim=0 280 0 0,clip,width=7.0cm]{/Users/javier/Desktop/Javier/PHD_RIT/ConferencesAndApplications/2015_SPIE_SanDiego/Images/2013262_ACOMOBSEAMUM_561_SeaDAS-SWIR_MoB-ELM.png}
      % \put (65,17) {\large D) $655nm$}
      \end{overpic}
  \end{minipage}

  \begin{minipage}[d]{1.0\linewidth}
    \centering
      \begin{overpic}[trim=70 0 0 1470,clip,width=8.0cm]{/Users/javier/Desktop/Javier/PHD_RIT/ConferencesAndApplications/2015_SPIE_SanDiego/Images/2013262_ACOMOBSEAMUM_655_Acolite-SWIR_SeaDAS-MUMM.png}
      \end{overpic}
  \end{minipage}    

% 
  \caption{Scatter plots showing the comparison of remote-sensing reflectance ($R_{rs}$) at 561 nm, derived from the 09-29-2013 image over the Rochester Embayment (scene LC80160302013262LGN00) using the the different methods. Colors denote pixel densities, the dashed black line is the 1:1 line, and the Reduced Major Axis (RMA) regression line is drawn in red. \label{fig:13262Rrs561} } 
\end{figure}

%^^^^^^^^^^^^^^^^^^^  FIGURE ^^^^^^^^^^^^^^^^^^^^^^^^^^^^^^^^^^^^^^^^^^^^
\begin{figure}[htbp!]
  \begin{minipage}[c]{0.48\linewidth}
      \centering
      \begin{overpic}[trim=0 280 0 0,clip,width=7.0cm]{/Users/javier/Desktop/Javier/PHD_RIT/ConferencesAndApplications/2015_SPIE_SanDiego/Images/2013262_ACOMOBSEAMUM_655_Acolite-SWIR_SeaDAS-MUMM.png}
      % \put (65,17) {\large A) $443nm$}
      \end{overpic}  
  \end{minipage}
  \hfill
  \begin{minipage}[d]{0.48\linewidth}
    \centering
      \begin{overpic}[trim=0 280 0 0,clip,width=7.0cm]{/Users/javier/Desktop/Javier/PHD_RIT/ConferencesAndApplications/2015_SPIE_SanDiego/Images/2013262_ACOMOBSEAMUM_655_Acolite-SWIR_MoB-ELM.png}
      % \put (65,17) {\large B) $483nm$}
      \end{overpic}
  \end{minipage}

  \begin{minipage}[c]{0.48\linewidth}
      \centering
      \begin{overpic}[trim=0 280 0 0,clip,width=7.0cm]{/Users/javier/Desktop/Javier/PHD_RIT/ConferencesAndApplications/2015_SPIE_SanDiego/Images/2013262_ACOMOBSEAMUM_655_Acolite-SWIR_SeaDAS-SWIR.png}
      % \put (65,17) {\large C) $655nm$}
      \end{overpic}  
  \end{minipage}
  \hfill
  \begin{minipage}[d]{0.48\linewidth}
    \centering
      \begin{overpic}[trim=0 280 0 0,clip,width=7.0cm]{/Users/javier/Desktop/Javier/PHD_RIT/ConferencesAndApplications/2015_SPIE_SanDiego/Images/2013262_ACOMOBSEAMUM_655_SeaDAS-MUMM_MoB-ELM.png}
      % \put (65,17) {\large D) $655nm$}
      \end{overpic}
  \end{minipage}

  \begin{minipage}[c]{0.48\linewidth}
      \centering
      \begin{overpic}[trim=0 280 0 0,clip,width=7.0cm]{/Users/javier/Desktop/Javier/PHD_RIT/ConferencesAndApplications/2015_SPIE_SanDiego/Images/2013262_ACOMOBSEAMUM_655_SeaDAS-SWIR_SeaDAS-MUMM.png}
      % \put (65,17) {\large C) $655nm$}
      \end{overpic}  
  \end{minipage}
  \hfill
  \begin{minipage}[d]{0.48\linewidth}
    \centering
      \begin{overpic}[trim=0 280 0 0,clip,width=7.0cm]{/Users/javier/Desktop/Javier/PHD_RIT/ConferencesAndApplications/2015_SPIE_SanDiego/Images/2013262_ACOMOBSEAMUM_655_SeaDAS-SWIR_MoB-ELM.png}
      % \put (65,17) {\large D) $655nm$}
      \end{overpic}
  \end{minipage}

  \begin{minipage}[d]{1.0\linewidth}
    \centering
      \begin{overpic}[trim=70 0 0 1470,clip,width=8.0cm]{/Users/javier/Desktop/Javier/PHD_RIT/ConferencesAndApplications/2015_SPIE_SanDiego/Images/2013262_ACOMOBSEAMUM_655_Acolite-SWIR_SeaDAS-MUMM.png}
      \end{overpic}
  \end{minipage}    

% 
  \caption{Scatter plots showing the comparison of remote-sensing reflectance ($R_{rs}$) at 655 nm, derived from the 09-29-2013 image over the Rochester Embayment (scene LC80160302013262LGN00) using the the different methods. Colors denote pixel densities, the dashed black line is the 1:1 line, and the Reduced Major Axis (RMA) regression line is drawn in red. \label{fig:13262Rrs655} } 
\end{figure}

%^^^^^^^^^^^^^^^^^^^  FIGURE ^^^^^^^^^^^^^^^^^^^^^^^^^^^^^^^^^^^^^^^^^^^^
\begin{figure}[htbp!]
  \begin{minipage}[c]{0.48\linewidth}
      \centering
      \begin{overpic}[trim=0 0 0 0,clip,width=7cm]{/Users/javier/Desktop/Javier/PHD_RIT/ConferencesAndApplications/2015_SPIE_SanDiego/Images/RrsCompONTNS.png}
      \put (20,60) {A) ONTNS} 
      \end{overpic}  
  \end{minipage}
  \hfill
  \begin{minipage}[d]{0.48\linewidth}
    \centering
      \begin{overpic}[trim=0 0 0 0,clip,width=7cm]{/Users/javier/Desktop/Javier/PHD_RIT/ConferencesAndApplications/2015_SPIE_SanDiego/Images/RrsCompONTOS.png}
      \put (20,14) {B) ONTOS}     
      \end{overpic}
  \end{minipage}
  
  \begin{minipage}[d]{0.48\linewidth}
    \centering
      \begin{overpic}[trim=0 0 0 0,clip,width=7cm]{/Users/javier/Desktop/Javier/PHD_RIT/ConferencesAndApplications/2015_SPIE_SanDiego/Images/RrsCompONTEX.png}
      \put (20,14) {C) ONTEX}   
      \end{overpic}
  \end{minipage}
  \hfill
  \begin{minipage}[c]{0.48\linewidth}
      \centering
      \begin{overpic}[trim=0 0 0 0,clip,width=7cm]{/Users/javier/Desktop/Javier/PHD_RIT/ConferencesAndApplications/2015_SPIE_SanDiego/Images/RrsCompRVRPL.png}
      \put (20,14) {D) RVRPLM}      
      \end{overpic}  
  \end{minipage}

  \begin{minipage}[c]{1.0\linewidth}
      \centering
      \begin{overpic}[trim=0 0 0 0,clip,width=7cm]{/Users/javier/Desktop/Javier/PHD_RIT/ConferencesAndApplications/2015_SPIE_SanDiego/Images/RrsCompRVRPI.png}
      \put (20,14) {E) RVRPIER}     
      \end{overpic}  
  \end{minipage}
  \caption{Comparison of $R_{rs}$ spectra with {\it in situ} data for the sites on the 09-19-2013 collection. \label{fig:13262RrsCompField}} 
\end{figure}

%^^^^^^^^^^^^^^^^^^^  FIGURE ^^^^^^^^^^^^^^^^^^^^^^^^^^^^^^^^^^^^^^^^^^^^
\begin{figure}[htbp!]
  \begin{minipage}[d]{0.48\linewidth}
    \centering
      \begin{overpic}[trim=0 0 0 0,clip,width=7cm]{/Users/javier/Desktop/Javier/PHD_RIT/ConferencesAndApplications/2015_SPIE_SanDiego/Images/RrsCompLONGN.png}
      \put (20,60) {A) LONGN}   
      \end{overpic}
  \end{minipage}
  \hfill
  \begin{minipage}[d]{0.48\linewidth}
    \centering
      \begin{overpic}[trim=0 0 0 0,clip,width=7cm]{/Users/javier/Desktop/Javier/PHD_RIT/ConferencesAndApplications/2015_SPIE_SanDiego/Images/RrsCompLONGS.png}
      \put (20,60) {B) LONGS}
      \end{overpic}
  \end{minipage}
  
   \begin{minipage}[c]{0.48\linewidth}
      \centering
      \begin{overpic}[trim=0 0 0 0,clip,width=7cm]{/Users/javier/Desktop/Javier/PHD_RIT/ConferencesAndApplications/2015_SPIE_SanDiego/Images/RrsCompCRANB.png}
      \put (45,14) {C) CRANB}     
      \end{overpic}  
  \end{minipage}
  \hfill
  \begin{minipage}[d]{0.48\linewidth}
    \centering
      \begin{overpic}[trim=0 0 0 0,clip,width=7cm]{/Users/javier/Desktop/Javier/PHD_RIT/ConferencesAndApplications/2015_SPIE_SanDiego/Images/RrsCompBRADI.png}
      \put (45,14) {D) BRADIN}
      \end{overpic}
  \end{minipage}
  
  \begin{minipage}[d]{1.0\linewidth}
    \centering
      \begin{overpic}[trim=0 0 0 0,clip,width=7cm]{/Users/javier/Desktop/Javier/PHD_RIT/ConferencesAndApplications/2015_SPIE_SanDiego/Images/RrsCompBRADO.png}
      \put (45,14) {E) BRADONT}
      \end{overpic}
  \end{minipage}    

% 
  \caption{Comparison of $R_{rs}$ spectra. \label{fig:13262RrsComp}} 
\end{figure}


%^^^^^^^^^^^^^^^^^^^  FIGURE ^^^^^^^^^^^^^^^^^^^^^^^^^^^^^^^^^^^^^^^^^^^^

\begin{figure}[htbp!]
  \begin{minipage}[c]{0.48\linewidth}
      \centering
      \begin{overpic}[trim=110 0 140 0,clip,width=7.0cm]{/Users/javier/Desktop/Javier/PHD_RIT/ConferencesAndApplications/2015_SPIE_SanDiego/Images/NRMSE_RRS_B1.png}
      \put (20,40) {A) 443nm} 
      \end{overpic}  
  \end{minipage}
  \hfill
  \begin{minipage}[d]{0.48\linewidth}
    \centering
      \begin{overpic}[trim=110 0 140 0,clip,width=7.0cm]{/Users/javier/Desktop/Javier/PHD_RIT/ConferencesAndApplications/2015_SPIE_SanDiego/Images/NRMSE_RRS_B2.png}
      \put (20,40) {B) 483nm}     
      \end{overpic}
  \end{minipage}

    \begin{minipage}[c]{0.48\linewidth}
      \centering
      \begin{overpic}[trim=110 0 140 0,clip,width=7.0cm]{/Users/javier/Desktop/Javier/PHD_RIT/ConferencesAndApplications/2015_SPIE_SanDiego/Images/NRMSE_RRS_B3.png}
      \put (20,40) {A) 561nm} 
      \end{overpic}  
  \end{minipage}
  \hfill
  \begin{minipage}[d]{0.48\linewidth}
    \centering
      \begin{overpic}[trim=110 0 140 0,clip,width=7.0cm]{/Users/javier/Desktop/Javier/PHD_RIT/ConferencesAndApplications/2015_SPIE_SanDiego/Images/NRMSE_RRS_B4.png}
      \put (20,40) {B) 655nm}     
      \end{overpic}
  \end{minipage}

  \caption{Comparison of the retrieved results for $R_{rs}$ with {\it in situ} data using the normalized root mean squared error (NRMSE) for $R_{rs}$ at 443, 483, 561 and 655nm. \label{fig:NRMSE130919_RRS} } 
\end{figure}



% ------------------------------
\section{Retrieval}

Preliminary results for concentration maps for each CPA over the Rochester Embayment, Rochester, NY are shown in Figure~\ref{fig:retrievalresults}. The expected trend of having low concentration of CPAs in the offshore of Lake Ontario and higher concentrations in the nearby ponds (Long Pond and Cranberry Pond) can be seen. 
\begin{figure}[htb]
\centering
\includegraphics[trim=200 100 180 0,clip,height=9cm]{/Users/javier/Desktop/Javier/PHD_RIT/ConferencesAndApplications/2014_RITResearchSymposium/Images/RetrievalResults.eps}
   \caption{Retrieval preliminary results.}
      \label{fig:retrievalresults}   
\end{figure}

Comparisons between retrieved CPAs concentrations and field measurements are shown in Figure~\ref{fig:chlcomp}, Figure~\ref{fig:tsscomp} and Figure~\ref{fig:cdomcomp} for four different stations in the area of study. This comparison with field measurements showed good agreement at low concentrations but differences at higher concentrations. Ongoing work is focusing on incorporation of the IOP differences between water bodies in the LUT optimization process.
\begin{figure}[htb]
\centering
    \includegraphics[height=7cm]{/Users/javier/Desktop/Javier/PHD_RIT/ConferencesAndApplications/IGARSS2014/paper/Images/chlcomp.eps} 
    \caption{Comparison between measured and retrieved chlorophyll {\it a} concentration.}
    \label{fig:chlcomp} 
\end{figure}     

\begin{figure}[htb]
\centering
    \includegraphics[height=7cm]{/Users/javier/Desktop/Javier/PHD_RIT/ConferencesAndApplications/IGARSS2014/paper/Images/tsscomp.eps}   
    \caption{Comparison between measured and retrieved TSS concentration.}
    \label{fig:tsscomp} 
\end{figure}  

\begin{figure}[htb]
\centering
    \includegraphics[height=7cm]{/Users/javier/Desktop/Javier/PHD_RIT/ConferencesAndApplications/IGARSS2014/paper/Images/cdomcomp.eps}    
    \caption{Comparison between measured and retrieved CDOM concentration.}
    \label{fig:cdomcomp} 
\end{figure}  

% ------------------------------
\subsection{$C_a$ Comparison}
The next step was to apply the algorithms described in Section~\ref{sec:chlretrieval} in order to retrieve chlorophyll-{\it a} concentration ($C_a$) from the $R_{rs}$ data. The spectral matching and LUT approach was applied to the $R_{rs}$ data retrieved from the MoB-ELM, while the OC3 algorithm was applied to data obtained from the rest of the algorithms. \autoref{fig:chlor_amaps} shows the $C_a$ maps obtained with the different methods. All the methods behave in a similar way for the clear water (lake), although some variability may not be visually noticeable due to the way the image is displayed. For the more turbid waters (ponds), the algorithms differ quite significantly. \autoref{fig:13262RrsMeaVSRet} shows a scatterplot of $C_a$ measured in the field versus the retrieved $C_a$ from the different algorithms. It can be seen that the retrieved values obtained from the MoB-ELM algorithm are closer to the measured values than the rest of the algorithms for the higher concentrations. However, the traditional algorithms generally perform well for the low concentration cases. Similarly, we calculated the normalized root mean squared error (NRMSE) for $C_a$, defined as

\begin{equation}
\label{eq:NRMSEchl}
  NRMSE =\frac{\sqrt{\frac{1}{N}\sum_{n=1}^N{\left[C_{ret}(n) - C_{mea}(n)\right]^2}}}{max\{C_{mea}(n)\} - min\{C_{mea}(n)\}}\times100 ~[\%]
\end{equation}

\noindent where $C_{ret}$ is the retrieved concentration, $C_{mea}$ is the measured concentration, and $n=1\dots N$ is the number of measured concentrations. The NRMSE for the $C_a$ for each algorithm are shown in \autoref{fig:NRMSE130919CHL}. The best results are obtained from the MoB-ELM method with overall values less than $10\%$.
%-------------
%^^^^^^^^^^^^^^^^^^^  FIGURE ^^^^^^^^^^^^^^^^^^^^^^^^^^^^^^^^^^^^^^^^^^^^
\begin{figure}[htbp!]
  \begin{minipage}[c]{0.48\linewidth}
      \centering
      \begin{overpic}[trim=0 0 40 0,clip,width=7.0cm]{/Users/javier/Desktop/Javier/PHD_RIT/ConferencesAndApplications/2015_SPIE_SanDiego/Images/Collocated13262_ACOSWIR_MOB_SEA5x5_MUMM45_chlor_MOB_D_R_R}
      \put (5,5) {A) MOB-ELM}
      \end{overpic}
    \end{minipage}
    \hfill
  \begin{minipage}[c]{0.48\linewidth}
      \centering
      \begin{overpic}[trim=0 0 40 0,clip,width=7.0cm]{/Users/javier/Desktop/Javier/PHD_RIT/ConferencesAndApplications/2015_SPIE_SanDiego/Images/LC80160302013262LGN00_L2_SWIR_FranzAve_chlor_a_ACO_OC3def}
      \put (5,5) {B) Acolite-SWIR}
      \end{overpic}
    \end{minipage}

    \vspace{0.7cm}

  \begin{minipage}[c]{0.48\linewidth}
      \centering
      \begin{overpic}[trim=0 0 40 0,clip,width=7.0cm]{/Users/javier/Desktop/Javier/PHD_RIT/ConferencesAndApplications/2015_SPIE_SanDiego/Images/Collocated13262_ACOSWIR_MOB_SEA5x5_MUMM45_chlor_a_SEA5x5_R}
      \put (5,5) {C) SeaDAS-SWIR}
      \end{overpic}
    \end{minipage}
    \hfill
  \begin{minipage}[c]{0.48\linewidth}
      \centering
      \begin{overpic}[trim=0 0 40 0,clip,width=7.0cm]{/Users/javier/Desktop/Javier/PHD_RIT/ConferencesAndApplications/2015_SPIE_SanDiego/Images/Collocated13262_ACOSWIR_MOB_SEA5x5_MUMM45_chlor_a_MUMM45}
      \put (5,5) {D) SeaDAS-MUMM}
      \end{overpic}
    \end{minipage}
    
    \begin{minipage}[c]{1.0\linewidth}
      \centering
      \vspace{0.5cm}
      \begin{overpic}[trim=0 0 0 0,clip,height=1.2cm]{/Users/javier/Desktop/Javier/PHD_RIT/ConferencesAndApplications/2015_SPIE_SanDiego/Images/Collocated13262_ACOSWIR_MOB_SEA5x5_MUMM45_colorbar_CHL_0_100}
      \put (35,16) {$C_a [mg/m^3]$}
      \end{overpic}
    \end{minipage}

  \caption{$C_a$ retrieved from the different atmospheric correction methods: A) MoB-ELM, B) Acolite-SWIR, C) SeaDAS-SWIR and D) SeaDAS-MUMM. The spectral matching and LUT approach was applied to $R_{rs}$ data from MoB-ELM method. The OC3 method was applied to the $R_{rs}$ data from the Acolite-SWIR, SeaDAS-SWIR and SeaDAS-MUMM.\label{fig:chlor_amaps} } 
\end{figure}

%^^^^^^^^^^^^^^^^^^^  FIGURE ^^^^^^^^^^^^^^^^^^^^^^^^^^^^^^^^^^^^^^^^^^^^
\begin{figure}[htbp!]
  \begin{minipage}[c]{1.0\linewidth}
    \centering
      \begin{overpic}[trim=0 0 0 0,clip,width=10cm]{/Users/javier/Desktop/Javier/PHD_RIT/ConferencesAndApplications/2015_SPIE_SanDiego/Images/CHLmeavsret.png}
      \put(17,62){\includegraphics[trim=10 80 0 0,clip,width=3.5cm]{/Users/javier/Desktop/Javier/PHD_RIT/ConferencesAndApplications/2015_SPIE_SanDiego/Images/CHLmeavsretZOOM.png}}
      \end{overpic}  
  \end{minipage}
  \caption{Comparison retrieved versus measured $R_{rs}$ for the sites on the 90-19-2013 collection. \label{fig:13262RrsMeaVSRet}} 
\end{figure}
% %^^^^^^^^^^^^^^^^^^^  FIGURE ^^^^^^^^^^^^^^^^^^^^^^^^^^^^^^^^^^^^^^^^^^^^
\begin{figure}[htbp!]
  \begin{minipage}[c]{0.48\linewidth}
      \centering
      \begin{overpic}[trim=0 250 0 0,clip,width=7cm]{/Users/javier/Desktop/Javier/PHD_RIT/ConferencesAndApplications/2015_SPIE_SanDiego/Images/2013262_ACOMOBSEAMUM_C_a_Acolite-SWIR_SeaDAS-MUMM.png}
      \end{overpic}  
  \end{minipage}
  \hfill
  \begin{minipage}[d]{0.48\linewidth}
    \centering
      \begin{overpic}[trim=0 250 0 0,clip,width=7cm]{/Users/javier/Desktop/Javier/PHD_RIT/ConferencesAndApplications/2015_SPIE_SanDiego/Images/2013262_ACOMOBSEAMUM_C_a_Acolite-SWIR_MoB-ELM.png}
      \end{overpic}
  \end{minipage}

  \begin{minipage}[c]{0.48\linewidth}
      \centering
      \begin{overpic}[trim=0 250 0 0,clip,width=7cm]{/Users/javier/Desktop/Javier/PHD_RIT/ConferencesAndApplications/2015_SPIE_SanDiego/Images/2013262_ACOMOBSEAMUM_C_a_Acolite-SWIR_SeaDAS-SWIR.png}
      \end{overpic}  
  \end{minipage}
  \hfill
  \begin{minipage}[d]{0.48\linewidth}
    \centering
      \begin{overpic}[trim=0 250 0 0,clip,width=7cm]{/Users/javier/Desktop/Javier/PHD_RIT/ConferencesAndApplications/2015_SPIE_SanDiego/Images/2013262_ACOMOBSEAMUM_C_a_SeaDAS-MUMM_MoB-ELM.png}
      \end{overpic}
  \end{minipage}

  \begin{minipage}[c]{0.48\linewidth}
      \centering
      \begin{overpic}[trim=0 250 0 0,clip,width=7cm]{/Users/javier/Desktop/Javier/PHD_RIT/ConferencesAndApplications/2015_SPIE_SanDiego/Images/2013262_ACOMOBSEAMUM_C_a_SeaDAS-SWIR_SeaDAS-MUMM.png}
      \end{overpic}  
  \end{minipage}
  \hfill
  \begin{minipage}[d]{0.48\linewidth}
    \centering
      \begin{overpic}[trim=0 250 0 0,clip,width=7cm]{/Users/javier/Desktop/Javier/PHD_RIT/ConferencesAndApplications/2015_SPIE_SanDiego/Images/2013262_ACOMOBSEAMUM_C_a_SeaDAS-SWIR_MoB-ELM.png}
      \end{overpic}
  \end{minipage}

  \begin{minipage}[d]{1.0\linewidth}
    \centering
      \begin{overpic}[trim=0 0 0 1500,clip,width=7cm]{/Users/javier/Desktop/Javier/PHD_RIT/ConferencesAndApplications/2015_SPIE_SanDiego/Images/2013262_ACOMOBSEAMUM_C_a_SeaDAS-SWIR_MoB-ELM.png}
      \end{overpic}
  \end{minipage}    

\vspace{.5cm}
  \caption{$C_a$ comparison among all the four method analyzed in this work. \label{fig:13262Chlor} } 
\end{figure}
%^^^^^^^^^^^^^^^^^^^  FIGURE ^^^^^^^^^^^^^^^^^^^^^^^^^^^^^^^^^^^^^^^^^^^^
\begin{figure}[htbp!]
  \centering
  \includegraphics[height=8.0cm]{/Users/javier/Desktop/Javier/PHD_RIT/ConferencesAndApplications/2015_SPIE_SanDiego/Images/13262_NRMSE_CHL.png}
  \caption{Comparison of the retrieved results for $C_a$ with {\it in situ} data using the normalized root mean squared error (NRMSE). The $R_{rs}$ results from the MoB-ELM were used for the Concha and Schott's retrieval of $C_a$, while the results of the rest of the atmospheric correction algorithms were used for the NASA's bio-optical algorithm. The MoB-ELM results combined with the Concha and Schott's $C_a$ retrieval give the best results.\label{fig:NRMSE130919CHL} } 
\end{figure}

% ------------------------------
\section{Concluding Remarks}
\include{Future_Work}
% !TEX root=Thesis_PhD.tex 
% the previous is to reference main.bib
%% CHAPTER
\begin{appendices}
% @@@@@@@@@@@@@@@@@@@@@@@@@@@@@@@@@@@@@@@@@@@@@@@@@@@@@@@@@@@@@@@@@@@@@@@@@@@@@
\chapter{Field Measurements}
% \addcontentsline{toc}{chapter}{Appendix}
% \renewcommand{\thesection}{\Alph{section}}
% @@@@@@@@@@@@@@@@@@@@@@@@@@@@@@@@@@@@@@@@@@@@@@@@@@@@@@@@@@@@@@@@@@@@@@@@@@@@@
\label{ch:fieldmea}
% -----------------------------------------------------------------------------
\section{Water Samples}


\subsection{Equipment}

\begin{multicols}{3}
\begin{itemize}
  \item Dark Nalgene bottles
  \item Monroe County Environmental Lab bottles
  \item Cooler
  \item Marker
  \item Bottle label
  \item Ice packs
  \item GPS
  \item Extra batteries GPS
  \item Data sheet
  \item Pen
  \item Back pen
  \item Canoe 
  \item Transport straps for canoe
  \item Paddles
  \item Life jacket
  \item Suncream 
  \item Drinking water
  \item Wipes to clean extra suncream from hands
  \item Bucket with rope (in case is not possible to take water samples due to bad condition weather, for example)
\end{itemize}
\end{multicols}


\subsection{Procedure}
\begin{enumerate}
  \item Throughly clean the bottles prior to collection by brushing them inside with tap water a couple of times and rinse with DIW water a couple of times
  \item Once in the site, press GPS button to save location
  \item Fill the log sheet with the ``Location Description'', ``GPS WAYPOINT'' and ``Time''
  \item Take a bottle from the cooler and write the bottle label down in the ``Bottle Number'' section on the data sheet along
  \item Rinse the Nalgene Bottle and cap at least 3 times with water before filling
  \item Submerge bottle with the cap on it in an undisturbed location.
  \item Uncap the submerged bottle  to take subsurface water sample (avoid to take water from the surface) \cite{Montana08} 
  \item Cap the bottle with the bottle still submerged
  \item Store bottle up-right immediately in the cooler in order to avoid direct sun light \cite{Mueller1995}
  \item Once off the water, text to Nina or person in charge of the collection for example: ``Safe, Long Pond team''
  \item Take water some water samples to the Monroe County Environmental Lab, if applicable.
  \item Place the sample bottles in the refrigerator as soon as possible
  \item Filter water samples right after collection to preserve the chlorophyll and storage filters in the freezer as soon as possible
\end{enumerate}
Notes: 
\begin{itemize}
  \item Do not take any personal electronic device with you (recommended) to avoid dropping it on the water
  \item Storage car keys in zipped bag in you packet
  \item In case of bad weather conditions that do not allow paddle the canoes, take at least water samples from the Charlotte pier with the bucket
\end{itemize}

% -----------------------------------------------------------------------------
\section{\texorpdfstring{$R_{rs}$}{Rrs}}

% method described by \cite{Mobley:1999} for measuring the spectra of the downwelling irradiance $E_d$, the surface reflected sky radiance $L_s$, and the water-leaving radiance $L_w$ for each site 

This section describes how to take the remote-sensing measurement using a single instrument that measures radiance (spectroradiometer or spectrometer) such as a SVC \cite{SVCHR1024i} or an ASD \cite{ASDManual2012} instrument. This procedure is taken from \cite{Mobley:1999} and \cite{Mueller1995}. 

Recall that the \gls{rrs} is defined as

\begin{equation}\label{eq:Rrs}
	R_{rs}(\theta,\phi,\lambda)=\frac{L_w(\theta,\phi,\lambda)}{E_d(\lambda)}
\end{equation}
where $L_w$ is the water-leaving radiance in the polar and azimuthal directions $\theta$ and $\phi$, respectively, and $E_d$ is the downwelling spectral plane irradiance incident onto the water surface. A radiometer pointing down toward the water surface in direction $(\pi-\theta,\phi)$ does not directly measure $L_w$. Instead, it measures $L_w$  plus any incident sky radiance reflected $L_r$ by the water surface into the field of view of the sensor. This total radiance at the sensor $L_t$ is define as

\begin{equation}\label{eq:Lt}
	L_t(\theta,\phi) = L_r(\theta,\phi)+L_w(\theta,\phi)\Rightarrow L_w(\theta,\phi)=L_t(\theta,\phi) - L_r(\theta,\phi)
\end{equation}

The term $L_r$ can be replaced by

\begin{equation}\label{eq:Lsky}
	L_r = \rho L_{sky}
\end{equation}
where $\rho$ is the proportionality factor that relates the radiance measured when the sensor views the sky to the reflected sky radiance measured when the sensor views the water surface. \cite{Mobley:1999} suggests to use $\rho \approx 0.028$ for a sensor view angle $\theta_v \approx 40^\circ$ from the nadir and  $\phi_v \approx 135^\circ$ from the Sun with the constraints of a clear sky and wind speed less than $5m/s$.

Although $E_d$ could be measured directly with an appropriate sensor, it will be estimated from the radiance measured from a Lambertian surface (Spectralon) because both instruments in this case (SVC and ASD) are set to measure radiance. When an irradiance $E_d$ falls in a Lambertian surface with a known irradiance reflectance $R_g$, the uniform radiance $L_g$ leaving the surface is given by 

\begin{equation}\label{eq:Lg}
	L_g = (R_g/\pi)E_d\Rightarrow E_d = L_g*\pi/R_g
\end{equation}

Applying \autoref{eq:Lt} ,\autoref{eq:Lsky} and \autoref{eq:Lg} in \autoref{eq:Rrs} yields

\begin{equation}
	R_{rs} = \frac{L_t-\rho L_{sky}}{\frac{\displaystyle \pi}{\displaystyle R_g}L_g}
\end{equation}

\subsection{ASD}

The ASD should be in ``radiance mode''. Three different radiance measurements need to be taken:
\begin{itemize}
	\item $L_g$: pointing the Spectralon

Description: $L_g$ is measured with the sensor pointing downward in the same direction as is used in viewing the water surface (see \autoref{fig:Ltmea}), while the Spectralon is inserted into the sensor FOV. The Spectralon should be normal to the water surface.

	\item $L_t$: pointing the water surface

Description: $L_t$ is measured with the sensor pointing downward toward the water surface in the direction $\approx \pi-\theta_v = 140^\circ$ from nadir with $\theta_v = 40^\circ$ and $\phi_v \approx 135^\circ$ or $\phi_v \approx -135^\circ$ from the Sun as illustrated in \autoref{fig:Ltmea}.

\begin{figure}[htb]
\centering
    \includegraphics[width=10cm]{/Users/javier/Desktop/Javier/PHD_RIT/LDCM/WaterQualityProtocols/Latex/Images/Lgmea.png}
    \vspace{0.5cm}
   \caption[]{\label{fig:Ltmea} $L_t$ measurement.}
\end{figure}

	\item $L_{sky}$: pointing the sky

Description: $L_{sky}$ is measured with the sensor pointing upward toward the sky in the direction $\approx \theta_v = 40^\circ$ from nadir and $\phi_v \approx 135^\circ$ or $\phi_v \approx -135^\circ$ from the Sun as illustrated in \autoref{fig:Lskymea}.

\begin{figure}[htb]
\centering
    \includegraphics[width=10cm]{/Users/javier/Desktop/Javier/PHD_RIT/LDCM/WaterQualityProtocols/Latex/Images/Lskymea.png}
    \vspace{0.5cm}
   \caption[]{\label{fig:Lskymea} $L_{sky}$ measurement.}
\end{figure}

\end{itemize}

\subsection{SVC}
The same three radiance measurements described above need to be taken:

\begin{itemize}
	\item $L_g$: pointing the Spectralon

Description: The measurement is taken in the same fashion described in the previous section and it is taken only once per site. When the SVC instrument is used in ``reflectance mode'', it is necessary to measure first an standard measurement (Spectralon measurement). This standard measurement is the $L_r$ and is recorded internally in the ``sig'' file. Therefore, $L_r$ needs to be extracted from the later from the ``sig'' file. 

	\item $L_t$: pointing the water surface

Description: $L_t$ is measured in the same fashion described in the previous section. However, this measurement is saved internally in the ``sig'' file after the standard measurement column ($L_g$).

	\item $L_{sky}$: pointing the sky	

Description: $L_{sky}$ is measured in the same fashion described in the previous section. However, this measurement is saved internally in the ``sig'' file after the standard measurement column ($L_g$).

\end{itemize}
{\bf Notes:}
\begin{itemize}
	\item Wear dark clothes, preferable black, to avoid contamination from adjacent objects.
	\item Avoid any reflection from nearby objects in the boat or ship by covering the ship's side with a black turf. 
\end{itemize}



% @@@@@@@@@@@@@@@@@@@@@@@@@@@@@@@@@@@@@@@@@@@@@@@@@@@@@@@@@@@@@@@@@@@@@@@@@@@@@
\chapter{Lab Measurements}
\label{ch:labmea} 


\begin{figure}[htb]
% \subfloat[]{
\centering
    \includegraphics[width=14cm]{/Users/javier/Desktop/Javier/PHD_RIT/LDCM/WaterQualityProtocols/Images/WaterQualityProtocolDiagram.png}%}\hspace{0.5cm}
% \subfloat[]{   
%     \includegraphics[width=8cm]{/Users/javier/Desktop/Javier/PHD_RIT/20122_Winter/Instrumentation/report3/Images/SideFluoSpec.jpg}}
    \vspace{0.5cm}
   \caption[]{\label{fig:ProtocolsDiagram} Lab measurement protocols diagram.}
\end{figure}
 %------------- 

% $a_{YS}$ cannot be determined directly. An approximation of $a_{YS}$ may be obtained by a spectrophotometer scan of a filtered sample (\cite{Bukata1995}, p.125). Spectrophotometer used in normal mode do not measure true absorbance but {\color{red} attenuance} because all the scattered light is measured. To overcome this, the cells can be placed close to a wide photomultiplier (\cite{Kirk1983}, p.51).
% -----------------------------------------------------------------------------
\section{IOPs}
%*******************************
\subsection{Chlorophyll absorption coefficients}
The spectrophotometric methods are described by \cite{Mitchell2002} and \cite{Cleveland1993}.
%*******************************
\subsection{Minerals absorption coefficients}

%*******************************
\subsubsection{Equipment}
%*******************************
\subsubsection*{Filtration}
\begin{itemize}
  \item Vacuum pump
  \item Filter tower (filter funnel stem, filter base, funnel, filter cup)
  \item Whatman Binder-Free Glass Microfiber Filters: Type GF/F - Diameter: 2.5cm
  \item Forceps
\end{itemize}
%*******************************
\subsubsection*{Measurement}
\begin{itemize}
  \item Spectrophotometer
  \item {\color{red} Two lenses support}
  \item Squirt bottle with {\color{red} DIW} or small pipette with {\color{red} DIW} 
  \item Methanol
\end{itemize}
%*******************************
\subsubsection{Procedure}
%*******************************
\begin{enumerate}
  \item Turn the spectrophotometer on at least 30 minutes before measuring
  \item Set the spectrophotometer parameters in the UV-2101PC software menu: Configure > Parameters...
  \item Select the Serial Port to be use for communication with the instrument. Go to: Configure > PC Configuration... In the PC Configuration Parameters, select Photometer Serial Port and click OK
  \item From the menu, go to Configure > Utilities. In the System Utilities window, Turn Photometer On and press OK
  \item Pour DIW water to two GF/F Whatman filters and stick them in the two lenses support. Both filter should have the same amount of water. Add water with the pipette or the squirt bottle
  \item Place the two lenses support in the spectrophotometer
  \item Press the Baseline button in the UV-2101PC software
  \item Perform an scanner to see the baseline level of the instrument by pressing the "Start" button in the software
  \item Press the "Go To WL" button of the software and type $850 [nm]$. Press the "Auto Zero" button of the software (optional)
  \item Invert water sample bottle a couple of times to mix by turbulence and ensure large particles that settle are re-suspended (\cite{Mitchell2002})
  \item \label{item:place_filter} Using the forceps, place the filter on the filter base and place the filter cup on the base. \textbf{Record volume filtered}. 
  \item \label{item:filtration} Turn the vacuum pump on and turn the knob $90^\circ$ to allow filtration. Once all the water pass through the filter, turn the know $90^\circ$ back and the turn the vacuum pump off
  \item Using the forceps, take the filter with just water from the two lenses support and storage it for future baselines. Do not remove the blank filter ({\color{red} OR reference filter}) for the whole measurement session
  \item \label{item:place_filter_spec} Using the forceps, take the sample filter from the filtering tower. Add one or a few water drops to the sample filter and stick in the two lenses support. 
  \item  Measure absorbance in the spectrophotometer by pressing the "Start" button of the software and save data. This will be the $OD_{filt}$ measurement
  \item Using the forceps, remove carefully the sample filter from two lenses support avoiding to break it and place in the filter tower as in step \ref{item:place_filter}
  \item Pour enough solvent {\color{red} to sumerge} the filter in the filter cup. {\color{red} Wait 5 min} and then filter as in step \ref{item:filtration}
  \item Repeat step \ref{item:place_filter_spec}
  \item Measure absorbance in the spectrophotometer by pressing the "Start" button of the software and save data. This will be the $OD_{no~pig}$ measurement 
  \item Record the area of filtration in the sample filter
  \item[]Note: The instrument only allows to save four measurement at the time. To save measurement, go to File > Data Translation > ASCII Export... {\color{red} Select channels to be saved, name files and press OK}
\end{enumerate}
%*******************************
\subsubsection{Calculations}


%*******************************
\subsection{CDOM absorption coefficients}
%*******************************
\subsubsection{Equipment}
%*******************************
\subsubsection*{Filtration}
\begin{itemize}
  \item Whatman GD/X 13 and 25mm Disposable Syringe Filters - Nylon $0.2[\mu m]$ Nylon
  \item {\color{red}Syringe}
\end{itemize}
\subsubsection*{Measurement}
\begin{itemize}
  \item Spectrophotometer (Shimadzu UV2100V - Dual beam spectrophotometer)
  \item Blank cell
  \item Sample cell
  \item Purified water
  \item Ethanol
\end{itemize}
%*******************************
\subsubsection{Procedure}
%*******************************
\begin{enumerate}
  \item \textbf{Turn the Spectrophotometer on at least one hour before measuring}. It needs to be warmed up for optimal measurements.
  \item Wash the syrenge filter out 3 times with purified water
  \item Rinse cells a couple of times with a small amount of ethanol by shaking it (optional, if the cells seem dirty)
  \item Use cotton sweep to clean internal face (optional, if face seems dirty)
  \item Rinse cell with purified water
  \item Clean and dry the external surface of the cells with optics paper. Be careful with scratching the surface, specially the front and bottom faces
  \item Select a Slit Width equal to $5.0~[nm]$ in the photometer software
  \item Select a Sampling Interval of $1~[nm]$ or $2~[nm]$ in the photometer software
  \item Fill both cells with purified water and extract bubbles
  \item Place both blank and sample cells filled with purified water in the sample compartment of the spectrophotometer
  \item Press ``Auto Zero'' button in the spectrophotometer software
  \item Press ``Baseline'' button in the spectrophotometer software
  \item Fill sample cell with filtered water from the syringe filter
  \item Press start button
  \item Save Channel in the spectrophotometer software
  \item Go to Data Translation > ASCII Export in spectrophotometer software
\end{enumerate}
\textbf{Important:} the samples should be at room temperature. The absorbance measurement is sensible to temperature changes.
%*******************************
\subsubsection{Data treatment}
%*******************************
\begin{itemize}
  \item Absorbance: $A=-\ln{\displaystyle\frac{1}{T}}$ 
  \item Substract bias before convert to coefficients
  \item $a_{CDOM}=2.303~A(\lambda)/L~~[m^{-1}]$ where $A(\lambda)$ is the absorbance and $L$ the pathlength of the absorbance cell in meters.
\end{itemize}
% -----------------------------------------------------------------------------
\section{Concentrations}
\subsection{Chlorophyll-{\it a} concentration}

\todo{Show comparison of Monroe county to RIT Lab}Methods described by \cite{Lorenzen:1967fk} and \cite{Ritchie:2008eu}.

\subsubsection{Calculations}

The calculations used \cite{Lorenzen:1967fk} are:

\begin{equation}
  C_a = \frac{26.7(655_o - 665_a)\times v}{V\times l}
\end{equation}

\begin{equation}
  Pheo = \frac{26.7([1.7\times 665_a]-665_o)\times v}{V\times l}
\end{equation}

\noindent where: \\
$665_o = 665 - (750-blank~value)~before~acidification$\\
$665_a = 665 - (750-blank~value)~after~acidification$  \\
$v = $ volume of extract in mililiters $[ml]$ \\
$V = $ volume of water filtered in liters $[L]$ \\
$l = $ pathlength of cuvette ($1cm$ for the cuvette used) \\

Note: concentrations are in $[mg/m^3]$ or $[\mu g/L]$.


\subsection{Total suspended solids (TSS)}

\subsubsection{Equipment}
%*******************************
% \subsubsection*{Filtration}
\begin{itemize}

  \item TCLP filters ($47 mm$, $0.7\mu m$)
  \item Vacuum
  \item Balance
  \item Graduate cylinder
  \item Forceps

\end{itemize}

%*******************************
\subsubsection{Procedure}
%*******************************
\begin{enumerate}
  \item Weight filters before filtering
  \item Record volume to filter. Use graduated cylinder
  \item Use vacuum to filter water with the TCLP filters to filter the particles
  \item Weight filter in balance
  \item Put in aluminium foil with weight
  \item Dry sample at $75^\circ C$ for a couple of hours
\end{enumerate}


\begin{equation}
SPM_{\displaystyle concentration} = \frac{[final~filter~weight~(mg) - tare~filter~weight~(mg)]}{volume~filtered~(L)}~~~\left[\frac{mg}{L}\right]
\end{equation}


% &&&&&&&&&&&&&&&&&&

% @@@@@@@@@@@@@@@@@@@@@@@@@@@@@@@@@@@@@@@@@@@@@@@@@@@@@@@@@@@@@@@@@@@@@@@@@@@@@
\chapter{HydroLight Specification}


\singlespacing
\lstset{language=bash,caption={Example of an input file used in Ecolight.},label=code:EcolightInput}
\renewcommand{\lstlistingname}{Code}
\begin{lstlisting}
0,400,2500,.02,488,.00026,1,5.3
FFbb determination for ONTOS
OutputEL
0,1,0,0,0,1
2,1,0,2,3
4,4
0,flaCH,flaCD,flaSM
0,2,440,0.1,0.014
0,0,440,0.1,0.014
0,4,440,1,0.01712
0,0,440,0.1,0.014
/home/jxc4005/hydrolight52Javier_install/data/H2OabDefaults_FRESHwater.txt
/home/jxc4005/HYDROLIGHT/EL5.2/user_inputs/astar_CH_ONTOS140929_CountyUncorr.txt
dummyastar.txt
/home/jxc4005/HYDROLIGHT/EL5.2/user_inputs/astar_SM_ONTOS140929_County.txt
4, 660, 0.189, 1, 0.751, -999
0,-999,-999,-999,-999,-999
-1,-999,0,-999,-999,-999
0,-999,-999,-999,-999,-999
bstarDummy.txt
/home/jxc4005/HYDROLIGHT/EL5.2/user_inputs/ChloroSct.txt
dummybstar.txt
/home/jxc4005/HYDROLIGHT/EL5.2/user_inputs/susmin.sct
0, 0, 550, 0.01, 0
0, 0, 0, 0, 0
-1, 0, 0, 0, 0
0, 0, 550, 0.01, 0
pureh2o.dpf
user_dpfCHL
isotrop.dpf
user_dpfTSS_b
 120
400, 405, 410, 415, 420, 425, 430, 435, 440, 445,
450, 455, 460, 465, 470, 475, 480, 485, 490, 495,
500, 505, 510, 515, 520, 525, 530, 535, 540, 545,
550, 555, 560, 565, 570, 575, 580, 585, 590, 595,
600, 605, 610, 615, 620, 625, 630, 635, 640, 645,
650, 655, 660, 665, 670, 675, 680, 685, 690, 695,
700, 705, 710, 715, 720, 725, 730, 735, 740, 745,
750, 755, 760, 765, 770, 775, 780, 785, 790, 795,
800, 805, 810, 815, 820, 825, 830, 835, 840, 845,
850, 855, 860, 865, 870, 875, 880, 885, 890, 895,
900, 905, 910, 915, 920, 925, 930, 935, 940, 945,
950, 955, 960, 965, 970, 975, 980, 985, 990, 995,
1000,
0,0,0,0,2
2, 3, 48, 0, 0
272, 43.28085,-77.61919, 29.92, 1, 80, 2.5, 15, 4.99746, 300
4.99746, 1.34, 20, 35
0, 0
0, 5, 0, 5, 10, 15, 20, 
/home/jxc4005/hydrolight52Javier_install/data/H2OabDefaults_FRESHwater.txt
1
/home/jxc4005/hydrolight5Aaron_install/data/user/mascot_ac9.txt
dummyFilteredAc9.txt
dummyHscat.txt
/home/jxc4005/hydrolight5Aaron_install/data/user/Chlzdata_10m.txt
dummyComp.txt
dummyR.bot
dummydata.txt
/home/jxc4005/hydrolight5Aaron_install/data/user/Chlzdata_10m.txt
dummyComp.txt
dummyComp.txt
/home/jxc4005/hydrolight5Aaron_install/data/user/Ed_total.txt
/home/jxc4005/hydrolight5Aaron_install/data/MyBiolumData.txt
\end{lstlisting}

\end{appendices}

\listoftodos

%\bibliographystyle{ieeetr}
%\bibliographystyle{unsrtnat}
\bibliographystyle{apalike}

\bibliography{/Users/javier/Desktop/Javier/PHD_RIT/Latex/javier_bib}

\printindex

\end{document}  