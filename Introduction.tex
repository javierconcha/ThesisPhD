% !TEX root=Thesis_PhD.tex  
% the previous is to reference main .bib
%% CHAPTER
\chapter{Introduction}
\label{ch:introduction} 
\pagenumbering{arabic} 
% What? and So what? What is the paper about, and why should the reader care?
% Topic, observation/discovery, explanation
% What research questions is this paper trying to answer?

% - Establish a territory (what is the field of the work, why is this field important, what has already been done?)
% - Establish a niche (indicate a gap, raise a question, or challenge prior work in this territory)
% - Occupy that niche (outline the purpose and announce the present research; optionally summarize the results).

% Not enough spatial resolution
Ocean color studies at a global scale, such as chlorophyll-{\it a} level trends in oceans, can be investigated using the heritage Ocean Color satellites (e.g. \acrfull{seawifs}, \acrfull{modis}). These satellites satisfied the spatial requirement for these kinds of studies. However, when the region of interests include coastal or inland waters, which could be considered Case 2 water, their spatial resolution of roughly a thousand meters are not enough to resolve smaller water bodies. These kinds of waters are important for us, because it is these kinds of waters that we have the most interaction with, like for drinking or recreation. 

% New medium spatial resolution
The launching of a new generation of high spatial resolution satellites (e.g. \acrshort{nasa}'s Landsat 8 \cite{Irons:2012}, \acrshort{esa}'s Sentinel 2 \cite{Malenovsky:2012}) is opening a complete new era in the remote sensing of coastal and inland waters. New sensor specifications could meet the requirements needed to have available the same kinds of tools (e.g. Chl-{\it a} product) that the first generation of ocean color satellites (a.k.a. heritage Ocean Color satellites), such as MODIS (\cite{Esaias1998}) and SeaWiFS (\cite{McClain2004}), made available for open ocean science more than a decade and half ago. The hope is to have these kinds of tools for coastal and inland waters available at a global scale and on a regular basis, similar to MODIS capabilities for open oceans (e.g. MODIS's Chl-{\it a} product). Although Landsat 8 does not have a daily frequency as MODIS does, its 16-day frequency makes it a good candidate to accomplish this endeavor, and most importantly, prepares the way for future missions with similar spatial resolution specifications (e.g. Sentinel 2, \acrfull{hyspiri}).

% Landsat
The Landsat project has been monitoring the earth for over four decades, being the longest uninterrupted data set available. The Landsat satellites' main mission is to image the land areas of the earth and therefore there are typically no open ocean (case 1 water) images available. This is one of the reason why Landsat satellites have been underestimated by the ocean color community for the study of water bodies. In addition, the Landsat instruments have generally had broad bands and low \acrfull{snr} when compared to heritage ocean color satellites such as SeaWiFS and MODIS. Carrying two instruments onboard, the \acrfull{oli} and the \acrfull{tirs}, Landsat 8 is the first of a new generation of Landsat satellite with state-of-the-art technology \cite{Irons:2012}. Since its launch in 2013, the OLI instrument onboard Landsat 8 has created high expectations in the ocean color community. With its 12-bit quantization and improved SNR, OLI is a big improvement to the Landsat mission. In addition, OLI includes a new coastal band that increased the spectral resolution of the instrument. These improvements are the main drivers to hypothesize that the Landsat 8 satellite will definitely have a better performance in water quality studies than its predecessors. \cite{Roy:2014} stated this potential use of Landsat 8 for fresh and coastal water studies, mainly due to a reported SNR that exceeded expectations and the new coastal band. Therefore, the overall objective of this research is to investigate the performance of Landsat 8 for accurately retrieving water constituents.

% Work done with Landsat 8
\cite{Gerace:2013} demonstrated that the spectral coverage and radiometric resolution of OLI should dramatically improve our ability to simultaneously retrieve the \acrfull{cpas} (chlorophyll-{\it a}, sediments and \acrfull{cdom}) concentrations from water bodies. \cite{Vanhellemont2014} and \cite{Vanhellemont:2015} created a tool to apply the standard algorithm for atmospheric correction over water developed by \cite{Gordon:1994} to Landsat 8 over turbid and extremely turbid waters. In \cite{Vanhellemont:2015}, after atmospheric correction, they used the end product for detection of high concentrations of black sediments, but neither concentration values nor comparison with field measurements were reported. \cite{Franz:2015} describes an implementation of the atmospheric corrections developed by \cite{Gordon:1994} applied to Landsat 8 in the \acrfull{seadas} software package (URL: \url{http://seadas.gsfc.nasa.gov/}). A comparison of the Landsat 8's retrieved \acrfull{rrs} and chlorophyll-{\it a} concentration over Chesapeake Bay with results from MODIS, SeaWiFS and {\it in situ} historical chlorophyll-{\it a} measurements is presented showing a relatively good agreement. Though again, no direct comparison to simultaneously measured values were available.

% Atmospheric Correction done
The retrieval of water components is in general performed in the reflectance domain, so the very first step in this work is to perform a high quality atmospheric correction to the radiance image from Landsat 8. This is a complex task to perform over water because the signal leaving the water that reaches the sensor is very small when compared to the signal reaching the sensor produced by atmospheric scattering. Most of the atmospheric correction algorithms for open oceans (Case 2 waters) are based on the methods developed for ocean color satellites by \cite{Gordon:1994}. These methods are based on the fact that the signal leaving the water does not contribute to the overall signal beyond the \acrfull{nir} part of the spectrum; so the signal reaching the sensor is caused only by atmospheric scattering (\cite{Gordon:1994}) in those wavelengths. This is known as the {\it black pixel assumption}. This concept can be expanded to the \acrfull{swir} bands when the black pixel assumption is not valid in the NIR bands, which is the case for Case 2 and highly productive Case 1 waters (\cite{Wang:2007}). Unfortunately, most of these methods are not suitable for highly turbid coastal water, although different modifications to these algorithms have been suggested (\cite{Patt2003}).

% Atmospheric Correction developed in this work
In this work, a different approach for atmospheric correction was developed based on \cite{Raqueno:2003}, \cite{Gerace:2013} and \cite{Pahlevan:2012}. This atmospheric correction algorithm is the \acrfull{mobelm} that uses a combination of an in-water radiative transfer model over water and a Landsat reflectance product to determine the bright and dark pixels in the image \cite{Concha2014SPIE,Concha2015_SPIE}. The MoB-ELM is compared with the standard algorithms based on \cite{Gordon:1994} and with {\it in situ} measurements.

% Retrieval: What it has be done
After having atmospherically corrected the image, the next step is to apply a retrieval algorithm that outputs maps of CPA concentrations. The standard algorithms for CPA retrieval used by the heritage ocean color missions do not retrieve the CPAs simultaneously. An example of these algorithms is the \acrfull{oc3} algorithm for retrieving chlorophyll-{\it a} concentrations \cite{OReilly2000}. This is an empirical algorithm that finds the best fit for a function between a ratio of two bands and a {\it in situ} chlorophyll-{\it a} concentration data set.

% Retrieval: What we are proposing
\cite{Gerace:2013} developed an algorithm for the retrieval of CPAs based on a \acrfull{lut} inversion using simulated Landsat 8 imagery. In this work, we extend Gerace's approach to real Landsat 8 imagery \cite{Concha2013IGARSS}. The retrieval algorithm uses a combination of least square error minimization algorithm and a non-linear optimization routine to find the best match for a specific reflectance signal in a LUT of spectral water-leaving reflectance curves. The LUT is created using HydroLight 5, an in-water radiative transfer model (\cite{Mobley:2005}). Each curve in the LUT has a specific set of water component concentrations. This is performed on a pixel-by-pixel basis. 

In order to have outputs that are representative of the water bodies that are being studied, \acrfull{iops} of those specific waters have to be input to the HydroLight model. To accomplish this, collections of water samples were conducted at the same time that the Landsat 8 satellite passed over the area of study (the Rochester Embayment, Rochester, NY). After collection, these water samples were analyzed in the lab to obtain the IOPs for the CPAs. Furthermore, \acrfull{aops}, specifically remote-sensing reflectances, and backscattering measurements were also collected for further comparison and to pursue closure between HydroLight AOP results and {\it in situ} AOP measurements.

% Data Collection and Lab Measurements
The concentration of CPAs obtained from the retrieval algorithm are validated by comparison with concentration measured in the lab from {\it in situ} water samples collected from water bodies present in the Landsat 8 image. Also, the chlorophyll-{\it a} retrieval is compared with the NASA's standard algorithms \cite{OReilly1998_Chl,OReilly2000,Hu:2012fv}.