% !TEX root=Thesis_PhD.tex 
% the previous is to reference main.bib
%% CHAPTER
\begin{appendices}
% @@@@@@@@@@@@@@@@@@@@@@@@@@@@@@@@@@@@@@@@@@@@@@@@@@@@@@@@@@@@@@@@@@@@@@@@@@@@@
\chapter{Field Measurements}
% \addcontentsline{toc}{chapter}{Appendix}
% \renewcommand{\thesection}{\Alph{section}}
% @@@@@@@@@@@@@@@@@@@@@@@@@@@@@@@@@@@@@@@@@@@@@@@@@@@@@@@@@@@@@@@@@@@@@@@@@@@@@
\label{ch:fieldmea}
% -----------------------------------------------------------------------------
\section{Water Samples}


\subsection{Equipment}

\begin{multicols}{3}
\begin{itemize}
  \item Dark Nalgene bottles
  \item Monroe County Environmental Lab bottles
  \item Cooler
  \item Marker
  \item Bottle label
  \item Ice packs
  \item GPS
  \item Extra batteries GPS
  \item Data sheet
  \item Pen
  \item Back pen
  \item Canoe 
  \item Transport straps for canoe
  \item Paddles
  \item Life jacket
  \item Suncream 
  \item Drinking water
  \item Wipes to clean extra suncream from hands
  \item Bucket with rope (in case is not possible to take water samples due to bad condition weather, for example)
\end{itemize}
\end{multicols}


\subsection{Procedure}
\begin{enumerate}
  \item Throughly clean the bottles prior to collection by brushing them inside with tap water a couple of times and rinse with DIW water a couple of times
  \item Once in the site, press GPS button to save location
  \item Fill the log sheet with the ``Location Description'', ``GPS WAYPOINT'' and ``Time''
  \item Take a bottle from the cooler and write the bottle label down in the ``Bottle Number'' section on the data sheet along
  \item Rinse the Nalgene Bottle and cap at least 3 times with water before filling
  \item Submerge bottle with the cap on it in an undisturbed location.
  \item Uncap the submerged bottle  to take subsurface water sample (avoid to take water from the surface) \cite{Montana08} 
  \item Cap the bottle with the bottle still submerged
  \item Store bottle up-right immediately in the cooler in order to avoid direct sun light \cite{Mueller1995}
  \item Once off the water, text to Nina or person in charge of the collection for example: ``Safe, Long Pond team''
  \item Take water some water samples to the Monroe County Environmental Lab, if applicable.
  \item Place the sample bottles in the refrigerator as soon as possible
  \item Filter water samples right after collection to preserve the chlorophyll and storage filters in the freezer as soon as possible
\end{enumerate}
Notes: 
\begin{itemize}
  \item Do not take any personal electronic device with you (recommended) to avoid dropping it on the water
  \item Storage car keys in zipped bag in you packet
  \item In case of bad weather conditions that do not allow paddle the canoes, take at least water samples from the Charlotte pier with the bucket
\end{itemize}

% -----------------------------------------------------------------------------
\section{\texorpdfstring{$R_{rs}$}{Rrs}}

% method described by \cite{Mobley:1999} for measuring the spectra of the downwelling irradiance $E_d$, the surface reflected sky radiance $L_s$, and the water-leaving radiance $L_w$ for each site 

This section describes how to take the remote-sensing measurement using a single instrument that measures radiance (spectroradiometer or spectrometer) such as a SVC \cite{SVCHR1024i} or an ASD \cite{ASDManual2012} instrument. This procedure is taken from \cite{Mobley:1999} and \cite{Mueller1995}. 

Recall that the \gls{rrs} is defined as

\begin{equation}\label{eq:Rrs}
	R_{rs}(\theta,\phi,\lambda)=\frac{L_w(\theta,\phi,\lambda)}{E_d(\lambda)}
\end{equation}
where $L_w$ is the water-leaving radiance in the polar and azimuthal directions $\theta$ and $\phi$, respectively, and $E_d$ is the downwelling spectral plane irradiance incident onto the water surface. A radiometer pointing down toward the water surface in direction $(\pi-\theta,\phi)$ does not directly measure $L_w$. Instead, it measures $L_w$  plus any incident sky radiance reflected $L_r$ by the water surface into the field of view of the sensor. This total radiance at the sensor $L_t$ is define as

\begin{equation}\label{eq:Lt}
	L_t(\theta,\phi) = L_r(\theta,\phi)+L_w(\theta,\phi)\Rightarrow L_w(\theta,\phi)=L_t(\theta,\phi) - L_r(\theta,\phi)
\end{equation}

The term $L_r$ can be replaced by

\begin{equation}\label{eq:Lsky}
	L_r = \rho L_{sky}
\end{equation}
where $\rho$ is the proportionality factor that relates the radiance measured when the sensor views the sky to the reflected sky radiance measured when the sensor views the water surface. \cite{Mobley:1999} suggests to use $\rho \approx 0.028$ for a sensor view angle $\theta_v \approx 40^\circ$ from the nadir and  $\phi_v \approx 135^\circ$ from the Sun with the constraints of a clear sky and wind speed less than $5m/s$.

Although $E_d$ could be measured directly with an appropriate sensor, it will be estimated from the radiance measured from a Lambertian surface (Spectralon) because both instruments in this case (SVC and ASD) are set to measure radiance. When an irradiance $E_d$ falls in a Lambertian surface with a known irradiance reflectance $R_g$, the uniform radiance $L_g$ leaving the surface is given by 

\begin{equation}\label{eq:Lg}
	L_g = (R_g/\pi)E_d\Rightarrow E_d = L_g*\pi/R_g
\end{equation}

Applying \autoref{eq:Lt} ,\autoref{eq:Lsky} and \autoref{eq:Lg} in \autoref{eq:Rrs} yields

\begin{equation}
	R_{rs} = \frac{L_t-\rho L_{sky}}{\frac{\displaystyle \pi}{\displaystyle R_g}L_g}
\end{equation}

\subsection{ASD}

The ASD should be in ``radiance mode''. Three different radiance measurements need to be taken:
\begin{itemize}
	\item $L_g$: pointing the Spectralon

Description: $L_g$ is measured with the sensor pointing downward in the same direction as is used in viewing the water surface (see \autoref{fig:Ltmea}), while the Spectralon is inserted into the sensor FOV. The Spectralon should be normal to the water surface.

	\item $L_t$: pointing the water surface

Description: $L_t$ is measured with the sensor pointing downward toward the water surface in the direction $\approx \pi-\theta_v = 140^\circ$ from nadir with $\theta_v = 40^\circ$ and $\phi_v \approx 135^\circ$ or $\phi_v \approx -135^\circ$ from the Sun as illustrated in \autoref{fig:Ltmea}.

\begin{figure}[htb]
\centering
    \includegraphics[width=10cm]{/Users/javier/Desktop/Javier/PHD_RIT/LDCM/WaterQualityProtocols/Latex/Images/Lgmea.png}
    \vspace{0.5cm}
   \caption[]{\label{fig:Ltmea} $L_t$ measurement.}
\end{figure}

	\item $L_{sky}$: pointing the sky

Description: $L_{sky}$ is measured with the sensor pointing upward toward the sky in the direction $\approx \theta_v = 40^\circ$ from nadir and $\phi_v \approx 135^\circ$ or $\phi_v \approx -135^\circ$ from the Sun as illustrated in \autoref{fig:Lskymea}.

\begin{figure}[htb]
\centering
    \includegraphics[width=10cm]{/Users/javier/Desktop/Javier/PHD_RIT/LDCM/WaterQualityProtocols/Latex/Images/Lskymea.png}
    \vspace{0.5cm}
   \caption[]{\label{fig:Lskymea} $L_{sky}$ measurement.}
\end{figure}

\end{itemize}

\subsection{SVC}
The same three radiance measurements described above need to be taken:

\begin{itemize}
	\item $L_g$: pointing the Spectralon

Description: The measurement is taken in the same fashion described in the previous section and it is taken only once per site. When the SVC instrument is used in ``reflectance mode'', it is necessary to measure first an standard measurement (Spectralon measurement). This standard measurement is the $L_r$ and is recorded internally in the ``sig'' file. Therefore, $L_r$ needs to be extracted from the later from the ``sig'' file. 

	\item $L_t$: pointing the water surface

Description: $L_t$ is measured in the same fashion described in the previous section. However, this measurement is saved internally in the ``sig'' file after the standard measurement column ($L_g$).

	\item $L_{sky}$: pointing the sky	

Description: $L_{sky}$ is measured in the same fashion described in the previous section. However, this measurement is saved internally in the ``sig'' file after the standard measurement column ($L_g$).

\end{itemize}
{\bf Notes:}
\begin{itemize}
	\item Wear dark clothes, preferable black, to avoid contamination from adjacent objects.
	\item Avoid any reflection from nearby objects in the boat or ship by covering the ship's side with a black turf. 
\end{itemize}



% @@@@@@@@@@@@@@@@@@@@@@@@@@@@@@@@@@@@@@@@@@@@@@@@@@@@@@@@@@@@@@@@@@@@@@@@@@@@@
\chapter{Lab Measurements}
\label{ch:labmea} 


\begin{figure}[htb]
% \subfloat[]{
\centering
    \includegraphics[width=14cm]{/Users/javier/Desktop/Javier/PHD_RIT/LDCM/WaterQualityProtocols/Images/WaterQualityProtocolDiagram.png}%}\hspace{0.5cm}
% \subfloat[]{   
%     \includegraphics[width=8cm]{/Users/javier/Desktop/Javier/PHD_RIT/20122_Winter/Instrumentation/report3/Images/SideFluoSpec.jpg}}
    \vspace{0.5cm}
   \caption[]{\label{fig:ProtocolsDiagram} Lab measurement protocols diagram.}
\end{figure}
 %------------- 

% $a_{YS}$ cannot be determined directly. An approximation of $a_{YS}$ may be obtained by a spectrophotometer scan of a filtered sample (\cite{Bukata1995}, p.125). Spectrophotometer used in normal mode do not measure true absorbance but {\color{red} attenuance} because all the scattered light is measured. To overcome this, the cells can be placed close to a wide photomultiplier (\cite{Kirk1983}, p.51).
% -----------------------------------------------------------------------------
\section{IOPs}
%*******************************
\subsection{Chlorophyll absorption coefficients}
The spectrophotometric methods are described by \cite{Mitchell2002} and \cite{Cleveland1993}.
%*******************************
\subsection{Minerals absorption coefficients}

%*******************************
\subsubsection{Equipment}
%*******************************
\subsubsection*{Filtration}
\begin{itemize}
  \item Vacuum pump
  \item Filter tower (filter funnel stem, filter base, funnel, filter cup)
  \item Whatman Binder-Free Glass Microfiber Filters: Type GF/F - Diameter: 2.5cm
  \item Forceps
\end{itemize}
%*******************************
\subsubsection*{Measurement}
\begin{itemize}
  \item Spectrophotometer
  \item {\color{red} Two lenses support}
  \item Squirt bottle with {\color{red} DIW} or small pipette with {\color{red} DIW} 
  \item Methanol
\end{itemize}
%*******************************
\subsubsection{Procedure}
%*******************************
\begin{enumerate}
  \item Turn the spectrophotometer on at least 30 minutes before measuring
  \item Set the spectrophotometer parameters in the UV-2101PC software menu: Configure > Parameters...
  \item Select the Serial Port to be use for communication with the instrument. Go to: Configure > PC Configuration... In the PC Configuration Parameters, select Photometer Serial Port and click OK
  \item From the menu, go to Configure > Utilities. In the System Utilities window, Turn Photometer On and press OK
  \item Pour DIW water to two GF/F Whatman filters and stick them in the two lenses support. Both filter should have the same amount of water. Add water with the pipette or the squirt bottle
  \item Place the two lenses support in the spectrophotometer
  \item Press the Baseline button in the UV-2101PC software
  \item Perform an scanner to see the baseline level of the instrument by pressing the "Start" button in the software
  \item Press the "Go To WL" button of the software and type $850 [nm]$. Press the "Auto Zero" button of the software (optional)
  \item Invert water sample bottle a couple of times to mix by turbulence and ensure large particles that settle are re-suspended (\cite{Mitchell2002})
  \item \label{item:place_filter} Using the forceps, place the filter on the filter base and place the filter cup on the base. \textbf{Record volume filtered}. 
  \item \label{item:filtration} Turn the vacuum pump on and turn the knob $90^\circ$ to allow filtration. Once all the water pass through the filter, turn the know $90^\circ$ back and the turn the vacuum pump off
  \item Using the forceps, take the filter with just water from the two lenses support and storage it for future baselines. Do not remove the blank filter ({\color{red} OR reference filter}) for the whole measurement session
  \item \label{item:place_filter_spec} Using the forceps, take the sample filter from the filtering tower. Add one or a few water drops to the sample filter and stick in the two lenses support. 
  \item  Measure absorbance in the spectrophotometer by pressing the "Start" button of the software and save data. This will be the $OD_{filt}$ measurement
  \item Using the forceps, remove carefully the sample filter from two lenses support avoiding to break it and place in the filter tower as in step \ref{item:place_filter}
  \item Pour enough solvent {\color{red} to sumerge} the filter in the filter cup. {\color{red} Wait 5 min} and then filter as in step \ref{item:filtration}
  \item Repeat step \ref{item:place_filter_spec}
  \item Measure absorbance in the spectrophotometer by pressing the "Start" button of the software and save data. This will be the $OD_{no~pig}$ measurement 
  \item Record the area of filtration in the sample filter
  \item[]Note: The instrument only allows to save four measurement at the time. To save measurement, go to File > Data Translation > ASCII Export... {\color{red} Select channels to be saved, name files and press OK}
\end{enumerate}
%*******************************
\subsubsection{Calculations}


%*******************************
\subsection{CDOM absorption coefficients}
%*******************************
\subsubsection{Equipment}
%*******************************
\subsubsection*{Filtration}
\begin{itemize}
  \item Whatman GD/X 13 and 25mm Disposable Syringe Filters - Nylon $0.2[\mu m]$ Nylon
  \item {\color{red}Syringe}
\end{itemize}
\subsubsection*{Measurement}
\begin{itemize}
  \item Spectrophotometer (Shimadzu UV2100V - Dual beam spectrophotometer)
  \item Blank cell
  \item Sample cell
  \item Purified water
  \item Ethanol
\end{itemize}
%*******************************
\subsubsection{Procedure}
%*******************************
\begin{enumerate}
  \item \textbf{Turn the Spectrophotometer on at least one hour before measuring}. It needs to be warmed up for optimal measurements.
  \item Wash the syrenge filter out 3 times with purified water
  \item Rinse cells a couple of times with a small amount of ethanol by shaking it (optional, if the cells seem dirty)
  \item Use cotton sweep to clean internal face (optional, if face seems dirty)
  \item Rinse cell with purified water
  \item Clean and dry the external surface of the cells with optics paper. Be careful with scratching the surface, specially the front and bottom faces
  \item Select a Slit Width equal to $5.0~[nm]$ in the photometer software
  \item Select a Sampling Interval of $1~[nm]$ or $2~[nm]$ in the photometer software
  \item Fill both cells with purified water and extract bubbles
  \item Place both blank and sample cells filled with purified water in the sample compartment of the spectrophotometer
  \item Press ``Auto Zero'' button in the spectrophotometer software
  \item Press ``Baseline'' button in the spectrophotometer software
  \item Fill sample cell with filtered water from the syringe filter
  \item Press start button
  \item Save Channel in the spectrophotometer software
  \item Go to Data Translation > ASCII Export in spectrophotometer software
\end{enumerate}
\textbf{Important:} the samples should be at room temperature. The absorbance measurement is sensible to temperature changes.
%*******************************
\subsubsection{Data treatment}
%*******************************
\begin{itemize}
  \item Absorbance: $A=-\ln{\displaystyle\frac{1}{T}}$ 
  \item Substract bias before convert to coefficients
  \item $a_{CDOM}=2.303~A(\lambda)/L~~[m^{-1}]$ where $A(\lambda)$ is the absorbance and $L$ the pathlength of the absorbance cell in meters.
\end{itemize}
% -----------------------------------------------------------------------------
\section{Concentrations}
\subsection{Chlorophyll-{\it a} concentration}

\todo{Show comparison of Monroe county to RIT Lab}Methods described by \cite{Lorenzen:1967fk} and \cite{Ritchie:2008eu}.

\subsubsection{Calculations}

The calculations used \cite{Lorenzen:1967fk} are:

\begin{equation}
  C_a = \frac{26.7(655_o - 665_a)\times v}{V\times l}
\end{equation}

\begin{equation}
  Pheo = \frac{26.7([1.7\times 665_a]-665_o)\times v}{V\times l}
\end{equation}

\noindent where: \\
$665_o = 665 - (750-blank~value)~before~acidification$\\
$665_a = 665 - (750-blank~value)~after~acidification$  \\
$v = $ volume of extract in mililiters $[ml]$ \\
$V = $ volume of water filtered in liters $[L]$ \\
$l = $ pathlength of cuvette ($1cm$ for the cuvette used) \\

Note: concentrations are in $[mg/m^3]$ or $[\mu g/L]$.


\subsection{Total suspended solids (TSS)}

\subsubsection{Equipment}
%*******************************
% \subsubsection*{Filtration}
\begin{itemize}

  \item TCLP filters ($47 mm$, $0.7\mu m$)
  \item Vacuum
  \item Balance
  \item Graduate cylinder
  \item Forceps

\end{itemize}

%*******************************
\subsubsection{Procedure}
%*******************************
\begin{enumerate}
  \item Weight filters before filtering
  \item Record volume to filter. Use graduated cylinder
  \item Use vacuum to filter water with the TCLP filters to filter the particles
  \item Weight filter in balance
  \item Put in aluminium foil with weight
  \item Dry sample at $75^\circ C$ for a couple of hours
\end{enumerate}


\begin{equation}
SPM_{\displaystyle concentration} = \frac{[final~filter~weight~(mg) - tare~filter~weight~(mg)]}{volume~filtered~(L)}~~~\left[\frac{mg}{L}\right]
\end{equation}


% &&&&&&&&&&&&&&&&&&

% @@@@@@@@@@@@@@@@@@@@@@@@@@@@@@@@@@@@@@@@@@@@@@@@@@@@@@@@@@@@@@@@@@@@@@@@@@@@@
\chapter{Main Codes}


\singlespacing
\lstset{language=bash,caption={Example of an input file used in Ecolight.},label=code:EcolightInput}
\renewcommand{\lstlistingname}{Code}
\begin{lstlisting}
0,400,2500,.02,488,.00026,1,5.3
FFbb determination for ONTOS
OutputEL
0,1,0,0,0,1
2,1,0,2,3
4,4
0,flaCH,flaCD,flaSM
0,2,440,0.1,0.014
0,0,440,0.1,0.014
0,4,440,1,0.01712
0,0,440,0.1,0.014
/home/jxc4005/hydrolight52Javier_install/data/H2OabDefaults_FRESHwater.txt
/home/jxc4005/HYDROLIGHT/EL5.2/user_inputs/astar_CH_ONTOS140929_CountyUncorr.txt
dummyastar.txt
/home/jxc4005/HYDROLIGHT/EL5.2/user_inputs/astar_SM_ONTOS140929_County.txt
4, 660, 0.189, 1, 0.751, -999
0,-999,-999,-999,-999,-999
-1,-999,0,-999,-999,-999
0,-999,-999,-999,-999,-999
bstarDummy.txt
/home/jxc4005/HYDROLIGHT/EL5.2/user_inputs/ChloroSct.txt
dummybstar.txt
/home/jxc4005/HYDROLIGHT/EL5.2/user_inputs/susmin.sct
0, 0, 550, 0.01, 0
0, 0, 0, 0, 0
-1, 0, 0, 0, 0
0, 0, 550, 0.01, 0
pureh2o.dpf
user_dpfCHL
isotrop.dpf
user_dpfTSS_b
 120
400, 405, 410, 415, 420, 425, 430, 435, 440, 445,
450, 455, 460, 465, 470, 475, 480, 485, 490, 495,
500, 505, 510, 515, 520, 525, 530, 535, 540, 545,
550, 555, 560, 565, 570, 575, 580, 585, 590, 595,
600, 605, 610, 615, 620, 625, 630, 635, 640, 645,
650, 655, 660, 665, 670, 675, 680, 685, 690, 695,
700, 705, 710, 715, 720, 725, 730, 735, 740, 745,
750, 755, 760, 765, 770, 775, 780, 785, 790, 795,
800, 805, 810, 815, 820, 825, 830, 835, 840, 845,
850, 855, 860, 865, 870, 875, 880, 885, 890, 895,
900, 905, 910, 915, 920, 925, 930, 935, 940, 945,
950, 955, 960, 965, 970, 975, 980, 985, 990, 995,
1000,
0,0,0,0,2
2, 3, 48, 0, 0
272, 43.28085,-77.61919, 29.92, 1, 80, 2.5, 15, 4.99746, 300
4.99746, 1.34, 20, 35
0, 0
0, 5, 0, 5, 10, 15, 20, 
/home/jxc4005/hydrolight52Javier_install/data/H2OabDefaults_FRESHwater.txt
1
/home/jxc4005/hydrolight5Aaron_install/data/user/mascot_ac9.txt
dummyFilteredAc9.txt
dummyHscat.txt
/home/jxc4005/hydrolight5Aaron_install/data/user/Chlzdata_10m.txt
dummyComp.txt
dummyR.bot
dummydata.txt
/home/jxc4005/hydrolight5Aaron_install/data/user/Chlzdata_10m.txt
dummyComp.txt
dummyComp.txt
/home/jxc4005/hydrolight5Aaron_install/data/user/Ed_total.txt
/home/jxc4005/hydrolight5Aaron_install/data/MyBiolumData.txt
\end{lstlisting}

\end{appendices}