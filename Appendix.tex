% !TEX root=Thesis_PhD.tex 
% the previous is to reference main.bib
%% CHAPTER
\begin{appendices}
% @@@@@@@@@@@@@@@@@@@@@@@@@@@@@@@@@@@@@@@@@@@@@@@@@@@@@@@@@@@@@@@@@@@@@@@@@@@@@
\chapter{Field Measurements}
% \addcontentsline{toc}{chapter}{Appendix}
% \renewcommand{\thesection}{\Alph{section}}
% @@@@@@@@@@@@@@@@@@@@@@@@@@@@@@@@@@@@@@@@@@@@@@@@@@@@@@@@@@@@@@@@@@@@@@@@@@@@@
\label{ch:fieldmea}
% -----------------------------------------------------------------------------
\section{$R_{rs}$}
% -----------------------------------------------------------------------------
\section{Water Samples}

% @@@@@@@@@@@@@@@@@@@@@@@@@@@@@@@@@@@@@@@@@@@@@@@@@@@@@@@@@@@@@@@@@@@@@@@@@@@@@
\chapter{Lab Measurements}
\label{ch:labmea} 
% -----------------------------------------------------------------------------
\section{IOPs}
\subsection{Chlorophyll}
Spectrophotometric methods
\subsection{Minerals}

\subsection{CDOM}
% -----------------------------------------------------------------------------
\section{Concentrations}
\subsection{Chlorophyll-{\it a}}

\subsection{TSS}



method described by \cite{Mobley:1999} for measuring the spectra of the downwelling irradiance $E_d$, the surface reflected sky radiance $L_s$, and the water-leaving radiance $L_w$ for each site 
% @@@@@@@@@@@@@@@@@@@@@@@@@@@@@@@@@@@@@@@@@@@@@@@@@@@@@@@@@@@@@@@@@@@@@@@@@@@@@
\chapter{HydroLight Specification}

\end{appendices}