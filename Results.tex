% !TEX root=Thesis_PhD.tex 
% the previous is to reference main .bib
%% CHAPTER
\chapter{Results}
\label{ch:results}
This chapter is separated in four sections. The first section explains the data collection process, giving some details in the kind of measurements taken in the field. Also, it includes a summary of the data collected as part of this research. The second section gives a brief description of the lab measurements. The atmospheric section shows results from the MoB-ELM atmospheric correction method and validation. Finally, the last section shows the results from the developed retrieval over the area of study.
% ------------------------------
\section{Data Collection}
An example of a Landsat 8 image that includes the area of study is illustrated in \autoref{fig:L8Image130919}.(a). This Landsat 8 image was taken on September 19, 2013 (scene LC80160302013262LGN00). This area includes part of Lake Ontario, the Genesee River, Irondequoit Bay and some nearby ponds (Long Pond and Cranberry Pond) (\autoref{fig:L8Image130919}.(b)). This area of study was chosen because of the very wide range of constituents between the ponds and the open water in the lake, which provides a very stressing case for the retrieval (\S\ref{subsec:areaofstudy}). \autoref{fig:0910913Sites} shows zoom over this area of study for the data collection done on September, $19^{th}$, 2013  with the different sites as an example. The data collections are divided in two crews. One crew, named ``Lake crew'', is in charge of the Irondequoit Bay, Ontario near shore, Ontario off shore, Genesee River plume, Genesee River pier sites (labeled in \autoref{fig:0910913Sites} as IBayN, OntNS, OntOS, RvrPLM and RvrPIER, respectively). The other crew, named ``Ponds crew'', is in charge of the Long Pond north and south, Cranberry Pond sites (labeled in \autoref{fig:0910913Sites} as LongN, LongS and Cranb, respectively).

\begin{figure}[htb]
    \begin{minipage}[c]{0.48\linewidth}
      \centering
      \includegraphics[trim=30 350 30 350,clip,width=7cm]{/Users/javier/Desktop/Javier/PHD_RIT/ConferencesAndApplications/2015_Landsat_Special_Issue/Images/WholeImage130919}
      \centerline{(a)}\medskip
    \end{minipage}
    \hfill
    \begin{minipage}[d]{0.48\linewidth}
      \centering
      \includegraphics[trim=0 150 30 50,width=7cm]{/Users/javier/Desktop/Javier/PHD_RIT/ConferencesAndApplications/2015_Landsat_Special_Issue/Images/WaterAndDTROC}
      \centerline{(b)}\medskip
    \end{minipage}  

    \caption{Landsat 8 image (scene LC80160302013262LGN00) acquired 09-19-2013 (a), with zoom over study region (b) showing the downtown of Rochester, NY and the Rochester Embayment. \label{fig:L8Image130919} }
\end{figure}

\begin{figure}[htb]
  \centering
  \includegraphics[width=12cm]{/Users/javier/Desktop/Javier/PHD_RIT/Latex/Proposal/Images/groundtruth-sitenames-no-ends.jpg}
  \caption{Sites in the Rochester Embayment for the water sample collection on September, $19^{th}$, 2013.\label{fig:0910913Sites} } 
  % \vspace{1cm}
\end{figure}


\begin{figure}[htb]
    \begin{minipage}[c]{0.48\linewidth}
      \centering
      \includegraphics[width=7cm]{/Users/javier/Desktop/Javier/PHD_RIT/Latex/ThesisPhD/Images/LC80170302014272LGN00_ENVI.jpg}
      \centerline{(a)}\medskip
    \end{minipage}
    \hfill
    \begin{minipage}[d]{0.48\linewidth}
      \centering
      \includegraphics[width=7cm]{/Users/javier/Desktop/Javier/PHD_RIT/Latex/ThesisPhD/Images/LC80170302015259LGN00_ENVI.jpg}
      \centerline{(b)}\medskip
    \end{minipage} 

      \caption{Landsat 8 images acquired on (a) 09-29-2014 (scene LC80170302014272LGN00) and (b) 09-16-2015 (scene LC80170302015259LGN00). \label{fig:L8Images1415} }
\end{figure}  

\begin{figure}[htb!]
    \begin{minipage}[c]{1.0\linewidth}
      \centering
      \begin{overpic}[width=12cm]{/Users/javier/Desktop/Javier/PHD_RIT/Latex/ThesisPhD/Images/LC80170302014272LGN00_AreaOfStudy.png}
       \put (95,45) {\Large (a)}
      \end{overpic} 
    \end{minipage}
    \begin{minipage}[d]{1.0\linewidth}
      \centering
      \begin{overpic}[width=12cm]{/Users/javier/Desktop/Javier/PHD_RIT/Latex/ThesisPhD/Images/LC80170302015259LGN00_AreaOfStudy.png}
        \put (95,45) {\Large (b)}
      \end{overpic}
    \end{minipage} 
      \caption{Area of study from the Landsat 8 images acquired on (a) 09-29-2014 (scene LC80170302014272LGN00) and (b) 09-16-2015 (scene LC80170302015259LGN00). \label{fig:ROI_L8Images1415} }
\end{figure}  



Water samples are collected for each site in dark Nalgene bottles. Additionally, remote-sensing reflectances $R_{rs}$ are measured using an ASD and a SVC instrument (one for each crew). Backscattering measurements are taken using a HydroScat-2. This measurement is taken by both crew only if the logistic of the particular day allows it. Otherwise priority is given to the Lake crew. For each site, a location is recorded using a GPS. \autoref{tab:collect} shows a summary of the data collections done in 2013, 2014 and 2015 seasons with the respective available data.

\begin{table}[htb]
  \caption{Summary of 2013, 2014 and 2015 data collections. Three images were used for testing the retrieval (green filling).}
  \centering
  \includegraphics[width=13cm]{/Users/javier/Desktop/Javier/PHD_RIT/Latex/ThesisPhD/Images/Collect131415.png}
  \label{tab:collect}
\end{table}

% ------------------------------
\section{Laboratory Measurements}

After collection, these water samples were analyzed in the lab at the Rochester Institute of Technology (RIT), and concentration and IOPs for each CPA were measured following the procedures described by \cite{Mueller1995}. These measurements are utilized as input to Ecolight for the LUT generation of the retrieval process and the black pixel determination for the MoB-ELM atmospheric correction algorithm. Also, some are utilized for validation of the retrieval process. The $C_a$ and $TSS$ measurements were validated by comparing with measurements analyzed by a credible lab (Monroe County Environmental Laboratory). This comparison between labs showed agreement in the measurements. \autoref{tab:collect} shows the measurements available.\todo{show IOPs figures}

\begin{figure}[htb]
    \centering
    \includegraphics[height=7cm]{/Users/javier/Desktop/Javier/PHD_RIT/Latex/Proposal/Images/CHLaastLONGSJavier.eps}
  \caption{Chlorophyll mass-specific absorption spectra used to create the LUT in HydroLight. \label{fig:CHLaastLONGS} } 
\end{figure}


\begin{figure}[htb]
    \centering
    \includegraphics[height=7cm]{/Users/javier/Desktop/Javier/PHD_RIT/Latex/Proposal/Images/astar_CDOM_LONGS091913.eps}
  \caption{CDOM mass-specific absorption spectra used to create the LUT in HydroLight. \label{fig:CDOMaastLONGS} } 
\end{figure}

\begin{figure}[htb]
    \centering
    \includegraphics[height=7cm]{/Users/javier/Desktop/Javier/PHD_RIT/Latex/Proposal/Images/astar_SM_LONGS091913.eps}
  \caption{SM mass-specific absorption spectra used to create the LUT in HydroLight. \label{fig:SMaastLONGS} } 
\end{figure}

% ------------------------------
\section{Atmospheric Correction}
\subsection{MoB-ELM}

The MoB-ELM atmospheric correction algorithm discussed above was applied to the Landsat 8 scene collected on 09-19-2013 (scene LC80160302013262LGN00) over the Rochester Embayment (latitude: $43^\circ15'32.5''$N and longitude: $77^\circ36'13.1''$W) shown in \autoref{fig:L8Image130919}. The radiance and reflectance spectra of the bright and dark pixels used for the MoB-ELM algorithm for the Landsat 8 scene acquired on 09-19-2013 used in this work are shown in \autoref{fig:MOBELMpxls}. \todo{connect with Methodology figure}. 

\begin{figure}[htbp!]
  \centering
  \includegraphics[height=6cm]{/Users/javier/Desktop/Javier/PHD_RIT/ConferencesAndApplications/2015_Landsat_Special_Issue/Images/ROI130919_150422}
  \caption{$R_{rs}$ for the sites for the 09-19-2013 collection after applying the MoB-ELM atmospheric correction (Labels: LONGS: Long Pond south, CRANB: Cranberry Pond, ONTOS: Lake Ontario off-shore, and ONTNS: Lake Ontario near-shore).\label{fig:RrsROIs130919} } 
\end{figure}

\begin{figure}[htb]
  \begin{minipage}[c]{0.48\linewidth}
    \centering
      \includegraphics[width=8cm]{/Users/javier/Desktop/Javier/PHD_RIT/ConferencesAndApplications/2015_SPIE_SanDiego/Images/ELMrad130929_150422}
    % \vspace{1.5cm}
    \centerline{(a)}\medskip
  \end{minipage}
  \hfill
  \begin{minipage}[d]{0.48\linewidth}
    \centering
      \includegraphics[width=8cm]{/Users/javier/Desktop/Javier/PHD_RIT/ConferencesAndApplications/2015_SPIE_SanDiego/Images/ELMRrs130929_150422}
    % \vspace{1.5cm}
    \centerline{(b)}\medskip
  \end{minipage}
  \caption{Example of the bright and dark pixel used in the MoB-ELM atmospheric correction method for Landsat 8 scene acquired on 09-19-2013. The radiance spectra are obtained from pseudo invariant features for the bright pixel and from a region in the lake with low level of concentrations. The remote-sensing reflectance ($R_{rs}$) spectra for the bright pixels are obtained from the Provisional Landsat 8 Surface Reflectance product from USGS over a bright object in the scene. The $R_{rs}$ spectra of the dark pixel are obtained from the Hydrolight simulating a region in the lake with low level of concentrations.\label{fig:MOBELMpxls} } 
\end{figure}

As an example of the MoB-ELM results, \autoref{fig:RrsROIs130919} shows the $R_{rs}$ spectrum for four different stations of the 09-19-2013 Landsat 8 scene (\autoref{fig:091913Sites}). The shape of these spectra resemble $R_{rs}$ of water pixels. Also, the spectrum corresponding to low concentration of CPAs (i.e. ONTOS: Lake Ontario off-shore and ONTNS: Lake Ontario near-shore) can be distinguished from the rest of the spectrum corresponding to higher concentration of CPAs (i.e. LONGS: Long Pond south, CRANB: Cranberry Pond).


Preliminary results of the model-based ELM atmospheric correction methods for different water bodies in the Rochester Embayment area are shown in \autoref{fig:waterpxs}. This figure shows the spectrum of the water pixels in reflectance values after applying the model-based ELM atmospheric correction. These curves exhibit shapes that correspond with the shapes of typical water pixels. \autoref{fig:refcomp} shows water reflectance spectra as preliminary results from the model-based ELM method (solid lines) compared with results from a traditional ELM method (dashed lines) for four different ROIs in the Rochester Embayment area (Cranberry Pond, Long Pond, and nearshore and offshore of the Lake Ontario, labeled as Cranb, LongS, OntNS and OntOS in \autoref{fig:0910913Sites}, respectively). 


The traditional ELM method was performed with reflectance measurements taken in the field. A reflectance measurement taken of the sand of Charlotte Beach, Rochester, NY (labeled as SandDry in \autoref{fig:0910913Sites}) was used for the bright pixel while a reflectance measurement taken at the site OntNS was used for the dark pixel. The radiance values were taken from the corresponding ROIs in the Landsat 8 image. 

As can be seen in \autoref{fig:refcomp}, the atmospheric correction algorithm proposed in this study exhibits less than one percent reflectance unit ($<0.01\Rightarrow <1\%$) of difference in comparison to the results from the traditional ELM algorithm.

\begin{figure}[htb]
    \centering
      \includegraphics[width=8cm]{/Users/javier/Desktop/Javier/PHD_RIT/ConferencesAndApplications/NESSF14/latex/WaterPixels_2.eps}
      \caption{Water pixel spectra after applying the model-based ELM atmospheric correction method.}
      \label{fig:waterpxs}
\end{figure}


\begin{figure}[htb]
    \centering
      \includegraphics[width=8cm]{/Users/javier/Desktop/Javier/PHD_RIT/ConferencesAndApplications/NESSF14/latex/WaterPixelComparisonELMELMbased}
      \caption{Comparison between traditional ELM (dashed lines) and model-based ELM (solid lines).}
      \label{fig:refcomp}
\end{figure}
\subsection{SeaDAS-SWIR}
\label{subsec:seadasswir}
The second method analyzed in this work was the Gordon and Wang \cite{Gordon:1994} approach for atmospherically correcting satellite data over water, hereafter SeaDAS-SWIR. This method is implemented in the l2gen tool of the the \gls{seadas} software package described in \S\ref{subsec:seadasrrs}.

% The l2gen tool processes heritage Ocean-Color sensor (e.g. MODIS, SeaWiFS, OCTS or CZCS) data (Level 1) and generates Level 2 geophysical products by applying atmospheric corrections and bio-optical algorithms to the sensor. 

The atmospheric correction methods based on the Gordon and Wang approach use standard radiative transfer methods to compute and remove scattering by the air molecules, and two bands where the water contribution is negligible to estimate the aerosol contribution for the rest of the bands. Formerly, two NIR bands were used for estimating the aerosol contribution, but a combination of NIR and SWIR bands could be used when there is some water contribution in the NIR wavelengths \cite{Wang2009}. For this analysis, The bands used for the atmospheric correction were the SWIR 1 and SWIR 2 (SeaDAS-SWIR, band 6 and 7). The OLI's NIR band was not used because the water contribution in the NIR band cannot be considered negligible due to the presence of highly turbid waters in the scene (i.e. ponds). Water has much stronger absorption at the SWIR wavelengths than at the NIR wavelengths. 

A full description of the OLI's processing in SeaDAS is described in \cite{Franz:2015}. The processing in SeaDAS includes a vicarious calibration derived from the marine optical buoy (MOBY) near Lanai, Hawaii. For the atmospheric correction in SeaDAS, the option used in this study was the operational atmospheric correction scheme (aer\_opt=-2), with the shortest and longest sensor wavelength for aerosol models selection equal to band 6 and band 7 (aer\_wave\_short=1609 and aer\_wave\_long=2201). Since a default land/water mask fine enough to resolve the ponds in the scene is not included in SeaDAS, all the pixels were forced to be processed as ocean (proc\_ocean=2) and the land mask was set to the ``landmask\_null.dat'' file. Additionally, in order to increase the SNR of SWIR bands and avoid $R_{rs}$ retrieval failing in the ponds, the SWIR bands were spatially averaged with a 5x5 window. This was accomplished by including a filter (filter\_opt=1) and modifying the default SeaDAS's file ``msl12\_filter.dat''.
% -----------------------------------------------------------
\subsection{Acolite-SWIR}
The third atmospheric correction method analyzed was the implementation of the \cite{Gordon:1994}'s approach in the atmospheric correction for OLI lite (Acolite) tool (\S\ref{subsec:acolite}). Acolite was used as a clone for the SeaDAS-SWIR method described above in \S~\ref{subsec:seadasswir}. Therefore, the option used was the default ``SWIR atmospheric correction'' with the ``Franz Ave'' gains for the vicarious calibration \cite{Franz:2015}.
% -----------------------------------------------------------
\subsection{SeaDAS-MUMM}
The fourth method evaluated in this work was the Ruddick {\it et al.}'s atmospheric correction approach \cite{Ruddick:2000bs}, hereafter SeaDAS-MUMM (\acrfull{mumm}). This method is a modification of \cite{Gordon:1994}'s algorithm for use over turbid coastal and inland waters (Case 2) or high productive Case 1 waters. In this study, this method was applied using its implementation available in SeaDAS (aer\_opt=-10), using bands 4 and 5 and setting the calibration parameter $\alpha=8.7$ for OLI, as suggested by \cite{Vanhellemont2014} and \cite{Vanhellemont2014a}. 
          % -10: Multi-scattering with MUMM correction
          %      and MUMM NIR calculation

% ------------------------------
\subsection{$R_{rs}$ Comparison}

\todo{Include new figures in the description} The four atmospheric correction methods (MoB-ELM, SeDAS-SWIR, Acolite-SWIR and SeaDAS-MUMM) described in Section~\ref{sec:atmcorr} \todo{Check section!} are compared in this section. These algorithm were applied to the 09-19-2013 Landsat 8 image. A mask was created to mask out all pixels but water pixels. This mask was created by thresholding the Landsat 8's SWIR 2 band. \autoref{fig:Rrs443}--\ref{fig:Rrs561} show the $R_{rs}$ for band 1--3 for these methods. When analyzed visually, the similarities among the methods are higher in band 3 than band 1. For band 1, SeaDAS-SWIR and SeaDAS-MUMM look similar in both the lake and ponds. Also, the SeaDAS-SWIR method exhibits some noise in band 1. This is caused due to the low SNR in SWIR bands used for the atmospheric correction. Note that there are some bottom effect in the lake's shoreline causing all methods to retrieve high $R_{rs}$ in those areas, which is not the case. This should be taken in account for future processing by masking the areas where the water is clear enough that the bottom can be seen.

Another comparison of $R_{rs}$ at $443nm$ and $561nm$ among all four atmospheric correction methods is shown in \autoref{fig:13262Rrs443}--\ref{fig:13262Rrs561} as scatter plots with a regression line in red solid lines. Again, all the methods show more similarities in band 3 than band 1, which is corroborated by the smaller offset values for band 3 than band 1, which translates to more closeness to the $1:1$ line. This indicates that there is a higher correlation for band 3 than band 1 for all methods. \autoref{tab:RrsCompMethod} shows the slope and offset for the regression lines and the goodness of fit $R^2$ values for the comparison among all methods and for all bands. The $R^2$ values show a good correlation for all cases in the range from $0.7090$ to $0.9302$ (underlined in \autoref{tab:RrsCompMethod}). \autoref{tab:RrsCompMethod} also shows the number of valid pixels retrieved $N$ (non negative) and the root mean squared error (RMSE) between the methods compared. 

All methods produced negative $R_{r}$ values. These negative values indicates that the atmosphere contribution has been over estimated. These negative values are shown in \autoref{tab:RrsCompMethod} as percentage of the total of water pixels for each method compared. These negative values were not used in the comparison. Finally, \autoref{tab:RrsCompMethod} shows the percentage of total number of water pixels used in the comparison (labeled as Used $[\%]$). It can be seen that the Gordon and Wang's based methods generate more negative values compared with the MoB-ELM method overall. 

\autoref{fig:13262RrsCompField} shows a comparison between $R_{rs}$ spectra measured in the field (blue dashed line) with retrieved $R_{rs}$ spectra from the four atmospheric correction methods for different sites within the study area. The spectra for each method tend to differ more from the field spectra in band 1 and 2 than in band 3 and 4, for all sites. We calculated the normalized root mean squared error (NRMSE) for $R_{rs}(\lambda)$ to evaluate the differences between field measurements and spectra retrieved from the atmospheric correction methods. The NRMSE is defined as

\begin{equation}
\label{eq:NRMSE}
  NRMSE =\frac{\sqrt{\frac{1}{N}\sum_{n=1}^N{\left[R_{rs}(\lambda)_{ret}(n) - R_{rs}(\lambda)_{mea}(n)\right]^2}}}{max\{R_{rs}(\lambda)_{mea}(n)\} - min\{R_{rs}(\lambda)_{mea}(n)\}}\times100 ~[\%]
\end{equation}

\noindent where $R_{rs}(\lambda)_{ret}$ is the retrieved $R_{rs}(\lambda)$, $R_{rs}(\lambda)_{mea}$ is the measured $R_{rs}(\lambda)$, and $n=1\dots N$ is the number of measured concentrations. \autoref{fig:NRMSE130919_RRS} shows the NRMSE for all the atmospheric correction methods for all bands. It can be seen that the MoB-ELM perform the best for all bands, followed by SeaDAS-SWIR. 
% The biggest errors are obtained when applying the Acolite-SWIR method, specially in band 1.



% \begin{figure}[htb]
%   \centering
%   \includegraphics[height=8.0cm]{/Users/javier/Desktop/Javier/PHD_RIT/ConferencesAndApplications/2015_SPIE_SanDiego/Images/Collated2013262_2_band_1_D.png}
%   \caption{$R_{rs}$ for the sites for the 09-19-2013 collection after applying the MoB-ELM atmospheric correction (Labels: LONGS: Long Pond south, CRANB: Cranberry Pond, ONTOS: Lake Ontario off-shore, and ONTNS: Lake Ontario near-shore).\label{fig:RrsROIs130919} } 
% \end{figure}

% \begin{overpic}[width=0.5\textwidth,grid,tics=10]{pictures/baum}
%  \put (20,85) {\huge$\displaystyle\gamma$}
% \end{overpic}
%^^^^^^^^^^^^^^^^^^^  FIGURE ^^^^^^^^^^^^^^^^^^^^^^^^^^^^^^^^^^^^^^^^^^^^
\begin{figure}[htb]
  \begin{minipage}[c]{0.48\linewidth}
      \centering
      \begin{overpic}[trim=0 200 0 0,clip,width=6.5cm]{/Users/javier/Desktop/Javier/PHD_RIT/ConferencesAndApplications/2015_SPIE_SanDiego/Images/subset_0_of_Collocated13262_ACOSWIR_MOB_SEA_Rrs_443MOBdivpi}
      \put (5,5) {MOB-ELM}
      \end{overpic}
    \end{minipage}
    \hfill
  \begin{minipage}[c]{0.48\linewidth}
      \centering
      \begin{overpic}[trim=0 200 0 0,clip,width=6.5cm]{/Users/javier/Desktop/Javier/PHD_RIT/ConferencesAndApplications/2015_SPIE_SanDiego/Images/subset_0_of_Collocated13262_ACOSWIR_MOB_SEA_Rrs_443ACO}
      \put (5,5) {Acolite-SWIR}
      \end{overpic}
    \end{minipage}

    \vspace{0.7cm}

  \begin{minipage}[c]{0.48\linewidth}
      \centering
      \begin{overpic}[trim=0 200 0 0,clip,width=6.5cm]{/Users/javier/Desktop/Javier/PHD_RIT/ConferencesAndApplications/2015_SPIE_SanDiego/Images/subset_0_of_Collocated13262_ACOSWIR_MOB_SEA_Rrs_443SEA}
      \put (5,5) {SeaDAS-SWIR}
      \end{overpic}
    \end{minipage}
    \hfill
  \begin{minipage}[c]{0.48\linewidth}
      \centering
      \begin{overpic}[trim=30 170 40 150,clip,width=6.5cm]{/Users/javier/Desktop/Javier/PHD_RIT/ConferencesAndApplications/2015_SPIE_SanDiego/Images/Collocated13262_ACOSWIR_MOB_SEA5x5_MUMM45_Rrs_443_MUMM45}
      \put (5,5) {SeaDAS-MUMM}
      \end{overpic}
    \end{minipage}

    \begin{minipage}[c]{1.0\linewidth}
      \centering
      \vspace{0.5cm}
      \begin{overpic}[trim=0 0 0 0,clip,height=1.2cm]{/Users/javier/Desktop/Javier/PHD_RIT/ConferencesAndApplications/2015_SPIE_SanDiego/Images/Collocated13262_ACOSWIR_MOB_SEA5x5_MUMM45_colorbar}
      \put (28,16) {$R_{rs}(443nm) [1/sr]$}
      \end{overpic}
    \end{minipage}

  \caption{Remote-sensing reflectance ($R_{rs}$) at $443nm$ from the 09-19-2013 image over the Rochester Embayment (scene LC80160302013262LGN00) processed using the MoB-ELM, SeaDAS-SWIR, Acolite-SWIR and SeaDAS-MUMM.\label{fig:Rrs443} } 
\end{figure}
% %^^^^^^^^^^^^^^^^^^^  FIGURE ^^^^^^^^^^^^^^^^^^^^^^^^^^^^^^^^^^^^^^^^^^^^
\begin{figure}[htb]
  \begin{minipage}[c]{0.48\linewidth}
      \centering
      \begin{overpic}[trim=0 150 40 150,clip,width=6.5cm]{/Users/javier/Desktop/Javier/PHD_RIT/ConferencesAndApplications/2015_SPIE_SanDiego/Images/Collocated13262_ACOSWIR_MOB_SEA5x5_MUMM45_Rrs_483_MOB}
      \put (5,6) {MOB-ELM}
      \end{overpic}
    \end{minipage}
    \hfill
  \begin{minipage}[c]{0.48\linewidth}
      \centering
      \begin{overpic}[trim=0 0 40 0,clip,width=6.5cm]{/Users/javier/Desktop/Javier/PHD_RIT/ConferencesAndApplications/2015_SPIE_SanDiego/Images/Collocated13262_ACOSWIR_MOB_SEA5x5_MUMM45_Rrs_483_ACO_R_R}
      \put (5,5) {Acolite-SWIR}
      \end{overpic}
    \end{minipage}

    \vspace{0.7cm}

  \begin{minipage}[c]{0.48\linewidth}
      \centering
      \begin{overpic}[trim=0 0 40 0,clip,width=6.5cm]{/Users/javier/Desktop/Javier/PHD_RIT/ConferencesAndApplications/2015_SPIE_SanDiego/Images/Collocated13262_ACOSWIR_MOB_SEA5x5_MUMM45_Rrs_482_SEA5x5_R}
      \put (5,5) {SeaDAS-SWIR}
      \end{overpic}
    \end{minipage}
    \hfill
  \begin{minipage}[c]{0.48\linewidth}
      \centering
      \begin{overpic}[trim=0 150 40 150,clip,width=6.5cm]{/Users/javier/Desktop/Javier/PHD_RIT/ConferencesAndApplications/2015_SPIE_SanDiego/Images/Collocated13262_ACOSWIR_MOB_SEA5x5_MUMM45_Rrs_482_MUMM45}
      \put (5,5) {SeaDAS-MUMM}
      \end{overpic}
    \end{minipage}
    

    \begin{minipage}[c]{1.0\linewidth}
      \centering
      \vspace{0.5cm}
      \begin{overpic}[trim=0 0 0 0,clip,height=1.2cm]{/Users/javier/Desktop/Javier/PHD_RIT/ConferencesAndApplications/2015_SPIE_SanDiego/Images/Collocated13262_ACOSWIR_MOB_SEA5x5_MUMM45_colorbar}
      \put (28,16) {$R_{rs}(482nm) [1/sr]$}
      \end{overpic}
    \end{minipage}

  \caption{Remote-sensing reflectance ($R_{rs}$) at $482nm$ from the 09-19-2013 image over the Rochester Embayment (scene LC80160302013262LGN00) processed using the MoB-ELM, SeaDAS-SWIR, Acolite-SWIR and SeaDAS-MUMM.\label{fig:Rrs482} } 
\end{figure}
%^^^^^^^^^^^^^^^^^^^  FIGURE ^^^^^^^^^^^^^^^^^^^^^^^^^^^^^^^^^^^^^^^^^^^^
\begin{figure}[htb]
  \begin{minipage}[c]{0.48\linewidth}
      \centering
      \begin{overpic}[trim=0 155 40 150,clip,width=6.5cm]{/Users/javier/Desktop/Javier/PHD_RIT/ConferencesAndApplications/2015_SPIE_SanDiego/Images/Collocated13262_ACOSWIR_MOB_SEA5x5_MUMM45_Rrs_561_MOB}
      \put (5,5) {MOB-ELM}
      \end{overpic}
    \end{minipage}
    \hfill
  \begin{minipage}[c]{0.48\linewidth}
      \centering
      \begin{overpic}[trim=0 150 40 150,clip,width=6.5cm]{/Users/javier/Desktop/Javier/PHD_RIT/ConferencesAndApplications/2015_SPIE_SanDiego/Images/Collocated13262_ACOSWIR_MOB_SEA5x5_MUMM45_Rrs_561_ACO_R_R}
      \put (5,5) {Acolite-SWIR}
      \end{overpic}
    \end{minipage}

    \vspace{0.7cm}

  \begin{minipage}[c]{0.48\linewidth}
      \centering
      \begin{overpic}[trim=0 150 40 150,clip,width=6.5cm]{/Users/javier/Desktop/Javier/PHD_RIT/ConferencesAndApplications/2015_SPIE_SanDiego/Images/Collocated13262_ACOSWIR_MOB_SEA5x5_MUMM45_Rrs_561_SEA5x5_R}
      \put (5,5) {SeaDAS-SWIR}
      \end{overpic}
    \end{minipage}
    \hfill
  \begin{minipage}[c]{0.48\linewidth}
      \centering
      \begin{overpic}[trim=0 150 40 150,clip,width=6.5cm]{/Users/javier/Desktop/Javier/PHD_RIT/ConferencesAndApplications/2015_SPIE_SanDiego/Images/Collocated13262_ACOSWIR_MOB_SEA5x5_MUMM45_Rrs_561_MUMM45}
      \put (5,5) {SeaDAS-MUMM}
      \end{overpic}
    \end{minipage}
    

    \begin{minipage}[c]{1.0\linewidth}
      \centering
      \vspace{0.5cm}
      \begin{overpic}[trim=0 0 0 0,clip,height=1.2cm]{/Users/javier/Desktop/Javier/PHD_RIT/ConferencesAndApplications/2015_SPIE_SanDiego/Images/Collocated13262_ACOSWIR_MOB_SEA5x5_MUMM45_colorbar}
      \put (28,16) {$R_{rs}(561nm) [1/sr]$}
      \end{overpic}
    \end{minipage}

  \caption{Remote-sensing reflectance ($R_{rs}$) at $561nm$ from the 09-19-2013 image over the Rochester Embayment (scene LC80160302013262LGN00) processed using the MoB-ELM, SeaDAS-SWIR, Acolite-SWIR and SeaDAS-MUMM..\label{fig:Rrs561} } 
\end{figure}
%^^^^^^^^^^^^^^^^^^^  FIGURE ^^^^^^^^^^^^^^^^^^^^^^^^^^^^^^^^^^^^^^^^^^^^
\begin{figure}[htb]
  \begin{minipage}[c]{0.48\linewidth}
      \centering
      \begin{overpic}[trim=0 0 40 0,clip,width=6.5cm]{/Users/javier/Desktop/Javier/PHD_RIT/ConferencesAndApplications/2015_SPIE_SanDiego/Images/Collocated13262_ACOSWIR_MOB_SEA5x5_MUMM45_Rrs_655_MOB}
      \put (5,5) {MOB-ELM}
      \end{overpic}
    \end{minipage}
    \hfill
  \begin{minipage}[c]{0.48\linewidth}
      \centering
      \begin{overpic}[trim=0 0 40 0,clip,width=6.5cm]{/Users/javier/Desktop/Javier/PHD_RIT/ConferencesAndApplications/2015_SPIE_SanDiego/Images/Collocated13262_ACOSWIR_MOB_SEA5x5_MUMM45_Rrs_655_ACO_R_R}
      \put (5,5) {Acolite-SWIR}
      \end{overpic}
    \end{minipage}

    \vspace{0.7cm}

  \begin{minipage}[c]{0.48\linewidth}
      \centering
      \begin{overpic}[trim=0 0 40 0,clip,width=6.5cm]{/Users/javier/Desktop/Javier/PHD_RIT/ConferencesAndApplications/2015_SPIE_SanDiego/Images/Collocated13262_ACOSWIR_MOB_SEA5x5_MUMM45_Rrs_655_SEA5x5_R}
      \put (5,5) {SEADAS-SWIR}
      \end{overpic}
    \end{minipage}
    \hfill
  \begin{minipage}[c]{0.48\linewidth}
      \centering
      \begin{overpic}[trim=0 0 40 0,clip,width=6.5cm]{/Users/javier/Desktop/Javier/PHD_RIT/ConferencesAndApplications/2015_SPIE_SanDiego/Images/Collocated13262_ACOSWIR_MOB_SEA5x5_MUMM45_Rrs_655_MUMM45}
      \put (5,5) {SeaDAS-MUMM}
      \end{overpic}
    \end{minipage}
    

    \begin{minipage}[c]{1.0\linewidth}
      \centering
      \vspace{0.5cm}
      \begin{overpic}[trim=0 0 0 0,clip,height=1.2cm]{/Users/javier/Desktop/Javier/PHD_RIT/ConferencesAndApplications/2015_SPIE_SanDiego/Images/Collocated13262_ACOSWIR_MOB_SEA5x5_MUMM45_colorbar}
      \put (28,16) {$R_{rs}(655nm) [1/sr]$}
      \end{overpic}
    \end{minipage}

  \caption{Remote-sensing reflectance ($R_{rs}$) at $655nm$ from the 09-19-2013 image over the Rochester Embayment (scene LC80160302013262LGN00) processed using the MoB-ELM, SeaDAS-SWIR, Acolite-SWIR and SeaDAS-MUMM.\label{fig:Rrs655} } 
\end{figure}

% Method 1    Method 2    wl    NegTool1  NegTool2    usable  rsq_SS      rsq_corr    slope   offset      R^2         N           RMSE
% Acolite     MoB-ELM     443   0.53      0.09        97      .0.8791     0.8828      0.9444  -0.0047     0.8791      144047     0.0054
% Acolite     SeaDAS      443   0.53      76.74       98      .0.7804     0.7924      1.1437  -0.0053     0.7804      145186     0.0038
% Acolite     MUMM        443   0.53      74.41       99      .0.8554     0.8606      1.0600  -0.0032     0.8554      147730     0.0026
% SeaDAS      MoB-ELM     443   43.98     0.05        55      .0.7637     0.7776      0.8664  -0.0007     0.7637      141563     0.0019
% SeaDAS      MUMM        443   43.98     42.64       56      .0.8456     0.8516      0.9175  +0.0018     0.8456      145315     0.0014
% MUMM        MoB-ELM     443   42.64     0.05        56      .0.7474     0.7634      0.8811  -0.0018     0.7474      144947     0.0029
% Acolite     MoB-ELM     483   0.51      0.00        97      .0.9302     0.9314      0.9353  -0.0030     0.9302      144077     0.0037
% Acolite     SeaDAS      483   0.51      76.36       98      .0.8739     0.8779      1.1269  -0.0041     0.8739      145692     0.0028
% Acolite     MUMM        483   0.51      74.29       99      .0.9175     0.9192      1.0693  -0.0024     0.9175      147837     0.0018
% SeaDAS      MoB-ELM     483   43.76     0.00        55      .0.8454     0.8514      0.8547  +0.0002     0.8454      142116     0.0014
% SeaDAS      MUMM        483   43.76     42.57       56      .0.9122     0.9141      0.9441  +0.0015     0.9122      145888     0.0012
% MUMM        MoB-ELM     483   42.57     0.00        56      .0.8408     0.8471      0.8639  -0.0007     0.8408      145138     0.0022
% Acolite     MoB-ELM     561   0.44      0.00        97      .0.9044     0.9067      1.0183  -0.0011     0.9044      144192     0.0011
% Acolite     SeaDAS      561   0.44      75.56       99      .0.8233     0.8311      1.0307  -0.0013     0.8233      146761     0.0013
% Acolite     MUMM        561   0.44      74.15       100     .0.8968     0.8995      0.9663  -0.0001     0.8968      148040     0.0007
% SeaDAS      MoB-ELM     561   43.30     0.00        55      .0.7828     0.7946      1.0043  +0.0001     0.7828      143288     0.0009
% SeaDAS      MUMM        561   43.30     42.49       57      .0.8728     0.8768      0.9374  +0.0011     0.8728      147116     0.0010
% MUMM        MoB-ELM     561   42.49     0.00        56      .0.8468     0.8527      1.0611  -0.0010     0.8468      145339     0.0009
% Acolite     MoB-ELM     655   0.49      0.00        97      .0.8592     0.8641      1.1703  -0.0017     0.8592      144108     0.0013
% Acolite     SeaDAS      655   0.49      76.22       98      .0.7090     0.7301      0.9628  -0.0011     0.7090      145798     0.0014
% Acolite     MUMM        655   0.49      74.15       99      .0.7880     0.7992      0.9700  -0.0008     0.7880      147956     0.0010
% SeaDAS      MoB-ELM     655   43.68     0.00        55      .0.7708     0.7840      1.2158  -0.0004     0.7708      142375     0.0007
% SeaDAS      MUMM        655   43.68     42.49       56      .0.7964     0.8067      1.0063  +0.0004     0.7964      146144     0.0007
% MUMM        MoB-ELM     655   42.49     0.00        56      .0.6841     0.7091      1.2403  -0.0008     0.6841      145340     0.0009   
 
\begin{table}[!ht]
\vspace{.3cm}
\caption{ Comparison of the different atmospheric correction methods for retrieving $R_{rs}$ with the slope and offset for the regression lines and their respective goodness of fit values. \label{tab:RrsCompMethod} } 
\centering
\vspace{.2cm}
\tiny
\begin{tabular}{cllcccccccc} 
 % \bfseries{Band n} & \bfseries{$m$}      & \bfseries{$y_0$}    & \bfseries{$R^2$}     & \bfseries{$RMSE$} & $y(x=45^\circ)$   \\ \hline \hline
Band    &   Method 1      &  Method 2   & Slope   & Offset  & $R^2 $  & N       & RMSE    &\multicolumn{2}{c}{$R_{rs}<0~[\%]$}   &   Used    \\ 
$[nm]$    &             &         &       &     &       &       &$[mg/m^3]$ & Method 1    & Method 2         & $[\%]$    \\ \hline \hline
\multirow{6}{*}{443}&Acolite-SWIR&MoB-ELM     & 0.9444  & -0.0047 & 0.8791  & 144047  & 0.0054  &  ~0.53      & ~0.09            &   97      \\
      &   SeaDAS-SWIR   &  MoB-ELM      & 0.8664  & -0.0007 & 0.7637  & 141563  & 0.0019  &  43.98      & ~0.05            &   55      \\
      &   SeaDAS-MUMM   &  MoB-ELM      & 0.8811  & -0.0018 & 0.7474  & 144947  & 0.0029  &  42.64      & ~0.05            &   56      \\
      &   Acolite-SWIR  &  SeaDAS-SWIR  & 1.1437  & -0.0053 & 0.7804  & 145186  & 0.0038  &  ~0.53      & 76.74            &   98      \\
      &   Acolite-SWIR  &  SeaDAS-MUMM  & 1.0600  & -0.0032 & 0.8554  & 147730  & 0.0026  &  ~0.53      & 74.41            &   99      \\
      &   SeaDAS-SWIR   &  SeaDAS-MUMM  & 0.9175  & ~0.0018 & 0.8456  & 145315  & 0.0014  &  43.98      & 42.64            &   56      \\  \hline
\multirow{6}{*}{448}&Acolite-SWIR&MoB-ELM     & 0.9353  & -0.0030 & \underline{0.9302}  & 144077  & 0.0037  &  ~0.51      & ~0.00            &   97      \\
      &   SeaDAS-SWIR   &  MoB-ELM      & 0.8547  & ~0.0002 & 0.8454  & 142116  & 0.0014  &  43.76      & ~0.00            &   55      \\
      &   SeaDAS-MUMM   &  MoB-ELM      & 0.8639  & -0.0007 & 0.8408  & 145138  & 0.0022  &  42.57      & ~0.00            &   56      \\ 
      &   Acolite-SWIR  &  SeaDAS-SWIR  & 1.1269  & -0.0041 & 0.8739  & 145692  & 0.0028  &  ~0.51      & 76.36            &   98      \\
      &   Acolite-SWIR  &  SeaDAS-MUMM  & 1.0693  & -0.0024 & 0.9175  & 147837  & 0.0018  &  ~0.51      & 74.29            &   99      \\
      &   SeaDAS-SWIR   &  SeaDAS-MUMM  & 0.9441  & ~0.0015 & 0.9122  & 145888  & 0.0012  &  43.76      & 42.57            &   56      \\ \hline
\multirow{6}{*}{561}&Acolite-SWIR&  MoB-ELM   & 1.0183  & -0.0011 & 0.9044  & 144192  & 0.0011  &  ~0.44      & ~0.00            &   97      \\
      &   SeaDAS-SWIR   &  MoB-ELM      & 1.0043  & ~0.0001 & 0.7828  & 143288  & 0.0009  &  43.30      & ~0.00            &   55      \\
      &   SeaDAS-MUMM   &  MoB-ELM      & 1.0611  & -0.0010 & 0.8468  & 145339  & 0.0009  &  42.49      & ~0.00            &   56      \\
      &   Acolite-SWIR  &  SeaDAS-SWIR  & 1.0307  & -0.0013 & 0.8233  & 146761  & 0.0013  &  ~0.44      & 75.56            &   99      \\
      &   Acolite-SWIR  &  SeaDAS-MUMM  & 0.9663  & -0.0001 & 0.8968  & 148040  & 0.0007  &  ~0.44      & 74.15            &   100     \\
        &   SeaDAS-SWIR   &  SeaDAS-MUMM  & 0.9374  & ~0.0011 & 0.8728  & 147116  & 0.0010  &  43.30      & 42.49            &   57      \\ \hline
\multirow{6}{*}{655}&Acolite-SWIR&MoB-ELM     & 1.1703  & -0.0017 & 0.8592  & 144108  & 0.0013  &  ~0.49      & ~0.00            &   97      \\
      &   SeaDAS-SWIR   &  MoB-ELM      & 1.2158  & -0.0004 & 0.7708  & 142375  & 0.0007  &  43.68      & ~0.00            &   55      \\
      &   SeaDAS-MUMM   &  MoB-ELM      & 1.2403  & -0.0008 & 0.6841  & 145340  & 0.0009  &  42.49      & ~0.00            &   56      \\ 
      &   Acolite-SWIR  &  SeaDAS-SWIR  & 0.9628  & -0.0011 & \underline{0.7090}  & 145798  & 0.0014  &  ~0.49      & 76.22            &   98      \\
      &   Acolite-SWIR  &  SeaDAS-MUMM  & 0.9700  & -0.0008 & 0.7880  & 147956  & 0.0010  &  ~0.49      & 74.15            &   99      \\
      &   SeaDAS-SWIR   &  SeaDAS-MUMM  & 1.0063  & ~0.0004 & 0.7964  & 146144  & 0.0007  &  43.68      & 42.49            &   56      \\
 \end{tabular}
\end{table}
%^^^^^^^^^^^^^^^^^^^  FIGURE ^^^^^^^^^^^^^^^^^^^^^^^^^^^^^^^^^^^^^^^^^^^^
\begin{figure}[htb]
  \begin{minipage}[c]{0.48\linewidth}
      \centering
      \begin{overpic}[trim=0 280 0 0,clip,width=6.5cm]{/Users/javier/Desktop/Javier/PHD_RIT/ConferencesAndApplications/2015_SPIE_SanDiego/Images/2013262_ACOMOBSEAMUM_443_Acolite-SWIR_SeaDAS-MUMM.png}
      % \put (65,17) {\large A) $443nm$}
      \end{overpic}  
  \end{minipage}
  \hfill
  \begin{minipage}[d]{0.48\linewidth}
    \centering
      \begin{overpic}[trim=0 280 0 0,clip,width=6.5cm]{/Users/javier/Desktop/Javier/PHD_RIT/ConferencesAndApplications/2015_SPIE_SanDiego/Images/2013262_ACOMOBSEAMUM_443_Acolite-SWIR_MoB-ELM.png}
      % \put (65,17) {\large B) $483nm$}
      \end{overpic}
  \end{minipage}

  \begin{minipage}[c]{0.48\linewidth}
      \centering
      \begin{overpic}[trim=0 280 0 0,clip,width=6.5cm]{/Users/javier/Desktop/Javier/PHD_RIT/ConferencesAndApplications/2015_SPIE_SanDiego/Images/2013262_ACOMOBSEAMUM_443_Acolite-SWIR_SeaDAS-SWIR.png}
      % \put (65,17) {\large C) $561nm$}
      \end{overpic}  
  \end{minipage}
  \hfill
  \begin{minipage}[d]{0.48\linewidth}
    \centering
      \begin{overpic}[trim=0 280 0 0,clip,width=6.5cm]{/Users/javier/Desktop/Javier/PHD_RIT/ConferencesAndApplications/2015_SPIE_SanDiego/Images/2013262_ACOMOBSEAMUM_443_SeaDAS-MUMM_MoB-ELM.png}
      % \put (65,17) {\large D) $655nm$}
      \end{overpic}
  \end{minipage}

  \begin{minipage}[c]{0.48\linewidth}
      \centering
      \begin{overpic}[trim=0 280 0 0,clip,width=6.5cm]{/Users/javier/Desktop/Javier/PHD_RIT/ConferencesAndApplications/2015_SPIE_SanDiego/Images/2013262_ACOMOBSEAMUM_443_SeaDAS-SWIR_SeaDAS-MUMM.png}
      % \put (65,17) {\large C) $561nm$}
      \end{overpic}  
  \end{minipage}
  \hfill
  \begin{minipage}[d]{0.48\linewidth}
    \centering
      \begin{overpic}[trim=0 280 0 0,clip,width=6.5cm]{/Users/javier/Desktop/Javier/PHD_RIT/ConferencesAndApplications/2015_SPIE_SanDiego/Images/2013262_ACOMOBSEAMUM_443_SeaDAS-SWIR_MoB-ELM.png}
      % \put (65,17) {\large D) $655nm$}
      \end{overpic}
  \end{minipage}

  \begin{minipage}[d]{1.0\linewidth}
    \centering
      \begin{overpic}[trim=70 0 0 1470,clip,width=6.5cm]{/Users/javier/Desktop/Javier/PHD_RIT/ConferencesAndApplications/2015_SPIE_SanDiego/Images/2013262_ACOMOBSEAMUM_655_Acolite-SWIR_SeaDAS-MUMM.png}
      \end{overpic}
  \end{minipage}    

% 
  \caption{Scatter plots showing the comparison of remote-sensing reflectance ($R_{rs}$) at 443 nm, derived from the 09-29-2013 image over the Rochester Embayment (scene LC80160302013262LGN00) using the different methods. Colors denote pixel densities, the dashed black line is the 1:1 line, and the Reduced Major Axis (RMA) regression line is drawn in red. \label{fig:13262Rrs443} } 
\end{figure}

%^^^^^^^^^^^^^^^^^^^  FIGURE ^^^^^^^^^^^^^^^^^^^^^^^^^^^^^^^^^^^^^^^^^^^^
\begin{figure}[htb]
  \begin{minipage}[c]{0.48\linewidth}
      \centering
      \begin{overpic}[trim=0 280 0 0,clip,width=6.5cm]{/Users/javier/Desktop/Javier/PHD_RIT/ConferencesAndApplications/2015_SPIE_SanDiego/Images/2013262_ACOMOBSEAMUM_483_Acolite-SWIR_SeaDAS-MUMM.png}
      % \put (65,17) {\large A) $443nm$}
      \end{overpic}  
  \end{minipage}
  \hfill
  \begin{minipage}[d]{0.48\linewidth}
    \centering
      \begin{overpic}[trim=0 280 0 0,clip,width=6.5cm]{/Users/javier/Desktop/Javier/PHD_RIT/ConferencesAndApplications/2015_SPIE_SanDiego/Images/2013262_ACOMOBSEAMUM_483_Acolite-SWIR_MoB-ELM.png}
      % \put (65,17) {\large B) $483nm$}
      \end{overpic}
  \end{minipage}

  \begin{minipage}[c]{0.48\linewidth}
      \centering
      \begin{overpic}[trim=0 280 0 0,clip,width=6.5cm]{/Users/javier/Desktop/Javier/PHD_RIT/ConferencesAndApplications/2015_SPIE_SanDiego/Images/2013262_ACOMOBSEAMUM_483_Acolite-SWIR_SeaDAS-SWIR.png}
      % \put (65,17) {\large C) $483nm$}
      \end{overpic}  
  \end{minipage}
  \hfill
  \begin{minipage}[d]{0.48\linewidth}
    \centering
      \begin{overpic}[trim=0 280 0 0,clip,width=6.5cm]{/Users/javier/Desktop/Javier/PHD_RIT/ConferencesAndApplications/2015_SPIE_SanDiego/Images/2013262_ACOMOBSEAMUM_483_SeaDAS-MUMM_MoB-ELM.png}
      % \put (65,17) {\large D) $655nm$}
      \end{overpic}
  \end{minipage}

  \begin{minipage}[c]{0.48\linewidth}
      \centering
      \begin{overpic}[trim=0 280 0 0,clip,width=6.5cm]{/Users/javier/Desktop/Javier/PHD_RIT/ConferencesAndApplications/2015_SPIE_SanDiego/Images/2013262_ACOMOBSEAMUM_483_SeaDAS-SWIR_SeaDAS-MUMM.png}
      % \put (65,17) {\large C) $483nm$}
      \end{overpic}  
  \end{minipage}
  \hfill
  \begin{minipage}[d]{0.48\linewidth}
    \centering
      \begin{overpic}[trim=0 280 0 0,clip,width=6.5cm]{/Users/javier/Desktop/Javier/PHD_RIT/ConferencesAndApplications/2015_SPIE_SanDiego/Images/2013262_ACOMOBSEAMUM_483_SeaDAS-SWIR_MoB-ELM.png}
      % \put (65,17) {\large D) $655nm$}
      \end{overpic}
  \end{minipage}

  \begin{minipage}[d]{1.0\linewidth}
    \centering
      \begin{overpic}[trim=70 0 0 1470,clip,width=6.5cm]{/Users/javier/Desktop/Javier/PHD_RIT/ConferencesAndApplications/2015_SPIE_SanDiego/Images/2013262_ACOMOBSEAMUM_655_Acolite-SWIR_SeaDAS-MUMM.png}
      \end{overpic}
  \end{minipage}    

% 
  \caption{Scatter plots showing the comparison of remote-sensing reflectance ($R_{rs}$) at 483 nm, derived from the 09-29-2013 image over the Rochester Embayment (scene LC80160302013262LGN00) using the the different methods. Colors denote pixel densities, the dashed black line is the 1:1 line, and the Reduced Major Axis (RMA) regression line is drawn in red. \label{fig:13262Rrs483} } 
\end{figure}

%^^^^^^^^^^^^^^^^^^^  FIGURE ^^^^^^^^^^^^^^^^^^^^^^^^^^^^^^^^^^^^^^^^^^^^
\begin{figure}[htb]
  \begin{minipage}[c]{0.48\linewidth}
      \centering
      \begin{overpic}[trim=0 280 0 0,clip,width=6.5cm]{/Users/javier/Desktop/Javier/PHD_RIT/ConferencesAndApplications/2015_SPIE_SanDiego/Images/2013262_ACOMOBSEAMUM_561_Acolite-SWIR_SeaDAS-MUMM.png}
      % \put (65,17) {\large A) $443nm$}
      \end{overpic}  
  \end{minipage}
  \hfill
  \begin{minipage}[d]{0.48\linewidth}
    \centering
      \begin{overpic}[trim=0 280 0 0,clip,width=6.5cm]{/Users/javier/Desktop/Javier/PHD_RIT/ConferencesAndApplications/2015_SPIE_SanDiego/Images/2013262_ACOMOBSEAMUM_561_Acolite-SWIR_MoB-ELM.png}
      % \put (65,17) {\large B) $483nm$}
      \end{overpic}
  \end{minipage}

  \begin{minipage}[c]{0.48\linewidth}
      \centering
      \begin{overpic}[trim=0 280 0 0,clip,width=6.5cm]{/Users/javier/Desktop/Javier/PHD_RIT/ConferencesAndApplications/2015_SPIE_SanDiego/Images/2013262_ACOMOBSEAMUM_561_Acolite-SWIR_SeaDAS-SWIR.png}
      % \put (65,17) {\large C) $561nm$}
      \end{overpic}  
  \end{minipage}
  \hfill
  \begin{minipage}[d]{0.48\linewidth}
    \centering
      \begin{overpic}[trim=0 280 0 0,clip,width=6.5cm]{/Users/javier/Desktop/Javier/PHD_RIT/ConferencesAndApplications/2015_SPIE_SanDiego/Images/2013262_ACOMOBSEAMUM_561_SeaDAS-MUMM_MoB-ELM.png}
      % \put (65,17) {\large D) $655nm$}
      \end{overpic}
  \end{minipage}

  \begin{minipage}[c]{0.48\linewidth}
      \centering
      \begin{overpic}[trim=0 280 0 0,clip,width=6.5cm]{/Users/javier/Desktop/Javier/PHD_RIT/ConferencesAndApplications/2015_SPIE_SanDiego/Images/2013262_ACOMOBSEAMUM_561_SeaDAS-SWIR_SeaDAS-MUMM.png}
      % \put (65,17) {\large C) $561nm$}
      \end{overpic}  
  \end{minipage}
  \hfill
  \begin{minipage}[d]{0.48\linewidth}
    \centering
      \begin{overpic}[trim=0 280 0 0,clip,width=6.5cm]{/Users/javier/Desktop/Javier/PHD_RIT/ConferencesAndApplications/2015_SPIE_SanDiego/Images/2013262_ACOMOBSEAMUM_561_SeaDAS-SWIR_MoB-ELM.png}
      % \put (65,17) {\large D) $655nm$}
      \end{overpic}
  \end{minipage}

  \begin{minipage}[d]{1.0\linewidth}
    \centering
      \begin{overpic}[trim=70 0 0 1470,clip,width=8.0cm]{/Users/javier/Desktop/Javier/PHD_RIT/ConferencesAndApplications/2015_SPIE_SanDiego/Images/2013262_ACOMOBSEAMUM_655_Acolite-SWIR_SeaDAS-MUMM.png}
      \end{overpic}
  \end{minipage}    

% 
  \caption{Scatter plots showing the comparison of remote-sensing reflectance ($R_{rs}$) at 561 nm, derived from the 09-29-2013 image over the Rochester Embayment (scene LC80160302013262LGN00) using the the different methods. Colors denote pixel densities, the dashed black line is the 1:1 line, and the Reduced Major Axis (RMA) regression line is drawn in red. \label{fig:13262Rrs561} } 
\end{figure}

%^^^^^^^^^^^^^^^^^^^  FIGURE ^^^^^^^^^^^^^^^^^^^^^^^^^^^^^^^^^^^^^^^^^^^^
\begin{figure}[htb]
  \begin{minipage}[c]{0.48\linewidth}
      \centering
      \begin{overpic}[trim=0 280 0 0,clip,width=6.5cm]{/Users/javier/Desktop/Javier/PHD_RIT/ConferencesAndApplications/2015_SPIE_SanDiego/Images/2013262_ACOMOBSEAMUM_655_Acolite-SWIR_SeaDAS-MUMM.png}
      % \put (65,17) {\large A) $443nm$}
      \end{overpic}  
  \end{minipage}
  \hfill
  \begin{minipage}[d]{0.48\linewidth}
    \centering
      \begin{overpic}[trim=0 280 0 0,clip,width=6.5cm]{/Users/javier/Desktop/Javier/PHD_RIT/ConferencesAndApplications/2015_SPIE_SanDiego/Images/2013262_ACOMOBSEAMUM_655_Acolite-SWIR_MoB-ELM.png}
      % \put (65,17) {\large B) $483nm$}
      \end{overpic}
  \end{minipage}

  \begin{minipage}[c]{0.48\linewidth}
      \centering
      \begin{overpic}[trim=0 280 0 0,clip,width=6.5cm]{/Users/javier/Desktop/Javier/PHD_RIT/ConferencesAndApplications/2015_SPIE_SanDiego/Images/2013262_ACOMOBSEAMUM_655_Acolite-SWIR_SeaDAS-SWIR.png}
      % \put (65,17) {\large C) $655nm$}
      \end{overpic}  
  \end{minipage}
  \hfill
  \begin{minipage}[d]{0.48\linewidth}
    \centering
      \begin{overpic}[trim=0 280 0 0,clip,width=6.5cm]{/Users/javier/Desktop/Javier/PHD_RIT/ConferencesAndApplications/2015_SPIE_SanDiego/Images/2013262_ACOMOBSEAMUM_655_SeaDAS-MUMM_MoB-ELM.png}
      % \put (65,17) {\large D) $655nm$}
      \end{overpic}
  \end{minipage}

  \begin{minipage}[c]{0.48\linewidth}
      \centering
      \begin{overpic}[trim=0 280 0 0,clip,width=6.5cm]{/Users/javier/Desktop/Javier/PHD_RIT/ConferencesAndApplications/2015_SPIE_SanDiego/Images/2013262_ACOMOBSEAMUM_655_SeaDAS-SWIR_SeaDAS-MUMM.png}
      % \put (65,17) {\large C) $655nm$}
      \end{overpic}  
  \end{minipage}
  \hfill
  \begin{minipage}[d]{0.48\linewidth}
    \centering
      \begin{overpic}[trim=0 280 0 0,clip,width=6.5cm]{/Users/javier/Desktop/Javier/PHD_RIT/ConferencesAndApplications/2015_SPIE_SanDiego/Images/2013262_ACOMOBSEAMUM_655_SeaDAS-SWIR_MoB-ELM.png}
      % \put (65,17) {\large D) $655nm$}
      \end{overpic}
  \end{minipage}

  \begin{minipage}[d]{1.0\linewidth}
    \centering
      \begin{overpic}[trim=70 0 0 1470,clip,width=8.0cm]{/Users/javier/Desktop/Javier/PHD_RIT/ConferencesAndApplications/2015_SPIE_SanDiego/Images/2013262_ACOMOBSEAMUM_655_Acolite-SWIR_SeaDAS-MUMM.png}
      \end{overpic}
  \end{minipage}    

% 
  \caption{Scatter plots showing the comparison of remote-sensing reflectance ($R_{rs}$) at 655 nm, derived from the 09-29-2013 image over the Rochester Embayment (scene LC80160302013262LGN00) using the the different methods. Colors denote pixel densities, the dashed black line is the 1:1 line, and the Reduced Major Axis (RMA) regression line is drawn in red. \label{fig:13262Rrs655} } 
\end{figure}

%^^^^^^^^^^^^^^^^^^^  FIGURE ^^^^^^^^^^^^^^^^^^^^^^^^^^^^^^^^^^^^^^^^^^^^
\begin{figure}[htb]
  \begin{minipage}[c]{0.48\linewidth}
      \centering
      \begin{overpic}[trim=0 0 0 0,clip,width=7cm]{/Users/javier/Desktop/Javier/PHD_RIT/ConferencesAndApplications/2015_SPIE_SanDiego/Images/RrsCompONTNS.png}
      \put (20,60) {A) ONTNS} 
      \end{overpic}  
  \end{minipage}
  \hfill
  \begin{minipage}[d]{0.48\linewidth}
    \centering
      \begin{overpic}[trim=0 0 0 0,clip,width=7cm]{/Users/javier/Desktop/Javier/PHD_RIT/ConferencesAndApplications/2015_SPIE_SanDiego/Images/RrsCompONTOS.png}
      \put (20,14) {B) ONTOS}     
      \end{overpic}
  \end{minipage}
  
  \begin{minipage}[d]{0.48\linewidth}
    \centering
      \begin{overpic}[trim=0 0 0 0,clip,width=7cm]{/Users/javier/Desktop/Javier/PHD_RIT/ConferencesAndApplications/2015_SPIE_SanDiego/Images/RrsCompONTEX.png}
      \put (20,14) {C) ONTEX}   
      \end{overpic}
  \end{minipage}
  \hfill
  \begin{minipage}[c]{0.48\linewidth}
      \centering
      \begin{overpic}[trim=0 0 0 0,clip,width=7cm]{/Users/javier/Desktop/Javier/PHD_RIT/ConferencesAndApplications/2015_SPIE_SanDiego/Images/RrsCompRVRPL.png}
      \put (20,14) {D) RVRPLM}      
      \end{overpic}  
  \end{minipage}

  \begin{minipage}[c]{1.0\linewidth}
      \centering
      \begin{overpic}[trim=0 0 0 0,clip,width=7cm]{/Users/javier/Desktop/Javier/PHD_RIT/ConferencesAndApplications/2015_SPIE_SanDiego/Images/RrsCompRVRPI.png}
      \put (20,14) {E) RVRPIER}     
      \end{overpic}  
  \end{minipage}
  \caption{Comparison of $R_{rs}$ spectra with {\it in situ} data for the sites on the 09-19-2013 collection. \label{fig:13262RrsCompField}} 
\end{figure}

%^^^^^^^^^^^^^^^^^^^  FIGURE ^^^^^^^^^^^^^^^^^^^^^^^^^^^^^^^^^^^^^^^^^^^^
\begin{figure}[htb]
  \begin{minipage}[d]{0.48\linewidth}
    \centering
      \begin{overpic}[trim=0 0 0 0,clip,width=7cm]{/Users/javier/Desktop/Javier/PHD_RIT/ConferencesAndApplications/2015_SPIE_SanDiego/Images/RrsCompLONGN.png}
      \put (20,60) {A) LONGN}   
      \end{overpic}
  \end{minipage}
  \hfill
  \begin{minipage}[d]{0.48\linewidth}
    \centering
      \begin{overpic}[trim=0 0 0 0,clip,width=7cm]{/Users/javier/Desktop/Javier/PHD_RIT/ConferencesAndApplications/2015_SPIE_SanDiego/Images/RrsCompLONGS.png}
      \put (20,60) {B) LONGS}
      \end{overpic}
  \end{minipage}
  
   \begin{minipage}[c]{0.48\linewidth}
      \centering
      \begin{overpic}[trim=0 0 0 0,clip,width=7cm]{/Users/javier/Desktop/Javier/PHD_RIT/ConferencesAndApplications/2015_SPIE_SanDiego/Images/RrsCompCRANB.png}
      \put (45,14) {C) CRANB}     
      \end{overpic}  
  \end{minipage}
  \hfill
  \begin{minipage}[d]{0.48\linewidth}
    \centering
      \begin{overpic}[trim=0 0 0 0,clip,width=7cm]{/Users/javier/Desktop/Javier/PHD_RIT/ConferencesAndApplications/2015_SPIE_SanDiego/Images/RrsCompBRADI.png}
      \put (45,14) {D) BRADIN}
      \end{overpic}
  \end{minipage}
  
  \begin{minipage}[d]{1.0\linewidth}
    \centering
      \begin{overpic}[trim=0 0 0 0,clip,width=7cm]{/Users/javier/Desktop/Javier/PHD_RIT/ConferencesAndApplications/2015_SPIE_SanDiego/Images/RrsCompBRADO.png}
      \put (45,14) {E) BRADONT}
      \end{overpic}
  \end{minipage}    

% 
  \caption{Comparison of $R_{rs}$ spectra. \label{fig:13262RrsComp}} 
\end{figure}


%^^^^^^^^^^^^^^^^^^^  FIGURE ^^^^^^^^^^^^^^^^^^^^^^^^^^^^^^^^^^^^^^^^^^^^

\begin{figure}[htb]
  \begin{minipage}[c]{0.48\linewidth}
      \centering
      \begin{overpic}[trim=110 0 140 0,clip,width=6.5cm]{/Users/javier/Desktop/Javier/PHD_RIT/ConferencesAndApplications/2015_SPIE_SanDiego/Images/NRMSE_RRS_B1.png}
      \put (20,40) {A) 443nm} 
      \end{overpic}  
  \end{minipage}
  \hfill
  \begin{minipage}[d]{0.48\linewidth}
    \centering
      \begin{overpic}[trim=110 0 140 0,clip,width=6.5cm]{/Users/javier/Desktop/Javier/PHD_RIT/ConferencesAndApplications/2015_SPIE_SanDiego/Images/NRMSE_RRS_B2.png}
      \put (20,40) {B) 483nm}     
      \end{overpic}
  \end{minipage}

    \begin{minipage}[c]{0.48\linewidth}
      \centering
      \begin{overpic}[trim=110 0 140 0,clip,width=6.5cm]{/Users/javier/Desktop/Javier/PHD_RIT/ConferencesAndApplications/2015_SPIE_SanDiego/Images/NRMSE_RRS_B3.png}
      \put (20,40) {A) 561nm} 
      \end{overpic}  
  \end{minipage}
  \hfill
  \begin{minipage}[d]{0.48\linewidth}
    \centering
      \begin{overpic}[trim=110 0 140 0,clip,width=6.5cm]{/Users/javier/Desktop/Javier/PHD_RIT/ConferencesAndApplications/2015_SPIE_SanDiego/Images/NRMSE_RRS_B4.png}
      \put (20,40) {B) 655nm}     
      \end{overpic}
  \end{minipage}

  \caption{Comparison of the retrieved results for $R_{rs}$ with {\it in situ} data using the normalized root mean squared error (NRMSE) for $R_{rs}$ at 443, 483, 561 and 655nm. \label{fig:NRMSE130919_RRS} } 
\end{figure}



% ------------------------------
\section{Retrieval of Color Producing Agents (CPAs)}

% ------------ added from SPIE SD
\subsection{Spectral-Matching and LUT Approach}
\label{subsec:LUTapproach}
The first chlorophyll-{\it a} retrieval algorithm used in this study was the algorithm developed by \cite{Concha2013IGARSS}. This algorithm uses a spectrum-matching and look-up-table (LUT) methodology to extract CPA concentrations from $R_{rs}$ data. A LUT of $R_{rs}$ spectra is created in Hydrolight \cite{MobleyHE} with IOPs and CPA concentrations measured in the field as inputs. The parameters used to create this LUT are shown in \autoref{tab:LUTconc}. These parameters are chlorophyll-{\it a} ($C_a$), total suspended solid ($TSS$), CDOM absorption coefficient at $\lambda=440nm$ ($a_{CDOM}(440nm)$), and the backscatter fraction ($b_b/b$) for the scattering phase function selection.

\begin{table}[htb]
\caption{Input parameters for the LUT generation in Hydrolight for the Landsat 8 image acquired on 09-19-2013. These parameters are inherent optical properties (IOPs) for the lake (ONTNS) and the ponds (LONGS), chlorophyll-{\it a} ($C_a$), total suspended solid ($TSS$), CDOM absorption coefficient at $\lambda=440nm$ ($a_{CDOM}(440nm)$), and the backscatter fraction ($b_b/b$). \label{tab:LUTconc}} 
\vspace{.07cm}
\small
\centering
    \begin{tabular}{ccccc}
    \hline \hline
    \multirow{2}{*}{IOPs Input} & \bfseries{$C_a$} & \bfseries{$TSS$} & \bfseries{$a_{CDOM}(440nm)$} & \bfseries{$b_b/b$}    \\
                   & $[mg~m^{-3}]$        & $[g~m^{-3}]$       &  $[1/m]$           & $[\%]$            \\ \hline \hline
\multirow{8}{*}{ONTNS}  &  0.1      & 1.0     &  0.11   &  0.3  \\
    &  0.5      & 2.0   &  0.15     &  0.4  \\
    &  1.0      & 5.0   &  0.21     &  0.5  \\
    &  3.0      & 10.0  &  0.6      &  0.6  \\ 
    &  10.0     & --    &  --     &  0.7  \\  
    &  20.0     & --    &  --     &  1.0  \\  
    &  40.0     & --    &  --     &  1.4  \\
    &  --       & --    &  --     &  2.0  \\ \hline

\multirow{8}{*}{LONGS}   &  60.0   & 25.0    &  1.0    &  0.3  \\
    &  90.0   & 45.0    &  1.2    &  0.4  \\
    &  110.0  & 50.0    &  --     &  0.5  \\
    &  --     & --      &  --     &  0.6  \\  
    &  --     & --      &  --     &  0.7  \\  
    &  --     & --      &  --     &  1.0  \\   
    &  --     & --      &  --     &  1.4  \\  
    &  --     & --      &  --     &  2.0  \\  \hline \hline
% --      &  135.0  & --      &  --     &  --   \\  
% --      &  150.0  & --      &  --     &  --   \\ \hline 
    \end{tabular}
  \end{table}

Then, this algorithm finds the closest match in the least-squared sense for the $R_{rs}$ of water pixels with unknown CPA concentrations in the LUT of water pixels with known CPA concentrations. At this point, the closest element has discrete CPA concentration values corresponding to values used for the LUT generation (see \autoref{tab:LUTconc}). Finally, a non-linear interpolation is used to interpolate between the closest match's neighbors in order to obtain continuous values for the three CPAs simultaneously. In this study, this algorithm was applied to the $R_{rs}$'s results from the MoB-ELM method only.

The results of the retrieval for the 09-19-2013 scene is illustrated in the left-hand side of \autoref{fig:CPAsMaps130919} as a concentration map for each CPA. These maps exhibit the expected trend of low concentrations in the lake and higher concentrations in the ponds. The right-hand side of \autoref{fig:CPAsMaps130919} shows a comparison between measured concentration from the field and retrieved concentration obtained from the retrieval for each CPA for the different sites in the area of study. The dashed line represent the 1:1 line. 

\begin{figure}[htbp!]
  \begin{minipage}[c]{0.55\linewidth}
      \centering
      \includegraphics[trim=0 20 0 30,clip,height=6.0cm]{/Users/javier/Desktop/Javier/PHD_RIT/ConferencesAndApplications/2015_Landsat_Special_Issue/Images/CHLmap130919_150420}  
  \end{minipage}
  \hfill
  \begin{minipage}[d]{0.35\linewidth}
      \includegraphics[trim=40 0 0 20,clip,height=5.0cm]{/Users/javier/Desktop/Javier/PHD_RIT/ConferencesAndApplications/2015_Landsat_Special_Issue/Images/CHLretvsmea130919_150420}
  \end{minipage}
% 
%% TSS
  \begin{minipage}[c]{0.55\linewidth}
      \centering
      \includegraphics[trim=0 30 0 10,clip,height=6.0cm]{/Users/javier/Desktop/Javier/PHD_RIT/ConferencesAndApplications/2015_Landsat_Special_Issue/Images/TSSmap130919_150420}  
  \end{minipage}
  \hfill
  \begin{minipage}[d]{0.35\linewidth}
      \includegraphics[trim=40 0 0 0,clip,height=5.0cm]{/Users/javier/Desktop/Javier/PHD_RIT/ConferencesAndApplications/2015_Landsat_Special_Issue/Images/TSSretvsmea130919_150420}
  \end{minipage}

%% CDOM
  \begin{minipage}[c]{0.55\linewidth}
      \centering
      \includegraphics[trim=0 0 0 30,clip,height=6.0cm]{/Users/javier/Desktop/Javier/PHD_RIT/ConferencesAndApplications/2015_Landsat_Special_Issue/Images/CDOMmap130919_150420}  
  \end{minipage}
  \hfill
  \begin{minipage}[d]{0.35\linewidth}
      \includegraphics[trim=40 0 0 0,clip,height=5.0cm]{/Users/javier/Desktop/Javier/PHD_RIT/ConferencesAndApplications/2015_Landsat_Special_Issue/Images/CDOMretvsmea130919_150420}
  \end{minipage}
% 
  \caption{CPA retrieval results for the Landsat 8 image acquired on 09-19-2013 (scene LC80160302013262LGN00). Concentration maps (left) and measured vs retrieved plots (right). The dashed line represent the 1:1 line. (Labels: LongS: Long Pond south, Cranb: Cranberry Pond, OntOS: Lake Ontario off-shore, and OntNS: Lake Ontario near-shore). \label{fig:CPAsMaps130919} } 
\end{figure}

% Preliminary results for concentration maps for each CPA over the Rochester Embayment, Rochester, NY are shown in \autoref{fig:CPAsMaps}  The expected trend of having low concentration of CPAs in the offshore of Lake Ontario and higher concentrations in the nearby ponds (Long Pond and Cranberry Pond) can be seen. 

% Comparisons between retrieved CPAs concentrations and field measurements are shown in \autoref{fig:CPAsMaps} for four different stations in the area of study. This comparison with field measurements showed good agreement at low concentrations but differences at higher concentrations. Ongoing work is focusing on incorporation of the IOP differences between water bodies in the LUT optimization process.
% Landsat 8 images from this area of study and corresponding water samples collected at the time of the satellite's overpass will be used to test the retrieval algorithm. So far, there are only three satisfactory images available from the summer 2013. This project contemplates performing one new ground truth data collection during 2014. Therefore, images from the 2013-2014 spring and summer collection seasons will be used to test the methodology. Note that a difficult challenge of this research is to obtain images with relatively clear weather conditions (i.e. cloud free) over the area of study.

% \autoref{fig:091913Sites} shows an image over this area of study for the data collection done on September, $19^{th}$, 2013  with the different sites as an example. The data collections are divided in two crews. One crew, named ``Lake crew'', is in charge of the Irondequoit Bay, Ontario near shore, Ontario off shore, Genesee River plume, Genesee River pier sites (labeled in \autoref{fig:091913Sites} as IBayN, OntNS, OntOS, RvrPLM and RvrPIER, respectively). The other crew, named ``Ponds crew'', is in charge of the Long Pond north and south, Cranberry Pond sites (labeled in \autoref{fig:091913Sites} as LongN, LongS and Cranb, respectively).
%% CHL

%%%%%%%%%%%%%%%%%%%%%%%%%%%%%%%%%%%%%%%%%%%%%%%%%%%%%%%%%%%%%%
\begin{figure}[htbp!]
  \begin{minipage}[c]{1.0\linewidth}
      \centering
      \includegraphics[trim=0 0 0 30,clip,height=6.5cm]{/Users/javier/Desktop/Javier/PHD_RIT/ConferencesAndApplications/2015_Landsat_Special_Issue/Images/CHLmap140929_150420}  
  \end{minipage}\\
  % \hfill
  % \begin{minipage}[d]{0.45\linewidth}
      % \includegraphics[height=5.5cm]{/Users/javier/Desktop/Javier/PHD_RIT/ConferencesAndApplications/2015_Landsat_Special_Issue/Images/CHLretvsmea140929_150420}
  % \end{minipage}

%% TSS
  \begin{minipage}[c]{1.0\linewidth}
      \centering
      \includegraphics[trim=0 0 0 30,clip,height=6.5cm]{/Users/javier/Desktop/Javier/PHD_RIT/ConferencesAndApplications/2015_Landsat_Special_Issue/Images/TSSmap140929_150420}  
  \end{minipage}\\
  % \hfill
  % \begin{minipage}[d]{0.45\linewidth}
      % \includegraphics[height=5.5cm]{/Users/javier/Desktop/Javier/PHD_RIT/ConferencesAndApplications/2015_Landsat_Special_Issue/Images/TSSretvsmea140929_150420}
  % \end{minipage}

%% CDOM
  \begin{minipage}[c]{1.0\linewidth}
      \centering
      \includegraphics[trim=0 0 0 30,clip,height=6.5cm]{/Users/javier/Desktop/Javier/PHD_RIT/ConferencesAndApplications/2015_Landsat_Special_Issue/Images/CDOMmap140929_150420}  
  \end{minipage}\\
  % \hfill
  % \begin{minipage}[d]{0.45\linewidth}
      % \includegraphics[height=5.5cm]{/Users/javier/Desktop/Javier/PHD_RIT/ConferencesAndApplications/2015_Landsat_Special_Issue/Images/CDOMretvsmea140929_150420}
  % \end{minipage}

  \caption{CPA concentration maps for Landsat 8 image acquired on 09-29-2014 (scene LC80170302014272LGN00).\label{fig:CPAsMaps140929} } 
\end{figure}

% %%%%%%%%%%%%%%%%%%% SECTION %%%%%%%%%%%%%%%%%%%%%%%%%%%%%%%%
% \section{Discussion}

% Note that the image to be analyzed needs to only include a bright target and not necessary a big urban area in order to determine the bright pixel from the Landsat surface reflectance product.\\

A second image acquired on 09-29-2014 (scene LC80170302014272LGN00) (\autoref{fig:092914Sites}) was analyzed in a similar fashion except that the reflectance of the bright pixel target was also predicted by Ecolight using data for the most eutrophic pond (Cranberry Pond). This approach is attractive when a wide range of water quality conditions are present and concentration measurements are available of bright and dark water targets. This approach may improve results where glint is present as it compensates for not only atmospheric effects but any other approximately linear phenomena that would modify the reflected energy leaving the water volume. Note that glint effects were not specifically compensated for in the atmospheric correction process and future efforts using the urban region bright target approach will need to compensate for glint when it is present. The concentration maps for this second scene are illustrated in \autoref{fig:CPAsMaps140929} again showing reasonable patterns in the lake and ponds. \autoref{fig:CPAsRetVSMea} shows the comparison between predicted and observed values for all sample sites on both days, along with a regression line fitted to the data and its respective goodness of fit values. The $R^2$ value of the regression for all CPAs are high. Note that dark targets on both days and the bright target on the 2014 date are forced to match by the MoB-ELM process. The results are very encouraging showing good quantitative agreement across a very wide range of concentrations. 

The numerical results are summarized in \autoref{fig:RMSE} where the root mean squared error (RMSE) in predicted concentration is normalized by the range in concentrations to estimate overall performance, i.e. the normalized RMSE (NRMSE), defined as

\begin{equation}
\label{eq:error_percentage}
  NRMSE =\frac{\sqrt{\frac{1}{N}\sum_{n=1}^N{\left[C_{ret}(n) - C_{mea}(n)\right]^2}}}{max\{C_{mea}(n)\} - min\{C_{mea}(n)\}}\times100 ~[\%]
\end{equation}

\noindent where $C_{ret}$ is the retrieved CPA concentration (i.e. $C_a$, $TSS$ or $a_{CDOM}(440nm)$), $C_{mea}$ is the measured CPA concentration, and $n=1,...,N$ is the $n$th site from a total of $N$ sites for both days. The RMSE percentage of range values are approximately $10\%$ for $C_a$ and $TSS$, and about $5\%$ for $a_{CDOM}(440nm)$. These errors are consistent with the expected errors reported by \cite{Gerace:2013}. 

%%%%%%%%%%%%%%%%%%%%%%%%%%%%%%%%%%%%%%%%%%%%%%%%%%%%%%%%%%%%%%
\begin{figure}[htb]
  \begin{minipage}[c]{0.32\linewidth}
  % CHL
      \includegraphics[trim=40 0 80 0,clip,height=5.1cm]{/Users/javier/Desktop/Javier/PHD_RIT/ConferencesAndApplications/2015_Landsat_Special_Issue/Images/CHLretvsmea150423}  
  \end{minipage}
  % TSS
  \begin{minipage}[d]{0.32\linewidth}
      \includegraphics[trim=40 0 80 0,clip,height=5.1cm]{/Users/javier/Desktop/Javier/PHD_RIT/ConferencesAndApplications/2015_Landsat_Special_Issue/Images/TSSretvsmea150423}
  \end{minipage}
  % CDOM
  \begin{minipage}[c]{0.32\linewidth}
      \includegraphics[trim=40 0 80 0,clip,height=5.1cm]{/Users/javier/Desktop/Javier/PHD_RIT/ConferencesAndApplications/2015_Landsat_Special_Issue/Images/CDOMretvsmea150423}  
  \end{minipage}

  \caption{Landsat 8's retrieved vs measured CPA concentration for the 09-19-2014 and 09-29-2015 scenes with regression line (solid red line) and goodness of fit values. The dashed line represents the 1:1 line. \label{fig:CPAsRetVSMea} } 
\end{figure}

%%%%%%%%%%%%%%%%%%%%%%%%%%%%%%%%%%%%%%%%%%%%%%%%%%%%%%%%%%%%%%
\begin{figure}[htb]
  \centering
      \includegraphics[height=5cm]{/Users/javier/Desktop/Javier/PHD_RIT/ConferencesAndApplications/2015_Landsat_Special_Issue/Images/RMSE_ret150421}
      % \vspace{-.4cm}
      \caption{RMSE expressed as percentage of range for each CPA. \label{fig:RMSE}}
      % \vspace{-.4cm}
\end{figure}





% -----------------------------------------------------------
\subsection{Bio-Optical Algorithm for $C_a$ Retrieval}
\label{subsec:bioopticalapproach}
The second chlorophyll-{\it a} retrieval algorithm was the standard NASA algorithm OC3 \cite{OReilly2000}, which is a three-band empirical $R_{rs}(\lambda)$ band ratio algorithm, as suggested by \cite{Franz:2015}. Band ratio algorithms try to find a good fit between a band ratio $R$ and $C_a$. The chlorophyll-{\it a} results from the OC3 algorithm were obtained from the l2gen tool in SeaDAS when the SeaDAS-SWIR and SeaDAS-MUMM results were generated. Additionally, The OC3 algorithm was applied to the Acolite-SWIR results using the band math tool in SeaDAS. For the OLI case, the OC3 algorithm uses band 1 ($443nm$) or band 2 ($483nm$), and band 3 ($561nm$) for the band ratio, i.e.

\begin{equation}
\begin{gathered}
  C_a = 10^{\gamma}\\
  \gamma = a_0+a_1\chi+a_2\chi^2+a_3\chi^3+a_4\chi^4\\
  \chi = log_{10}(R)\\
  R = \frac{max(R_{rs}(443,483))}{R_{rs}(561)}
\end{gathered}
\end{equation}

\noindent where $a_0-a_4$ are the empirical regression coefficients. The empirical coefficients used in this study were the tuned values for OLI provided in SeaDAS (chloc3\_coef = [0.2412,-2.0546,1.1776,-0.5538,-0.4570]).

% --------------- end SPIE SD

\todo{Connect with previous paragraph} Preliminary results for concentration maps for each CPA over the Rochester Embayment, Rochester, NY are shown in Figure~\ref{fig:retrievalresults}. The expected trend of having low concentration of CPAs in the offshore of Lake Ontario and higher concentrations in the nearby ponds (Long Pond and Cranberry Pond) can be seen. 
\begin{figure}[htb]
\centering
\includegraphics[trim=200 100 180 0,clip,height=9cm]{/Users/javier/Desktop/Javier/PHD_RIT/ConferencesAndApplications/2014_RITResearchSymposium/Images/RetrievalResults.eps}
   \caption{Retrieval preliminary results.}
      \label{fig:retrievalresults}   
\end{figure}

Comparisons between retrieved CPAs concentrations and field measurements are shown in Figure~\ref{fig:chlcomp}, Figure~\ref{fig:tsscomp} and Figure~\ref{fig:cdomcomp} for four different stations in the area of study. This comparison with field measurements showed good agreement at low concentrations but differences at higher concentrations. Ongoing work is focusing on incorporation of the IOP differences between water bodies in the LUT optimization process.
\begin{figure}[htb]
\centering
    \includegraphics[height=7cm]{/Users/javier/Desktop/Javier/PHD_RIT/ConferencesAndApplications/IGARSS2014/paper/Images/chlcomp.eps} 
    \caption{Comparison between measured and retrieved chlorophyll {\it a} concentration.}
    \label{fig:chlcomp} 
\end{figure}     

\begin{figure}[htb]
\centering
    \includegraphics[height=7cm]{/Users/javier/Desktop/Javier/PHD_RIT/ConferencesAndApplications/IGARSS2014/paper/Images/tsscomp.eps}   
    \caption{Comparison between measured and retrieved TSS concentration.}
    \label{fig:tsscomp} 
\end{figure}  

\begin{figure}[htb]
\centering
    \includegraphics[height=7cm]{/Users/javier/Desktop/Javier/PHD_RIT/ConferencesAndApplications/IGARSS2014/paper/Images/cdomcomp.eps}    
    \caption{Comparison between measured and retrieved CDOM concentration.}
    \label{fig:cdomcomp} 
\end{figure}  

% ------------------------------
\subsection{$C_a$ Comparison}
The next step was to apply the algorithms described in Section~\ref{sec:retrieval} in order to retrieve chlorophyll-{\it a} concentration ($C_a$) from the $R_{rs}$ data. The spectral matching and LUT approach was applied to the $R_{rs}$ data retrieved from the MoB-ELM, while the OC3 algorithm was applied to data obtained from the rest of the algorithms. \autoref{fig:chlor_amaps} shows the $C_a$ maps obtained with the different methods. All the methods behave in a similar way for the clear water (lake), although some variability may not be visually noticeable due to the way the image is displayed. For the more turbid waters (ponds), the algorithms differ quite significantly. \autoref{fig:13262RrsMeaVSRet} shows a scatterplot of $C_a$ measured in the field versus the retrieved $C_a$ from the different algorithms. It can be seen that the retrieved values obtained from the MoB-ELM algorithm are closer to the measured values than the rest of the algorithms for the higher concentrations. However, the traditional algorithms generally perform well for the low concentration cases. Similarly, we calculated the normalized root mean squared error (NRMSE) for $C_a$, defined as

\begin{equation}
\label{eq:NRMSEchl}
  NRMSE =\frac{\sqrt{\frac{1}{N}\sum_{n=1}^N{\left[C_{ret}(n) - C_{mea}(n)\right]^2}}}{max\{C_{mea}(n)\} - min\{C_{mea}(n)\}}\times100 ~[\%]
\end{equation}

\noindent where $C_{ret}$ is the retrieved concentration, $C_{mea}$ is the measured concentration, and $n=1\dots N$ is the number of measured concentrations. The NRMSE for the $C_a$ for each algorithm are shown in \autoref{fig:NRMSE130919CHL}. The best results are obtained from the MoB-ELM method with overall values less than $10\%$.
%-------------
%^^^^^^^^^^^^^^^^^^^  FIGURE ^^^^^^^^^^^^^^^^^^^^^^^^^^^^^^^^^^^^^^^^^^^^
\begin{figure}[htb]
  \begin{minipage}[c]{0.48\linewidth}
      \centering
      \begin{overpic}[trim=0 0 40 0,clip,width=6.5cm]{/Users/javier/Desktop/Javier/PHD_RIT/ConferencesAndApplications/2015_SPIE_SanDiego/Images/Collocated13262_ACOSWIR_MOB_SEA5x5_MUMM45_chlor_MOB_D_R_R}
      \put (5,5) {A) MOB-ELM}
      \end{overpic}
    \end{minipage}
    \hfill
  \begin{minipage}[c]{0.48\linewidth}
      \centering
      \begin{overpic}[trim=0 0 40 0,clip,width=6.5cm]{/Users/javier/Desktop/Javier/PHD_RIT/ConferencesAndApplications/2015_SPIE_SanDiego/Images/LC80160302013262LGN00_L2_SWIR_FranzAve_chlor_a_ACO_OC3def}
      \put (5,5) {B) Acolite-SWIR}
      \end{overpic}
    \end{minipage}

    \vspace{0.7cm}

  \begin{minipage}[c]{0.48\linewidth}
      \centering
      \begin{overpic}[trim=0 0 40 0,clip,width=6.5cm]{/Users/javier/Desktop/Javier/PHD_RIT/ConferencesAndApplications/2015_SPIE_SanDiego/Images/Collocated13262_ACOSWIR_MOB_SEA5x5_MUMM45_chlor_a_SEA5x5_R}
      \put (5,5) {C) SeaDAS-SWIR}
      \end{overpic}
    \end{minipage}
    \hfill
  \begin{minipage}[c]{0.48\linewidth}
      \centering
      \begin{overpic}[trim=0 0 40 0,clip,width=6.5cm]{/Users/javier/Desktop/Javier/PHD_RIT/ConferencesAndApplications/2015_SPIE_SanDiego/Images/Collocated13262_ACOSWIR_MOB_SEA5x5_MUMM45_chlor_a_MUMM45}
      \put (5,5) {D) SeaDAS-MUMM}
      \end{overpic}
    \end{minipage}
    
    \begin{minipage}[c]{1.0\linewidth}
      \centering
      \vspace{0.5cm}
      \begin{overpic}[trim=0 0 0 0,clip,height=1.2cm]{/Users/javier/Desktop/Javier/PHD_RIT/ConferencesAndApplications/2015_SPIE_SanDiego/Images/Collocated13262_ACOSWIR_MOB_SEA5x5_MUMM45_colorbar_CHL_0_100}
      \put (35,16) {$C_a [mg/m^3]$}
      \end{overpic}
    \end{minipage}

  \caption{$C_a$ retrieved from the different atmospheric correction methods: A) MoB-ELM, B) Acolite-SWIR, C) SeaDAS-SWIR and D) SeaDAS-MUMM. The spectral matching and LUT approach was applied to $R_{rs}$ data from MoB-ELM method. The OC3 method was applied to the $R_{rs}$ data from the Acolite-SWIR, SeaDAS-SWIR and SeaDAS-MUMM.\label{fig:chlor_amaps} } 
\end{figure}

%^^^^^^^^^^^^^^^^^^^  FIGURE ^^^^^^^^^^^^^^^^^^^^^^^^^^^^^^^^^^^^^^^^^^^^
\begin{figure}[htb]
  \begin{minipage}[c]{1.0\linewidth}
    \centering
      \begin{overpic}[trim=0 0 0 0,clip,width=10cm]{/Users/javier/Desktop/Javier/PHD_RIT/ConferencesAndApplications/2015_SPIE_SanDiego/Images/CHLmeavsret.png}
      \put(17,62){\includegraphics[trim=10 80 0 0,clip,width=3.5cm]{/Users/javier/Desktop/Javier/PHD_RIT/ConferencesAndApplications/2015_SPIE_SanDiego/Images/CHLmeavsretZOOM.png}}
      \end{overpic}  
  \end{minipage}
  \caption{Comparison retrieved versus measured $R_{rs}$ for the sites on the 90-19-2013 collection. \label{fig:13262RrsMeaVSRet}} 
\end{figure}
% %^^^^^^^^^^^^^^^^^^^  FIGURE ^^^^^^^^^^^^^^^^^^^^^^^^^^^^^^^^^^^^^^^^^^^^
\begin{figure}[htb]
  \begin{minipage}[c]{0.48\linewidth}
      \centering
      \begin{overpic}[trim=0 250 0 0,clip,width=6.5cm]{/Users/javier/Desktop/Javier/PHD_RIT/ConferencesAndApplications/2015_SPIE_SanDiego/Images/2013262_ACOMOBSEAMUM_C_a_Acolite-SWIR_SeaDAS-MUMM.png}
      \end{overpic}  
  \end{minipage}
  \hfill
  \begin{minipage}[d]{0.48\linewidth}
    \centering
      \begin{overpic}[trim=0 250 0 0,clip,width=6.5cm]{/Users/javier/Desktop/Javier/PHD_RIT/ConferencesAndApplications/2015_SPIE_SanDiego/Images/2013262_ACOMOBSEAMUM_C_a_Acolite-SWIR_MoB-ELM.png}
      \end{overpic}
  \end{minipage}

  \begin{minipage}[c]{0.48\linewidth}
      \centering
      \begin{overpic}[trim=0 250 0 0,clip,width=6.5cm]{/Users/javier/Desktop/Javier/PHD_RIT/ConferencesAndApplications/2015_SPIE_SanDiego/Images/2013262_ACOMOBSEAMUM_C_a_Acolite-SWIR_SeaDAS-SWIR.png}
      \end{overpic}  
  \end{minipage}
  \hfill
  \begin{minipage}[d]{0.48\linewidth}
    \centering
      \begin{overpic}[trim=0 250 0 0,clip,width=6.5cm]{/Users/javier/Desktop/Javier/PHD_RIT/ConferencesAndApplications/2015_SPIE_SanDiego/Images/2013262_ACOMOBSEAMUM_C_a_SeaDAS-MUMM_MoB-ELM.png}
      \end{overpic}
  \end{minipage}

  \begin{minipage}[c]{0.48\linewidth}
      \centering
      \begin{overpic}[trim=0 250 0 0,clip,width=6.5cm]{/Users/javier/Desktop/Javier/PHD_RIT/ConferencesAndApplications/2015_SPIE_SanDiego/Images/2013262_ACOMOBSEAMUM_C_a_SeaDAS-SWIR_SeaDAS-MUMM.png}
      \end{overpic}  
  \end{minipage}
  \hfill
  \begin{minipage}[d]{0.48\linewidth}
    \centering
      \begin{overpic}[trim=0 250 0 0,clip,width=6.5cm]{/Users/javier/Desktop/Javier/PHD_RIT/ConferencesAndApplications/2015_SPIE_SanDiego/Images/2013262_ACOMOBSEAMUM_C_a_SeaDAS-SWIR_MoB-ELM.png}
      \end{overpic}
  \end{minipage}

  \begin{minipage}[d]{1.0\linewidth}
    \centering
      \begin{overpic}[trim=0 0 0 1500,clip,width=6.5cm]{/Users/javier/Desktop/Javier/PHD_RIT/ConferencesAndApplications/2015_SPIE_SanDiego/Images/2013262_ACOMOBSEAMUM_C_a_SeaDAS-SWIR_MoB-ELM.png}
      \end{overpic}
  \end{minipage}    

\vspace{.5cm}
  \caption{$C_a$ comparison among all the four method analyzed in this work. \label{fig:13262Chlor} } 
\end{figure}
%^^^^^^^^^^^^^^^^^^^  FIGURE ^^^^^^^^^^^^^^^^^^^^^^^^^^^^^^^^^^^^^^^^^^^^
\begin{figure}[htb]
  \centering
  \includegraphics[height=8.0cm]{/Users/javier/Desktop/Javier/PHD_RIT/ConferencesAndApplications/2015_SPIE_SanDiego/Images/13262_NRMSE_CHL.png}
  \caption{Comparison of the retrieved results for $C_a$ with {\it in situ} data using the normalized root mean squared error (NRMSE). The $R_{rs}$ results from the MoB-ELM were used for the Concha and Schott's retrieval of $C_a$, while the results of the rest of the atmospheric correction algorithms were used for the NASA's bio-optical algorithm. The MoB-ELM results combined with the Concha and Schott's $C_a$ retrieval give the best results.\label{fig:NRMSE130919CHL} } 
\end{figure}

% ------------------------------
\section{Concluding Remarks}